\subsection{
    Нахождение собственных значений линейного оператора (вывести характеристическое уравнение).
}

\begin{theorem}
    Для того чтобы действительное число $\lambda$ являлось собственным значением линейного оператора, необходимо и достаточно, чтобы оно было корнем характеристического уравнения этого оператора.
\end{theorem}

\begin{proof}~
    \begin{description}
        \item[$(\implies)$] 
            Пусть число $\lambda$ является собственным значением линейного оператора $\mathscr{A} \colon \mathcal{L} \to \mathcal{L}$. Это значит, что существует вектор $\vec{x} \ne \vec{0}$, для которого 

            $$\mathscr{A}\vec{x} = \lambda \vec{x}.$$

            Используя тождественный оператор $\mathscr{I}\vec{x} = \vec{x}$, преобразуем равенство: $\mathscr{A}\vec{x} = \lambda\mathscr{I}\vec{x}$, или
            
            $$(\mathscr{A} - \lambda\mathscr{I})\vec{x} = \vec{0}.$$

            Запишем векторное равенство выше в каком-либо базисе $b$. Матрицей линейного оператора $\mathscr{A} - \lambda\mathscr{I}$ будет матрица $A - \lambda E$, где $A$ - матрица линейного оператора $\mathscr{A}$ в базисе $b$, а $E$ - единичная матрица, и пусть $x$ - столбец координат собственного вектора $\vec{x}$. Тогда $x \ne 0$, а векторное равентсво выше равносильно матричному

            $$(A - \lambda E) = 0,$$

            которое представляет собой матричную форму записи ОСЛАУ с квадратной матрицей $A - \lambda E$ порядка $n$. Эта система имеет ненулевое решение, являющееся столбцом координат $x$ собственного вектора $\vec{x}$. Поэтому $\det(A - \lambda E) = 0$. А это означает, что $\lambda$ является корнем характеристического уравнения линейного оператора $\mathscr{A}$.
        \item[$(\impliedby)$]
            Приведенные рассуждения можно привести в обратном порядке. Если $\lambda$ является корнем характеристического уравнения, то в заданном базисе $b$ выполняется равенство $\det (A - \lambda E) = 0$. Следовательно, матрица ОСЛАУ, записанной в матричной форме, вырождена, и система имеет ненулевое решение $x$. Это ненулевое решение $x$ представляет собой набор координат в базисе $b$ некоторого ненулевого вектора $\vec{x}$, для которого выполняется равенство $(\mathscr{A} - \lambda\mathscr{I})\vec{x} = \vec{0}$. Значит, число $\lambda$ - собственное значение линейного оператора $\mathscr{A}$.
    \end{description}
\end{proof}
