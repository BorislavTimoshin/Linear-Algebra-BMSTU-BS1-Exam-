\subsection{
    *Полезные факты, которые тоже могут быть на экзамене.
}

\begin{definition}
    Квадратные матрицы $A$ и $B$ порядка $n$ называются \textit{\textbf{подобными}}, если существует такая невырожденная матрица $P$, что $P^{-1}AP = B$.
    \label{def:definition_17_1}
\end{definition}

\begin{theorem}
    Если матрицы $A$ и $B$ подобны, то $\det A = \det B$.
    \label{thm:theorem_25_1}
\end{theorem}

\begin{proof}~

    Если матрицы подобны, то согласно определению \eqref{def:definition_17_1}, существует такая невырожденная матрица $P$, что $B = P^{-1}AP$. Так как определитель произведения квадратных матриц равен произведению определителей этих матриц, а $\det(P^{-1}) = (\det P)^{-1}$, то получаем
    $$\det B = \det(P^{-1}AP) = \det(P^{-1})\det A \det P = \det(P)^{-1}\det A \det P = \det A.$$
\end{proof}

\begin{corollary}
    Определитель матрицы линейного оператора не зависит от выбора базиса.
\end{corollary}

\begin{proof}
    Действительно, возьмем матрицы $A_b$ и $A_e$ линейного оператора $\mathscr{A}$ в двух различных базисах $b$ и $e$.

    $$A_e = T^{-1}_{b \to e}A_bT_{b \to e}.$$

    Согласно определению \eqref{def:definition_17_1}, матрицы $A_b$ и $A_e$ подобны. Поэтому $\det A_b = \det A_e$ по теореме \ref{thm:theorem_25_1}.
\end{proof}
