\subsection{
    Характеристическое уравнение и характеристический многочлен линейного оператора.
}

Для произвольной квадратной матрицы $A = (a_{ij})$ порядка $n$ рассмотрим определитель

$$\det(A - \lambda E) = \begin{vmatrix} 
    a_{11} - \lambda & a_{12} & \ldots & a_{1n} \\
    a_{21} & a_{22} - \lambda & \ldots & a_{2n} \\
    \vdots & \vdots & \ddots & \vdots \ \\
    a_{n1} & a_{12} & \ldots & a_{nn} - \lambda \\
\end{vmatrix},$$

где $E$ - единичная матрица, а $\lambda$ - действительное переменное.

\begin{definition}
    Многочлен $\chi_A(\lambda) = \det(A - \lambda E)$ называют \textbf{\textit{характеристическим многочленом}} матрицы $A$, а уравнение $\chi_A(\lambda) = 0$ — \textbf{\textit{характеристическим уравнением}} матрицы $A$.
\end{definition}

\begin{definition}
    \textbf{\textit{Характеристическим многочленом линейного оператора}} $\mathscr{A} \colon \mathcal{L} \to \mathcal{L}$ называют характеристический многочлен его матрицы $A$, записанной в некотором базисе, а \textbf{\textit{характеристическим уравнением}} этого \textbf{\textit{оператора}} - характеристическое уравнение матрицы $A$.
\end{definition}
