$A' = U^TAU$, где $U$ - матрица перехода. Если рассматривается евклидово пространство, а старый и новый базис - ОНБ, то матрица $U$ является ортогональной и мы имеем ортогональное преобразование КФ.

\begin{theorem} 
    Любую квадратичную форму ортогональным преобразованием можно привести к диагональному виду.
\end{theorem}

\begin{proof}~

    Матрица $A$ любой КФ - симметричная. Любая симметричная матрица подобна некоторой диагональной (\textbf{*}желательно уметь доказывать), т.е. существует такая невырожденная матрица $P$, что матрица $A' = P^{-1}AP$ является диагональной. Остается убедиться, что в качестве $P$ можно выбрать ортогональную матрицу. Тогда $A' = P^TAP$ и диагональная $A'$ является матрицей квадратичной формы, полученной из исходной при помощи ортогонального преобразования.

    Рассмотрим произвольное $n$-мерное евклидово пространство $\mathcal{E}$ и некоторый ОНБ $b$ в нем. Матрица $A$ является матрицей некоторого самосопряженного оператора $\mathscr{A}$ в базисе $b$. Тогда существует такой ОНБ $e$, что матрица $A'$ оператора $\mathscr{A}$ в этом базисе диагональна (\textbf{*}желательно знать формулировку теоремы). Согласно формуле преобразования матрицы линейного оператора, имеем $A' = P_{b \to e}^{-1}AP_{b \to e}$. Так как оба базиса являются ОНБ, то матрица $P$ является ортогональной.
\end{proof}
