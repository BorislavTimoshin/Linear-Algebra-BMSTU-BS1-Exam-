\subsection{
    Инвариантное подпространство линейного оператора.
}

\begin{definition}
    Пусть $\mathscr{A} \colon \mathcal{L} \to \mathcal{L}$ - линейный оператор. Подпространство $\mathcal{V} \in \mathcal{L}$ называется \textbf{\textit{инвариантным подпространством}} оператора $\mathscr{A}$, если оператор $\mathscr{A}$ отображает всякий вектор $\vec{x} \in \mathcal{V}$, в вектор, также принадлежащий подпространству $\mathcal{V}$, то есть $\forall \vec{x} \in \mathcal{V} \colon \vec{y} = \mathscr{A}\vec{x} \in \mathcal{V}$.
\end{definition}

\begin{example}~

    \begin{itemize}
        \item Тривиальными примерами являются: само пространство $\mathcal{L}$ и нулевое подпространство (состоящее из единственного нулевого вектора).
        \item Любой собственный вектор оператора порождает его одномерное инвариантное подпространство.
        \item Ядро линейного оператора $\ker \mathcal{L}$. 
    \end{itemize}
\end{example}
