\subsection{
    Ранг и дефект, ядро и образ линейного оператора.
}

Пусть $\mathscr{A} \colon \mathcal{V} \to \mathcal{W}$ - линейный оператор.

\begin{definition}
    \textbf{\textit{Образом}} оператора $\mathscr{A}$ называется множество всех векторов $\vec{y} \in \mathcal{W}$, представимых в виде $\vec{y} = \mathscr{A}\vec{x}$.
\end{definition}

\begin{designation}
    $\Im \mathscr{A}$.
\end{designation}

\begin{definition}
    \textbf{\textit{Ядром}} оператора $\mathscr{A}$ называется множество всех векторов $\vec{x} \in \mathcal{V} \colon \mathscr{A}\vec{x} = \vec{0}_{\mathcal{W}}$.
\end{definition}

\begin{designation}
    $\ker \mathscr{A}$.
\end{designation}

\begin{definition}
    Размерность образа линейного оператора $\mathscr{A}$ называется \textbf{\textit{рангом}} линейного оператора $\mathscr{A}$. 
\end{definition}

\begin{designation}
    $\rank \mathscr{A}$.
\end{designation}

\begin{definition}
    Размерность ядра линейного оператора $\mathscr{A}$ называется \textbf{\textit{дефектом}} линейного оператора $\mathscr{A}$. 
\end{definition}

\begin{designation}
    $\defect \mathscr{A}$.
\end{designation}
