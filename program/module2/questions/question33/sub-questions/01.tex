\begin{definition}
    Квадратную матрицу $O$ называют \textbf{\textit{ортогональной}}, если она удовлетворяет условию

    \begin{equation}
        O^TO = E,
        \label{eq:equation_33_1}
    \end{equation}

    где $E$ — единичная матрица.
\end{definition}

\begin{example}
    Простейший пример - единичная матрица $E$, так как $E^TE = EE = E$.
\end{example}

\begin{example}
    $U = \begin{pmatrix}
    \cos \varphi & -\sin \varphi \\
    \sin \varphi & \cos \varphi
    \end{pmatrix}$.
\end{example}

\subsection*{Свойства ортогональных матриц.}

Пусть $O$ - ортогональная матрица.

\begin{enumerate}[label={\arabic*°.}]
    \item $\det O = \pm 1$.
    
    \begin{proof}~
    
        $\det(O^TO) = \det O^T \det O = (\det O)^2$.
        
        Так как $\det E = 1$, то и $(\det O)^2 = 1$. Следовательно, $\det O = \pm 1$.
    \end{proof}
    
    \item $O^{-1} = O^T$.

    \begin{proof}~
    
        Согласно свойству 1, ортогональная матрица невырождена и поэтому имеет обратную $O^{-1}$. Умножая равенство \eqref{eq:equation_33_1} справа на $O^{-1}$, получаем

        $$(O^TO)O^{-1} = EO^{-1},$$

        откуда $O^T(OO^{-1}) = O^{-1}$. Но $OO^{-1} = E$, поэтому $O^T = O^{-1}$.
    \end{proof}

    \item $OO^T = E$.

    \begin{proof}
        Согласно свойству 2 и определению обратной матрицы, $OO^T = OO^{-1} = E$.
    \end{proof}

    \item $O^T$ - тоже ортогональная.

    \begin{proof}~
    
        Нужно для произвольной ортогональной матрицы $O$ доказать равенство

        $$(O^T)^TO^T = E,$$

        представляющее собой запись соотношения \eqref{eq:equation_33_1} для предполагаемой ортогональной матрицы $O^T$ (\textbf{*}вместо $O$). Так как, согласно свойству операции транспонирования, $(O^T)^T = O$, равенство выше эквивалентно $OO^T = E$, которое верно в силу свойства 3.
    \end{proof}

    \item Произведение двух ортогональных матриц $O$ и $Q$ одного порядка является ортогональной матрицей.

    \begin{proof}
        Для доказательства достаточно проверить выполнение равенства \eqref{eq:equation_33_1} для матрицы $OQ$:

        $$(OQ)^T(OQ) = (Q^TO^T)OQ = Q^T(O^TO)Q = Q^TEQ = Q^TQ = E.$$
    \end{proof}

    \item $O^{-1}$ - тоже ортогональная.

    \begin{proof}~
    
        Согласно свойству 1, ортогональная матрица вырождена, а потому имеет обратную. Согласно свойству 2, матрица, обратная к ортогональной, совпадает с транспонированной. Наконец, согласно свойству 4, матрица, транспонированная к ортогональной, является ортогональной.
    \end{proof}
\end{enumerate}
