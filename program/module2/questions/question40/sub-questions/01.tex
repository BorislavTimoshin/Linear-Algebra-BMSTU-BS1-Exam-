Сначала сформулируем и докажем теоремы Фредгольма для операторов в линейных пространствах:

\subsection{
    Альтернатива Фредгольма.
}

\begin{theorem}["Альтернатива Фредгольма"] Пусть

    \begin{enumerate}
        \item $\left.\begin{array}{l}
            \mathcal{V} \text{ - линейное пространство, } \dim \mathcal{V} = n \\
            \mathcal{W} \text{ - линейное пространство, } \dim \mathcal{W} = m
        \end{array}\right\}$ \text{В обоих задано скалярное произведение.}
        \item $\mathscr{A} \colon \mathcal{V} \to \mathcal{W}$ - линейный оператор.
    \end{enumerate}

    Тогда справедливо ровно одно из двух:

    \begin{itemize}[nosep]
        \item либо уравнение $\mathscr{A}\vec{v} = \vec{w}$ имеет решение при любом $\vec{w} \in \mathcal{W}$,
        \item либо уравнение $\mathscr{A^*}\vec{w} = \vec{0}$ имеет нетривиальное (ненулевое) решение.
    \end{itemize}
\end{theorem}

\begin{proof}~

    Обозначим $r = \rank(\mathscr{A})$.

    \textbf{1 случай:}
    
    \begin{gather*}
        r = m \\
        \downimplies \\
        \dim(\Im \mathscr{A}) = m \\
        \downimplies \\
        \text{Т.к. } \Im \mathscr{A} \text{— } m\text{-мерное подпространство } m\text{-мерного пространства } \mathcal{W}, \text{то } \Im \mathscr{A} = \mathcal{W}. \\
        \downimplies \\
        \forall \vec{w} \in \mathcal{W} \exists \vec{v} \in \mathcal{V} \colon \mathscr{A}\vec{v} = \vec{w}. \\
        \text{Также заметим, что }\dim(\Im \mathscr{A^*}) + \dim(\ker \mathscr{A^*}) = m \\
        \downimplies \\
        \dim(\Im \mathscr{A}) + \dim(\ker \mathscr{A^*}) = m \\
        \downimplies \\
        m + \dim(\ker \mathscr{A^*}) = m \\
        \downimplies \\
        \dim(\ker \mathscr{A^*}) = 0 \\
        \downimplies \\
        \ker \mathscr{A^*} = \{\vec{0}\} \\
        \downimplies \\
        \text{Уравнение } \mathscr{A^*}\vec{w} = \vec{0} \text{ имеет только тривиальное решение.}
    \end{gather*}

    Получается, что второе "либо" не выполнено!
    
    \textbf{2 случай:}

    \begin{gather*}
        r < m \\
        \downimplies \\
        \dim(\Im \mathscr{A}) < \dim \mathcal{W} \\
        \downimplies \\
        \dim(\Im \mathscr{A^*}) < \dim \mathcal{W} \\
        \downimplies \\
        \text{Т.к. } \dim(\Im \mathscr{A^*}) + \dim(\ker \mathscr{A^*}) = \dim \mathcal{W}, \ker \mathscr{A^*} \ne \{\vec{0}\} \\
        \downimplies \\
        \exists \vec{w} \ne 0 \in \ker \mathscr{A^*} \\
        \downimplies \\
        \exists \vec{w} \ne 0 \colon \mathscr{A^*}\vec{w} = \vec{0}
    \end{gather*}
\end{proof}
