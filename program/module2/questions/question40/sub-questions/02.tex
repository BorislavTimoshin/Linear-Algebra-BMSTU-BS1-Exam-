\subsection{
    Теорема Фредгольма.
}


\begin{theorem}[Фредгольма] Пусть
    \begin{enumerate}
        \item $\left.\begin{array}{l}
                \mathcal{V} \text{ - линейное пространство, } \dim \mathcal{V} = n \\
                \mathcal{W} \text{ - линейное пространство, } \dim \mathcal{W} = m
            \end{array}\right\}$ \text{оба конечномерны, в обоих задано скалярное произведение.}   
        \item $\mathscr{A} \colon \mathcal{V} \to \mathcal{W}$ - линейный оператор.
        \item $\vec{w} \in \mathcal{W}$ - фиксированный элемент, причем $\vec{w} \ne \vec{0}$.
    \end{enumerate}

    Тогда уравнение $\mathscr{A}\vec{v} = \vec{w}$ имеет решение $\iff$ вектор $\vec{w}$ ортогонален всем решением уравнения $\mathscr{A^*}\vec{u} = \vec{0}$.
\end{theorem}

\begin{proof}~

    \begin{description}
        \item[$(\implies)$]~
        
        Пусть $\mathscr{A}\vec{v} = \vec{w}$ имеет решение, т.е. $\exists \vec{v}_0 \in \mathcal{V} \colon \mathscr{A}\vec{v}_0 = \vec{w}$.

            Рассмотрим произвольное решение $\vec{u}_0$ уравнения $\mathscr{A^*}\vec{u} = \vec{0}$ (т.е. $\mathscr{A^*}\vec{u}_0 = \vec{0}$).

            Получим:

            $$(\vec{w}, \vec{u}_0) = (\mathscr{A}\vec{v}_0, \vec{u}_0) = (\vec{v}_0, \mathscr{A^*}\vec{u}_0) = (\vec{v}_0, \vec{0}) = 0.$$

            Значит, $\vec{w} \perp \vec{u}_0$, ч.т.д.
        \item[$(\impliedby)$]~
        
        \textbf{($\star$)} Пусть $\forall \vec{u} \in \mathcal{W} \left(\mathscr{A^*}\vec{u} = \vec{0} \Rightarrow (\vec{u}, \vec{w}) = 0\right)$.

        Докажем, что $\mathscr{A}\vec{v} = \vec{w}$ имеет хотя бы одно решение.

        \textbf{($\star\star\star$)} Предположим, что $\mathscr{A}\vec{v} = \underbrace{\vec{w}}_{\mathclap{\substack{\text{фикс. в условии} \\ \text{теоремы} \\ \text{ненулевой} \\ \text{элемент}.}}}$ несовместна.

        Так как $\mathcal{W}$ - конечномерное (пусть $m$-мерное), то в нем найдется некоторый базис $\vec{e}_1, \ldots, \vec{e}_m$. Так как в $\mathcal{W}$ задано скалярное произведение, то базис $e$ можно ортогонализовать, а потом нормировать, т.е. в $\mathcal{W}$ найдется ОНБ $\vec{f}_1, \vec{f}_2, \ldots, \vec{f}_m$.

        Рассмотрим какой-нибудь $\vec{u} \in \mathcal{W}$ такой, что $\mathscr{A}\vec{u} = \vec{0}$.

         По условию \textbf{$(\star)$} из этого будет следовать, что $(\vec{u}, \underbrace{\vec{w}}_{\mathclap{\substack{\text{фикс. в условии} \\ \text{теоремы} \\ \text{ненулевой} \\ \text{элемент}.}}}) = 0$.

         Пусть в базисе $\vec{f}_1, \ldots, \vec{f}_m$ вектор $\vec{u}$, вектор $\vec{w}$ и оператор $\mathscr{A^*}$ имеют координаты $\vec{u}_f = \begin{pmatrix}
             u_1 \\
             u_2 \\
             \vdots \\
             u_m
         \end{pmatrix}$, $\vec{w}_f = \begin{pmatrix}
             w_1 \\
             w_2 \\
             \vdots \\
             w_m
         \end{pmatrix}$ и матрицу $A^*_f = \begin{pmatrix}
             A_1 & A_2 & \ldots & A_m
         \end{pmatrix} = \begin{pmatrix}
             a_{11} & a_{12} & \ldots & a_{1m} \\
             a_{21} & a_{22} & \cdots & a_{2m} \\
             \vdots & \vdots & \ddots & \vdots \\
             a_{n1} & a_{n2} & \ldots & a_{nm}
         \end{pmatrix}$.

         Так как $\mathscr{A^*}\vec{u} = \vec{0}$, то $A^*_f\vec{u}_f = \begin{pmatrix}
             0 \\
             0 \\
             \vdots \\
             0
         \end{pmatrix} \in \RR^n$, т.е.

         $$\begin{pmatrix}
             a_{11} & a_{12} & \ldots & a_{1m} \\
             a_{21} & a_{22} & \cdots & a_{2m} \\
             \vdots & \vdots & \ddots & \vdots \\
             a_{n1} & a_{n2} & \ldots & a_{nm}
         \end{pmatrix}\begin{pmatrix}
             u_1 \\
             u_2 \\
             \vdots \\
             u_m
         \end{pmatrix} = \begin{pmatrix}
             0 \\
             0 \\
             \vdots \\
             0
         \end{pmatrix}.$$

        $$\textbf{(}\star\star\textbf{)} \thinspace u_1 \cdot A^T_1 + u_2 \cdot A^T_2 + \ldots + u_m \cdot A^T_m = \underbrace{\begin{pmatrix}
             0 & \ldots & 0
         \end{pmatrix}}_{n}.$$

         Так как $(\vec{u}, \vec{w}) = 0$, то в силу ортонормированности $\vec{f}_1, \vec{f}_2, \ldots, \vec{f}_m$ имеем
         $$u_1w_1 + u_2w_2 + \ldots + u_mw_m = 0$$

         Значит, соотношение \textbf{$(\star \star)$} можно дополнить:

         $$u_1\cdot(A^T_1, \vec{w}_1) + u_2\cdot(A^T_2, \vec{w}_2) + \ldots + u_m\cdot(A^T_m, \vec{w}_m) = \underbrace{\begin{pmatrix}
             0 & 0 & \ldots & 0
         \end{pmatrix}}_{n + 1}.$$

         Это можно переписать иначе

         $$\begin{pmatrix}
                u_1 & u_2 & \ldots & u_m
            \end{pmatrix}
            \cdot
            \left(
                \begin{array}{c|c}
                    A_1^T & w_1 \\
                    A_2^T & w_2 \\
                    \vdots & \vdots \\
                    A_m^T & w_m
                \end{array}
            \right)
            =
            \underbrace{
                \begin{pmatrix}
                    0 & 0 & \ldots & 0
                \end{pmatrix}
            }_{n + 1}.$$
            Вспомним, что $\mathscr{A}\vec{v} = \vec{w}$ несовместно (предположение \textbf{$(\star\star\star)$} метода от противного). По т. Кронекера-Капелли 

            $$\rank\begin{pmatrix}
                    A^T_1 \\ 
                    A^T_2 \\ 
                    \vdots \\
                    A^T_m
                \end{pmatrix} \ne \rank \left(
                \begin{array}{c|c}
                    A_1^T & w_1 \\
                    A_2^T & w_2 \\
                    \vdots & \vdots \\
                    A_m^T & w_m
                \end{array}
            \right).$$

            Это означает, что если приводить обе матрицы одинаковыми преобразованиями над строками к ступенчатому виду, то количество "ступенек" у левой будет на 1 меньше, чем количество ступенек у правой. Это означает, что в ступенчатом виде правой матрицы будет строка

            $$\underbrace{\left(
                \begin{array}{cccc|c}
                    0 & 0 & \ldots & 0 & 1
                \end{array}
            \right)}_{n \text{ нулей и одна единица.}}.$$

            Мы знаем, что если над строками матрицы были совершены элементарные преобразования, то любую строку полученной матрицы можно представить в виде линейной комбинации строк исходной матрицы. Следовательно, найдутся такие $\lambda_1, \dots, \lambda_m$, что 

            \begin{equation}
                \lambda_1\cdot(A^T_1, \vec{w}_1) + \lambda_2\cdot(A^T_2, \vec{w}_2) + \ldots + \lambda_m\cdot(A^T_m, \vec{w}_m) = \begin{pmatrix}
                     0 & 0 & \ldots & 0 & 1
                 \end{pmatrix}.
                 \label{eq:equation_40_1}
            \end{equation}

            Откуда, беря только последний элемент строк, получим

            \begin{equation}
                \lambda_1\vec{w}_1 + \lambda_2\vec{w}_2 + \ldots + \lambda_m\vec{w}_m = 1.
                \label{eq:equation_40_2}
            \end{equation}
            Возьмем $\vec{z}_f = \begin{pmatrix}
                     \lambda_1 \\
                     \lambda_2 \\
                     \vdots \\
                     \lambda_m
                 \end{pmatrix}.$
            С одной стороны, из \eqref{eq:equation_40_1} следует, что $A^*_f\vec{z}_f = \left. \begin{pmatrix}
                 0 \\
                 0 \\
                 \vdots \\
                 0
             \end{pmatrix} \right\} n$, т.е. $A^*\vec{z} = \vec{0}$. А значит, согласно условию $(\vec{z}, \vec{w}) = 0$. С другой стороны, из \eqref{eq:equation_40_2} следует, что $(\vec{z}, \vec{w}) = 1$. 
             
             Противоречие.
    \end{description}
\end{proof}
