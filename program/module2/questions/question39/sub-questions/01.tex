По определению, квадратичной формой называется сумма вида

$$f(x_1, \ldots, x_n) = \sum_{i=1}^n b_{ii}x_i^2 + \sum_{1 \leq i < j \leq n} 2b_{ij}x_ix_j,$$

где $b_{ij}$ - заданные числа, $1 \leq i < j < n$.

Таким образом, у нас есть упорядоченный набор переменных $x_1, x_2, \ldots, x_n$, причем каждая переменная одного из трех типов:

\begin{itemize}
    \item \textbf{1 типа:} есть ненулевое слагаемое с квадратом это переменной и хотя бы одно слагаемое с первой степенью этой переменной;
    \item \textbf{2 типа:} нет слагаемого с квадратом этой переменной, но есть хотя бы одно ненулевое слагаемое с первой степенью этой переменной;
    \item \textbf{3 типа:} есть ненулевое слагаемое с квадратом этой переменной, но нет слагаемых с первой степенью этой переменной.
\end{itemize}

Суммарное количество переменных 1-го и 2-го типа назовем \textit{дефектом КФ}.

Например, 

\begin{enumerate}
    \item $f(x, y, z) = 2x^2 - y^2 + xz$.

    \begin{itemize}[nosep]
        \item $x$ — 1-го типа.
        \item $z$ — 2-го типа.
        \item $y$ — 3-го типа.
    \end{itemize}
    Дефект КФ равен 2.

    \item $f(x, y, z, t) = 4y^2-z^2-xy+3xt+2yt$.

    \begin{itemize}[nosep]
        \item $y$ — 1-го типа.
        \item $x, t$ — 2-го типа.
        \item $z$ — 3-го типа.
    \end{itemize}
    Дефект КФ равен 3.
\end{enumerate}

Докажем, что если дефект КФ больше нуля (т.е. в КФ есть хотя бы одна переменная 1-го или 2-го типа), то его можно понизить, сделав подходящую линейную замену.

\textbf{Случай 1: В КФ есть переменные 1-го типа.}

Соберем все слагаемые с этой переменной в скобку и выделим полный квадрат:

\begin{align}
    \text{КФ } &= (\underbrace{a}_{\ne 0}x^2 + \underbrace{2b_1xy_1 + 2b_2xy_2 + \ldots + 2b_kxy_k}_{\text{все эти слагаемые (их }\geq \text{ 1) ненулевые.}}) + \underbrace{\ldots}_{\star} = \\
    &= a(x^2 + 2\cdot x\cdot\frac{b_1}{a}y_1 + 2\cdot x\cdot\frac{b_2}{a}y_2 + \ldots + 2\cdot x\cdot\frac{b_k}{a}y_k) + \underbrace{\ldots}_{\star} = \\
    &= a\left((x + \frac{b_1}{a}y_1 + \ldots + \frac{b_k}{a}y_k)^2 - \sum_{i = 1}^k(\frac{b_i}{a}y_i)^2 - \sum_{1 \leq i < j \leq k}2\cdot\frac{b_i}{a}y_i\cdot \frac{b_j}{a}y_j\right) + \underbrace{\ldots}_{\star} = \\
    &= a(x + \frac{b_1}{a}y_1 + \ldots + \frac{b_k}{a}y_k)^2 + \sum_{i = 1}^k\underbrace{\left(-\frac{b^2_i}{a}\right)}_{\lambda_i \ne 0}y^2_i + \sum_{1 \leq i < j \leq k}\underbrace{\left(-\frac{2b_ib_j}{a}\right)}_{\mu_{ij} \ne 0}y_iy_j + \underbrace{\ldots}_{\star} = \\
    &=\left\{ 
            \begin{array}{l}
                \text{Замена} \\
                x' = x + \frac{b_1}{a}y_1 + \ldots + \frac{b_k}{a}y_k
            \end{array} \right\} = \\
    &= a(x')^2 + \underbrace{\sum_{i = 1}^k\lambda_iy^2_i}_{\star\star} + \underbrace{\sum_{1 \leq i < j \leq k}\mu_{ij}y_iy_j}_{\star\star\star} + \underbrace{\ldots}_{\text{нет }x'}.
\end{align}

\newpage

"$\star$" - В этом месте:

\begin{enumerate}
    \item нет $x$.
    \item могут быть $Ay_iy_j, 1\leq i < j \leq k$.
    \item могут быть $Ay^2_i, 1 \leq i \leq k$.
    \item могут быть слагаемые $Az_pz_q, 1 \leq p \leq q \leq m$ целиком из незадействованных в скобке переменных $z_1, z_2, \ldots, z_m$.
    \item могут быть $Ay_iz_j, 1\leq i\leq k, 1\leq j\leq m$.
\end{enumerate}

В новой квадратичной форме $x'$ - переменная 3-го типа.

\textbf{А что произошло с другими переменными - т.е. с $y_1, y_2, \ldots, y_k$ и "незадействованными" переменными $z_1, z_2, \ldots, z_m$?}

\begin{enumerate}
    \item Если в исходной КФ $y_1$ была переменной 1-го типа, то она станет либо 1-го, либо 2-го, либо 3-го типа (в зависимости от результата сложения $\star\star$ и $\star\star\star$);
    \item Если в исходной КФ $y_1$ была переменной 2-го типа, то она станет переменной либо 1-го, либо 3-го типа;
    \item $y_1$ не могла быть переменной 3-го типа;
    \item Если $z_1$ была переменной 1-го типа, то она ею и останется;
    \item Если $z_1$ была переменной 2-го типа, то она ею и останется;
    \item Если $z_1$ была переменной 3-го типа, то она ею и останется;
\end{enumerate}

Получается, что наша процедура как минимум на 1 увеличило число переменных 3-го типа $\Rightarrow$ дефект КФ уменьшился на 1.

\bigbreak

\textbf{Случай 2: В КФ нет переменных 1-го типа.} 

\bigbreak

\textbf{Подслучай 2а: В КФ есть переменная 2-го типа.}

В этом случае КФ содержит только слагаемые с разноименными переменными. Так как КФ $\ne 0$, то в ней есть хотя бы одно такое слагаемое; пусть, это в примеру, $ax_1x_2 (a \ne 0)$. Тогда КФ будет иметь вид:

\begin{align*}
    \text{КФ} &= ax_1x_2 + \\
    &+ b_3x_1x_3 + b_4x_1x_4 + \ldots + b_nx_1x_n + \\
    &+ c_3x_2x_3 + c_4x_2x_4 + \ldots + c_nx_2x_n + \\
    &+ \underbrace{\ldots}_{\text{Здесь нет }x_1, x_2\text{ но, возм., есть какие-то другие переменные.}} = \\
    &= \left\{\begin{array}{c}
        x_1 = v_1 + v_2 \\
        x_2 = v_1 - v_2 
    \end{array}\right\} = \\
    &= av^2_1 + (-a)v^2_2 + \\
    &+ b_3(v_1 + v_2)x_3 + \ldots + b_n(v_1 + v_2)x_n + \\
    &+ c_3(v_1 - v_2)x_3 + \ldots + c_n(v_1 - v_2)x_n + \\
    & + \underbrace{\ldots}_{\text{ничего не изменится.}} = \\
    &= av^2_1 + (-a)v^2_2 + \\
    &+ (b_3 - c_3)v_1x_3 + (b_4 - c_4)z_1x_4 + \ldots + (b_n - c_n)z_1x_n + \\ &+ \underbrace{\ldots}_{\text{ничего не изменится.}}
\end{align*}

где $b_i, c_j$ - любые (в том числе и нулевые)

Видим, что количество переменных 3-го типа выросло $\Rightarrow$ дефект КФ уменьшился.

\bigbreak

\textbf{Подслучай 2б: В КФ нет переменных 2-го типа.}

Этот случай невозможен, т.к. по условию дефект КФ положителен.
\bigbreak

Итак, мы предположили процедуру, благодаря которой в любой КФ с положительным дефектом можно уменьшить этот дефект. Последовательно применяя эту процедуру, можно получить КФ с нулевым дефектом, т.е. КФ канонического вида.
