\subsection{
    Линейный оператор, определение, три примера.
}

\begin{definition}
    Пусть $\mathcal{V}, \mathcal{W}$ - 
    линейные пространства над полем $\PP$, 
    $\dim \mathcal{V} = n, \dim \mathcal{W} = m$. 
    Отображение $\mathscr{A} \colon \mathcal{V} 
    \to \mathcal{W}$ называется 
    \textit{\textbf{линейным оператором}}, 
    если $\forall \vec{x}, \vec{y} \in 
    \mathcal{V}$ и $\forall \lambda \in \PP$ 
    выполнены следующие условия:
    
    \begin{enumerate}[nosep]
        \item $\mathscr{A}(\vec{x} + \vec{y}) = 
        \mathscr{A}(\vec{x}) + \mathscr{A}(\vec{y}),$
        \item $\mathscr{A}(\alpha\vec{x}) = 
        \alpha\mathscr{A}(\vec{x})$,
    \end{enumerate}
    где $\vec{x}$ - прообраз $\vec{y} = \mathscr{A}\vec{x}$,
    $\vec{y} = \mathscr{A}\vec{x}$ - образ $\vec{x}$.
\end{definition}

\begin{example}~

    \begin{enumerate}[nosep]
        \item Нулевой: $\zeroperator\vec{x} = \vec{0}$.
        \item Тождественный/единичный: 
        $\identityoperator\vec{x} = \vec{x}$.
        \item Оператор дифференцирования $\mathscr{D} \colon P_n(x) \to P_n(x)$, действующий по правилу $\mathscr{D}f = f'$.
    \end{enumerate}
\end{example}
