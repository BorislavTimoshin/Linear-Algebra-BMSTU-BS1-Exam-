\subsection{
    Матрица линейного оператора.
}

Пусть $\mathcal{V}$ и $\mathcal{W}$ - два линейных пространства.

Пусть $e = (\vec{e_1}, \ldots, \vec{e_n})$ - некоторый базис в $\mathcal{V}$, $f = (\vec{f_1}, \ldots, \vec{f_m})$ - некоторый базис в $\mathcal{W}$. 

Тогда $\mathscr{A}\vec{e_1}, \ldots, \mathscr{A}\vec{e_n}$ - это некоторые векторы в $\mathcal{W}$; значит, их можно, причем единственным образом, разложить по базису $\vec{f_1}, \ldots, \vec{f_m}$:

$$\mathscr{A}\vec{e_1} = a_{11}\vec{f_1} + \ldots + a_{m1}\vec{f_m}$$
$$\ldots \ldots \ldots \ldots \ldots \ldots \ldots \ldots \ldots$$
$$\mathscr{A}\vec{e_n} = a_{1n}\vec{f_1} + \ldots + a_{mn}\vec{f_m},$$

где $(a_{ij} \in \PP)$.

\begin{definition}
    Матрицу, составленную из координатных столбцов векторов $\mathscr{A}\vec{e_1}, \ldots, \mathscr{A}\vec{e_n}$ в базисе $f = (\vec{f_1}, \ldots, \vec{f_m})$, называют \textit{\textbf{матрицей линейного оператора}} $\mathscr{A}$ в базисе $f$.
\end{definition}
