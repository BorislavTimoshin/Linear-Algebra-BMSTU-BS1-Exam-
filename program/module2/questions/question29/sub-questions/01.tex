\subsection{
    Свойство собственных векторов линейного оператора, соответствующих одному и тому же собственному значению (с доказательством).
}

Если $\vec{x}$ - собственный вектор оператора $\mathscr{A}$, отвечающий собственному значению, то для любого числа $k \ne 0$ вектор $k\vec{x}$ также является собственным вектором оператора $\mathscr{A}$, отвечающим собственному значению.

\begin{proof}~

    $\mathscr{A}\vec{x} = \lambda\vec{x}$,
    
    $\mathscr{A}(k\vec{x}) = k\mathscr{A}\vec{x} = k\lambda\vec{x}$.
\end{proof}

Если $\vec{x}$ и $\vec{y}$ - собственные векторы оператора $\mathscr{A}$, отвечающие собственному значению, то вектор $\vec{x} + \vec{y} \ne \vec{0}$ также является собственным вектором, отвечающим собственному значению.

\begin{proof}~

    $\mathscr{A}\vec{x} = \lambda\vec{x}$,
    
    $\mathscr{A}\vec{y} = \lambda\vec{y}$,
    
    $\mathscr{A}(\vec{x} + \vec{y}) = \mathscr{A}\vec{x} + \mathscr{A}\vec{y} = \lambda\vec{x} + \lambda\vec{y} = \lambda(\vec{x} + \vec{y})$.
\end{proof}

\begin{corollary}[свойство]~

    Множество всех собственных векторов, отвечающих данному собственному значению, с добавлением $\vec{0}$ является линейным подпространством для данного пространства $\mathcal{L}$. Такое подпространство называется \textbf{\textit{собственным подпространством $\mathcal{L}$}}.
\end{corollary}
