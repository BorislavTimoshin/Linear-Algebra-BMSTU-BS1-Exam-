\subsection{
    Свойство собственных векторов линейного оператора, соответствующих различным собственным значениям (с доказательством).
}

\begin{theorem}
    Пусть собственные значения $\lambda_1, \ldots, \lambda_r$ линейного оператора $\mathscr{A}$ попарно различны. Тогда система соответствующих им собственных векторов $\vec{e}_1, \ldots, \vec{e}_r$ линейно независима.
\end{theorem}

\begin{proof}~

    Воспользуемся методом математической индукции.

    При $r = 1$ утверждение теоремы верно, так как линейная независимость системы из одного вектора означает, что этот вектор ненулевой, а собственный вектор, согласно его определению, является ненулевым.

    Пусть утверждение верно при $r = m$, т.е. для произвольной системы из $m$ собственных векторов $\vec{e}_1, \ldots, \vec{e}_m$. Добавим к системе векторов еще один собственный вектор $\vec{e}_{m + 1}$, отвечающий собственному значению $\lambda_{m + 1}$, и докажем, что расширенная таким способом система векторов останется линейно независимой. Рассмотрим произвольную линейную комбинацию полученной системы векторов и предположим, что она равна нулевому вектору:

    \begin{equation}
        \alpha_1\vec{e}_1 + \ldots + \alpha_m\vec{e}_m + \alpha_{m + 1}\vec{e}_{m + 1} = \vec{0}.
        \label{eq:equation_30_1}
    \end{equation}

    $$\alpha_1\mathscr{A}\vec{e}_1 + \ldots + \alpha_m\mathscr{A}\vec{e}_m + \alpha_{m + 1}\mathscr{A}\vec{e}_{m + 1} = \vec{0}.$$

    \begin{equation}
        \alpha_1\lambda_1\vec{e}_1 + \ldots + \alpha_m\lambda_m\vec{e}_m + \alpha_{m + 1}\lambda_{m + 1}\vec{e}_{m + 1} = \vec{0}.
        \label{eq:equation_30_2}
    \end{equation}

    Умножим равенство \eqref{eq:equation_30_1} на $\lambda_{m + 1}$ и вычтем из него равенство \eqref{eq:equation_30_2}

    $$\alpha_1(\lambda_1 - \lambda_{m + 1})\vec{e}_1 + \ldots + \alpha_m(\lambda_m - \lambda_{m + 1})\vec{e}_m = \vec{0}.$$

    Так как система векторов $\vec{e}_1, \ldots, \vec{e}_m$ по предположению, линейно независима, то у полученной линейной комбинации все коэффициенты равны нулю:

    \begin{equation}
        \alpha_k(\lambda_k - \lambda_{m + 1}) = 0, k = \overline{1, m}.
        \label{eq:equation_30_3}
    \end{equation}

    Поскольку все собственные значения $\lambda_i$ попарно различны, то из равенств \eqref{eq:equation_30_3} следует, что $\alpha_1 = \alpha_2 = \ldots = \alpha_m = 0$. Значит, соотношение \eqref{eq:equation_30_1} можно записать в виде $\alpha_{m + 1}\vec{e}_{m + 1} = \vec{0}$, а так как вектор $\vec{e}_{m + 1}$ ненулевой (как собственный вектор), то $\alpha_{m + 1} = 0$. В итоге получаем, что равенство \eqref{eq:equation_30_1} выполняется лишь в случае тривиальной линейной комбинации. Тем самым мы доказали, что система векторов $\vec{e}_1, \ldots, \vec{e}_m, \vec{e}_{m + 1}$ линейно независима.
\end{proof}
