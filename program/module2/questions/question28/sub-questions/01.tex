\begin{theorem}
    Характеристические многочлены (уравнения) подобных матриц совпадают.
    \label{thm:theorem_28_1}
\end{theorem}

\begin{proof}~

    Пусть квадратные матрицы $A$ и $A'$ одного порядка подобны, т.е. существует такая невырожденная матрица $P$ того же порядка, что $A' = P^{-1}AP$. Тогда в силу свойств определителей имеем

    \begin{align*}
        \chi_{A'}&(\lambda) = \det(A' - \lambda E) = \det(P^{-1}AP - \lambda P^{-1}EP) = \\
        &=\det(P^{-1}(A - \lambda E)P) = \det P^{-1}\det(A - \lambda E)\det P = \\
        &=\det(A - \lambda E) = \chi_A(\lambda).
    \end{align*}
\end{proof}

\begin{theorem}
    Характеристический многочлен и характеристическое уравнение линейного оператора не зависят от выбора базиса.
\end{theorem}

\begin{proof} Пусть 

    \begin{enumerate}[nosep]
        \item $\mathscr{A} \colon \mathcal{L} \to \mathcal{L}$ - линейный оператор,
        \item $A_b$ - матрица линейного оператора $\mathscr{A}$ в некотором ''старом'' базисе $b$, 
        \item $A_e$ - матрица линейного оператора $\mathscr{A}$  в некотором ''новом'' базисе $e$.
    \end{enumerate}
    
    Тогда 
    
    $$A_e = T^{-1}_{b \to e}A_bT_{b \to e},$$ где $T_{b \to e}$ - матрица перехода от базиса $b$ к базису $e$.

    $A_e$ и $A_b$ - две подобные матрицы.

    Значит, по теореме \ref{thm:theorem_28_1} характеристические многочлены двух подобных матриц $A_e$ и $A_b$ равны. Следовательно, и характеристический многочлен линейного оператора не зависит от выбора базиса.
\end{proof}
