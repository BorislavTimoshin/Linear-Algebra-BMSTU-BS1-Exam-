\subsection{
    Операции с линейными операторами.
}

\begin{definition}
    Операторы $\mathscr{A} \colon \mathcal{V} \to \mathcal{W}$ и $\mathscr{B} \colon \mathcal{V} \to \mathcal{W}$ называются \textbf{\textit{равными}}, если $\mathscr{A}\vec{x} = \mathscr{B}\vec{x}, \forall \vec{x} \in \mathcal{V}$.
\end{definition}

\begin{definition}
    \textbf{\textit{Суммой операторов}} $\mathscr{A} \colon \mathcal{V} \to \mathcal{W}$ и $\mathscr{B} \colon \mathcal{V} \to \mathcal{W}$ называется оператор $(\mathscr{A} + \mathscr{B}) \colon \mathcal{V} \to \mathcal{W}$, действующий по правилу $(\mathscr{A} + \mathscr{B})\vec{x} = \mathscr{A}\vec{x} + \mathscr{B}\vec{x}, \forall \vec{x} \in \mathcal{V}$.
\end{definition}

\begin{definition}
    \textbf{\textit{Произведением оператора}} $\mathscr{A} \colon \mathcal{V} \to \mathcal{W}$ \textbf{\textit{на действительное число}} $\lambda$ называется оператор $(\lambda\mathscr{A}) \colon \mathcal{V} \to \mathcal{W}$, действующий по правилу $(\lambda\mathscr{A})\vec{x} = \lambda(\mathscr{A}\vec{x}), \forall \vec{x} \in \mathcal{V}$.
\end{definition}

\begin{definition}
    \textbf{\textit{Произведением операторов}} $\mathscr{A} \colon \mathcal{V} \to \mathcal{W}$ и $\mathscr{B} \colon \mathcal{L} \to \mathcal{V}$ называется оператор $(\mathscr{A}\mathscr{B}) \colon \mathcal{L} \to \mathcal{W}$, действующий по правилу $(\mathscr{A}\mathscr{B})\vec{x} = \mathscr{A}(\mathscr{B}\vec{x}), \forall \vec{x} \in \mathcal{L}$.
\end{definition}
