\subsection{
    Ранг произведения операторов.
}

Для любых двух линейных операторов $\mathscr{A}$ и $\mathscr{B}$, действующих в линейном пространстве $\mathcal{L}$, выполняется соотношение

$$\rank(\mathscr{A}\mathscr{B}) \leq \min\{\rank\mathscr{A}, \rank\mathscr{B}\}$$

\begin{proof}~

    Рассмотрим оператор $\mathscr{A}$ как линейный оператор $\mathscr{A}\colon \Im\mathscr{B} \to \mathcal{L}$. Размерность образа оператора не превосходит размерности линейного пространства, из которого он действует, так как сумма и дефекта и ранга совпадает с размерностью этого пространства.

    $$\rank(\mathscr{A}\mathscr{B}) = \dim \Im (\mathscr{A}\mathscr{B}) \leq \dim \Im \mathscr{B} = \rank \mathscr{B}.$$
    
    Так как образ линейного оператора $\mathscr{A}\mathscr{B}$ является линейным подпространством образа линейного оператора $\mathscr{A}$, то
    
    $$\rank(\mathscr{A}\mathscr{B}) \leq \rank \mathscr{A}.$$
\end{proof}


\begin{comment}~

    Доказанное соотношение можно перенести на квадратные матрицы. 
    
    Получаем, 
    $$\rank{(AB)} \leq \min\{\rank A, \rank B\}.$$

    Пусть $B$ - невырожденная. То есть ее ранг равен размерности матрицы. 
    
    Тогда $\rank{(AB)} \leq \rank A$ и одновременно $\rank A = \rank ((AB)B^{-1}) \leq \rank (AB)$.

    То есть 
    
    $$\rank (AB) \leq \rank A \leq \rank (AB).$$
    
    Следовательно, при умножении матрицы $A$ справа на невырожденную матрицу ее ранг не изменяется. 
    
    При умножении матрицы $A$ слева на невырожденную матрицу ранг также не изменяется, что доказывается аналогично.
    \label{comment:comment_24_2}
\end{comment}
