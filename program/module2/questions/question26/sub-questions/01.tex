\subsection{
    Формулировка теоремы Гамильтона – Кэли.
}

Квадратную матрицу можно использовать в качестве значения переменного в произвольном многочлене. Тогда значением многочлена от матрицы будет матрица того же порядка, что и исходная. Интерес представляют такие многочлены, значение которых от данной матрицы есть нулевая матрица. Их называют аннулирующими многочленами. Оказывается, что одним из таких аннулирующих многочленов для матрицы является ее характеристический многочлен.

\begin{theorem}
    Для любой квадратной матрицы характеристический многочлен является ее аннулирующим многочленом.
\end{theorem}
