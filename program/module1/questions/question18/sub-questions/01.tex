\subsection{
    Евклидовы и унитарные пространства. Три примера.
}

\textbf{Примеры $\mathcal{E}$:}
\begin{enumerate}
    \item В линейных пространствах $\mathcal{V}_2$ и $\mathcal{V}_3 \colon (\vec{x}, \vec{y}) = |\vec{x}||\vec{y}|\cos \widehat{(\vec{x}, \vec{y})}$.
    \item В арифметическом линейном пространстве $\RR^n \colon (\vec{x}, \vec{y}) = x_1y_1 + \ldots + x_ny_n$. 
    \item Линейное пространство $C[0, 1]$ всех функций, непрерывных на отрезке $[0, 1]$ становится евклидовым, если в нем ввести скалярное произведение:
    $$(\vec{f}, \vec{g}) = \int_{0}^{1} f(x)g(x) \dd x.$$
\end{enumerate}

\textbf{Примеры $\mathcal{U}$:}

\begin{enumerate}
    \item $\CC^n \colon (\vec{x}, \vec{y}) = x_1\overline{y}_1 + x_2\overline{y}_2 + \ldots + x_n\overline{y}_n$.
\end{enumerate}
