\subsection{
    Неравенство Коши-Буняковского (доказательство для $\mathcal{E}$).
}


\begin{theorem}
    Для любых векторов $\vec{x}, \vec{y}$ евклидова пространства справедливо неравенство Коши-Буняковского

    \begin{equation}
        (\vec{x}, \vec{y})^2 \leq (\vec{x}, \vec{x}) (\vec{y}, \vec{y}).
        \label{equition:equition_18_1}
    \end{equation}
\end{theorem}

\begin{proof}~

    При $\vec{x} = \vec{0}$ обе части неравенства \eqref{equition:equition_18_1} равны нулю согласно свойствам скалярного произведения, значит, неравенство выполняется. Отбрасывая этот очевидный случай, будем считать, что $\vec{x} \ne \vec{0}$. Для любого действительного числа, в силу аксиомы 4 скалярного произведения, выполняется неравенство

    $$(\lambda \vec{x} - \vec{y}, \lambda \vec{x} - \vec{y}) \geq 0.$$

    Преобразуем левую часть неравенства, используя аксиомы и свойства скалярного произведения:

    \begin{align*}
        (\lambda \vec{x} - \vec{y}, \lambda \vec{x} - \vec{y}) &= \lambda (\vec{x}, \lambda \vec{x} - \vec{y}) - (\vec{y}, \lambda\vec{x} - \vec{y}) = \\
        &= \lambda^2\underbrace{(\vec{x}, \vec{x})}_{\ne 0} - 2\lambda(\vec{x}, \vec{y}) + (\vec{y}, \vec{y}) \geq 0.
    \end{align*}

    Мы получили квадратным трехчлен относительно параметра $\lambda$, неотрицательный при всех действительных значениях параметра. Следовательно, его дискриминант равен нулю или отрицательный, т.е.

    $$(\vec{x}, \vec{y})^2 - (\vec{x}, \vec{x})(\vec{y}, \vec{y}) \leq 0.$$
\end{proof}
