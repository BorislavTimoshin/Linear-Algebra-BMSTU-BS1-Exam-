\subsection{
    Неравенство Коши-Буняковского (доказательство для $\mathcal{U}$).
}


\begin{proof}~
    
    При $\vec{x} = \vec{0}$ обе части неравенства равны нулю, значит, неравенство выполняется. Отбрасывая этот очевидный случай, будем считать, что $\vec{x}, \vec{y} \ne \vec{0}$, $(\vec{x}, \vec{y}) \ne 0$, $t \in \CC$.
    
    $$(\vec{x} + t\vec{y}, \vec{x} + t\vec{y}) = (\vec{x}, \vec{x}) + \underbrace{t(\vec{y}, \vec{x}) + \overline{t}(\vec{x}, \vec{y})}_{\star} + t \cdot \overline{t}(\vec{y}, \vec{y}) = (\vec{x}, \vec{x}) + \underbrace{2 \text{Re}(t(\vec{y}, \vec{x}))}_{\star} + |t|^2(\vec{y}, \vec{y}) = \underbrace{(\vec{y}, \vec{y})|t|^2}_{\in \RR} + \underbrace{2|(\vec{x}, \vec{y})||t|}_{\in \RR} + \underbrace{(\vec{x}, \vec{x})}_{\in \RR}.$$

    Выберем $t$ так, чтобы $\arg{t} = -\arg{(y, x)}$, тогда $\arg{(y, x)} = -\arg{t}$.

    \begin{gather*}
        t = |t|e^{i\varphi} \\
        \text{Так как }\arg{(y, x)} = -\arg{t}, \text{ то } \\
        (y, x) = |(x, y)|e^{i(-\varphi)}.
    \end{gather*}

    Итак, 
    
    $$2 \cdot \text{Re} (t \cdot (\vec{y}, \vec{x})) = 2 \cdot \text{Re} (|t|e^{i\varphi} \cdot |(\vec{x}, \vec{y})|e^{-i\varphi}) = 2|t||(\vec{y}, \vec{x})|.$$

    Тогда

    \begin{gather*}
        \underbrace{(\vec{y}, \vec{y})|t|^2}_{\in \RR} + \underbrace{2|(\vec{x}, \vec{y})||t|}_{\in \RR} + \underbrace{(\vec{x}, \vec{x})}_{\in \RR} \geq 0 \\
        D = |(\vec{x}, \vec{y})|^2 - (\vec{x}, \vec{x})(\vec{y}, \vec{y}) \leq 0.
    \end{gather*}

    \bigbreak
    
    $\text{\textbf{Примечание} } (\star) \colon$ Пусть $t = p + iq, (\vec{y}, \vec{x}) = a + ib$, $\overline{t} = p - iq, \overline{(\vec{y}, \vec{x})} = a - ib$. Тогда

    \begin{enumerate}
        \item \begin{align*}
            t(\vec{y}, \vec{x}) &+ \overline{t} \cdot (\vec{x}, \vec{y}) = t(\vec{y}, \vec{x}) + \overline{t} \cdot \overline{(\vec{y}, \vec{x})} = \\
            &= (p + iq)(a + ib) + (p - iq)(a - ib) = \\ 
            &= pa + pib + aiq - qb + pa - pib - aiq - qb = 2pa - 2qb = \\
            &= 2 \cdot (pa - qb).
        \end{align*}
        \item \begin{align*}
            2 \cdot \text{Re} (t \cdot (\vec{y}, \vec{x})) &= 2 \cdot \text{Re}((p + iq)(a + ib)) = \\
            &= 2 \cdot \text{Re}(pa - qb + i(pb + aq)) = \\
            &= 2 \cdot (pa - qb).
        \end{align*}
        \item \begin{align*}
            t(\vec{y}, \vec{x}) + \overline{t} \cdot (\vec{x}, \vec{y}) = 2 \cdot \text{Re} (t \cdot (\vec{y}, \vec{x})).
        \end{align*}
    \end{enumerate}
\end{proof}
