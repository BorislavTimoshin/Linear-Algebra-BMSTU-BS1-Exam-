\subsection{
    Неравенство Коши-Буняковского (Шварца).
}

\begin{theorem}
    Для любых векторов $\vec{x}, \vec{y} \in \mathcal{E}$ (или $\mathcal{U}$) справедливо неравенство Коши-Буняковского
    $$|(\vec{x}, \vec{y})|^2 \leq (\vec{x}, \vec{x}) (\vec{y}, \vec{y}),$$
    причем $|(\vec{x}, \vec{y})|^2 = (\vec{x}, \vec{x}) (\vec{y}, \vec{y}) \iff \vec{x} \parallel \vec{y}.$
\end{theorem}

\begin{corollary}~

    В случае линейного арифметического пр-ва $\RR^n$ неравенство Коши-Буняковского трансформируется в \textbf{неравенство Коши}:
    $$(a_1b_1 + \ldots + a_nb_n)^2 \leq (a_1^2 + \ldots + a_n^2)(b_1^2 + \ldots + b_n^2).$$
    Равенство достигается при линейной зависимости векторов, т.е. $\frac{a_i}{b_i} = const, \quad i = \overline{1, n}$.
\end{corollary}

\begin{corollary}~

    В евклидовом пространстве $C[0, 1]$, скалярное произведение в котором выражается определенным интегралом, неравенство Коши-Буняковского превращается в неравенство Буняковского-Шварца:
    $$\left( \int_0^1 f(x)g(x) \, dx \right)^2 \le \left( \int_0^1 f(x)^2 \, dx \right) \left( \int_0^1 g(x)^2 \, dx \right).$$
\end{corollary}
