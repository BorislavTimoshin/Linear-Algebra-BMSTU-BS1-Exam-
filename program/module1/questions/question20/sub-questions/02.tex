\subsection{
    Три примера задания нормы.
}


\begin{definition}
    Норма вида $\norm{\vec{x}}_2 = \sqrt{(\vec{x}, \vec{x})}$ называется \textbf{\textit{евклидовой}} $(l_2)$
\end{definition}

\begin{definition}
    Норма вида $\norm{\vec{x}}_1 = |x_1| + \dots + |x_n|$ называется \textbf{\textit{октаэдрической}} $(l_1)$
\end{definition}

\begin{definition}
    Норма вида $\norm{\vec{x}}_{\infty} = max\{|x_1|, \dots, |x_n|\}$ называется \textbf{\textit{кубической}} $(l_{\infty})$
\end{definition}
