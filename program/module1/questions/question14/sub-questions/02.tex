\subsection{
    Пример решения.
}

$\vec{a}_1 = \begin{pmatrix} 1 \\ -2 \\ 2 \end{pmatrix}$, $\vec{a}_2 = \begin{pmatrix} 3 \\ 1 \\ -3 \end{pmatrix}$, $\vec{b} = \begin{pmatrix} 16 \\ 6 \\ 4 \end{pmatrix}$

$A = \begin{pmatrix} \vec{a}_1 & \vec{a}_2 \end{pmatrix} = \begin{pmatrix} 1 & 3 \\ -2 & 1 \\ 2 & -3 \end{pmatrix}.$

Найдем элементы матрицы Грама $\Gamma(\vec{a}_1, \vec{a}_2) = A^TA$:

\begin{gather*}
    (\vec{a}_1, \vec{a}_1) = 1 + 4 + 4 = 9 \\
    (\vec{a}_1, \vec{a}_2) = 3 - 2 - 6 = -5 \\
    (\vec{a}_2, \vec{a}_2) = 9 + 1 + 9 = 19
\end{gather*}


Найдем элементы матрицы $A\vec{b}$:

\begin{gather*}
    (\vec{a}_1, \vec{b}) = 16 - 12 + 8 = 12 \\
    (\vec{a}_2, \vec{b}) = 48 + 6 - 12 = 42
\end{gather*}

Найдем решение СЛАУ $A^TA\vec{x} = A\vec{b}$:

\begin{equation*}
    \left(\begin{array}{cc|c}
        9 & -5 & 12 \\
        -5 & 19 & 42
    \end{array}\right)
    \sim
    \left(\begin{array}{cc|c}
        -1 & 33 & 96 \\
        0 & -146 & -438
    \end{array}\right)
    \sim
    \left(\begin{array}{cc|c}
        1 & -33 & -96 \\
        0 & 1 & 3
    \end{array}\right)
    \sim
    \left(\begin{array}{cc|c}
        1 & 0 & 3 \\
        0 & 1 & 3
    \end{array}\right)
.\end{equation*}

$\vec{x} = 3\vec{a}_1 + 3\vec{a}_2 = \begin{pmatrix} 12 \\ -3 \\ -3 \end{pmatrix}$ – наилучшее приближение вектора $\vec{b}$.

Абсолютная ошибка $= \norm{\vec{b} - \vec{x}} = \norm{\begin{pmatrix} 16 \\ 6 \\ 4 \end{pmatrix} - \begin{pmatrix} 12 \\ -3 \\ -3 \end{pmatrix}} =  \norm{\begin{pmatrix} 4 \\ 9 \\ 7 \end{pmatrix}} =  \sqrt{146}$.

Относительная ошибка $= \frac{\norm{\vec{b} - \vec{x}}}{\norm{\vec{b}}} = \frac{\sqrt{146}}{\sqrt{308}} = \frac{\sqrt{11242}}{154}$.
