\subsection{
    Линейные подпространства. Определения, теоремы, примеры и контрпримеры.
}

\begin{definition}
    Подмножество $\mathcal{L}_1 \subset \mathcal{L}$ называется \textbf{\textit{линейным подпространством}} над полем $\PP$, если \\ $\forall \vec{a}, \vec{b} \in \mathcal{L}_1, \forall \alpha, \beta \in \PP \colon \alpha \vec{a} + \beta \vec{b} \in \mathcal{L}$.
\end{definition}

\begin{theorem}
    Линейное подпространство является линейным пространством.
\end{theorem}

\begin{example}~

    \begin{enumerate}
        \item В любом линейном пространстве $\mathcal{L}$ всегда имеются два линейных подпространства: само пространство $\mathcal{L}$ и нулевое подпространство, состоящее из одного нулевого элемента. Эти подпространства называются \textbf{\textit{несобственными}}. Все остальные линейные пространства называются \textbf{\textit{собственными}}.
        \item Множество всех свободных векторов, параллельных данной плоскости, образуют линейное подпространство пространства $\mathcal{V}_3$ всех свободных векторов трехмерного пространства.
        \item В линейном пространстве $M_n(\RR)$ всех квадратных матриц порядка $n$ линейное подпространство образуют все симметрические матрицы.
    \end{enumerate}
\end{example}

\begin{counterexample}~

    \begin{enumerate}
        \item Множество всех векторов на плоскости, у которых первая координата положительна. Это множество не является подпространством, потому что оно не замкнуто относительно умножения на скаляр. Например, вектор $(1, 1)$ принадлежит этому множеству, но вектор $-1 \cdot (1, 1) = (-1, -1)$ — нет, так как первая координата отрицательна.
        \item Множество всех векторов в $\RR^3$, лежащих в первой октанте (где все координаты неотрицательны). Это множество не замкнуто относительно умножения на скаляр. Например, вектор $(1, 1, 1)$ принадлежит этому множеству, но вектор $-1 \cdot (1, 1, 1) = (-1, -1, -1)$ не принадлежит.
    \end{enumerate}
\end{counterexample}
