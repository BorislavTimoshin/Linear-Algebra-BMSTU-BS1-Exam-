\subsection{
    Привести примеры задания пространств и подпространств без использования матриц и СЛАУ.
}

\begin{enumerate}
    \item В любом линейном пространстве $\mathcal{L}$ всегда имеются два линейных подпространства: само пространство $\mathcal{L}$ и нулевое подпространство, состоящее из одного нулевого элемента.
    \item Множество всех свободных векторов, параллельных данной плоскости, образуют линейное подпространство пространства $\mathcal{V}_3$ всех свободных векторов трехмерного пространства.
    \item Любая прямая, проходящая через начало координат $(0, 0, 0)$ в $\RR^3$.
    \item Линейное пространство - множество $P_n[x]$ многочленов переменного $x$ степени, не превышающей $n$. Для данного линейного пространства линейным подпространтсвом является множество $K_m[x]$ многочленов переменного $x$ степени, не превышающей $m$, где $m \leq n$. При этом подпространством не является множество всех многочленов степени ровно $m$. Например, сумма двух многочленов степени $m$ может иметь степень меньше $m$ (например, x² + (-x²) = 0).
    \item Множество функций, непрерывных на отрезке, с обычными операциями сложения функций и умножения функции на число.
\end{enumerate}
