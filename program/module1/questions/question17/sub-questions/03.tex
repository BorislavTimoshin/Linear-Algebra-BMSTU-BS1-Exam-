\subsection{
    *Полезное дополнение про преобразование координат вектора при переходе от старого базиса к новому.
    \label{subsection:subsection_17_3}
}

Выберем произвольный вектор $\vec{x} \in \mathcal{L}$ и разложим его в старом базисе $b$:

$$\vec{x} = bx_b, \quad \quad x_b = \begin{pmatrix} x_1 \\ \vdots \\ x_n \end{pmatrix}.$$

Разложение того же вектора в новом базисе $c$ имеет вид:

$$\vec{x} = cx_c, \quad \quad x_c = \begin{pmatrix} x_1' \\ \vdots \\ x_n' \end{pmatrix}.$$

Найдем связь между старыми координатами $x_b$ вектора $\vec{x}$ и его новыми координатами $x_c$. Из соотношений выше следует, что $bx_b = cx_c$. Учитывая, что $c = bT_{b \to c}$, получаем 
$$bx_b = (bT_{b \to c})x_c,$$ или 
$$bx_b = b(T_{b \to c}x_c).$$ 
Последнее равенство можно рассматривать как запись двух разложений одного и того же вектора $\vec{x}$ в базисе $b$. Разложениями соответствуют столбцы координат $x_b$ и $T_{b \to c}x_c$, которые, согласно теореме \ref{thm:theorem_2_3} о единственности разложения вектора по базису, должны быть равны:
$$x_b = T_{b \to c}x_c, \quad \quad \text{или} \quad \quad x_c = T^{-1}_{b \to c}x_b.$$
