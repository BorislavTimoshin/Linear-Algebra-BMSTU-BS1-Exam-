\subsection{
    Обратный переход. Работа с тремя и более базисами.
}

\textbf{Свойства.}

\begin{enumerate}[label={\arabic*°.}]
    \item Матрица перехода невырождена и всегда имеет обратную.
    \begin{proof}~

        Столбцы матрицы перехода - столбцы координат векторов нового \textbf{базиса} в старом. Следовательно, они, как и векторы базиса, линейно независимы. Значит, матрица $T$ невырожденная и имеет обратную матрицу $T^{-1}$.
    \end{proof}
    
    \item Если в $n$-мерном линейном пространстве задан базис $b$, то для любой невырожденной квадратной матрицы $T$ порядка $n$ существует такой базис $c$ в этом линейном пространстве, что $T$ будет матрицей перехода то базиса $b$ к базису $c$.
    \begin{proof}~

        Из невырожденности матрицы $T$ следует, что ее ранг равен $n$, и поэтому ее столбцы, будучи базисными, линейно независимы. Эти столбцы являются столбцами координат векторов системы $c = bT_{b \to c}$. Линейная независимость столбцов матрицы $T$ равносильна линейной независимости системы векторов $c$. Так как система $c$ содержит $n$ векторов, причем линейное пространство $n$-мерно, то согласно теореме $\eqref{thm:theorem_2_1}$, эта система является базисом.
    \end{proof}
    
    \item Если $T_{b \to c}$ - матрица перехода от старого базиса $b$ к новому базису $c$ линейного пространства, то $T^{-1}_{b \to c}$ - матрица перехода от базиса $c$ к базису $b$.
    \begin{proof}~

        Матрица $T_{b \to c}$ невырождена, и поэтому из равенства $c = bT_{b \to c}$ следует, что $cT^{-1}_{b \to c} = b$. Последнее равенство означает, что столбцы матрицы $T^{-1}_{b \to c}$ являются столбцами координат векторов $b$ относительно базиса $c$, т.е. согласно определению $\eqref{fig:definition_17_1}$ $T^{-1}_{b \to c}$ - это матрица перехода от базиса $c$ к базису $b$.
    \end{proof}

    \item Если в линейном пространстве заданы базисы $b, c$ и $d$, причем $T_{b \to c}$ - матрица перехода от базиса $b$ к новому базису $c$, а $T_{c \to d}$ - матрица перехода от базиса $c$ к базису $d$, то произведение этих матриц $T_{b \to c}T_{c \to d}$ - матрица перехода от базиса $b$ к базису $d$.
    \begin{proof}~

        Согласно определению $\eqref{fig:definition_17_1}$ матрицы перехода, имеем равенства
        $$c = bT_{b \to c}, \quad d = cT_{c \to d},$$
        откуда
        $$d = cT_{c \to d} = (b T_{b \to c}) \cdot T_{c \to d} = b(T_{b \to c} \cdot T_{c \to d}),$$
        т.е. $T_{b \to c} \cdot T_{c \to d} = T_{b \to d}$ - матрица перехода от базиса $b$ к базису $d$.
    \end{proof}

    \item Пусть $b_1, b_2, \ldots, b_n$ - это $n$ базисов линейного пространства $\mathcal{V}$ ($n \geq 4$). $T_k$ - матрица перехода от $b_k$ к $b_{k + 1}$, $k = \overline{1, n - 1}$. Тогда матрица перехода от $b_1$ к $b_n$ равна $T_1\cdot T_2 \cdot \ldots \cdot T_{n - 1}$.
    \begin{proof}
        Последовательное применение свойства 4°.
    \end{proof}
\end{enumerate}
