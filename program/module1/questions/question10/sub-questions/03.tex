\subsection{
    *Полезные факты, которые тоже будут на экзамене.
}

\begin{definition}
    Линейные подпространства $\mathcal{L}_1$ и $\mathcal{L}_2$ называются \textbf{\textit{ортогональными}}, если $\forall \vec{x} \in \mathcal{L}_1, \forall \vec{y} \in \mathcal{L}_2 \colon \vec{x} \perp \vec{y}$.
\end{definition}

\begin{theorem}
    Любая ортогональная система ненулевых векторов линейно независима.
\end{theorem}

\begin{proof}
    Рассмотрим произвольную ортогональную систему ненулевых векторов $\vec{e_1}, \ldots, \vec{e_m}$. Предположим, что для действительных коэффициентов $\alpha_1, \ldots, \alpha_m$ выполняется равенство
    \begin{equation}
        \alpha_1\vec{e_1} + \ldots + \alpha_m\vec{e_m} = \vec{0}.
        \label{eq:theorem_10_1_1}
    \end{equation}
    Умножим это равенство скалярно на какой-либо вектор $\vec{e_i}$:
    $$(\alpha_1\vec{e_1} + \ldots + \alpha_m\vec{e_m}, \vec{e_i}) = (\vec{0}, \vec{e_i}).$$
    $$\alpha_1(\vec{e_1}, \vec{e_i}) + \ldots + \alpha_i(\vec{e_i}, \vec{e_i}) + \ldots + \alpha_m(\vec{e_m}, \vec{e_i}) = 0.$$
    Так как система ортогональна, то все слагаемые слева, кроме одного, равны нулю, т.е.
    \begin{equation}
        \alpha_i(\vec{e_i}, \vec{e_i}) = 0.
        \label{eq:theorem_10_1_2}
    \end{equation}
    Так как вектор $\vec{e_i}$ ненулевой, то $(\vec{e_i}, \vec{e_i}) \ne 0$. Поэтому из $\eqref{eq:theorem_10_1_2}$ следует, что $\alpha_i = 0$. Индекс $i$ можно было выбирать произвольно, так что на самом деле все коэффициенты $a_i$ являются нулевыми. Значит, равенство $\eqref{eq:theorem_10_1_1}$ возможно лишь при нулевых коэффициентах. Значит, система векторов $\vec{e_1}, \ldots, \vec{e_m}$ линейно независима.
\end{proof}

\begin{corollary}
    Ортонормированная система векторов линейно независима. 
\end{corollary}

\begin{corollary}
    В $n$-мерном пространстве ортогональная/ортонормированная система из $n$ векторов является базисом.
\end{corollary}

\begin{definition}
    Если базис евклидова пространства представляет собой ортогональную систему векторов, то этот базис называют \textbf{\textit{ортогональным}}.
\end{definition}

\begin{definition}
    Ортогональный базис называется \textbf{\textit{ортонормированным}}, если каждый вектор этого базиса имеет норму (длину), равную единице.
\end{definition}

\begin{theorem}
    В конечномерном евклидовом пространстве существует ортонормированный базис.
\end{theorem}
