\subsection{
    Процесс ортогонализации Грама-Шмидта, вывод формулы. Построение ортонормированного базиса.
}

Построить ортонормированный базис можно, отталкиваясь от некоторого исходного базиса, при помощи алгоритма, который называют процессом ортогонализации Грама-Шмидта:

\bigbreak

Пусть $f = (\vec{f_1}, \ldots, \vec{f_n})$ - некоторый базис в евклидовом $n$-мерном пространстве $\mathcal{E}$. Модифицируя этот базис, мы будем строить новый базис $e = (\vec{e_1}, \ldots, \vec{e_n})$, который будет ортонормированным. Последовательно вычисляем векторы $\vec{g_1}$ и $\vec{e_1}$, $\vec{g_2}$ и $\vec{e_2}$ и т.д. по формулам:

\begin{align*}
    \vec{g_1} = \vec{f_1}, \quad \quad \quad \quad \quad \quad \quad \quad \quad \quad \quad \quad \quad \quad \quad
    \quad \quad \quad \quad \quad \thinspace \thinspace \thinspace \thinspace \thinspace \vec{e_1} = \frac{\vec{g_1}}{\norm{\vec{g_1}}};\\
    \vec{g_2} = \vec{f_2} - (\vec{f_2}, \vec{e_1}) \cdot \vec{e_1}, \quad \quad \quad \quad \quad \quad \quad \quad \quad \quad
    \quad \quad \quad \quad \thinspace \thinspace \thinspace \thinspace \thinspace \vec{e_2} = \frac{\vec{g_2}}{\norm{\vec{g_2}}};\\
    \vec{g_3} = \vec{f_3} - (\vec{f_3}, \vec{e_1}) \cdot \vec{e_1} - (\vec{f_3}, \vec{e_2}) \cdot \vec{e_2}, \quad \quad \quad \quad \quad \quad \quad \quad \thinspace \thinspace \thinspace \thinspace \thinspace \vec{e_3} = \frac{\vec{g_3}}{\norm{\vec{g_3}}};\\
    \ldots \ldots \ldots \ldots \ldots \ldots \ldots
    \ldots \ldots \ldots \ldots
    \quad \quad \quad \quad \quad \quad
    \quad \quad \quad \quad
    \ldots \ldots \ldots \\
    \vec{g_n} = \vec{f_n} - (\vec{f_n}, \vec{e_1}) \cdot \vec{e_1} - \ldots - (\vec{f_n}, \vec{e}_{n - 1}) \cdot \vec{e}_{n - 1}, \quad \quad \quad \thinspace \thinspace \thinspace \vec{e_n} = \frac{\vec{g_n}}{\norm{\vec{g_n}}}.
\end{align*}

\begin{proof}~

    Рассмотрим индукцию по количеству векторов $n$.
    \begin{enumerate}
        \item При $n = 1$ утверждение очевидно.
        \item Пусть это утверждение выполнено для количества векторов, равного $n$, докажем его для $n + 1$. 
        
        Т.к. утверждение верно для $n$ векторов, то мы можем считать, что векторы $\vec{g_1}, \ldots, \vec{g_n}$ с указанными свойствами уже построены. Построим вектор $\vec{g}_{n + 1}$ в виде 
        $$\vec{g}_{n + 1} = \vec{f}_{n + 1} + \lambda_1 \vec{g_1} + \ldots + \lambda_n\vec{g_n}.$$ 
        Линейная оболочка векторов $\vec{g_1}, \ldots,  \vec{g}_{n + 1}$ совпадает с $\vec{f_1}, \ldots, \vec{f}_{n + 1}$ при любых $\lambda_i$, поэтому мы будем подбирать коэффициенты $\lambda_i$ так, чтобы выполнялось условие $(\vec{g}_{n + 1},  \vec{g_i}) = 0$ для всех $i = 1, \ldots, n$. Рассмотрим скалярное произведение $$0 = (\vec{g}_{n + 1}, \vec{g_i}) = (\vec{f}_{n + 1}, \vec{g_i})+ \lambda_1(\vec{g_1}, \vec{g_i}) + \ldots + \lambda_n(\vec{g_n}, \vec{g_i}).$$
        Поскольку $(\vec{g_j}, \vec{g_i}) = 0$ при $j \ne i$ по предположению индукции, то 
        $$0 = (\vec{f}_{n + 1}, \vec{g_i}) + \lambda_i(\vec{g_i}, \vec{g_i}),$$ следовательно 
        $$\lambda_i = -\frac{(\vec{f}_{n + 1}, \vec{g_i})}{(\vec{g_i}, \vec{g_i})} \text{ (знаменатель отличен от нуля).}$$ 
        
        Таким образом, чтобы получить вектор $\vec{g}_{n + 1}$, надо из вектора $\vec{f}_{n + 1}$ вычесть его ортогональные проекции на векторы $\vec{g_1}, \ldots, \vec{g_n}$:
        $$\vec{g}_{n + 1} = \vec{f}_{n + 1} -\frac{(\vec{f}_{n + 1}, \vec{g_1})}{(\vec{g_1}, \vec{g_1})} \cdot \vec{g_1} - \ldots -\frac{(\vec{f}_{n + 1}, \vec{g_n})}{(\vec{g_n}, \vec{g_n})} \cdot \vec{g_n}.$$
    \end{enumerate}
\end{proof}
