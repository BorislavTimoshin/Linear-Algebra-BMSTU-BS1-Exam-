\subsection{
    Евклидово пространство.
}

\begin{definition}
    Отображение $\mathcal{V} \times \mathcal{V} \to \RR$, где $\mathcal{V}$ - линейное пространство над полем $\RR$, называется \textbf{\textit{скалярным произведением}}, если выполнены 4 аксиомы:
    \begin{enumerate}[nosep]
        \item $(\vec{x}, \vec{y}) = (\vec{y}, \vec{x})$.
        \item $(\vec{x} + \vec{y}, \vec{z}) = (\vec{x}, \vec{z}) + (\vec{y}, \vec{z})$ - аддитивность по первому аргументу.
        \item $(\alpha \vec{x}, \vec{y}) = \alpha(\vec{x}, \vec{y})$ - однородность по первому аргументу.
        \item $(\vec{x}, \vec{x}) \geq 0$, причем $(\vec{x}, \vec{x}) = 0 \iff \vec{x} = 0$.
    \end{enumerate}
\end{definition}

\begin{definition}
    Вещественное линейное пространство с так введенным скалярным произведением называется \textbf{\textit{евклидовым пространством}}.
\end{definition}

\subsection{
    Ортонормированный базис.
}

\begin{definition}
    Векторы $\vec{x}$ и $\vec{y}$ называются \textbf{\textit{ортогональными}}, если $(\vec{x}, \vec{y}) = 0$.
\end{definition}

\begin{definition}
    Система векторов называется \textit{\textbf{ортогональной}}, если все векторы в ней попарно ортогональны.
\end{definition}

\begin{definition}
    Система векторов называется \textit{\textbf{ортонормированной}}, если она ортогональна и норма каждого вектора равна 1.
\end{definition}
