\begin{definition}
    Пусть даны векторы \( \vec{x}_1, \vec{x}_2, \dots, \vec{x}_m \) в некотором евклидовом пространстве. \textit{\textbf{Матрицей Грама}} этой системы называется квадратная матрица \( \Gamma \) размера \( m \times m \), элементы которой задаются скалярными произведениями:

    \[
    \Gamma = \begin{pmatrix}
    (\vec{x}_1, \vec{x}_1) & (\vec{x}_1, \vec{x}_2) & \cdots & (\vec{x}_1, \vec{x}_m) \\
    (\vec{x}_2, \vec{x}_1) & (\vec{x}_2, \vec{x}_2) & \cdots & (\vec{x}_2, \vec{x}_m) \\
    \vdots     & \vdots     & \ddots & \vdots     \\
    (\vec{x}_m, \vec{x}_1) & (\vec{x}_m, \vec{x}_2) & \cdots & (\vec{x}_m, \vec{x}_m)
    \end{pmatrix},
    \]
    
    где \( (\vec{x}_i, \vec{x}_j) \) обозначает скалярное произведение векторов \( \vec{x}_i \) и \( \vec{x}_j \).
\end{definition}


\subsection*{Свойства матрицы.}

\begin{enumerate}[label={\arabic*°.}]
    \item $\Gamma = \Gamma^T$ (матрица Грама симметрическая).
    \item Матрица Грама для ОНБ: $\Gamma = \begin{pmatrix}
    1 & 0 & \cdots & 0 \\
    0 & 1 & \cdots & 0 \\
    \vdots     & \vdots     & \ddots & \vdots     \\
    0 & 0 & \cdots & 1
    \end{pmatrix}$.

    \item $\det \Gamma = 0 \iff \vec{x}_1, \vec{x}_2, \dots, \vec{x}_m$ - линейно зависимые.
    \item $\det \Gamma \geq 0$.
    \item $\det \Gamma$ равен квадрату объёма параллелепипеда, натянутого на векторы.
    \item Пусть $X = \begin{pmatrix}
    \vec{x}_1 & \vec{x}_2 & \cdots & \vec{x}_m
    \end{pmatrix}$, тогда $X^TX = \Gamma(\vec{x}_1, \vec{x}_2, \ldots, \vec{x}_m).$
\end{enumerate}
