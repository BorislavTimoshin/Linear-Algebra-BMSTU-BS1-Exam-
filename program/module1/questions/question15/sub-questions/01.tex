\begin{definition}
    Для любой матрицы $A$ существует такая матрица $A^+$, что нормальное псевдорешение СЛАУ $A\vec{x} = \vec{b}$ с произвольным вектор-столбцом $\vec{b}$ имеет вид $\vec{x} = A^+\vec{b}$. Матрицу $A^+$ называют \textit{\textbf{псевдообратной}} по отношению к матрице $A$.
\end{definition}

\subsection{
    Алгоритм нахождения, опираясь на нормальное псевдорешение.
}

Рассмотрим СЛАУ $A\vec{x} = \vec{e}_i$, в правой части которой записан $i$-й вектор стандартного базиса в пространстве $\RR^n$. Ее нормальное псевдорешение $A^+\vec{e}_i$ - это $i$-й столбец псевдообратной матрицы. Вычислив все $n$ столбцов, получим $A^+$. Таким образом, $i$-й столбец псевдообратной матрицы $a_i$ можно найти из системы

\begin{equation*}
    \left(\begin{array}{c}
        A^TA \\
        F^T
    \end{array}\right)a_i
    =
    \left(\begin{array}{c}
        A^T\vec{e}_i \\
        \vec{0}
    \end{array}\right)
,\end{equation*}

где $F$ - матрица, составленная из столбцов ФСР СЛАУ $A\vec{x} = \vec{0}$. Объединив по правилам действий с блочными матрицами $n$ систем в одно матричное уравнение, получим

\begin{equation*}
    \left(\begin{array}{c}
        A^TA \\
        F^T
    \end{array}\right)A^+
    =
    \left(\begin{array}{c}
        A^T \\
        \vec{0}
    \end{array}\right)
.\end{equation*}

\textbf{Итак, алгоритм состоит в следующем:}

\begin{enumerate}
    \item Найти $F$ - матрицу, составленную из столбцов ФСР СЛАУ $A\vec{x} = \vec{0}$.
    \item Транспонируем матрицу $F$, чтобы в дальнейшем добавить $F^T$ как строки новой матрицы.
    \item Найдем $A^T$.
    \item Найдем $A^TA$.
    \item Составим матричное уравнение ниже и решим его методом элементарных преобразований.

    \begin{equation*}
        \left(\begin{array}{c}
            A^TA \\
            F^T
        \end{array}\right)A^+
        =
        \left(\begin{array}{c}
            A^T \\
            \vec{0}
        \end{array}\right)
    .\end{equation*}
    \begin{equation*}
        \left(\begin{array}{c|c}
            A^TA & A^T \\
            F^T & \vec{0}
        \end{array}\right)
        \sim \ldots \sim
        \left(\begin{array}{c|c}
            E & A^+
        \end{array}\right)
    .\end{equation*}
\end{enumerate}
