\subsection{
    Теорема о единственности разложения вектора по базису.
}

\begin{theorem}
    В линейном пространстве разложение любого вектора по данному базису единственно.
    \label{thm:theorem_2_3}
\end{theorem}

\begin{proof}
    Выберем в линейном пространстве $\mathcal{L}$ произвольный базис $\vec{b_1}, \ldots, \vec{b_n}$ и предположим, что вектор $\vec{x}$ имеет в этом базисе два разложения
    \begin{align*}
        \vec{x} = x_1\vec{b_1} + \ldots + x_n\vec{b_n},\\
        \vec{x} = x_1'\vec{b_1} + \ldots + x_n'\vec{b_n}.
    \end{align*}
    Воспользуемся тем, что аксиомы линейного пространства позволяют преобразовывать линейные комбинации так же, как и обычные алгебраические выражения. Вычитая записанные равенства почленно, получим
    $$(x_1 - x_1')\vec{b_1} + \ldots + (x_n - x_n')\vec{b_n} = 0$$
    Так как базис - это линейно независимая система векторов, ее линейная комбинация равна $0$, лишь если она тривиальная. Значит, все коэффициенты этой линейной комбинации равны нулю: $x_1 - x_1' = 0, \ldots, x_n - x_n' = 0$. Таким образом, $x_1 = x_1', \ldots, x_n = x_n'$ и два разложения вектора $\vec{x}$ в базисе $\vec{b_1}, \ldots, \vec{b_n}$ совпадают.
\end{proof}
