\subsection{
    Базис и размерность пространства.
}

\begin{definition}
    \textbf{\textit{Базисом линейного пространства}} $\mathcal{L}$ называют любую упорядоченную систему векторов, для которой выполнены два условия:
    \begin{enumerate}[nosep]
        \item эта система векторов линейно независима.
        \item каждый вектор в линейном пространстве может быть представлен в виде линейной комбинации векторов этой системы.
    \end{enumerate}
\end{definition}

\begin{definition}
    Максимальное количество линейно независимых векторов в данном линейном пространстве называют \textbf{\textit{размерностью линейного пространства}}.
\end{definition}

\begin{designation}
    $n = \dim \mathcal{L}$, где $n$ - размерность линейного пространства $\mathcal{L}$.
\end{designation}
