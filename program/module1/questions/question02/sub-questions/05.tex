\subsection{
    Линейная оболочка системы векторов. Способы задания линейного подпространства и переход между разными способами задания.
}

\begin{definition}
    \textbf{\textit{Линейной оболочкой}} системы векторов $\vec{a_1}, \ldots, \vec{a_n}$ называется множество всех линейных комбинаций этой системы.
\end{definition}

\begin{theorem}
    Линейная оболочка является линейным пространством.
\end{theorem}

\begin{designation}
    $\Span(\vec{a_1}, \ldots, \vec{a_n})$.
\end{designation}

Существует 2 способа задания линейного подпространства:

\begin{enumerate}
    \item явное - $\mathcal{L} = \Span(\vec{a_1}, \ldots, \vec{a_k})$.
    \item неявное - $\mathcal{L}$ - решение однородной СЛАУ.
\end{enumerate}

\subsection*{Переход между разными способами задания.
}

\begin{itemize}
    \item Если подпространство задано неявно, то для перехода к явному способу достаточно решить однородную систему уравнений, выбрав какую-либо ФСР. Столбцы ФСР - это столбцы координат векторов некоторого базиса рассматриваемого подпространства. Следовательно, подпространство можно задать как линейную оболочку системы этих векторов.
    \item Опишем два способа перехода от явного описания к неявному.

    \begin{enumerate}
        \item Выберем в линейном пространстве какой-либо базис и запишем векторы заданной системы векторов $\vec{a}_1, \vec{a}_2, \ldots, \vec{a}_k$ через координаты в выбранном базисе. Вектор $\vec{b}$ с координатами $b = \begin{pmatrix}
            b_1 \\
            b_2 \\
            \vdots \\
            b_n
        \end{pmatrix}$ является линейной комбинацией заданной системы векторов тогда и только тогда, когда СЛАУ $\begin{pmatrix}
            \vec{a}_1 & \vec{a}_2 & \ldots & \vec{a}_k & \vec{b}
        \end{pmatrix}$ совместна. Записывая условие совместности с помощью теоремы Кронекера-Капелли, получим уравнения, связывающие координаты вектора $\vec{b}$. Эти уравнения составляют СЛАУ, неявно описывающую подпространство $\mathcal{H} = \Span(\vec{a}_1, \vec{a}_2, \ldots, \vec{a}_k)$.
    
    \item Пусть $\mathcal{H} = \Span(\vec{a}_1, \vec{a}_2, \ldots, \vec{a}_k)$. Составим матрицу $A$ из столбцов координат векторов $\vec{a}_1, \vec{a}_2, \ldots, \vec{a}_k$ в некотором базисе. Решим однородную СЛАУ $A^T\vec{x} = \vec{0}$, найдя какую-либо ФСР этой СЛАУ. Из столбцов ФСР составим матрицу $F$. Однородная СЛАУ $F^T\vec{x} = \vec{0}$ неявно описывает подпространство $\mathcal{H}$.
    \end{enumerate}
\end{itemize}
