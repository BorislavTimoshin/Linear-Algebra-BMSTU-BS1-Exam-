\subsection{
    Понятие метрики.
}

\begin{definition}
    Пусть $M$ - произвольное непустое множество. Отображение декартова квадрата $M \times M$ на поле $\RR$ называется метрикой, если оно удовлетворяет трем аксиомам:
    \begin{enumerate}
        \item $\rho(\vec{x}, \vec{y}) = \rho(\vec{y}, \vec{x})$.
        \item $\rho(\vec{x}, \vec{y}) \geq 0$, причем $\rho(\vec{x}, \vec{y}) = 0 \iff \vec{x} = \vec{y}$.
        \item $\rho(\vec{x}, \vec{y}) \leq \rho(\vec{x}, \vec{z}) + \rho(\vec{z}, \vec{y})$ - неравенство треугольника.
    \end{enumerate}
\end{definition}

\begin{example}~

    \begin{enumerate}
        \item $M = \RR, \rho(\vec{x}, \vec{y}) = |\vec{x} - \vec{y}|$.
        \item $M$ - произвольное непустое множество. Тогда дискретная метрика: 
        
        $\rho(\vec{x}, \vec{y}) = 
        \begin{cases}
        1, & \vec{x} \ne \vec{y} \\
        0, & \vec{x} = \vec{y}
        \end{cases}$.
    \end{enumerate}
\end{example}
