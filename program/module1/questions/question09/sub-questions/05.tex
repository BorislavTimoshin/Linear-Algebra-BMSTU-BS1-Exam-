\subsection{
    *Полезные факты, которые тоже могут быть на экзамене.
}

\begin{lemma}~

    Пусть

    \begin{enumerate}
        \item $\mathcal{E}$ - $n$-мерное евклидово пространство;
        \item $\vec{e}_1, \ldots, \vec{e}_n$ - некоторый базис $\mathcal{E}$;
        \item $\vec{x}_e = \begin{pmatrix}
            x_1 \\
            \vdots \\
            x_n
        \end{pmatrix}$ - столбец координат $\vec{x}$ в базисе $\vec{e}_1, \ldots, \vec{e}_n$;
        \item $\vec{y}_e = \begin{pmatrix}
            y_1 \\
            \vdots \\
            y_n
        \end{pmatrix}$ - столбец координат $\vec{y}$ в базисе $\vec{e}_1, \ldots, \vec{e}_n$;
    \end{enumerate}

    Тогда 
    $$(\vec{x}, \vec{y}) = \vec{x}^T_e\Gamma_e\vec{y}_e.$$
    \label{lemma:lemma_1}
\end{lemma}

\begin{proof}~

    \begin{align*}
        (\vec{x}, \vec{y}) &= (\sum_{i = 1}^nx_i\vec{e}_i, \sum_{j = 1}^ny_j\vec{e}_j) = \\ 
        &=\sum_{i = 1}^n\sum_{j =  1}^n(x_iy_j(\vec{e}_i, \vec{e}_j)) = \\
        &= \vec{x}^T_e\Gamma_e\vec{y}_e,
    \end{align*}

    где $\Gamma_e$ - матрица Грама для системы векторов $\vec{e}_1, \ldots, \vec{e}_n$.
\end{proof}

\begin{corollary}~

    Пусть $e$ - ОНБ $\mathcal{E}$. Тогда матрица Грама для этого базиса является единичной. Поэтому

    $$(\vec{x}, \vec{y}) = \vec{x}^T_eE\vec{y}_e = \vec{x}^T_e\vec{y}_e = x_1y_1 + x_2y_2 + \ldots + x_ny_n.$$

    В частности,

    $$\norm{\vec{x}} = \sqrt{(\vec{x}, \vec{x})} = \sqrt{x^T_ex_e} = \sqrt{x^2_1 + \ldots + x^2_n}.$$

    \label{corollary:corollary_1}
\end{corollary}
