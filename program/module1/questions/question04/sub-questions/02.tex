\subsection{
    Доказать, что указанные множества (сумма и пересечение ЛПП) являются линейными подпространствами.
}

\begin{theorem}
    Сумма линейных подпространств данного линейного пространства является линейным подпространством в том же линейном пространстве.
\end{theorem}

\begin{proof}~

    Проверим, выполняются ли условия определения линейного подпространства:
    \begin{enumerate}[nosep]
        \item Рассмотрим два вектора $\vec{v}$ и $\vec{w}$ из множества $\mathcal{H}_1 + \mathcal{H}_2$. Согласно определению суммы линейных подпространств, имеют место представления $\vec{v} = \vec{x_1} + \vec{x_2}$, $\vec{w} = \vec{y_1} + \vec{y_2}$, где векторы $\vec{x_i}$, $\vec{y_i}$ принадлежат $\mathcal{H}_i$, $i = 1, 2$. Складывая эти равенства, получаем
        $$\vec{v} + \vec{w} = (\vec{x_1} + \vec{y_1}) + (\vec{x_2} + \vec{y_2}).$$
        Сумма $\vec{x_1} + \vec{y_1}$ векторов $\vec{x_1}$ и $\vec{y_1}$ линейного подпространства $\mathcal{H}_1$ принадлежит $\mathcal{H}_1$. Точно так же сумма $\vec{x_2} + \vec{y_2}$ векторов $\vec{x_2}$ и $\vec{y_2}$ линейного подпространства $\mathcal{H}_2$ принадлежит $\mathcal{H}_2$. Поэтому вектор $\vec{v} + \vec{w}$ принадлежит множеству $\mathcal{H}_1 + \mathcal{H}_2$.
        \item Произвольный вектор $\vec{v} \in \mathcal{H}_1 + \mathcal{H}_2$ имеет представление $\vec{v} = \vec{x_1} + \vec{x_2}$, где $\vec{x_1} \in \mathcal{H}_1$, $\vec{x_2} \in \mathcal{H}_2$. Для любого действительного числа $\lambda$ получаем равенства 
        $$\lambda \vec{v} = \lambda (\vec{x_1} + \vec{x_2}) = \lambda \vec{x_1} + \lambda \vec{x_2}.$$
        Так как вектор $\lambda \vec{x_1}$ принадлежит $\mathcal{H}_1$, а вектор $\lambda \vec{x_2}$ - $\mathcal{H}_2$, то вектор $\lambda \vec{u}$ является элементом множества $\mathcal{H}_1 + \mathcal{H}_2$.
    \end{enumerate}
    Мы доказали, что множество $\mathcal{H}_1 + \mathcal{H}_2$ замкнуто относительно линейных операций объемлющего линейного пространства и поэтому, согласно определению линейного подпространства, оно является линейным подпространством.
\end{proof}

\begin{theorem}
    Пересечение $\mathcal{H}_1 \cap \mathcal{H}_2$ двух линейных подпространств $\mathcal{H}_1$ и $\mathcal{H}_2$ в линейном пространстве $\mathcal{L}$ является линейным подпространством в $\mathcal{L}$.
\end{theorem}

\begin{proof}~

    Проверим, выполняются ли условия определения линейного подпространства:
    \begin{enumerate}[nosep]
        \item Если векторы $\vec{x_1}$ и $\vec{x_2}$ принадлежат $\mathcal{H}_1 \cap \mathcal{H}_2$, то каждый из этих векторов принадлежит как $\mathcal{H}_1$, так и $\mathcal{H}_2$. Поскольку $\mathcal{H}_1$ - линейное подпространство, то согласно определению линейного подпространства, заключаем, что вектор $\vec{x_1} + \vec{x_2}$, равный сумме векторов этого линейного подпространства, тоже принадлежит $\mathcal{H}_1$. Аналогично $\vec{x_1} + \vec{x_2} \in \mathcal{H}_2$, так как каждое из слагаемых является элементом линейного подпространства $\mathcal{H}_2$. Следовательно, $\vec{x_1} + \vec{x_2} \in \mathcal{H}_1 \cap \mathcal{H}_2$.
        \item Выберем произвольный вектор $\vec{x} \in \mathcal{H}_1 \cap \mathcal{H}_2$. Тогда $\vec{x} \in \mathcal{H}_1$ и $\vec{x} \in \mathcal{H}_2$. Так как $\mathcal{H}_1$ является линейным подпространством, то произведение элемента $\vec{x}$ этого линейного подпространства на произвольное действительное число $\lambda$ принадлежит $\mathcal{H}_1$. Но совершенно аналогично вектор $\lambda \vec{x}$ принадлежит и $\mathcal{H}_2$. Поэтому $\lambda \vec{x} \in \mathcal{H}_1 \cap \mathcal{H}_2$. 
    \end{enumerate}
    Итак, оба условия определения линейного подпространства выполнены. Следовательно, $\mathcal{H}_1 \cap \mathcal{H}_2$ является линейным подпространством.
\end{proof}
