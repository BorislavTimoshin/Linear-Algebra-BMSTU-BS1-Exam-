\subsection{
    Связь с решениями неоднородной СЛАУ.
}

Рассмотрим неоднородную СЛАУ

$$A\vec{x} = \vec{b},$$

где $A \in \RR^{m \times n}, \vec{x} = \begin{pmatrix} x_1 \\ \vdots \\ x_n \end{pmatrix} \in \RR^n, \vec{b} = \begin{pmatrix} b_1 \\ \vdots \\ b_m \end{pmatrix} \in \RR^m$.

Из курса Ан. Геом. известно, что если СЛАУ имеет бесконечно много решений, то они задаются так:

$$\vec{x} = \vec{x}_0 + c_1\vec{f}_1 + \ldots + c_k\vec{f}_k,$$

где $\vec{x}_0 \in \RR^n$ - вектор с постоянными коэффициентами, $\vec{f}_1, \ldots, \vec{f}_k \in \RR^n$ - векторы, образующие ФСР, $c_1, \ldots, c_k$ - произвольные константы.

Если рассмотреть $\mathcal{W} = \Span \{\vec{f}_1, \ldots, \vec{f}_k\}$, то получится, что множество $\mathcal{U}$ всех решений СЛАУ будет представлять из себя ЛАМ:

$$\mathcal{U} = \vec{x}_0 + \mathcal{W}.$$

Представляете!
