\subsection{
    Нормальное псевдорешение, алгоритм нахождения.
}


Среди всех псевдорешений СЛАУ $A\vec{x} = \vec{b}$ есть единственное псевдорешение, имеющее наименьшую норму. Оно называется \textit{нормальным}. Итак,

\begin{definition}
    Наименьшее по норме псевдорешение СЛАУ $A\vec{x} = \vec{b}$ называется \textbf{\textit{нормальным псевдорешением}}.
\end{definition}

Нормальное псевдорешение СЛАУ $A\vec{x} = \vec{b}$ является решением системы уравнений, которая получается, если к нормальной системе $A^TA\vec{x} = A^T\vec{b}$ добавить уравнения СЛАУ $F^T\vec{x} = \vec{0}$, где $F$ - матрица, составленная из столбцов ФСР СЛАУ $A\vec{x} = \vec{0}$.

\begin{equation*}
    \left(\begin{array}{c}
        A^TA \\
        F^T
    \end{array}\right)\vec{x}
    =
    \left(\begin{array}{c}
        A^T\vec{b} \\
        \vec{0}
    \end{array}\right)
.\end{equation*}
