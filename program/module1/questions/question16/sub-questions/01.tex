\subsection{
    Псевдорешение, алгоритм нахождения.
}

Рассмотрим СЛАУ 

$$A\vec{x} = \vec{b},$$

где $A$ - матрица размера $m \times n, \vec{x} \in \RR^n, \vec{b} \in \RR^m$. Каждому столбцу $\vec{x}$ можно сопоставить столбец $\vec{b} - A\vec{x}$, называемый \textit{вектором невязок}. Евклидову норму этого столбца $\norm{\vec{b} - A\vec{x}} = \sqrt{(\vec{b} - A\vec{x}, \vec{b} - A\vec{x})}$ называют \textit{нормой невязки}.

Для любой СЛАУ $A\vec{x} = \vec{b}$ найдется хотя бы один столбец $\vec{x}$, на котором норма невязки $v(x) = \norm{\vec{b} - A\vec{x}}$ принимает наименьшее значение $v_0$. Каждый такой столбец называется \textit{псевдорешением} СЛАУ $A\vec{x} = \vec{b}$. Итак,

\begin{definition}
    Вектор-столбец $\vec{x}$, такой, что норма невязки $\norm{\vec{b} - A\vec{x}}$ СЛАУ $A\vec{x} = \vec{b}$ принимает наименьшее значение, называется \textit{\textbf{псевдорешением}}.
\end{definition}

Если система совместна, то минимальное значение нормы невязки равно нулю и множество псевдорешений совпадает с множеством решений системы. Множество псевдорешений СЛАУ совпадает с множеством решений соответствующей \textit{нормальной системы} $A^TA\vec{x} = A^T\vec{b}$ (\textbf{*}см. 14 "билет"). 
