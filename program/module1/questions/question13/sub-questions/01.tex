\subsection{
    Прямое дополнение.
}

\begin{definition}
    Если линейные подпространства $\mathcal{H}_1$ и $\mathcal{H}_2$ в линейном пространстве $\mathcal{L}$ образуют прямую сумму, причем $\mathcal{H}_1 \oplus \mathcal{H}_2 = \mathcal{L}$, то говорят, что $\mathcal{H}_1$ является \textit{\textbf{прямым дополнением}} для $\mathcal{H}_2$.
\end{definition}

\begin{theorem}
    Любое линейное подпространство $\mathcal{H}$ в линейном пространстве $\mathcal{L}$ имеет прямое дополнение.
\end{theorem}

\begin{proof}~

    Если линейное подпространство $\mathcal{H}$  совпадает со всем линейным пространством $\mathcal{L}$, то в качестве его прямого дополнения следует взять другое несобственное подпространство: $\mathcal{H}_1 = \{\vec{0}\}$. Точно так же прямым дополнением к нулевому подпространству $\{\vec{0}\}$ является само линейное пространство $\mathcal{L}$. Опуская эти два тривиальных случая, полагаем, что линейное подпространство $\mathcal{H}_1$ является собственным.

    Выберем в $\mathcal{H}$ какой-либо базис $e = (\vec{e_1}, \ldots, \vec{e_k})$ и дополним его системой векторов $f = (\vec{f_1}, \ldots, \vec{f_m})$ до базиса в $\mathcal{L}$. Положим, $\mathcal{H}_1 = \Span(f)$. Тогда $\mathcal{H} + \mathcal{H}_1 = \mathcal{L}$, так как сумма $\mathcal{H} + \mathcal{H}_1$ содержит все векторы системы $(e, f)$, являющейся базисом в $\mathcal{L}$, а значит, и любой другой вектор линейного пространства. Остается доказать, что сумма $\mathcal{H} + \mathcal{H}_1$ является прямой.

    Выберем произвольный вектор $\vec{y} \in \mathcal{H} \cap \mathcal{H}_1$. Тогда, с одной стороны, $\vec{y} = \alpha_1\vec{e_1} + \ldots + \alpha_k\vec{e_k}$, так как $\vec{y}$ принадлежит линейному подпространству $\mathcal{H}$, а с другой стороны, $\vec{y} = \beta_1\vec{f_1} + \ldots + \beta_m\vec{f_m}$, так как $\vec{y}$ принадлежит линейному подпространству $\mathcal{H}_1$. Эти две линейные комбинации есть два разложения вектора в базисе $(e, f)$ линейного пространства $\mathcal{L}$ и, следовательно, должны совпадать:
    $$\alpha_1\vec{e_1} + \ldots + \alpha_k\vec{e_k} = \beta_1\vec{f_1} + \ldots + \beta_m\vec{f_m},$$
    или
    $$\alpha_1\vec{e_1} + \ldots + \alpha_k\vec{e_k} - \beta_1\vec{f_1} - \ldots - \beta_m\vec{f_m} = \vec{0}.$$
    Система векторов $(e, f)$ линейно независима, так как является базисом. Поэтому из последнего равенства векторов следует, что в нем все коэффициенты нулевые. Значит, вектор $\vec{y}$ является нулевым, а так как он выбирался произвольно, то $\mathcal{H} \cap \mathcal{H}_1 = \{\vec{0}\}$. Поэтому линейные подпространства $\mathcal{H}$ и $\mathcal{H}_1$ образуют прямую сумму.
\end{proof}
