\subsection{
    Линейное аффинное многообразие. Вектор сдвига.
}

\begin{definition}
    Пусть $\mathcal{V}$ - линейное пространство над полем $\PP$, $\mathcal{W}$ - это его подпространство. Зафиксируем вектор $\vec{a} \in \mathcal{V}$. Тогда множество $\vec{a} + \mathcal{W} = \{\, \vec{a} + \vec{x} \mid \vec{x} \in \mathcal{W} \,\}$ называется \textbf{\textit{линейным аффинным многообразием}}.

    При этом подпространство $\mathcal{W}$ называется \textbf{\textit{направляющим подпространством}}, а вектор $\vec{a}$ называется \textbf{\textit{вектором сдвига}}.
\end{definition}

Заметим, что ЛАМ, вообще говоря, не является подпространством (например, потому что оно может не содержать $\vec{0} \in \mathcal{V}$.)

\begin{proof}~

    Если $\vec{a} \notin \mathcal{W}$, то согласно аксиоме линейного пространства, и $(-\vec{a}) \notin \mathcal{W}$. 

    \begin{gather*}
        \text{Предположим, что } \underbrace{\vec{a} + (-\vec{a})}_{\vec{0}} \in \vec{a} + \mathcal{W}. \\
        \exists \vec{w} \in \mathcal{W} (\vec{a} + (-\vec{a}) = \vec{a} + \underbrace{\vec{w}}_{\in \mathcal{W}}). \\
        \exists \vec{w} \in \mathcal{W} ((-\vec{a}) = \vec{w}). \\
        \downimplies \\
        -\vec{a} \notin \mathcal{W}\text{. Противоречие.}
    \end{gather*}
    Это и означает, что если $\vec{a} \notin \mathcal{W}$, то $\vec{a} + \mathcal{W}$ не является подпространством.
\end{proof}

\begin{definition}
    \textbf{\textit{Размерностью}} линейного аффинного многообразия $\vec{a} + \mathcal{W}$ называется размерность $\mathcal{W}$, т.е. $$\dim (\vec{a} + \mathcal{W}) = \dim \mathcal{W}.$$
\end{definition}

\begin{definition}
    Пусть 

    \begin{enumerate}
        \item $\mathcal{V}$ - линейное пространство,
        \item $\dim \mathcal{V} = n$,
        \item $\mathcal{W}$ - некоторое подпространство $\mathcal{V}$.
    \end{enumerate}
    
    Тогда \textbf{\textit{гиперплоскостью}} в нем будет называться ЛАМ вида $(\vec{a} + \mathcal{W})$, где $\dim \mathcal{W} = n - 1$.
\end{definition}
