\documentclass[a4papper]{article}
\usepackage[left=0.5cm,right=0.5cm,top=2cm,bottom=0.5cm,bindingoffset=0cm]{geometry}
\usepackage{header-linear-algebra}

\title{\Huge Линейная алгебра, Экзамен}

\author{
    Написано простым \href{https://t.me/Borislav_Timoshin}{BIT}-ом на основе лекций Гордеевой Н.М., \\ 
    учебника Канатникова А.Н., Крищенко А.П., \\
    методичек Власова П.А., Соболева С.К., \\
    записей одногруппников и старшекурсников,\\ некоторых методичек Мехмата МГУ и, конечно, \\ с помощью одного загадочного, вдохновляющего на \\ изучение математики путешественника по млечному пути, \\
    приносящего умные мысли в процессе \\
    нашего общего мозгового штурма. \\
    \href{https://github.com/BorislavTimoshin/Linear-Algebra-BMSTU-BS1-Exam-}{GitHub}.
}

\date{} % Очищаем стандартную дату

\begin{document}
    \pagestyle{fancy}
    \fancyhead[L]{\thepage}
    \fancyhead[R]{\hyperlink{toc}{\large Содержание}}

    \maketitle

    \epigraph{
        ``Верю, что проникну в пространство, невидимое доселе, и буду творить в нем чудеса!!''.
    }{\rightline{{\rm --- BIT}}}


    \vfill % Добавляем вертикальное заполнение, чтобы сдвинуть вниз

    \begin{center}
        \Large МГТУ им. Н.Э. Баумана, кафедра ФН1. \\
        \emph{2024 — 2025}
    \end{center}

    \newpage
        
    \tableofcontents

    \Hide

\part{Модуль 1.}

\section{
    Комплексные числа. Арифметические действия над комплексными числами. Возведение в степень и извлечение корня. Комплексная плоскость, модуль, аргумент. Формула Эйлера. Формула Муавра.
}

% Комплексные числа
\subsection{
    Комплексные числа.
}

\begin{definition}
    \textbf{\textit{Комплексным числом $z$}} называется пара $(x, y)$ действительных чисел $x, y \in \RR$, для которой определены понятие равенства и операции сложения и умножения следующим образом:
    
    \begin{enumerate}[nosep]
        \item $z_1 = z_2 \iff x_1 = x_2$ и $y_1 = y_2$.
        \item $z_1 + z_2 = (x_1 + x_2, \thinspace y_1 + y_2)$.
        \item $z_1 \cdot z_2 = (x_1x_2 - y_1y_2, \thinspace \thinspace x_1y_2 + x_2y_1)$
    \end{enumerate}
\end{definition}

\begin{designation}
    $\CC$.
\end{designation}

\begin{definition}
    \textbf{\textit{Мнимой единицей}} называется число $i = (0, 1)$, причем ее квадратом, является пара $(-1, 0)$.
\end{definition}

\begin{definition}
    Комплексное число $\overline{z}$ называется \textbf{\textit{сопряженным}} к $z$, если $\overline{z} = x - iy$, $z = x + iy$.
\end{definition}

\textbf{Свойства комплексного сопряжения.}

\begin{enumerate}[label={\arabic*°.}]
    \item $\overline{z + w} = \overline{z} + \overline{w}.$
    
    $\overline{z + w} = \overline{(a_1 + b_1 i) + (a_2 + b_2 i)} = \overline{(a_1 + a_2) + (b_1 + b_2) i} = (a_1 + a_2) - (b_1 + b_2)i = (a_1 - b_1 i) + (a_2 - b_2 i) = \overline{z} + \overline{w}.$
    
    \item $\overline{zw} = \overline{z} \cdot \overline{w}.$
    
    $\overline{z} \cdot \overline{w} = (a_1 - b_1 i) (a_2 - b_2 i) = (a_1 a_2 - b_1 b_2) - (a_1 b_2 + a_2 b_1) i = \overline{zw}.$

    \item $\overline{\overline{z}} = z.$

    $\overline{\overline{z}} = \overline{\overline{a + bi}} = \overline{a - bi} = a + bi = z.$
\end{enumerate}



\newpage


% Арифметические действия над комплексными числами
\subsection{
    Арифметические действия над комплексными числами.
}

\begin{itemize}[nosep]
    \item $z_1 + z_2 = x_1 + x_2 + i(y_1 + y_2)$.
    \item $z_1 - z_2 = x_1 - x_2 + i(y_1 - y_2)$.
    \item $z_1 \cdot z_2 = (x_1 + iy_1)(x_2 + iy_2) = x_1x_2 + ix_1y_2 + ix_2y_1 - y_1y_2 = x_1x_2 - y_1y_2 + i(x_1y_2 + x_2y_1)$.
    \item $\frac{z_1}{z_2} = \frac{x_1 + iy_1}{x_2 + iy_2} = \frac{(x_1 + iy_1)(x_2 - iy_2)}{x_2^2 + y_2^2} = \underbrace{\frac{x_1x_2 + y_1y_2}{x_2^2 + y_2^2}}_{\text{Re}(\frac{z_1}{z_2})} + i\underbrace{\frac{x_2y_1 - x_1y_2}{x_2^2 + y_2^2}}_{\text{Im}(\frac{z_1}{z_2})}$.
\end{itemize}

% Возведение в степень и извлечение корня
\subsection{
    Возведение в степень и извлечение корня.
}

$z^n = w$

$w = |w|e^{i\varphi}$

$z = \sqrt[n]{|w|} \cdot e^{i \frac{\varphi + 2 \pi k}{n}}, \quad k = 0, 1, \ldots, n - 1.$

При нахождении корня $n$ степени из комплексного числа получается ровно $n$ чисел, лежащих на окружности радиусом $\sqrt[n]{|w|}$ и образующих правильный $n$-угольник.

% Формула Эйлера
\subsection{
    Формула Эйлера.
}

$z = |z|(\cos \varphi + i \sin \varphi)$ — тригонометрическая форма записи.

$e^{i\varphi} = \cos \varphi + i \sin \varphi$ — \textbf{формула Эйлера}.

$z = |z|e^{i\varphi}$ — показательная форма записи.

% Формула Муавра
\subsection{
    Формула Муавра.
}

$z^n = |z|^n e^{in\varphi} = r^n(\cos n \varphi + i \sin n \varphi)$ — \textbf{формула Муавра}.



\newpage


% Комплексная плоскость, модуль, аргумент
\subsection{
    Комплексная плоскость, модуль, аргумент.
}

\begin{definition}
    \textbf{\textit{Комплексной плоскостью}} называется плоскость, образованная комплексными числами, у которой ось $Ox$ образована действительными числами, а ось $Oy$ - мнимыми числами.
\end{definition}

\begin{figure}[H]
    \centering
    \includegraphics[scale=0.55]{images/module1/question01/1.jpg}
    \label{fig:picture_01_1}
\end{figure}


\begin{definition}
    \textbf{\textit{Модулем числа $z = x + iy$}} называется вещественное число $|z| = \sqrt{x^2 + y^2}$ (то есть длина соответствующего вектора).
\end{definition}

\textbf{Свойства.}

\begin{enumerate}[label={\arabic*°.}]
    \item $|z| \geq 0$, причем $|z| = 0 \iff z = 0$.
    \item $|z + w| \leq |z| + |w|$ (неравенство треугольника).

    Пусть $z = a + bi$, $w = c + di$.
    \begin{align*}
        |z + w| &\leq |z| + |w| \\
        \sqrt{(a + c)^2 + (b + d)^2} &\leq \sqrt{a^2 + b^2} + \sqrt{c^2 + d^2} \\
        (a + c)^2 + (b + d)^2 &\leq a^2 + b^2 + c^2 + d^2 + 2\sqrt{(a^2 + b^2)(c^2 + d^2)} \\
        ac + bd &\leq\sqrt{(a^2 + b^2)(c^2 + d^2)} \\
        ac + bd &\leq\sqrt{(ac)^2 + (ad)^2 + (bc)^2 + (bd)^2} \\
        (ac)^2 + (bd)^2 + 2acbd &\leq (ac)^2 + (ad)^2 + (bc)^2 + (bd)^2 \\
        2acbd &\leq (ad)^2 + (bc)^2 \\
        0 &\leq (ad)^2 + (bc)^2 - 2abcd \\
        0 &\leq (ad - bc)^2
    \end{align*}
    \item $z \overline{z} = |z|^2$.

    $z \overline{z} = (a + bi)(a - bi) = a^2 - (bi)^2 = a^2 + b^2 = |z|^2$
    \item $|zw| = |z||w|$.

    $|zw|^2 = (zw) \cdot (\overline{zw}) = z \cdot w \cdot \overline{z} \cdot \overline{w} = |z|^2 |w|^2$.
\end{enumerate}


\subsubsection*{
    Аргумент комплексного числа.
}


Пусть $z = a + bi \in \CC$, $z \neq 0$.

Тогда, $z = |z| \left(\frac{a}{|z|} + \frac{b}{|z|}i\right)$, при этом $\left(\frac{a}{|z|}\right)^2 + \left(\frac{b}{|z|}\right)^2 = (\frac{a}{\sqrt{a^2 + b^2}})^2 + (\frac{b}{\sqrt{a^2 + b^2}})^2 = \frac{a^2}{a^2 + b^2} + \frac{b^2}{a^2 + b^2} = \frac{a^2 + b^2}{a^2 + b^2} = 1.$

Значит, $\frac{a}{|z|}$ и $\frac{b}{|z|}$ являются синусом и косинусом некоторого угла.

\begin{definition}
    \textbf{\textit{Аргументом числа}} $z = a + bi \in \CC \setminus \{0\}$ называется любое число $\phi \in \RR$, такое что
    \begin{equation*}
        \cos \phi = \frac{a}{|z|} = \frac{a} {\sqrt{a^2 + b^2}}.
    \end{equation*}

    \begin{equation*}
        \sin \phi = \frac{b}{|z|} = \frac{b}{\sqrt{a^2 + b^2}}.
    \end{equation*}

    В геометрических терминах, $\phi$ есть угол между положительным направлением оси $Ox$ и вектором с началом в точке $0$ и концом в точке $z$.
\end{definition}

\begin{comment}
    Таких чисел бесконесно много, причем такие числа отличаются на $2\pi k, k \in \ZZ$.
\end{comment}


\newpage
\section{
    Линейные пространства. Определение, примеры. Базис и размерность пространства. Линейная оболочка системы векторов. Способы задания и переход между разными способами задания. Дать определение базиса и размерности линейного пространства. Связь между этими понятиями. Доказать теорему о единственности разложения вектора по базису. Привести пять примеров различных линейных пространств.
}

% Линейные пространства. Определение, примеры
\subsection{
    Линейные пространства. Определение, примеры.
}

\begin{definition}
    Непустое множество $\mathcal{L}$ называется \textbf{\textit{линейным пространством}} (афинным, векторным) над полем $\PP$, если $\forall \alpha, \beta \in \PP$ и $\forall \vec{a}, \vec{b} \in \mathcal{L}$ выполнены  линейность:  $(\alpha \vec{a} + \beta \vec{b}) \in \mathcal{L}$, и следующие аксиомы векторного пространства:

    \begin{enumerate}[nosep]
        \item $\vec{a} + \vec{b} = \vec{b} + \vec{a}$.
        \item $\forall \vec{c} \in \mathcal{L} \colon (\vec{a} + \vec{b}) + \vec{c} = \vec{a} + (\vec{b} + \vec{c})$.
        \item $\forall \vec{a} \thinspace \thinspace \exists \overrightarrow{0} \in \mathcal{L} \colon \vec{a} + \overrightarrow{0} = \vec{a}$.
        \item $\forall \vec{a} \thinspace \thinspace \exists \vec{a}' \in \mathcal{L} \colon \vec{a} + \vec{a'} = \overrightarrow{0}$.
        \item $(\alpha \beta) \vec{a} = \alpha(\beta \vec{a})$.
        \item $(\alpha + \beta)\vec{a} = \alpha \vec{a} + \beta \vec{a}$.
        \item $\alpha(\vec{a} + \vec{b}) = \alpha \vec{a} + \alpha \vec{b}$.
        \item $1 \cdot \vec{a} = \vec{a}$, $\forall \vec{a}$.
    \end{enumerate}
\end{definition}

\begin{definition}
    \textbf{\textit{Полем}} называется множество $\PP$ произвольной природы, на котором заданы две бинарные операции ($+$ и $\cdot$) и которое подчиняется следующим аксиомам:
    \begin{enumerate}[nosep]
        \item $\vec{a} + \vec{b} = \vec{b} + \vec{a}$.
        \item $\forall \vec{c} \colon (\vec{a} + \vec{b}) + \vec{c} = \vec{a} + (\vec{b} + \vec{c})$.
        \item $\forall \vec{a} \thinspace \thinspace \exists \overrightarrow{0} \in \PP \colon \vec{a} + \overrightarrow{0} = \vec{a}$.
        \item $\forall \vec{a} \thinspace \thinspace \exists \vec{a}' \in \PP \colon \vec{a} + \vec{a'} = \overrightarrow{0}$.
        \item $\vec{a} \cdot \vec{b} = \vec{b} \cdot \vec{a}$.
        \item $(\vec{a} \cdot \vec{b}) \cdot \vec{c} = \vec{a} \cdot (\vec{b} \cdot \vec{c})$.
        \item $\forall \vec{a} \thinspace \thinspace \exists \vec{e} \in \PP \thinspace \text{(\textbf{единичный})} \colon \vec{a} \cdot \vec{e} = \vec{a}$.
        \item $\forall \vec{a} \in \PP, \vec{a} \ne \vec{0} \thinspace \thinspace \exists \vec{a}^{-1} \in \PP \thinspace \colon \vec{a} \cdot \vec{a}^{-1} = \vec{e}$.
        \item $(\vec{a} + \vec{b}) \cdot \vec{c} = (\vec{a} \cdot \vec{c}) + (\vec{b} \cdot \vec{c})$.
    \end{enumerate}
\end{definition}

\begin{definition}
    Элементы линейного пространства называются (абстрактными) \textbf{\textit{векторами}}.
\end{definition}

\begin{example}~
    \begin{itemize}[nosep]
        \item множество $\mathcal{V}_3 (\mathcal{V}_2)$ всех \textit{свободных векторов} в пространстве (на плоскости) с линейными операциями над векторами - линейное пространство.
        
        \item множество всех \textit{геометрических векторов} в пространстве с началом в данной точке и параллельных данной плоскости (рис. \ref{fig:picture_02_1}) с линейными операциями над векторами.
        \begin{figure}[H]
            \centering
            \includegraphics[scale=0.5]{images/module1/question02/1.jpg}
            \label{fig:picture_02_1}
            \caption{}
        \end{figure}

        \item множество $M_{mn}(\RR)$ матриц типа $m \times n$, элементами которых являются действительные числа, с линейными операциями над матрицами.

        \item множество $K_n[x]$ многочленов переменного $x$ степени, не превышающей $n$, которые как функции можно складывать и умножать на действительные числа.

        \item множество всех решений данной ОСЛАУ (решения можно рассматривать как матрицы-столбцы, складывать и умножать на числа по законам матричных операций).

        \item множество функций, непрерывных на отрезке, с обычными операциями сложения функций и умножения функции на число.

        \item $\RR$ является линейным пространством над полем $\QQ$.
        
        \item $\CC$ является линейным пространством над полем $\RR$.
    \end{itemize}
\end{example}



\newpage


% Базис и размерность пространства
\subsection{
    Базис и размерность пространства.
}

\begin{definition}
    \textbf{\textit{Базисом линейного пространства}} $\mathcal{L}$ называют любую упорядоченную систему векторов, для которой выполнены два условия:
    \begin{enumerate}[nosep]
        \item эта система векторов линейно независима.
        \item каждый вектор в линейном пространстве может быть представлен в виде линейной комбинации векторов этой системы.
    \end{enumerate}
\end{definition}

\begin{definition}
    Максимальное количество линейно независимых векторов в данном линейном пространстве называют \textbf{\textit{размерностью линейного пространства}}.
\end{definition}

\begin{designation}
    $n = \dim \mathcal{L}$, где $n$ - размерность линейного пространства $\mathcal{L}$.
\end{designation}



\newpage


% Связь между базисом и размерностью пространства
\subsection{
    Связь между базисом и размерностью пространства.
}

\begin{theorem}
    Если $\dim \mathcal{L} = n$, то любая линейно независимая система из $n$ векторов является его базисом.
    \label{thm:theorem_2_1}
\end{theorem}

\begin{proof}~

    Пусть система векторов $\vec{b_1}, \ldots, \vec{b_n} \in \mathcal{L}$ линейно независима. Тогда для любого вектора $\vec{x} \in \mathcal{L}$ система векторов $\vec{x}, \vec{b_1}, \ldots, \vec{b_n}$ линейно зависима, так как она содержит $n + 1$ вектор, т.е. количество большее, чем размерность линейного пространства. Это значит, что существуют такие коэффициенты $\alpha_0, \alpha_1, \ldots, \alpha_n$, одновременно не равные нулю, что

    \begin{equation}
        \alpha_0\vec{x} + \alpha_1\vec{b_1} + \ldots + \alpha_n\vec{b_n} = \vec{0}
        \label{eq:theorem_2_1_1}
    \end{equation}

    Заметим, что $\alpha_0 \ne 0$, так как в противном случае равенство \eqref{eq:theorem_2_1_1} сводится к равенству

    \begin{equation}
        \alpha_1\vec{b_1} + \ldots + \alpha_n\vec{b_n} = \vec{0},
    \end{equation}

    причем среди коэффициентов $\alpha_1, \ldots, \alpha_n$ есть хотя бы один ненулевой (так как $\alpha_0 = 0$). Но это означало бы, что система векторов $\vec{b_1}, \ldots, \vec{b_n}$ линейно зависима. 
    
    Учитывая, что $\alpha_0 \ne 0$, из \eqref{eq:theorem_2_1_1} находим

    \begin{equation}
        \vec{x} = -\frac{\alpha_1}{\alpha_0}\vec{b_1} - \ldots - \frac{\alpha_n}{\alpha_0}\vec{b_n}.
    \end{equation}

    Так как вектор $\vec{x}$ был выбран произвольно, заключаем, что любой вектор в линейном пространстве $\mathcal{L}$ можно представить в виде линейной комбинации системы векторов $\vec{b_1}, \ldots, \vec{b_n}$.

    Поэтому эта система векторов, по предположению линейно независимая, является базисом в $\mathcal{L}$.
\end{proof}

\begin{theorem}[обратная]
    Если в линейном пространстве $\mathcal{L}$ существует базис из $n$ векторов, то $\dim \mathcal{L} = n$.
    \label{thm:theorem_2_2}
\end{theorem}


\newpage


% Теорема о единственности разложения вектора по базису
\subsection{
    Теорема о единственности разложения вектора по базису.
}

\begin{theorem}
    В линейном пространстве разложение любого вектора по данному базису единственно.
    \label{thm:theorem_2_3}
\end{theorem}

\begin{proof}
    Выберем в линейном пространстве $\mathcal{L}$ произвольный базис $\vec{b_1}, \ldots, \vec{b_n}$ и предположим, что вектор $\vec{x}$ имеет в этом базисе два разложения
    \begin{align*}
        \vec{x} = x_1\vec{b_1} + \ldots + x_n\vec{b_n},\\
        \vec{x} = x_1'\vec{b_1} + \ldots + x_n'\vec{b_n}.
    \end{align*}
    Воспользуемся тем, что аксиомы линейного пространства позволяют преобразовывать линейные комбинации так же, как и обычные алгебраические выражения. Вычитая записанные равенства почленно, получим
    $$(x_1 - x_1')\vec{b_1} + \ldots + (x_n - x_n')\vec{b_n} = 0$$
    Так как базис - это линейно независимая система векторов, ее линейная комбинация равна $0$, лишь если она тривиальная. Значит, все коэффициенты этой линейной комбинации равны нулю: $x_1 - x_1' = 0, \ldots, x_n - x_n' = 0$. Таким образом, $x_1 = x_1', \ldots, x_n = x_n'$ и два разложения вектора $\vec{x}$ в базисе $\vec{b_1}, \ldots, \vec{b_n}$ совпадают.
\end{proof}



\newpage


% Линейная оболочка системы векторов. Способы задания линейного подпространства и переход между разными способами задания
\subsection{
    Линейная оболочка системы векторов. Способы задания линейного подпространства и переход между разными способами задания.
}

\begin{definition}
    \textbf{\textit{Линейной оболочкой}} системы векторов $\vec{a_1}, \ldots, \vec{a_n}$ называется множество всех линейных комбинаций этой системы.
\end{definition}

\begin{theorem}
    Линейная оболочка является линейным пространством.
\end{theorem}

\begin{designation}
    $\Span(\vec{a_1}, \ldots, \vec{a_n})$.
\end{designation}

Существует 2 способа задания линейного подпространства:

\begin{enumerate}
    \item явное - $\mathcal{L} = \Span(\vec{a_1}, \ldots, \vec{a_k})$.
    \item неявное - $\mathcal{L}$ - решение однородной СЛАУ.
\end{enumerate}

\subsection*{Переход между разными способами задания.
}

\begin{itemize}
    \item Если подпространство задано неявно, то для перехода к явному способу достаточно решить однородную систему уравнений, выбрав какую-либо ФСР. Столбцы ФСР - это столбцы координат векторов некоторого базиса рассматриваемого подпространства. Следовательно, подпространство можно задать как линейную оболочку системы этих векторов.
    \item Опишем два способа перехода от явного описания к неявному.

    \begin{enumerate}
        \item Выберем в линейном пространстве какой-либо базис и запишем векторы заданной системы векторов $\vec{a}_1, \vec{a}_2, \ldots, \vec{a}_k$ через координаты в выбранном базисе. Вектор $\vec{b}$ с координатами $b = \begin{pmatrix}
            b_1 \\
            b_2 \\
            \vdots \\
            b_n
        \end{pmatrix}$ является линейной комбинацией заданной системы векторов тогда и только тогда, когда СЛАУ $\begin{pmatrix}
            \vec{a}_1 & \vec{a}_2 & \ldots & \vec{a}_k & \vec{b}
        \end{pmatrix}$ совместна. Записывая условие совместности с помощью теоремы Кронекера-Капелли, получим уравнения, связывающие координаты вектора $\vec{b}$. Эти уравнения составляют СЛАУ, неявно описывающую подпространство $\mathcal{H} = \Span(\vec{a}_1, \vec{a}_2, \ldots, \vec{a}_k)$.
    
    \item Пусть $\mathcal{H} = \Span(\vec{a}_1, \vec{a}_2, \ldots, \vec{a}_k)$. Составим матрицу $A$ из столбцов координат векторов $\vec{a}_1, \vec{a}_2, \ldots, \vec{a}_k$ в некотором базисе. Решим однородную СЛАУ $A^T\vec{x} = \vec{0}$, найдя какую-либо ФСР этой СЛАУ. Из столбцов ФСР составим матрицу $F$. Однородная СЛАУ $F^T\vec{x} = \vec{0}$ неявно описывает подпространство $\mathcal{H}$.
    \end{enumerate}
\end{itemize}


\newpage
\section{
    Линейные подпространства. Определения, теоремы, примеры и контрпримеры. Базис и размерность. Привести примеры задания пространств и подпространств без использования матриц и СЛАУ.
}

% Линейные подпространства. Определения, теоремы, примеры и контрпримеры
\subsection{
    Линейные подпространства. Определения, теоремы, примеры и контрпримеры.
}

\begin{definition}
    Подмножество $\mathcal{L}_1 \subset \mathcal{L}$ называется \textbf{\textit{линейным подпространством}} над полем $\PP$, если \\ $\forall \vec{a}, \vec{b} \in \mathcal{L}_1, \forall \alpha, \beta \in \PP \colon \alpha \vec{a} + \beta \vec{b} \in \mathcal{L}$.
\end{definition}

\begin{theorem}
    Линейное подпространство является линейным пространством.
\end{theorem}

\begin{example}~

    \begin{enumerate}
        \item В любом линейном пространстве $\mathcal{L}$ всегда имеются два линейных подпространства: само пространство $\mathcal{L}$ и нулевое подпространство, состоящее из одного нулевого элемента. Эти подпространства называются \textbf{\textit{несобственными}}. Все остальные линейные пространства называются \textbf{\textit{собственными}}.
        \item Множество всех свободных векторов, параллельных данной плоскости, образуют линейное подпространство пространства $\mathcal{V}_3$ всех свободных векторов трехмерного пространства.
        \item В линейном пространстве $M_n(\RR)$ всех квадратных матриц порядка $n$ линейное подпространство образуют все симметрические матрицы.
    \end{enumerate}
\end{example}

\begin{counterexample}~

    \begin{enumerate}
        \item Множество всех векторов на плоскости, у которых первая координата положительна. Это множество не является подпространством, потому что оно не замкнуто относительно умножения на скаляр. Например, вектор $(1, 1)$ принадлежит этому множеству, но вектор $-1 \cdot (1, 1) = (-1, -1)$ — нет, так как первая координата отрицательна.
        \item Множество всех векторов в $\RR^3$, лежащих в первой октанте (где все координаты неотрицательны). Это множество не замкнуто относительно умножения на скаляр. Например, вектор $(1, 1, 1)$ принадлежит этому множеству, но вектор $-1 \cdot (1, 1, 1) = (-1, -1, -1)$ не принадлежит.
    \end{enumerate}
\end{counterexample}



\newpage


% Базис и размерность
\subsection{
    Базис и размерность.
}

\begin{definition}
    \textbf{\textit{Базисом линейного подпространства}} $\mathcal{L}$ называют любую упорядоченную систему векторов, для которой выполнены два условия:
    \begin{enumerate}[nosep]
        \item эта система векторов линейно независима.
        \item каждый вектор в линейном подпространстве может быть представлен в виде линейной комбинации векторов этой системы.
    \end{enumerate}
\end{definition}

\begin{definition}
    Максимальное количество линейно независимых векторов в данном линейном подпространстве называют \textbf{\textit{размерностью линейного подпространства}}.
\end{definition}



\newpage


% Привести примеры задания пространств и подпространств без использования матриц и СЛАУ
\subsection{
    Привести примеры задания пространств и подпространств без использования матриц и СЛАУ.
}

\begin{enumerate}
    \item В любом линейном пространстве $\mathcal{L}$ всегда имеются два линейных подпространства: само пространство $\mathcal{L}$ и нулевое подпространство, состоящее из одного нулевого элемента.
    \item Множество всех свободных векторов, параллельных данной плоскости, образуют линейное подпространство пространства $\mathcal{V}_3$ всех свободных векторов трехмерного пространства.
    \item Любая прямая, проходящая через начало координат $(0, 0, 0)$ в $\RR^3$.
    \item Линейное пространство - множество $P_n[x]$ многочленов переменного $x$ степени, не превышающей $n$. Для данного линейного пространства линейным подпространтсвом является множество $K_m[x]$ многочленов переменного $x$ степени, не превышающей $m$, где $m \leq n$. При этом подпространством не является множество всех многочленов степени ровно $m$. Например, сумма двух многочленов степени $m$ может иметь степень меньше $m$ (например, x² + (-x²) = 0).
    \item Множество функций, непрерывных на отрезке, с обычными операциями сложения функций и умножения функции на число.
\end{enumerate}


\newpage
\section{
    Сумма и пересечение линейных подпространств. Доказать, что указанные множества являются линейными подпространствами. Нахождение базисов для суммы и пересечения подпространств. Теорема о размерностях подпространств, суммы и пересечения.
}

% Сумма и пересечение линейных подпространств
\subsection{
    Сумма и пересечение линейных подпространств.
}   

\begin{definition}
    Пусть $\mathcal{H}_1$ и $\mathcal{H}_2$ - линейные подпространства в линейном пространстве $\mathcal{L}$. Множество $\mathcal{H}_1 + \mathcal{H}_2$ всех векторов $\vec{x}$ вида $\vec{x} = \vec{x_1} + \vec{x_2}$, где $\vec{x_1} \in \mathcal{H}_1$, $\vec{x_2} \in \mathcal{H}_2$, называют \textbf{\textit{суммой линейных подпространств}} $\mathcal{H}_1$ и $\mathcal{H}_2$. 
\end{definition}

\begin{figure}[H]
    \centering
    \includegraphics[scale=0.7]{images/module1/question04/1.jpg}
    \label{fig:picture_04_1}
    \caption{Сумма линейных подпространств.}
\end{figure}

\begin{definition}
    Пусть $\mathcal{H}_1$ и $\mathcal{H}_2$ - линейные подпространства в линейном пространстве $\mathcal{L}$. Множество $\mathcal{H}_1 \cap \mathcal{H}_2$ всех векторов $\vec{x}$, где $\vec{x} \in \mathcal{H}_1$ и $\vec{x} \in \mathcal{H}_2$, называют \textbf{\textit{пересечением линейных подпространств}} $\mathcal{H}_1$ и $\mathcal{H}_2$.
\end{definition}

\begin{figure}[H]
    \centering
    \includegraphics[scale=0.4]{images/module1/question04/2.jpg}
    \label{fig:picture_04_2}
    \caption{Пересечение линейных подпространств.}
\end{figure}




\newpage


% Доказать, что указанные множества (сумма и пересечение ЛПП) являются линейными подпространствами
\subsection{
    Доказать, что указанные множества (сумма и пересечение ЛПП) являются линейными подпространствами.
}

\begin{theorem}
    Сумма линейных подпространств данного линейного пространства является линейным подпространством в том же линейном пространстве.
\end{theorem}

\begin{proof}~

    Проверим, выполняются ли условия определения линейного подпространства:
    \begin{enumerate}[nosep]
        \item Рассмотрим два вектора $\vec{v}$ и $\vec{w}$ из множества $\mathcal{H}_1 + \mathcal{H}_2$. Согласно определению суммы линейных подпространств, имеют место представления $\vec{v} = \vec{x_1} + \vec{x_2}$, $\vec{w} = \vec{y_1} + \vec{y_2}$, где векторы $\vec{x_i}$, $\vec{y_i}$ принадлежат $\mathcal{H}_i$, $i = 1, 2$. Складывая эти равенства, получаем
        $$\vec{v} + \vec{w} = (\vec{x_1} + \vec{y_1}) + (\vec{x_2} + \vec{y_2}).$$
        Сумма $\vec{x_1} + \vec{y_1}$ векторов $\vec{x_1}$ и $\vec{y_1}$ линейного подпространства $\mathcal{H}_1$ принадлежит $\mathcal{H}_1$. Точно так же сумма $\vec{x_2} + \vec{y_2}$ векторов $\vec{x_2}$ и $\vec{y_2}$ линейного подпространства $\mathcal{H}_2$ принадлежит $\mathcal{H}_2$. Поэтому вектор $\vec{v} + \vec{w}$ принадлежит множеству $\mathcal{H}_1 + \mathcal{H}_2$.
        \item Произвольный вектор $\vec{v} \in \mathcal{H}_1 + \mathcal{H}_2$ имеет представление $\vec{v} = \vec{x_1} + \vec{x_2}$, где $\vec{x_1} \in \mathcal{H}_1$, $\vec{x_2} \in \mathcal{H}_2$. Для любого действительного числа $\lambda$ получаем равенства 
        $$\lambda \vec{v} = \lambda (\vec{x_1} + \vec{x_2}) = \lambda \vec{x_1} + \lambda \vec{x_2}.$$
        Так как вектор $\lambda \vec{x_1}$ принадлежит $\mathcal{H}_1$, а вектор $\lambda \vec{x_2}$ - $\mathcal{H}_2$, то вектор $\lambda \vec{u}$ является элементом множества $\mathcal{H}_1 + \mathcal{H}_2$.
    \end{enumerate}
    Мы доказали, что множество $\mathcal{H}_1 + \mathcal{H}_2$ замкнуто относительно линейных операций объемлющего линейного пространства и поэтому, согласно определению линейного подпространства, оно является линейным подпространством.
\end{proof}

\begin{theorem}
    Пересечение $\mathcal{H}_1 \cap \mathcal{H}_2$ двух линейных подпространств $\mathcal{H}_1$ и $\mathcal{H}_2$ в линейном пространстве $\mathcal{L}$ является линейным подпространством в $\mathcal{L}$.
\end{theorem}

\begin{proof}~

    Проверим, выполняются ли условия определения линейного подпространства:
    \begin{enumerate}[nosep]
        \item Если векторы $\vec{x_1}$ и $\vec{x_2}$ принадлежат $\mathcal{H}_1 \cap \mathcal{H}_2$, то каждый из этих векторов принадлежит как $\mathcal{H}_1$, так и $\mathcal{H}_2$. Поскольку $\mathcal{H}_1$ - линейное подпространство, то согласно определению линейного подпространства, заключаем, что вектор $\vec{x_1} + \vec{x_2}$, равный сумме векторов этого линейного подпространства, тоже принадлежит $\mathcal{H}_1$. Аналогично $\vec{x_1} + \vec{x_2} \in \mathcal{H}_2$, так как каждое из слагаемых является элементом линейного подпространства $\mathcal{H}_2$. Следовательно, $\vec{x_1} + \vec{x_2} \in \mathcal{H}_1 \cap \mathcal{H}_2$.
        \item Выберем произвольный вектор $\vec{x} \in \mathcal{H}_1 \cap \mathcal{H}_2$. Тогда $\vec{x} \in \mathcal{H}_1$ и $\vec{x} \in \mathcal{H}_2$. Так как $\mathcal{H}_1$ является линейным подпространством, то произведение элемента $\vec{x}$ этого линейного подпространства на произвольное действительное число $\lambda$ принадлежит $\mathcal{H}_1$. Но совершенно аналогично вектор $\lambda \vec{x}$ принадлежит и $\mathcal{H}_2$. Поэтому $\lambda \vec{x} \in \mathcal{H}_1 \cap \mathcal{H}_2$. 
    \end{enumerate}
    Итак, оба условия определения линейного подпространства выполнены. Следовательно, $\mathcal{H}_1 \cap \mathcal{H}_2$ является линейным подпространством.
\end{proof}



\newpage

% Нахождение базисов для суммы и пересечения подпространств
\subsection{
    Нахождение базисов для суммы и пересечения подпространств.
}

\begin{theorem}
    Если $\{e\}$ - базис $\mathcal{L}_1$, $\{f\}$ - базис $\mathcal{L}_2$, $\ldots$, $\{g\}$ - базис $\mathcal{L}_k$, то $\sum_{j=1}^{k} L_j = span(e, f, \ldots, g)$.
\end{theorem}

\begin{proof}
    $\vec{x} = \underbrace{\vec{x_1}}_{\text{расклад. по $e$}} + \underbrace{\vec{x_2}}_{\text{расклад. по $f$}} + \ldots + \underbrace{\vec{x_k}}_{\text{расклад. по $g$}}$
\end{proof}

\begin{comment}
    Набор $(e, f, \ldots, g)$ может быть избыточен; нужны только ЛНЗ векторы.
\end{comment}

\begin{proposition}
    $\dim \sum_{j=1}^{k} \mathcal{L}_j = rank(e, f, \ldots, g)$.
\end{proposition}

Пусть $\mathcal{L}_1, \mathcal{L}_2, \ldots, \mathcal{L}_k$ заданы с помощью СЛАУ. Базисом суммы подпространств $\mathcal{L}_1 + \mathcal{L}_2 + \ldots + \mathcal{L}_k$ будет любая её ФСР. Базисом пересечения подпространств $\mathcal{L}_1 \cap \mathcal{L}_2 \cap \ldots \cap \mathcal{L}_k$ будет любая его ФСР.



\newpage


% Теорема о размерностях подпространств, суммы и пересечения
\subsection{
    Теорема о размерностях подпространств, суммы и пересечения.
}

\begin{theorem}
    Если $\mathcal{H}$ - линейное подпространство линейного пространства $\mathcal{L}$, то $\dim \mathcal{H} \leq \dim \mathcal{L}$. Если к тому же $\mathcal{H} \ne \mathcal{L}$, то $\dim \mathcal{H} < \dim \mathcal{L}$.
\end{theorem}

\begin{proof}~

    Любой базис линейного подпространства $\mathcal{H}$ является ЛНЗ системой векторов в линейном пространстве $\mathcal{L}$. Если этот базис из $\mathcal{H}$ является базисом и в $\mathcal{L}$, то согласно теореме \eqref{thm:theorem_2_2}, $\dim \mathcal{H} = \dim \mathcal{L}$ и ясно, что в этом случае $\mathcal{H} = \mathcal{L}$, так как у них общий базис. Если базис $\mathcal{H}$ не является базисом $\mathcal{L}$, то $\exists \vec{x} \in \mathcal{L}$, который не является линейной комбинацией векторов этого базиса $\implies \mathcal{H} \ne \mathcal{L}$. Добавив вектор $x$ к векторам базиса, получим ЛНЗ систему векторов. Значит, в $\mathcal{L}$ больше ЛНЗ векторов, чем в $\mathcal{H} \implies \dim \mathcal{H} < \dim \mathcal{L}$.
\end{proof}

\begin{theorem}
    Если $\mathcal{H}_1$ и $\mathcal{H}_2$ - линейные подпространства линейного пространства $\mathcal{L}$, то
    $$\dim(\mathcal{H}_1 + \mathcal{H}_2) = \dim\mathcal{H}_1 + \dim\mathcal{H}_2 - \dim(\mathcal{H}_1 \cap \mathcal{H}_2).$$
    \label{thm:theorem_4_6}
\end{theorem}

\begin{proof}~

    В линейном подпространстве $\mathcal{H}_1 \cap \mathcal{H}_2$ выберем некоторый базис $\vec{e} = (\vec{e_1}, \ldots, \vec{e_m})$. Множество $\mathcal{H}_1 \cap \mathcal{H}_2$ является линейным подпространством не только в $\mathcal{L}$, но и в его части $\mathcal{H}_1$. Поэтому выбранный базис можно дополнить некоторой системой векторов $f = (\vec{f_1}, \ldots, \vec{f_l})$ до базиса $(e, f)$ в линейном подпространстве $\mathcal{H}_1$. Аналогично систему $e$ можно дополнить некоторым набором векторов $g = (\vec{g_1}, \ldots, \vec{g_k})$ до базиса $(e, g)$ в $\mathcal{H}_2$. Докажем, что система векторов
    $$(e, f, g) = (\vec{e_1}, \ldots, \vec{e_m}, \vec{f_1}, \ldots, \vec{f_l}, \vec{g_1}, \ldots, \vec{g_k})$$
    является базисом в линейном пространстве $\mathcal{H}_1 + \mathcal{H}_2$.

    \textbf{Во-первых}, установим, что указанная система линейно независима. Пусть имеет место равенство
    $$\alpha_1\vec{e_1} + \ldots + \alpha_m\vec{e_m} + \beta_1\vec{f_1} + \ldots + \beta_l\vec{f_l} + \gamma_1\vec{g_1} + \ldots \gamma_k\vec{g_k} = \vec{0}$$
    Тогда для вектора
    \begin{equation}
        \vec{y} = \beta_1\vec{f_1} + \ldots + \beta_l\vec{f_l}.
        \label{eq:theorem_4_6_1}
    \end{equation}
    выполнено равенство
    \begin{equation}
        \vec{y} = -\alpha_1\vec{e_1} - \ldots -\alpha_m\vec{e_m} - \gamma_1\vec{g_1} - \ldots - \gamma_k\vec{g_k}.
        \label{eq:theorem_4_6_2}
    \end{equation}
    Согласно равенству \eqref{eq:theorem_4_6_1} заключаем, что $\vec{y} \in \mathcal{H}_1$, а согласно \eqref{eq:theorem_4_6_2} делаем вывод, что $\vec{y} \in \mathcal{H}_2$. Следовательно, $\vec{y} \in \mathcal{H}_1 \cap \mathcal{H}_2$ и потому имеет единственное разложение
    \begin{equation}
        \vec{y} = \delta_1\vec{e_1} + \ldots + \delta_m\vec{e_m}.
        \label{fig:theorem_4_6_3}
    \end{equation}
    по базису $e$ линейного пространства $\mathcal{H}_1 \cap \mathcal{H}_2$.

    Рассмотрев разложение по $(e, g)$ и $(e, f)$, получим, что все коэффициенты равны нулю. Значит, система векторов $(e, f, g)$ линейно независима.

    \textbf{Во-вторых}, всякий вектор $\vec{y} \in \mathcal{H}_1 + \mathcal{H}_2$ есть линейная комбинация системы векторов $(e, f, g)$. Действительно, такой вектор представим в виде $\vec{y} = \vec{y_1} + \vec{y_2}$, где $\vec{y_1} \in \mathcal{H}_1$, $\vec{y_2} \in \mathcal{H}_2$. Вектор $\vec{y_1}$ представляется линейной комбинацией системы векторов $(e, f)$, а $\vec{y_2}$ - линейной комбинацией системы векторов $(e, g)$. Поэтому $\vec{y}$ разлагается по системе векторов $(e, f, g)$.

    Итак, система векторов $(e, f, g)$ линейно независима и любой вектор из $\mathcal{H}_1 + \mathcal{H}_2$ разлагается по этой системе. Следовательно, $(e, f, g)$ - базис $\mathcal{H}_1 + \mathcal{H}_2$. Остается подсчитать размерности:
    
    \begin{table}[H]
        \centering
        \begin{tabular}{|c|c|c|}
            \hline
            Линейное подпространство & базис & размерность\\
            \hline
            $\mathcal{H}_1$ & $(e, f)$ & $m + l$\\
            \hline
            $\mathcal{H}_2$ & $(e, g)$ & $m + k$\\
            \hline
            $\mathcal{H}_1 \cap \mathcal{H}_2$ & $e$ & $m$\\
            \hline
            $\mathcal{H}_1 + \mathcal{H}_2$ & $(e, f, g)$ & $m + l + k$\\
            \hline
        \end{tabular}
        \label{tab:my_label}
    \end{table}
    Таким образом, получаем утверждение теоремы.
\end{proof}

\begin{corollary}
    $\dim(\mathcal{H}_1 \oplus \mathcal{H}_2) = \dim \mathcal{H}_1 + \dim \mathcal{H}_2.$
    \label{cor:corollary_4_6}
\end{corollary}


\newpage
\section{
    Прямая сумма подпространств. Критерий прямой суммы.
}

% Прямая сумма подпространств. Критерий прямой суммы
Сначала сформулируем и докажем теоремы Фредгольма для операторов в линейных пространствах:

\subsection{
    Альтернатива Фредгольма.
}

\begin{theorem}["Альтернатива Фредгольма"] Пусть

    \begin{enumerate}
        \item $\left.\begin{array}{l}
            \mathcal{V} \text{ - линейное пространство, } \dim \mathcal{V} = n \\
            \mathcal{W} \text{ - линейное пространство, } \dim \mathcal{W} = m
        \end{array}\right\}$ \text{В обоих задано скалярное произведение.}
        \item $\mathscr{A} \colon \mathcal{V} \to \mathcal{W}$ - линейный оператор.
    \end{enumerate}

    Тогда справедливо ровно одно из двух:

    \begin{itemize}[nosep]
        \item либо уравнение $\mathscr{A}\vec{v} = \vec{w}$ имеет решение при любом $\vec{w} \in \mathcal{W}$,
        \item либо уравнение $\mathscr{A^*}\vec{w} = \vec{0}$ имеет нетривиальное (ненулевое) решение.
    \end{itemize}
\end{theorem}

\begin{proof}~

    Обозначим $r = \rank(\mathscr{A})$.

    \textbf{1 случай:}
    
    \begin{gather*}
        r = m \\
        \downimplies \\
        \dim(\Im \mathscr{A}) = m \\
        \downimplies \\
        \text{Т.к. } \Im \mathscr{A} \text{— } m\text{-мерное подпространство } m\text{-мерного пространства } \mathcal{W}, \text{то } \Im \mathscr{A} = \mathcal{W}. \\
        \downimplies \\
        \forall \vec{w} \in \mathcal{W} \exists \vec{v} \in \mathcal{V} \colon \mathscr{A}\vec{v} = \vec{w}. \\
        \text{Также заметим, что }\dim(\Im \mathscr{A^*}) + \dim(\ker \mathscr{A^*}) = m \\
        \downimplies \\
        \dim(\Im \mathscr{A}) + \dim(\ker \mathscr{A^*}) = m \\
        \downimplies \\
        m + \dim(\ker \mathscr{A^*}) = m \\
        \downimplies \\
        \dim(\ker \mathscr{A^*}) = 0 \\
        \downimplies \\
        \ker \mathscr{A^*} = \{\vec{0}\} \\
        \downimplies \\
        \text{Уравнение } \mathscr{A^*}\vec{w} = \vec{0} \text{ имеет только тривиальное решение.}
    \end{gather*}

    Получается, что второе "либо" не выполнено!
    
    \textbf{2 случай:}

    \begin{gather*}
        r < m \\
        \downimplies \\
        \dim(\Im \mathscr{A}) < \dim \mathcal{W} \\
        \downimplies \\
        \dim(\Im \mathscr{A^*}) < \dim \mathcal{W} \\
        \downimplies \\
        \text{Т.к. } \dim(\Im \mathscr{A^*}) + \dim(\ker \mathscr{A^*}) = \dim \mathcal{W}, \ker \mathscr{A^*} \ne \{\vec{0}\} \\
        \downimplies \\
        \exists \vec{w} \ne 0 \in \ker \mathscr{A^*} \\
        \downimplies \\
        \exists \vec{w} \ne 0 \colon \mathscr{A^*}\vec{w} = \vec{0}
    \end{gather*}
\end{proof}


\newpage
\section{
    Линейное аффинное многообразие. Вектор сдвига. Пересечение линейных аффинных многообразий. 
}

% Линейное аффинное многообразие. Вектор сдвига
\subsection{
    Линейное аффинное многообразие. Вектор сдвига.
}

\begin{definition}
    Пусть $\mathcal{V}$ - линейное пространство над полем $\PP$, $\mathcal{W}$ - это его подпространство. Зафиксируем вектор $\vec{a} \in \mathcal{V}$. Тогда множество $\vec{a} + \mathcal{W} = \{\, \vec{a} + \vec{x} \mid \vec{x} \in \mathcal{W} \,\}$ называется \textbf{\textit{линейным аффинным многообразием}}.

    При этом подпространство $\mathcal{W}$ называется \textbf{\textit{направляющим подпространством}}, а вектор $\vec{a}$ называется \textbf{\textit{вектором сдвига}}.
\end{definition}

Заметим, что ЛАМ, вообще говоря, не является подпространством (например, потому что оно может не содержать $\vec{0} \in \mathcal{V}$.)

\begin{proof}~

    Если $\vec{a} \notin \mathcal{W}$, то согласно аксиоме линейного пространства, и $(-\vec{a}) \notin \mathcal{W}$. 

    \begin{gather*}
        \text{Предположим, что } \underbrace{\vec{a} + (-\vec{a})}_{\vec{0}} \in \vec{a} + \mathcal{W}. \\
        \exists \vec{w} \in \mathcal{W} (\vec{a} + (-\vec{a}) = \vec{a} + \underbrace{\vec{w}}_{\in \mathcal{W}}). \\
        \exists \vec{w} \in \mathcal{W} ((-\vec{a}) = \vec{w}). \\
        \downimplies \\
        -\vec{a} \notin \mathcal{W}\text{. Противоречие.}
    \end{gather*}
    Это и означает, что если $\vec{a} \notin \mathcal{W}$, то $\vec{a} + \mathcal{W}$ не является подпространством.
\end{proof}

\begin{definition}
    \textbf{\textit{Размерностью}} линейного аффинного многообразия $\vec{a} + \mathcal{W}$ называется размерность $\mathcal{W}$, т.е. $$\dim (\vec{a} + \mathcal{W}) = \dim \mathcal{W}.$$
\end{definition}

\begin{definition}
    Пусть 

    \begin{enumerate}
        \item $\mathcal{V}$ - линейное пространство,
        \item $\dim \mathcal{V} = n$,
        \item $\mathcal{W}$ - некоторое подпространство $\mathcal{V}$.
    \end{enumerate}
    
    Тогда \textbf{\textit{гиперплоскостью}} в нем будет называться ЛАМ вида $(\vec{a} + \mathcal{W})$, где $\dim \mathcal{W} = n - 1$.
\end{definition}



\newpage


% Пересечение линейных аффинных многообразий
\subsection{
    Пересечение линейных аффинных многообразий.
}

\begin{theorem}
    Пересечение двух линейных аффинных многообразий одного линейного пространства либо пусто, либо является линейным аффинным многообразием.
\end{theorem}

\begin{proof}~

    Пусть дано линейное пространство $\mathcal{V}$, в нем есть некоторые линейные подпространства $\mathcal{U}$ и $\mathcal{W}$. Зафиксируем $\vec{a}, \vec{b} \in \mathcal{V}$. Тогда возникают ЛАМ-я $(\vec{a} + \mathcal{U})$ и $(\vec{b} + \mathcal{W})$.

    Рассмотрим их пересечение $(\vec{a} + \mathcal{U}) \cap (\vec{b} + \mathcal{W})$.

    \textbf{1 случай}: $(\vec{a} + \mathcal{U}) \cap (\vec{b} + \mathcal{W}) = \emptyset$.

    Тогда всё доказано. Достаточно привести пример, когда действительно $(\vec{a} + \mathcal{U}) \cap (\vec{b} + \mathcal{W}) = \emptyset$.

    \bigbreak

    \textbf{2 случай}: $(\vec{a} + \mathcal{U}) \cap (\vec{b} + \mathcal{W}) \ne \emptyset$.

    Зафиксируем некоторый вектор $\vec{c} \in \vec{a} + \mathcal{U}$ и $\vec{c} \in \vec{b} + \mathcal{W}$, т.е. $\vec{c} = \vec{a} + \underbrace{\vec{u}}_{\in \mathcal{U}} = \vec{b} + \underbrace{\vec{w}}_{\in \mathcal{W}}$.

    Докажем, что $(\vec{a} + \mathcal{U}) \cap (\vec{b} + \mathcal{W}) = \vec{c} + \underbrace{(\mathcal{U} \cap \mathcal{W})}_{\text{явл. ЛПП } \mathcal{V}.}$.

    \begin{enumerate}
        \item[$\subseteq$] Рассмотрим произвольный вектор $\vec{\varphi} = \vec{a} + \underbrace{\vec{x}}_{\in \mathcal{U}} =\vec{b} + \underbrace{\vec{y}}_{\in \mathcal{W}}.$
    
        Докажем, что он лежит в $\vec{c} + (\mathcal{U} \cap \mathcal{W})$.
    
        $$\vec{\varphi} = \vec{a} + \vec{x} = (\vec{a} + \vec{u}) + ((-\vec{u}) + \vec{x}) = \vec{c} + \underbrace{(\underbrace{(-\vec{u})}_{\in \mathcal{U}} + \underbrace{\vec{x}}_{\in \mathcal{U}})}_{\in \mathcal{U}} \in \vec{c} +\mathcal{U}.$$
    
        Аналогично,
    
        $$\vec{\varphi} = \vec{b} + \vec{y} = (\vec{b} + \vec{w}) + ((-\vec{w}) + \vec{y}) = \vec{c} + \underbrace{(\underbrace{(-\vec{w})}_{\in \mathcal{W}} + \underbrace{\vec{y}}_{\in \mathcal{W}})}_{\in \mathcal{W}} \in \vec{c} +\mathcal{W}.$$
    
        Итак, $\vec{\varphi} - \vec{c} \in \mathcal{U}$ и $\vec{\varphi} - \vec{c} \in \mathcal{W}$. Значит, $\vec{\varphi} - \vec{c} \in \mathcal{U} \cap \mathcal{W} \Rightarrow \vec{\varphi} \in \vec{c} + (\mathcal{U} \cap \mathcal{W})$, ч.т.д.
        
        \item[$\supseteq$] Возьмем произвольный элемент $\vec{c} + \vec{z}$, где $\vec{z} \in \mathcal{U}$ и $\vec{z} \in \mathcal{W}$.

        Докажем, что он лежит в $(\vec{a} + \mathcal{U})$. Действительно, 
    
        $$\vec{c} + \vec{z} = \vec{a} + \underbrace{(\underbrace{(-\vec{a}) + \vec{c}}_{\vec{u} \in \mathcal{U}} + \underbrace{\vec{z}}_{\in \mathcal{U}})}_{\in \mathcal{U}} \in \vec{a} + \mathcal{U.}$$
    
        Докажем, что он лежит в $(\vec{b} + \mathcal{W})$. Действительно, 
    
        $$\vec{c} + \vec{z} = \vec{b} + \underbrace{(\underbrace{(-\vec{b}) + \vec{c}}_{\vec{w} \in \mathcal{W}} + \underbrace{\vec{z}}_{\in \mathcal{W}})}_{\in \mathcal{W}} \in \vec{b} + \mathcal{W.}$$
    
        Таким образом, $\vec{c} + \vec{z} \in (\vec{a} + \mathcal{U}) \cap (\vec{b} + \mathcal{W})$.
    \end{enumerate}
\end{proof}


\newpage
\section{
    Линейное аффинное многообразие. Вектор сдвига. Представление $k$-мерного линейного аффинного многообразия. Решения неоднородной СЛАУ. 
}

% Представление $k$-мерного линейного аффинного многообразия
\subsection{
    Представление $k$-мерного линейного аффинного многообразия.
}

Пусть даны $\mathcal{V}$ - линейное пространство над $\PP$, $\mathcal{W}$ - его подпространство, $\vec{a} \in \mathcal{V}$ - некоторый вектор.

\begin{definition}
    \textbf{\textit{Аффинной линейной комбинацией}} произвольных $s$ векторов $\vec{b}_1, \vec{b}_2, \ldots, \vec{b}_s \in \mathcal{V}$ называется вектор

    $$\lambda_0\vec{a} + \lambda_1\vec{b}_1 + \lambda_2\vec{b}_2 +  \ldots + \lambda_s\vec{b}_s,$$

    где числа $\lambda_i$ удовлетворяют соотношению $\lambda_0 + \lambda_1 + \ldots + \lambda_s = 1$.
\end{definition}

\begin{definition}
    \textbf{\textit{Аффинной оболочкой заданных векторов $\vec{b}_1, \vec{b}_2, \ldots, \vec{b}_s \in \mathcal{V}$}} называется множество всех их аффинных линейных комбинаций.
\end{definition}

\begin{designation}
    $\Aff(\vec{b}_1, \vec{b}_2, \ldots, \vec{b}_s)$.
\end{designation}

\begin{theorem}
    Всякое $k$-мерное ЛАМ $(\vec{a} + \mathcal{W})$ линейного пространства $\mathcal{V}$ может быть представлено как аффинная линейная оболочка $\leq k$ векторов.
\end{theorem}

\begin{proof}~

    \begin{gather*}
        (\vec{a} + \mathcal{W})\text{ - это $k$-мерное ЛАМ.} \\
        \downimplies \\
        \dim \mathcal{W} = k. \\
        \downimplies \\
        \text{Можно зафиксировать базис }\mathcal{W}\text{ - векторы }\vec{e}_1, \vec{e}_2, \ldots, \vec{e}_k. \\
        \text{Тогда рассмотрим векторы: } \\
        \vec{v}_1 = \vec{a} + \vec{e}_1 \\
        \vec{v}_2 = \vec{a} + \vec{e}_2 \\
        \vdots \\
        \vec{v}_k = \vec{a} + \vec{e}_k. \\
        \text{И докажем, что } \vec{a} + \mathcal{W} = \Aff(\vec{v}_1, \vec{v}_2, \ldots, \vec{v}_k).
    \end{gather*}

    \begin{enumerate}
        \item[$\subseteq$] Возьмем любой $\vec{a} + \vec{w} \in \vec{a} + \mathcal{W}$.

        Тогда 

        \begin{align*}
            \vec{a} + \vec{w} &= \vec{a} + \lambda_1\vec{e}_1 + \ldots + \lambda_k\vec{e}_k = \\
            &= \vec{a} + \lambda_1(\vec{v}_1 - \vec{a}) + \ldots + \lambda_k(\vec{v}_k - \vec{a}) = \\
            &= \vec{a} + \lambda_1\vec{v}_1 - \lambda_1\vec{a} + \ldots + \lambda_k\vec{v}_k - \lambda_k\vec{a} = \\
            &= (\underbrace{1 - \lambda_1 - \lambda_2 - \ldots - \lambda_k}_{\lambda_0})\vec{a} + \lambda_1\vec{v}_1 + \ldots + \lambda_k\vec{v}_k = \\
            &= \lambda_0\vec{a} + \lambda_1\vec{v}_1 + \ldots + \lambda_k\vec{v}_k,
        \end{align*}
        при этом $\lambda_0 + \lambda_1 + \ldots + \lambda_k = 1$.
        
        Значит, $\vec{a} + \vec{w} \in \Aff(\vec{v}_1, \vec{v}_2, \ldots, \vec{v}_k)$.
        
        \item[$\supseteq$] Рассмотрим любой $\vec{x} \in \Aff(\vec{v}_1, \vec{v}_2, \ldots, \vec{v}_k)$.

        Тогда $\vec{x} = \lambda_0\vec{a} + \lambda_1\vec{v}_1 + \ldots + \lambda_k\vec{v}_k$, где $\lambda_0 + \lambda_1 + \ldots + \lambda_k = 1$. Выразим $\lambda_0 = 1 - \lambda_1 - \lambda_2 - \ldots - \lambda_k$.

        Тогда

        \begin{align*}
            \vec{x} &= (1 - \lambda_1 - \lambda_2 - \ldots - \lambda_k)\vec{a} + \lambda_1\vec{v}_1 + \ldots + \lambda_k\vec{v}_k = \\
            &= \vec{a} + \lambda_1(\vec{v}_1 - \vec{a}) + \ldots + \lambda_k(\vec{v}_k - \vec{a}) = \\
            &= \vec{a} + \underbrace{\lambda_1\vec{e}_1 + \ldots + \lambda_k\vec{e}_k}_{\in \mathcal{W}} \in \vec{a} + \mathcal{W}.
        \end{align*}
    \end{enumerate}
\end{proof}



\newpage


% Связь с решениями неоднородной СЛАУ
\subsection{
    Связь с решениями неоднородной СЛАУ.
}

Рассмотрим неоднородную СЛАУ

$$A\vec{x} = \vec{b},$$

где $A \in \RR^{m \times n}, \vec{x} = \begin{pmatrix} x_1 \\ \vdots \\ x_n \end{pmatrix} \in \RR^n, \vec{b} = \begin{pmatrix} b_1 \\ \vdots \\ b_m \end{pmatrix} \in \RR^m$.

Из курса Ан. Геом. известно, что если СЛАУ имеет бесконечно много решений, то они задаются так:

$$\vec{x} = \vec{x}_0 + c_1\vec{f}_1 + \ldots + c_k\vec{f}_k,$$

где $\vec{x}_0 \in \RR^n$ - вектор с постоянными коэффициентами, $\vec{f}_1, \ldots, \vec{f}_k \in \RR^n$ - векторы, образующие ФСР, $c_1, \ldots, c_k$ - произвольные константы.

Если рассмотреть $\mathcal{W} = \Span \{\vec{f}_1, \ldots, \vec{f}_k\}$, то получится, что множество $\mathcal{U}$ всех решений СЛАУ будет представлять из себя ЛАМ:

$$\mathcal{U} = \vec{x}_0 + \mathcal{W}.$$

Представляете!


\newpage
\section{
    Линейное аффинное многообразие. Вектор сдвига. Пересечение линейного аффинного многообразия с подпространством, дополнительным к его направляющему подпространству.
}

% Пересечение линейного аффинного многообразия с подпространством, дополнительным к его направляющему подпространству
Сначала сформулируем и докажем теоремы Фредгольма для операторов в линейных пространствах:

\subsection{
    Альтернатива Фредгольма.
}

\begin{theorem}["Альтернатива Фредгольма"] Пусть

    \begin{enumerate}
        \item $\left.\begin{array}{l}
            \mathcal{V} \text{ - линейное пространство, } \dim \mathcal{V} = n \\
            \mathcal{W} \text{ - линейное пространство, } \dim \mathcal{W} = m
        \end{array}\right\}$ \text{В обоих задано скалярное произведение.}
        \item $\mathscr{A} \colon \mathcal{V} \to \mathcal{W}$ - линейный оператор.
    \end{enumerate}

    Тогда справедливо ровно одно из двух:

    \begin{itemize}[nosep]
        \item либо уравнение $\mathscr{A}\vec{v} = \vec{w}$ имеет решение при любом $\vec{w} \in \mathcal{W}$,
        \item либо уравнение $\mathscr{A^*}\vec{w} = \vec{0}$ имеет нетривиальное (ненулевое) решение.
    \end{itemize}
\end{theorem}

\begin{proof}~

    Обозначим $r = \rank(\mathscr{A})$.

    \textbf{1 случай:}
    
    \begin{gather*}
        r = m \\
        \downimplies \\
        \dim(\Im \mathscr{A}) = m \\
        \downimplies \\
        \text{Т.к. } \Im \mathscr{A} \text{— } m\text{-мерное подпространство } m\text{-мерного пространства } \mathcal{W}, \text{то } \Im \mathscr{A} = \mathcal{W}. \\
        \downimplies \\
        \forall \vec{w} \in \mathcal{W} \exists \vec{v} \in \mathcal{V} \colon \mathscr{A}\vec{v} = \vec{w}. \\
        \text{Также заметим, что }\dim(\Im \mathscr{A^*}) + \dim(\ker \mathscr{A^*}) = m \\
        \downimplies \\
        \dim(\Im \mathscr{A}) + \dim(\ker \mathscr{A^*}) = m \\
        \downimplies \\
        m + \dim(\ker \mathscr{A^*}) = m \\
        \downimplies \\
        \dim(\ker \mathscr{A^*}) = 0 \\
        \downimplies \\
        \ker \mathscr{A^*} = \{\vec{0}\} \\
        \downimplies \\
        \text{Уравнение } \mathscr{A^*}\vec{w} = \vec{0} \text{ имеет только тривиальное решение.}
    \end{gather*}

    Получается, что второе "либо" не выполнено!
    
    \textbf{2 случай:}

    \begin{gather*}
        r < m \\
        \downimplies \\
        \dim(\Im \mathscr{A}) < \dim \mathcal{W} \\
        \downimplies \\
        \dim(\Im \mathscr{A^*}) < \dim \mathcal{W} \\
        \downimplies \\
        \text{Т.к. } \dim(\Im \mathscr{A^*}) + \dim(\ker \mathscr{A^*}) = \dim \mathcal{W}, \ker \mathscr{A^*} \ne \{\vec{0}\} \\
        \downimplies \\
        \exists \vec{w} \ne 0 \in \ker \mathscr{A^*} \\
        \downimplies \\
        \exists \vec{w} \ne 0 \colon \mathscr{A^*}\vec{w} = \vec{0}
    \end{gather*}
\end{proof}


\newpage
\section{
    Скалярное произведение, примеры (привести три примера). Косинус. Евклидовы пространства. Понятие метрики и нормы, способы задания норм (привести три примера). 
}

% Скалярное произведение, примеры (привести три примера). Евклидовы пространства
\subsection{
    Скалярное произведение, примеры (привести три примера). Евклидовы пространства.
}

\begin{definition}
    Пусть дано линейное пространство $\mathcal{V} = \{\vec{a}, \vec{b}, \vec{c}, \vec{d}, \dots\}$. Множество вида 
    
    $\{(\vec{a}, \vec{a}), (\vec{a}, \vec{b}), (\vec{a}, \vec{c}), (\vec{a}, \vec{d}), \dots, (\vec{b}, \vec{a}), (\vec{b}, \vec{b}), (\vec{b}, \vec{c}), (\vec{b}, \vec{d}), \dots\}$ называется \textbf{\textit{декартовым квадратом}} $\mathcal{V} \times \mathcal{V}$.
\end{definition}

\begin{definition}
    Отображение $\mathcal{V} \times \mathcal{V} \to \RR$, где $\mathcal{V}$ - линейное пространство над полем $\RR$, называется \textbf{\textit{скалярным произведением}}, если выполнены 4 аксиомы:
    \begin{enumerate}[nosep]
        \item $(\vec{x}, \vec{y}) = (\vec{y}, \vec{x})$.
        \item $(\vec{x} + \vec{y}, \vec{z}) = (\vec{x}, \vec{z}) + (\vec{y}, \vec{z})$ - аддитивность по первому аргументу.
        \item $(\alpha \vec{x}, \vec{y}) = \alpha(\vec{x}, \vec{y})$ - однородность по первому аргументу.
        \item $(\vec{x}, \vec{x}) \geq 0$, причем $(\vec{x}, \vec{x}) = 0 \iff \vec{x} = 0$.
    \end{enumerate}
\end{definition}

\begin{definition}
    Вещественное линейное пространство с так введенным скалярным произведением называется \textbf{\textit{евклидовым пространством}}.
\end{definition}

\begin{designation}
    $\mathcal{E}$.
\end{designation}

\begin{definition}
    Отображение $\mathcal{V} \times \mathcal{V} \to \CC$, где $\mathcal{V}$ - линейное пространство над полем $\CC$, называется \textbf{\textit{скалярным произведением}}, если выполнены 4 аксиомы:
    \begin{enumerate}[nosep]
        \item $(\vec{x}, \vec{y}) = \overline{(\vec{y}, \vec{x})}$.
        \item $(\vec{x} + \vec{y}, \vec{z}) = (\vec{x}, \vec{z}) + (\vec{y}, \vec{z})$ - аддитивность по первому аргументу.
        \item $(\alpha \vec{x}, \vec{y}) = \alpha(\vec{x}, \vec{y})$ - однородность по первому аргументу.
        \item $(\vec{x}, \vec{x}) \geq 0$, причем $(\vec{x}, \vec{x}) = 0 \iff \vec{x} = 0$.
    \end{enumerate}
\end{definition}

\begin{definition}
    Комплексное линейное пространство с так введенным скалярным произведением называется \textbf{\textit{унитарным пространством}}.
\end{definition}

\begin{designation}
    $\mathcal{U}$.
\end{designation}

\begin{example}~
    \begin{enumerate}[nosep]
        \item В линейных пространствах $\mathcal{V}_2$ и $\mathcal{V}_3 \colon (\vec{x}, \vec{y}) = |\vec{x}||\vec{y}|\cos \widehat{(\vec{x}, \vec{y})}$.
        \item В арифметическом линейном пространстве $\RR^n \colon (\vec{x}, \vec{y}) = x_1y_1 + \dots + x_ny_n$. 
        \item Линейное пространство $C[0, 1]$ всех функций, непрерывных на отрезке $[0, 1]$ становится евклидовым, если в нем ввести скалярное произведение:
        $$(\vec{f}, \vec{g}) = \int_{0}^{1} f(x)g(x) \dd x.$$
    \end{enumerate}
\end{example}

\textbf{Свойства скалярного произведения.}

\begin{enumerate}[label={\arabic*°.}]
    \item $(\vec{x}, \vec{y} + \vec{z}) = (\vec{x}, \vec{y}) + (\vec{x}, \vec{z}).$
    
    $(\vec{x}, \vec{y} + \vec{z}) = (\vec{y} + \vec{z}, \vec{x}) = (\vec{y}, \vec{x}) + (\vec{z}, \vec{x}) = (\vec{x}, \vec{y}) + (\vec{x}, \vec{z}).$
    
    \item $(\vec{x}, \lambda \vec{y}) = \overline{\lambda}(\vec{x}, \vec{y}).$

    $(\vec{x}, \lambda \vec{y}) = \overline{(\lambda \vec{y}, \vec{x})} = \overline{\lambda \cdot (\vec{y}, \vec{x})} = \overline{\lambda} \cdot \overline{(\vec{y}, \vec{x})} = \overline{\lambda} (\vec{x}, \vec{y}).$
    
    \item $(\vec{x}, \vec{0}) = 0.$

    $(\vec{x}, \vec{0}) = (\vec{x}, 0 \cdot \vec{0}) = \overline{0} \cdot (\vec{x}, \vec{0}) = 0 \cdot (\vec{x}, \vec{0}) = 0.$
    
    \item $(\forall \vec{y} \colon(\vec{x}, \vec{y}) = 0 )\implies \vec{x} = \vec{0}.$
    
    Возьмем $\vec{y} = \vec{x}$. Тогда $(\vec{x}, \vec{x}) = 0$. Значит, по определению $(\vec{x}, \vec{x}) = 0$.
    
    \item Любое подпространство $\mathcal{E}$ ($\mathcal{U}$) само является евклидовым (унитарным).
    
    Непосредственная проверка всех аксиом ленейного пространства и скалярного произведения.
\end{enumerate}



\newpage


% Понятие нормы, способы задания норм (привести три примера)
\subsection{
    Понятие нормы, способы задания норм (привести три примера). 
}

\begin{definition}
    Функция, заданная на линейном пространстве $\mathcal{V}$, которая каждому вектору ставит в соответствие вещественное число, называется \textbf{\textit{нормой}}, если выполнены 3 аксиомы:
    \begin{enumerate}[nosep]
        \item $\norm{\vec{x}} \geq 0$, причем $\norm{\vec{x}} = 0 \iff \vec{x} = \vec{0}$;
        \item $\norm{\lambda \vec{x}} = |\lambda| \cdot  \norm{\vec{x}}, \thinspace \lambda \in \RR$;
        \item $\norm{\vec{x} + \vec{y}} \leq \norm{\vec{x}} + \norm{\vec{y}}$ (неравенство треугольника).
    \end{enumerate}
\end{definition}

\begin{definition}
    Линейное пространство с заданной нормой называется \textbf{\textit{нормированным}}.
\end{definition}

\begin{theorem}
    Всякое скалярное произведение в евклидовом пространстве определяет норму $\norm{\vec{x}} = \sqrt{(\vec{x}, \vec{x})}$.
\end{theorem}

\begin{proof}~

    Проверим норму с помощью трех аксиом:
    \begin{enumerate}[nosep]
        \item $(\vec{x}, \vec{x}) \geq 0 \implies$ заданная функция определена для любого вектора $\vec{x}$ евклидова пространства.
        \item $\norm{\lambda \vec{x}} = \sqrt{(\lambda \vec{x}, \lambda \vec{x})} = \sqrt{\lambda^2(\vec{x}, \vec{x})} = \sqrt{\lambda^2}\sqrt{(\vec{x}, \vec{x})} = |\lambda| \cdot \norm{\vec{x}}$.
        \item Воспользуемся неравенством Коши-Буняковского: 

        \begin{gather*}
            (\vec{x}, \vec{y}) \leq \sqrt{(\vec{x}, \vec{x})}\cdot\sqrt{(\vec{y}, \vec{y})} \\
            (\vec{x}, \vec{y}) \leq \norm{\vec{x}} \cdot \norm{\vec{y}}.
        \end{gather*}
        
        Используя это неравенство, получаем:

        \begin{align*}
            &\norm{\vec{x} + \vec{y}} ^2 = (\vec{x} + \vec{y}, \vec{x} + \vec{y}) = \\
            &= (\vec{x}, \vec{x}) + 2(\vec{x}, \vec{y}) + (\vec{y}, \vec{y}) \leq (\vec{x}, \vec{x}) + 2 \norm{\vec{x}} \cdot \norm{\vec{y}} + (\vec{y}, \vec{y}) = \\
            &= (\norm{\vec{x}} + \norm{\vec{y}})^2 \implies \norm{\vec{x} + \vec{y}} \leq \norm{\vec{x}} + \norm{\vec{y}}.
        \end{align*}
    \end{enumerate}
\end{proof}

\subsubsection*{
    Способы задания норм (привести три примера).
}

\begin{definition}
    Норма вида $\norm{\vec{x}}_2 = \sqrt{(\vec{x}, \vec{x})}$ называется \textbf{\textit{евклидовой}} $(l_2)$
\end{definition}

\begin{definition}
    Норма вида $\norm{\vec{x}}_1 = |x_1| + \dots + |x_n|$ называется \textbf{\textit{октаэдрической}} $(l_1)$
\end{definition}

\begin{definition}
    Норма вида $\norm{\vec{x}}_{\infty} = max\{|x_1|, \dots, |x_n|\}$ называется \textbf{\textit{кубической}} $(l_{\infty})$
\end{definition}



\newpage


% Понятие метрики
\subsection{
    Понятие метрики.
}

\begin{definition}
    Пусть $M$ - произвольное непустое множество. Отображение декартова квадрата $M \times M$ на поле $\RR$ называется метрикой, если оно удовлетворяет трем аксиомам:
    \begin{enumerate}
        \item $\rho(\vec{x}, \vec{y}) = \rho(\vec{y}, \vec{x})$.
        \item $\rho(\vec{x}, \vec{y}) \geq 0$, причем $\rho(\vec{x}, \vec{y}) = 0 \iff \vec{x} = \vec{y}$.
        \item $\rho(\vec{x}, \vec{y}) \leq \rho(\vec{x}, \vec{z}) + \rho(\vec{z}, \vec{y})$ - неравенство треугольника.
    \end{enumerate}
\end{definition}

\begin{example}~

    \begin{enumerate}
        \item $M = \RR, \rho(\vec{x}, \vec{y}) = |\vec{x} - \vec{y}|$.
        \item $M$ - произвольное непустое множество. Тогда дискретная метрика: 
        
        $\rho(\vec{x}, \vec{y}) = 
        \begin{cases}
        1, & \vec{x} \ne \vec{y} \\
        0, & \vec{x} = \vec{y}
        \end{cases}$.
    \end{enumerate}
\end{example}



\newpage


% Косинус
\subsection{
    Косинус.
}

\begin{definition}
    \textbf{\textit{Косинусом угла между}} $\vec{x}$ и $\vec{y} \in \mathcal{V}$ называется величина $\cos \widehat{(\vec{x}, \vec{y})} = \frac{(\vec{x}, \vec{y})}{\norm{\vec{x}} \cdot\norm{\vec{y}}}, \thinspace \varphi \in [0, \pi]$.
\end{definition}

\begin{definition}
    Пусть $\vec{x} \ne \vec{0}$ и $\vec{y} \ne \vec{0}$. Тогда \textbf{\textit{углом}} $\widehat{(\vec{x}, \vec{y})}$ называется число $\arccos{\frac{(\vec{x}, \vec{y})}{\sqrt{(\vec{x}, \vec{x})} \sqrt{(\vec{y}, \vec{y})}}}$.
\end{definition}



\newpage


% *Полезные факты, которые тоже могут быть на экзамене
\subsection{
    *Полезные факты, которые тоже могут быть на экзамене.
}

\begin{lemma}~

    Пусть

    \begin{enumerate}
        \item $\mathcal{E}$ - $n$-мерное евклидово пространство;
        \item $\vec{e}_1, \ldots, \vec{e}_n$ - некоторый базис $\mathcal{E}$;
        \item $\vec{x}_e = \begin{pmatrix}
            x_1 \\
            \vdots \\
            x_n
        \end{pmatrix}$ - столбец координат $\vec{x}$ в базисе $\vec{e}_1, \ldots, \vec{e}_n$;
        \item $\vec{y}_e = \begin{pmatrix}
            y_1 \\
            \vdots \\
            y_n
        \end{pmatrix}$ - столбец координат $\vec{y}$ в базисе $\vec{e}_1, \ldots, \vec{e}_n$;
    \end{enumerate}

    Тогда 
    $$(\vec{x}, \vec{y}) = \vec{x}^T_e\Gamma_e\vec{y}_e.$$
    \label{lemma:lemma_1}
\end{lemma}

\begin{proof}~

    \begin{align*}
        (\vec{x}, \vec{y}) &= (\sum_{i = 1}^nx_i\vec{e}_i, \sum_{j = 1}^ny_j\vec{e}_j) = \\ 
        &=\sum_{i = 1}^n\sum_{j =  1}^n(x_iy_j(\vec{e}_i, \vec{e}_j)) = \\
        &= \vec{x}^T_e\Gamma_e\vec{y}_e,
    \end{align*}

    где $\Gamma_e$ - матрица Грама для системы векторов $\vec{e}_1, \ldots, \vec{e}_n$.
\end{proof}

\begin{corollary}~

    Пусть $e$ - ОНБ $\mathcal{E}$. Тогда матрица Грама для этого базиса является единичной. Поэтому

    $$(\vec{x}, \vec{y}) = \vec{x}^T_eE\vec{y}_e = \vec{x}^T_e\vec{y}_e = x_1y_1 + x_2y_2 + \ldots + x_ny_n.$$

    В частности,

    $$\norm{\vec{x}} = \sqrt{(\vec{x}, \vec{x})} = \sqrt{x^T_ex_e} = \sqrt{x^2_1 + \ldots + x^2_n}.$$

    \label{corollary:corollary_1}
\end{corollary}


\newpage
\section{
    Евклидово пространство. Ортонормированный базис. Процесс ортогонализации Грама-Шмидта, вывод формулы. Построение ортонормированного базиса.
}

% Евклидово пространство
\subsection{
    Евклидово пространство.
}

\begin{definition}
    Отображение $\mathcal{V} \times \mathcal{V} \to \RR$, где $\mathcal{V}$ - линейное пространство над полем $\RR$, называется \textbf{\textit{скалярным произведением}}, если выполнены 4 аксиомы:
    \begin{enumerate}[nosep]
        \item $(\vec{x}, \vec{y}) = (\vec{y}, \vec{x})$.
        \item $(\vec{x} + \vec{y}, \vec{z}) = (\vec{x}, \vec{z}) + (\vec{y}, \vec{z})$ - аддитивность по первому аргументу.
        \item $(\alpha \vec{x}, \vec{y}) = \alpha(\vec{x}, \vec{y})$ - однородность по первому аргументу.
        \item $(\vec{x}, \vec{x}) \geq 0$, причем $(\vec{x}, \vec{x}) = 0 \iff \vec{x} = 0$.
    \end{enumerate}
\end{definition}

\begin{definition}
    Вещественное линейное пространство с так введенным скалярным произведением называется \textbf{\textit{евклидовым пространством}}.
\end{definition}

\subsection{
    Ортонормированный базис.
}

\begin{definition}
    Векторы $\vec{x}$ и $\vec{y}$ называются \textbf{\textit{ортогональными}}, если $(\vec{x}, \vec{y}) = 0$.
\end{definition}

\begin{definition}
    Система векторов называется \textit{\textbf{ортогональной}}, если все векторы в ней попарно ортогональны.
\end{definition}

\begin{definition}
    Система векторов называется \textit{\textbf{ортонормированной}}, если она ортогональна и норма каждого вектора равна 1.
\end{definition}



\newpage


% Процесс ортогонализации Грама-Шмидта, вывод формулы. Построение ортонормированного базиса
% Евклидово пространство
\subsection{
    Процесс ортогонализации Грама-Шмидта, вывод формулы. Построение ортонормированного базиса.
}

Построить ортонормированный базис можно, отталкиваясь от некоторого исходного базиса, при помощи алгоритма, который называют процессом ортогонализации Грама-Шмидта:

\bigbreak

Пусть $f = (\vec{f_1}, \ldots, \vec{f_n})$ - некоторый базис в евклидовом $n$-мерном пространстве $\mathcal{E}$. Модифицируя этот базис, мы будем строить новый базис $e = (\vec{e_1}, \ldots, \vec{e_n})$, который будет ортонормированным. Последовательно вычисляем векторы $\vec{g_1}$ и $\vec{e_1}$, $\vec{g_2}$ и $\vec{e_2}$ и т.д. по формулам:

\begin{align*}
    \vec{g_1} = \vec{f_1}, \quad \quad \quad \quad \quad \quad \quad \quad \quad \quad \quad \quad \quad \quad \quad
    \quad \quad \quad \quad \quad \thinspace \thinspace \thinspace \thinspace \thinspace \vec{e_1} = \frac{\vec{g_1}}{\norm{\vec{g_1}}};\\
    \vec{g_2} = \vec{f_2} - (\vec{f_2}, \vec{e_1}) \cdot \vec{e_1}, \quad \quad \quad \quad \quad \quad \quad \quad \quad \quad
    \quad \quad \quad \quad \thinspace \thinspace \thinspace \thinspace \thinspace \vec{e_2} = \frac{\vec{g_2}}{\norm{\vec{g_2}}};\\
    \vec{g_3} = \vec{f_3} - (\vec{f_3}, \vec{e_1}) \cdot \vec{e_1} - (\vec{f_3}, \vec{e_2}) \cdot \vec{e_2}, \quad \quad \quad \quad \quad \quad \quad \quad \thinspace \thinspace \thinspace \thinspace \thinspace \vec{e_3} = \frac{\vec{g_3}}{\norm{\vec{g_3}}};\\
    \ldots \ldots \ldots \ldots \ldots \ldots \ldots
    \ldots \ldots \ldots \ldots
    \quad \quad \quad \quad \quad \quad
    \quad \quad \quad \quad
    \ldots \ldots \ldots \\
    \vec{g_n} = \vec{f_n} - (\vec{f_n}, \vec{e_1}) \cdot \vec{e_1} - \ldots - (\vec{f_n}, \vec{e}_{n - 1}) \cdot \vec{e}_{n - 1}, \quad \quad \quad \thinspace \thinspace \thinspace \vec{e_n} = \frac{\vec{g_n}}{\norm{\vec{g_n}}}.
\end{align*}

\begin{proof}~

    Рассмотрим индукцию по количеству векторов $n$.
    \begin{enumerate}
        \item При $n = 1$ утверждение очевидно.
        \item Пусть это утверждение выполнено для количества векторов, равного $n$, докажем его для $n + 1$. 
        
        Т.к. утверждение верно для $n$ векторов, то мы можем считать, что векторы $\vec{g_1}, \ldots, \vec{g_n}$ с указанными свойствами уже построены. Построим вектор $\vec{g}_{n + 1}$ в виде 
        $$\vec{g}_{n + 1} = \vec{f}_{n + 1} + \lambda_1 \vec{g_1} + \ldots + \lambda_n\vec{g_n}.$$ 
        Линейная оболочка векторов $\vec{g_1}, \ldots,  \vec{g}_{n + 1}$ совпадает с $\vec{f_1}, \ldots, \vec{f}_{n + 1}$ при любых $\lambda_i$, поэтому мы будем подбирать коэффициенты $\lambda_i$ так, чтобы выполнялось условие $(\vec{g}_{n + 1},  \vec{g_i}) = 0$ для всех $i = 1, \ldots, n$. Рассмотрим скалярное произведение $$0 = (\vec{g}_{n + 1}, \vec{g_i}) = (\vec{f}_{n + 1}, \vec{g_i})+ \lambda_1(\vec{g_1}, \vec{g_i}) + \ldots + \lambda_n(\vec{g_n}, \vec{g_i}).$$
        Поскольку $(\vec{g_j}, \vec{g_i}) = 0$ при $j \ne i$ по предположению индукции, то 
        $$0 = (\vec{f}_{n + 1}, \vec{g_i}) + \lambda_i(\vec{g_i}, \vec{g_i}),$$ следовательно 
        $$\lambda_i = -\frac{(\vec{f}_{n + 1}, \vec{g_i})}{(\vec{g_i}, \vec{g_i})} \text{ (знаменатель отличен от нуля).}$$ 
        
        Таким образом, чтобы получить вектор $\vec{g}_{n + 1}$, надо из вектора $\vec{f}_{n + 1}$ вычесть его ортогональные проекции на векторы $\vec{g_1}, \ldots, \vec{g_n}$:
        $$\vec{g}_{n + 1} = \vec{f}_{n + 1} -\frac{(\vec{f}_{n + 1}, \vec{g_1})}{(\vec{g_1}, \vec{g_1})} \cdot \vec{g_1} - \ldots -\frac{(\vec{f}_{n + 1}, \vec{g_n})}{(\vec{g_n}, \vec{g_n})} \cdot \vec{g_n}.$$
    \end{enumerate}
\end{proof}



\newpage


% *Полезные факты, которые тоже будут на экзамене
\subsection{
    *Полезные факты, которые тоже будут на экзамене.
}

\begin{definition}
    Линейные подпространства $\mathcal{L}_1$ и $\mathcal{L}_2$ называются \textbf{\textit{ортогональными}}, если $\forall \vec{x} \in \mathcal{L}_1, \forall \vec{y} \in \mathcal{L}_2 \colon \vec{x} \perp \vec{y}$.
\end{definition}

\begin{theorem}
    Любая ортогональная система ненулевых векторов линейно независима.
\end{theorem}

\begin{proof}
    Рассмотрим произвольную ортогональную систему ненулевых векторов $\vec{e_1}, \ldots, \vec{e_m}$. Предположим, что для действительных коэффициентов $\alpha_1, \ldots, \alpha_m$ выполняется равенство
    \begin{equation}
        \alpha_1\vec{e_1} + \ldots + \alpha_m\vec{e_m} = \vec{0}.
        \label{eq:theorem_10_1_1}
    \end{equation}
    Умножим это равенство скалярно на какой-либо вектор $\vec{e_i}$:
    $$(\alpha_1\vec{e_1} + \ldots + \alpha_m\vec{e_m}, \vec{e_i}) = (\vec{0}, \vec{e_i}).$$
    $$\alpha_1(\vec{e_1}, \vec{e_i}) + \ldots + \alpha_i(\vec{e_i}, \vec{e_i}) + \ldots + \alpha_m(\vec{e_m}, \vec{e_i}) = 0.$$
    Так как система ортогональна, то все слагаемые слева, кроме одного, равны нулю, т.е.
    \begin{equation}
        \alpha_i(\vec{e_i}, \vec{e_i}) = 0.
        \label{eq:theorem_10_1_2}
    \end{equation}
    Так как вектор $\vec{e_i}$ ненулевой, то $(\vec{e_i}, \vec{e_i}) \ne 0$. Поэтому из $\eqref{eq:theorem_10_1_2}$ следует, что $\alpha_i = 0$. Индекс $i$ можно было выбирать произвольно, так что на самом деле все коэффициенты $a_i$ являются нулевыми. Значит, равенство $\eqref{eq:theorem_10_1_1}$ возможно лишь при нулевых коэффициентах. Значит, система векторов $\vec{e_1}, \ldots, \vec{e_m}$ линейно независима.
\end{proof}

\begin{corollary}
    Ортонормированная система векторов линейно независима. 
\end{corollary}

\begin{corollary}
    В $n$-мерном пространстве ортогональная/ортонормированная система из $n$ векторов является базисом.
\end{corollary}

\begin{definition}
    Если базис евклидова пространства представляет собой ортогональную систему векторов, то этот базис называют \textbf{\textit{ортогональным}}.
\end{definition}

\begin{definition}
    Ортогональный базис называется \textbf{\textit{ортонормированным}}, если каждый вектор этого базиса имеет норму (длину), равную единице.
\end{definition}

\begin{theorem}
    В конечномерном евклидовом пространстве существует ортонормированный базис.
\end{theorem}


\newpage
\section{
    Матрица Грама. Свойства матрицы.
}

% Матрица Грама. Свойства матрицы
Сначала сформулируем и докажем теоремы Фредгольма для операторов в линейных пространствах:

\subsection{
    Альтернатива Фредгольма.
}

\begin{theorem}["Альтернатива Фредгольма"] Пусть

    \begin{enumerate}
        \item $\left.\begin{array}{l}
            \mathcal{V} \text{ - линейное пространство, } \dim \mathcal{V} = n \\
            \mathcal{W} \text{ - линейное пространство, } \dim \mathcal{W} = m
        \end{array}\right\}$ \text{В обоих задано скалярное произведение.}
        \item $\mathscr{A} \colon \mathcal{V} \to \mathcal{W}$ - линейный оператор.
    \end{enumerate}

    Тогда справедливо ровно одно из двух:

    \begin{itemize}[nosep]
        \item либо уравнение $\mathscr{A}\vec{v} = \vec{w}$ имеет решение при любом $\vec{w} \in \mathcal{W}$,
        \item либо уравнение $\mathscr{A^*}\vec{w} = \vec{0}$ имеет нетривиальное (ненулевое) решение.
    \end{itemize}
\end{theorem}

\begin{proof}~

    Обозначим $r = \rank(\mathscr{A})$.

    \textbf{1 случай:}
    
    \begin{gather*}
        r = m \\
        \downimplies \\
        \dim(\Im \mathscr{A}) = m \\
        \downimplies \\
        \text{Т.к. } \Im \mathscr{A} \text{— } m\text{-мерное подпространство } m\text{-мерного пространства } \mathcal{W}, \text{то } \Im \mathscr{A} = \mathcal{W}. \\
        \downimplies \\
        \forall \vec{w} \in \mathcal{W} \exists \vec{v} \in \mathcal{V} \colon \mathscr{A}\vec{v} = \vec{w}. \\
        \text{Также заметим, что }\dim(\Im \mathscr{A^*}) + \dim(\ker \mathscr{A^*}) = m \\
        \downimplies \\
        \dim(\Im \mathscr{A}) + \dim(\ker \mathscr{A^*}) = m \\
        \downimplies \\
        m + \dim(\ker \mathscr{A^*}) = m \\
        \downimplies \\
        \dim(\ker \mathscr{A^*}) = 0 \\
        \downimplies \\
        \ker \mathscr{A^*} = \{\vec{0}\} \\
        \downimplies \\
        \text{Уравнение } \mathscr{A^*}\vec{w} = \vec{0} \text{ имеет только тривиальное решение.}
    \end{gather*}

    Получается, что второе "либо" не выполнено!
    
    \textbf{2 случай:}

    \begin{gather*}
        r < m \\
        \downimplies \\
        \dim(\Im \mathscr{A}) < \dim \mathcal{W} \\
        \downimplies \\
        \dim(\Im \mathscr{A^*}) < \dim \mathcal{W} \\
        \downimplies \\
        \text{Т.к. } \dim(\Im \mathscr{A^*}) + \dim(\ker \mathscr{A^*}) = \dim \mathcal{W}, \ker \mathscr{A^*} \ne \{\vec{0}\} \\
        \downimplies \\
        \exists \vec{w} \ne 0 \in \ker \mathscr{A^*} \\
        \downimplies \\
        \exists \vec{w} \ne 0 \colon \mathscr{A^*}\vec{w} = \vec{0}
    \end{gather*}
\end{proof}


\newpage
\section{
    Проекция вектора на подпространство вдоль другого подпространства. Расстояние от вектора до подпространства и угол между вектором и подпространством (для случая евклидовых пространств). 
}

% Проекция вектора на подпространство вдоль другого подпространства
Рассмотрим $\mathcal{L} = \mathcal{L}_1 \oplus \mathcal{L}_2$.

\subsection{
    Проекция вектора на подпространство вдоль другого подпространства.
}

\begin{definition}
    \textit{\textbf{Проекцией вектора $\vec{x}$ на подпространство $\mathcal{L}_1$ вдоль подпространства $\mathcal{L}_2$}} называется вектор $x_1$ из представления $\vec{x} = \vec{x_1} + \vec{x_2}$, $\vec{x_1} \in \mathcal{L}_1, \vec{x_2} \in \mathcal{L}_2$.
\end{definition}
% Расстояние от вектора до подпространства и угол между вектором и подпространством (для случая евклидовых пространств)
\subsection{
    Расстояние от вектора до подпространства и угол между вектором и подпространством (для случая евклидовых пространств). 
}

\begin{definition}
    \textit{\textbf{Расстоянием от вектора $\vec{a}$ до подпространства $\mathcal{L}_1$}} называется норма вектора, опущенного из конца вектора $\vec{a}$ на линейное подпространство $\mathcal{L}$ и ортогонального ему, т.е. норма ортогональной составляющей вектора $\vec{a}$ относительно подпространства $\mathcal{L}$.
\end{definition}

\begin{definition}
    \textit{\textbf{Углом между вектором $\vec{a}$ и подпространством $\mathcal{L}_1$}} называется угол между вектором $\vec{a}$ и его ортогональной проекцией на подпространство $\mathcal{L}_1$. (имеется в виду векторной ортогональной проекцией, не скалярной, как в АГ).

    $\varphi = \widehat{\vec{a}, \mathcal{L}} = \widehat{\vec{a}, \vec{b}} = \arccos(\cos(\widehat{\vec{a}, \vec{b}})) = \arccos \frac{(\vec{a}, \vec{b})}{\norm{\vec{a}} \cdot \norm{\vec{b}}}$, где $\vec{b}$ - ортогональная проекция вектора $\vec{a}$ на подпространство $\mathcal{L}_1$.
\end{definition}


\newpage
\section{
    Прямое дополнение. Ортогональное дополнение.
}

% Прямое дополнение
\subsection{
    Прямое дополнение.
}

\begin{definition}
    Если линейные подпространства $\mathcal{H}_1$ и $\mathcal{H}_2$ в линейном пространстве $\mathcal{L}$ образуют прямую сумму, причем $\mathcal{H}_1 \oplus \mathcal{H}_2 = \mathcal{L}$, то говорят, что $\mathcal{H}_1$ является \textit{\textbf{прямым дополнением}} для $\mathcal{H}_2$.
\end{definition}

\begin{theorem}
    Любое линейное подпространство $\mathcal{H}$ в линейном пространстве $\mathcal{L}$ имеет прямое дополнение.
\end{theorem}

\begin{proof}~

    Если линейное подпространство $\mathcal{H}$  совпадает со всем линейным пространством $\mathcal{L}$, то в качестве его прямого дополнения следует взять другое несобственное подпространство: $\mathcal{H}_1 = \{\vec{0}\}$. Точно так же прямым дополнением к нулевому подпространству $\{\vec{0}\}$ является само линейное пространство $\mathcal{L}$. Опуская эти два тривиальных случая, полагаем, что линейное подпространство $\mathcal{H}_1$ является собственным.

    Выберем в $\mathcal{H}$ какой-либо базис $e = (\vec{e_1}, \ldots, \vec{e_k})$ и дополним его системой векторов $f = (\vec{f_1}, \ldots, \vec{f_m})$ до базиса в $\mathcal{L}$. Положим, $\mathcal{H}_1 = \Span(f)$. Тогда $\mathcal{H} + \mathcal{H}_1 = \mathcal{L}$, так как сумма $\mathcal{H} + \mathcal{H}_1$ содержит все векторы системы $(e, f)$, являющейся базисом в $\mathcal{L}$, а значит, и любой другой вектор линейного пространства. Остается доказать, что сумма $\mathcal{H} + \mathcal{H}_1$ является прямой.

    Выберем произвольный вектор $\vec{y} \in \mathcal{H} \cap \mathcal{H}_1$. Тогда, с одной стороны, $\vec{y} = \alpha_1\vec{e_1} + \ldots + \alpha_k\vec{e_k}$, так как $\vec{y}$ принадлежит линейному подпространству $\mathcal{H}$, а с другой стороны, $\vec{y} = \beta_1\vec{f_1} + \ldots + \beta_m\vec{f_m}$, так как $\vec{y}$ принадлежит линейному подпространству $\mathcal{H}_1$. Эти две линейные комбинации есть два разложения вектора в базисе $(e, f)$ линейного пространства $\mathcal{L}$ и, следовательно, должны совпадать:
    $$\alpha_1\vec{e_1} + \ldots + \alpha_k\vec{e_k} = \beta_1\vec{f_1} + \ldots + \beta_m\vec{f_m},$$
    или
    $$\alpha_1\vec{e_1} + \ldots + \alpha_k\vec{e_k} - \beta_1\vec{f_1} - \ldots - \beta_m\vec{f_m} = \vec{0}.$$
    Система векторов $(e, f)$ линейно независима, так как является базисом. Поэтому из последнего равенства векторов следует, что в нем все коэффициенты нулевые. Значит, вектор $\vec{y}$ является нулевым, а так как он выбирался произвольно, то $\mathcal{H} \cap \mathcal{H}_1 = \{\vec{0}\}$. Поэтому линейные подпространства $\mathcal{H}$ и $\mathcal{H}_1$ образуют прямую сумму.
\end{proof}



\newpage


% Ортогональное дополнение
\subsection{
    Связь с решениями неоднородной СЛАУ.
}

Рассмотрим неоднородную СЛАУ

$$A\vec{x} = \vec{b},$$

где $A \in \RR^{m \times n}, \vec{x} = \begin{pmatrix} x_1 \\ \vdots \\ x_n \end{pmatrix} \in \RR^n, \vec{b} = \begin{pmatrix} b_1 \\ \vdots \\ b_m \end{pmatrix} \in \RR^m$.

Из курса Ан. Геом. известно, что если СЛАУ имеет бесконечно много решений, то они задаются так:

$$\vec{x} = \vec{x}_0 + c_1\vec{f}_1 + \ldots + c_k\vec{f}_k,$$

где $\vec{x}_0 \in \RR^n$ - вектор с постоянными коэффициентами, $\vec{f}_1, \ldots, \vec{f}_k \in \RR^n$ - векторы, образующие ФСР, $c_1, \ldots, c_k$ - произвольные константы.

Если рассмотреть $\mathcal{W} = \Span \{\vec{f}_1, \ldots, \vec{f}_k\}$, то получится, что множество $\mathcal{U}$ всех решений СЛАУ будет представлять из себя ЛАМ:

$$\mathcal{U} = \vec{x}_0 + \mathcal{W}.$$

Представляете!


\newpage
\section{
    Метод наименьших квадратов, обоснование, подробное описание, пример решения.
}

% Метод наименьших квадратов, обоснование, подробное описание
Сначала сформулируем и докажем теоремы Фредгольма для операторов в линейных пространствах:

\subsection{
    Альтернатива Фредгольма.
}

\begin{theorem}["Альтернатива Фредгольма"] Пусть

    \begin{enumerate}
        \item $\left.\begin{array}{l}
            \mathcal{V} \text{ - линейное пространство, } \dim \mathcal{V} = n \\
            \mathcal{W} \text{ - линейное пространство, } \dim \mathcal{W} = m
        \end{array}\right\}$ \text{В обоих задано скалярное произведение.}
        \item $\mathscr{A} \colon \mathcal{V} \to \mathcal{W}$ - линейный оператор.
    \end{enumerate}

    Тогда справедливо ровно одно из двух:

    \begin{itemize}[nosep]
        \item либо уравнение $\mathscr{A}\vec{v} = \vec{w}$ имеет решение при любом $\vec{w} \in \mathcal{W}$,
        \item либо уравнение $\mathscr{A^*}\vec{w} = \vec{0}$ имеет нетривиальное (ненулевое) решение.
    \end{itemize}
\end{theorem}

\begin{proof}~

    Обозначим $r = \rank(\mathscr{A})$.

    \textbf{1 случай:}
    
    \begin{gather*}
        r = m \\
        \downimplies \\
        \dim(\Im \mathscr{A}) = m \\
        \downimplies \\
        \text{Т.к. } \Im \mathscr{A} \text{— } m\text{-мерное подпространство } m\text{-мерного пространства } \mathcal{W}, \text{то } \Im \mathscr{A} = \mathcal{W}. \\
        \downimplies \\
        \forall \vec{w} \in \mathcal{W} \exists \vec{v} \in \mathcal{V} \colon \mathscr{A}\vec{v} = \vec{w}. \\
        \text{Также заметим, что }\dim(\Im \mathscr{A^*}) + \dim(\ker \mathscr{A^*}) = m \\
        \downimplies \\
        \dim(\Im \mathscr{A}) + \dim(\ker \mathscr{A^*}) = m \\
        \downimplies \\
        m + \dim(\ker \mathscr{A^*}) = m \\
        \downimplies \\
        \dim(\ker \mathscr{A^*}) = 0 \\
        \downimplies \\
        \ker \mathscr{A^*} = \{\vec{0}\} \\
        \downimplies \\
        \text{Уравнение } \mathscr{A^*}\vec{w} = \vec{0} \text{ имеет только тривиальное решение.}
    \end{gather*}

    Получается, что второе "либо" не выполнено!
    
    \textbf{2 случай:}

    \begin{gather*}
        r < m \\
        \downimplies \\
        \dim(\Im \mathscr{A}) < \dim \mathcal{W} \\
        \downimplies \\
        \dim(\Im \mathscr{A^*}) < \dim \mathcal{W} \\
        \downimplies \\
        \text{Т.к. } \dim(\Im \mathscr{A^*}) + \dim(\ker \mathscr{A^*}) = \dim \mathcal{W}, \ker \mathscr{A^*} \ne \{\vec{0}\} \\
        \downimplies \\
        \exists \vec{w} \ne 0 \in \ker \mathscr{A^*} \\
        \downimplies \\
        \exists \vec{w} \ne 0 \colon \mathscr{A^*}\vec{w} = \vec{0}
    \end{gather*}
\end{proof}

% Пример решения
\subsection{
    Пример решения.
}

$\vec{a}_1 = \begin{pmatrix} 1 \\ -2 \\ 2 \end{pmatrix}$, $\vec{a}_2 = \begin{pmatrix} 3 \\ 1 \\ -3 \end{pmatrix}$, $\vec{b} = \begin{pmatrix} 16 \\ 6 \\ 4 \end{pmatrix}$

$A = \begin{pmatrix} \vec{a}_1 & \vec{a}_2 \end{pmatrix} = \begin{pmatrix} 1 & 3 \\ -2 & 1 \\ 2 & -3 \end{pmatrix}.$

Найдем элементы матрицы Грама $\Gamma(\vec{a}_1, \vec{a}_2) = A^TA$:

\begin{gather*}
    (\vec{a}_1, \vec{a}_1) = 1 + 4 + 4 = 9 \\
    (\vec{a}_1, \vec{a}_2) = 3 - 2 - 6 = -5 \\
    (\vec{a}_2, \vec{a}_2) = 9 + 1 + 9 = 19
\end{gather*}


Найдем элементы матрицы $A\vec{b}$:

\begin{gather*}
    (\vec{a}_1, \vec{b}) = 16 - 12 + 8 = 12 \\
    (\vec{a}_2, \vec{b}) = 48 + 6 - 12 = 42
\end{gather*}

Найдем решение СЛАУ $A^TA\vec{x} = A\vec{b}$:

\begin{equation*}
    \left(\begin{array}{cc|c}
        9 & -5 & 12 \\
        -5 & 19 & 42
    \end{array}\right)
    \sim
    \left(\begin{array}{cc|c}
        -1 & 33 & 96 \\
        0 & -146 & -438
    \end{array}\right)
    \sim
    \left(\begin{array}{cc|c}
        1 & -33 & -96 \\
        0 & 1 & 3
    \end{array}\right)
    \sim
    \left(\begin{array}{cc|c}
        1 & 0 & 3 \\
        0 & 1 & 3
    \end{array}\right)
.\end{equation*}

$\vec{x} = 3\vec{a}_1 + 3\vec{a}_2 = \begin{pmatrix} 12 \\ -3 \\ -3 \end{pmatrix}$ – наилучшее приближение вектора $\vec{b}$.

Абсолютная ошибка $= \norm{\vec{b} - \vec{x}} = \norm{\begin{pmatrix} 16 \\ 6 \\ 4 \end{pmatrix} - \begin{pmatrix} 12 \\ -3 \\ -3 \end{pmatrix}} =  \norm{\begin{pmatrix} 4 \\ 9 \\ 7 \end{pmatrix}} =  \sqrt{146}$.

Относительная ошибка $= \frac{\norm{\vec{b} - \vec{x}}}{\norm{\vec{b}}} = \frac{\sqrt{146}}{\sqrt{308}} = \frac{\sqrt{11242}}{154}$.


\newpage
\section{
    Псевдообратная матрица. Алгоритм нахождения.
}

% Алгоритм нахождения, опираясь на нормальное псевдорешение
\begin{definition}
    Для любой матрицы $A$ существует такая матрица $A^+$, что нормальное псевдорешение СЛАУ $A\vec{x} = \vec{b}$ с произвольным вектор-столбцом $\vec{b}$ имеет вид $\vec{x} = A^+\vec{b}$. Матрицу $A^+$ называют \textit{\textbf{псевдообратной}} по отношению к матрице $A$.
\end{definition}

\subsection{
    Алгоритм нахождения, опираясь на нормальное псевдорешение.
}

Рассмотрим СЛАУ $A\vec{x} = \vec{e}_i$, в правой части которой записан $i$-й вектор стандартного базиса в пространстве $\RR^n$. Ее нормальное псевдорешение $A^+\vec{e}_i$ - это $i$-й столбец псевдообратной матрицы. Вычислив все $n$ столбцов, получим $A^+$. Таким образом, $i$-й столбец псевдообратной матрицы $a_i$ можно найти из системы

\begin{equation*}
    \left(\begin{array}{c}
        A^TA \\
        F^T
    \end{array}\right)a_i
    =
    \left(\begin{array}{c}
        A^T\vec{e}_i \\
        \vec{0}
    \end{array}\right)
,\end{equation*}

где $F$ - матрица, составленная из столбцов ФСР СЛАУ $A\vec{x} = \vec{0}$. Объединив по правилам действий с блочными матрицами $n$ систем в одно матричное уравнение, получим

\begin{equation*}
    \left(\begin{array}{c}
        A^TA \\
        F^T
    \end{array}\right)A^+
    =
    \left(\begin{array}{c}
        A^T \\
        \vec{0}
    \end{array}\right)
.\end{equation*}

\textbf{Итак, алгоритм состоит в следующем:}

\begin{enumerate}
    \item Найти $F$ - матрицу, составленную из столбцов ФСР СЛАУ $A\vec{x} = \vec{0}$.
    \item Транспонируем матрицу $F$, чтобы в дальнейшем добавить $F^T$ как строки новой матрицы.
    \item Найдем $A^T$.
    \item Найдем $A^TA$.
    \item Составим матричное уравнение ниже и решим его методом элементарных преобразований.

    \begin{equation*}
        \left(\begin{array}{c}
            A^TA \\
            F^T
        \end{array}\right)A^+
        =
        \left(\begin{array}{c}
            A^T \\
            \vec{0}
        \end{array}\right)
    .\end{equation*}
    \begin{equation*}
        \left(\begin{array}{c|c}
            A^TA & A^T \\
            F^T & \vec{0}
        \end{array}\right)
        \sim \ldots \sim
        \left(\begin{array}{c|c}
            E & A^+
        \end{array}\right)
    .\end{equation*}
\end{enumerate}



\newpage


% *Алгоритм, основанный на понятии скелетного разложения
\subsection{
    Связь с решениями неоднородной СЛАУ.
}

Рассмотрим неоднородную СЛАУ

$$A\vec{x} = \vec{b},$$

где $A \in \RR^{m \times n}, \vec{x} = \begin{pmatrix} x_1 \\ \vdots \\ x_n \end{pmatrix} \in \RR^n, \vec{b} = \begin{pmatrix} b_1 \\ \vdots \\ b_m \end{pmatrix} \in \RR^m$.

Из курса Ан. Геом. известно, что если СЛАУ имеет бесконечно много решений, то они задаются так:

$$\vec{x} = \vec{x}_0 + c_1\vec{f}_1 + \ldots + c_k\vec{f}_k,$$

где $\vec{x}_0 \in \RR^n$ - вектор с постоянными коэффициентами, $\vec{f}_1, \ldots, \vec{f}_k \in \RR^n$ - векторы, образующие ФСР, $c_1, \ldots, c_k$ - произвольные константы.

Если рассмотреть $\mathcal{W} = \Span \{\vec{f}_1, \ldots, \vec{f}_k\}$, то получится, что множество $\mathcal{U}$ всех решений СЛАУ будет представлять из себя ЛАМ:

$$\mathcal{U} = \vec{x}_0 + \mathcal{W}.$$

Представляете!


\newpage
\section{
    Псевдорешение, алгоритм нахождения. Нормальное псевдорешение, алгоритм нахождения. Определение, смысл. Рассмотреть случаи для совместных и несовместных систем.
}

% Псевдорешение, алгоритм нахождения
\subsection{
    Псевдорешение, алгоритм нахождения.
}

Рассмотрим СЛАУ 

$$A\vec{x} = \vec{b},$$

где $A$ - матрица размера $m \times n, \vec{x} \in \RR^n, \vec{b} \in \RR^m$. Каждому столбцу $\vec{x}$ можно сопоставить столбец $\vec{b} - A\vec{x}$, называемый \textit{вектором невязок}. Евклидову норму этого столбца $\norm{\vec{b} - A\vec{x}} = \sqrt{(\vec{b} - A\vec{x}, \vec{b} - A\vec{x})}$ называют \textit{нормой невязки}.

Для любой СЛАУ $A\vec{x} = \vec{b}$ найдется хотя бы один столбец $\vec{x}$, на котором норма невязки $v(x) = \norm{\vec{b} - A\vec{x}}$ принимает наименьшее значение $v_0$. Каждый такой столбец называется \textit{псевдорешением} СЛАУ $A\vec{x} = \vec{b}$. Итак,

\begin{definition}
    Вектор-столбец $\vec{x}$, такой, что норма невязки $\norm{\vec{b} - A\vec{x}}$ СЛАУ $A\vec{x} = \vec{b}$ принимает наименьшее значение, называется \textit{\textbf{псевдорешением}}.
\end{definition}

Если система совместна, то минимальное значение нормы невязки равно нулю и множество псевдорешений совпадает с множеством решений системы. Множество псевдорешений СЛАУ совпадает с множеством решений соответствующей \textit{нормальной системы} $A^TA\vec{x} = A^T\vec{b}$ (\textbf{*}см. 14 "билет"). 



\newpage


% Нормальное псевдорешение, алгоритм нахождения
\subsection{
    Нормальное псевдорешение, алгоритм нахождения.
}


Среди всех псевдорешений СЛАУ $A\vec{x} = \vec{b}$ есть единственное псевдорешение, имеющее наименьшую норму. Оно называется \textit{нормальным}. Итак,

\begin{definition}
    Наименьшее по норме псевдорешение СЛАУ $A\vec{x} = \vec{b}$ называется \textbf{\textit{нормальным псевдорешением}}.
\end{definition}

Нормальное псевдорешение СЛАУ $A\vec{x} = \vec{b}$ является решением системы уравнений, которая получается, если к нормальной системе $A^TA\vec{x} = A^T\vec{b}$ добавить уравнения СЛАУ $F^T\vec{x} = \vec{0}$, где $F$ - матрица, составленная из столбцов ФСР СЛАУ $A\vec{x} = \vec{0}$.

\begin{equation*}
    \left(\begin{array}{c}
        A^TA \\
        F^T
    \end{array}\right)\vec{x}
    =
    \left(\begin{array}{c}
        A^T\vec{b} \\
        \vec{0}
    \end{array}\right)
.\end{equation*}



\newpage


% Пример для несовместной системы
\subsection{
    Пример для несовместной системы.
}

\begin{example}
    Рассмотрим простейшую систему

    \begin{equation}
        \begin{cases}
            x + y = 0 \\
            x + y = 1
        \end{cases}
    \end{equation}

    двух уравнений с двумя неизвестными. Видно, что эта система несовместна. Последовательно вычисляем

    \begin{equation}
        A = \left(\begin{array}{cc}
            1 & 1 \\
            1 & 1
        \end{array}\right), \thinspace \thinspace
        A^TA = \left(\begin{array}{cc}
            1 & 1 \\
            1 & 1
        \end{array}\right)^2 = \left(\begin{array}{cc}
            2 & 2 \\
            2 & 2
        \end{array}\right), \thinspace \thinspace
        A^Tb = \left(\begin{array}{cc}
            1 & 1 \\
            1 & 1
        \end{array}\right)\left(\begin{array}{c}
            0 \\
            1
        \end{array}\right) = \left(\begin{array}{c}
            1 \\
            1
        \end{array}\right)
    \end{equation}

    Таким образом, нормальная СЛАУ в этом случае состоит из двух одинаковых уравнений:

    \begin{equation}
        \begin{cases}
            2x + 2y = 1 \\
            2x + 2y = 1
        \end{cases}
    \end{equation}

    Множество решений нормальной системы, т.е. множество пар $x, y$, дающих минимальную невязку в исходной системе, на плоскости изображаются прямой $x + y = 0.5$ (рис. \ref{fig:picture_16_1}), а нормальным псевдорешением будет точка этой прямой, ближайшая к началу координат, т.е. точка с координатами $x = 0.25, y = 0.25$. Этой точке соответствует радиус-вектор с наименьшей нормой среди всех радиус-векторов точек прямой $x + y = 0.5$.
    
    К слову, если одно из уравнений исходной системы умножить на коэффициент, то и множество решений нормальной системы, и нормальное псевдорешение данной системы изменятся, так как умножение на коэффициент, вообще говоря, меняет его невязку.

    \begin{figure}[H]
        \centering
        \includegraphics[scale=0.6]{images/module1/question16/1.jpg}
        \caption{}
        \label{fig:picture_16_1}
    \end{figure}

\end{example}


\newpage
\section{
    Матрица перехода от базиса к базису, вывод формулы для преобразования координат вектора при переходе к новому базису. Обратный переход. Работа с тремя и более базисами.
}

% Матрица перехода от базиса к базису
\subsection{
    Матрица перехода от базиса к базису.
}

\begin{definition}
    \textit{\textbf{Матрицей перехода}} от старого базиса к новому называется матрица, элементами \textit{\textbf{столбцов}} которой являются координаты векторов нового базиса, разложенных по старому базису.
    \label{fig:definition_17_1}
\end{definition}

% Вывод формулы для преобразования координат вектора при переходе к новому базису
\subsection{
    Вывод формулы для преобразования координат вектора при переходе к новому базису.
}

Пусть в $n$-мерном линейном пространстве $\mathcal{L}$ заданы два базиса: старый $b = (\vec{b_1}, \ldots, \vec{b_n})$ и новый $c = (\vec{c_1}, \ldots, \vec{c_n})$.

Разложим векторы базиса $c$ по базису $b$:

$$\vec{c_i} = \alpha_{1i}\vec{b_1} + \ldots + \alpha_{ni}\vec{b_n}, \quad i = \overline{1, n}.$$

Запишем эти представления в матричной форме:

$$\vec{c_i} = b \begin{pmatrix} \alpha_{1i} \\ \vdots \\ \alpha_{ni} \end{pmatrix}, \quad  i = \overline{1, n},$$

или

$$c = bT_{b \to c},$$

где

\begin{equation*}
    T_{b \to c} = \left(\begin{array}{ccc}
        \alpha_{11} & \ldots & \alpha_{1n} \\
        \hdotsfor{3} \\
        \alpha_{n1} & \ldots & \alpha_{nn}
    \end{array}\right).
\end{equation*}

% *Полезное дополнение про преобразование координат вектора при переходе от старого базиса к новому
\subsection{
    *Полезное дополнение про преобразование координат вектора при переходе от старого базиса к новому.
    \label{subsection:subsection_17_3}
}

Выберем произвольный вектор $\vec{x} \in \mathcal{L}$ и разложим его в старом базисе $b$:

$$\vec{x} = bx_b, \quad \quad x_b = \begin{pmatrix} x_1 \\ \vdots \\ x_n \end{pmatrix}.$$

Разложение того же вектора в новом базисе $c$ имеет вид:

$$\vec{x} = cx_c, \quad \quad x_c = \begin{pmatrix} x_1' \\ \vdots \\ x_n' \end{pmatrix}.$$

Найдем связь между старыми координатами $x_b$ вектора $\vec{x}$ и его новыми координатами $x_c$. Из соотношений выше следует, что $bx_b = cx_c$. Учитывая, что $c = bT_{b \to c}$, получаем 
$$bx_b = (bT_{b \to c})x_c,$$ или 
$$bx_b = b(T_{b \to c}x_c).$$ 
Последнее равенство можно рассматривать как запись двух разложений одного и того же вектора $\vec{x}$ в базисе $b$. Разложениями соответствуют столбцы координат $x_b$ и $T_{b \to c}x_c$, которые, согласно теореме \ref{thm:theorem_2_3} о единственности разложения вектора по базису, должны быть равны:
$$x_b = T_{b \to c}x_c, \quad \quad \text{или} \quad \quad x_c = T^{-1}_{b \to c}x_b.$$



\newpage


% Обратный переход. Работа с тремя и более базисами
\subsection{
    Обратный переход. Работа с тремя и более базисами.
}

\textbf{Свойства.}

\begin{enumerate}[label={\arabic*°.}]
    \item Матрица перехода невырождена и всегда имеет обратную.
    \begin{proof}~

        Столбцы матрицы перехода - столбцы координат векторов нового \textbf{базиса} в старом. Следовательно, они, как и векторы базиса, линейно независимы. Значит, матрица $T$ невырожденная и имеет обратную матрицу $T^{-1}$.
    \end{proof}
    
    \item Если в $n$-мерном линейном пространстве задан базис $b$, то для любой невырожденной квадратной матрицы $T$ порядка $n$ существует такой базис $c$ в этом линейном пространстве, что $T$ будет матрицей перехода то базиса $b$ к базису $c$.
    \begin{proof}~

        Из невырожденности матрицы $T$ следует, что ее ранг равен $n$, и поэтому ее столбцы, будучи базисными, линейно независимы. Эти столбцы являются столбцами координат векторов системы $c = bT_{b \to c}$. Линейная независимость столбцов матрицы $T$ равносильна линейной независимости системы векторов $c$. Так как система $c$ содержит $n$ векторов, причем линейное пространство $n$-мерно, то согласно теореме $\eqref{thm:theorem_2_1}$, эта система является базисом.
    \end{proof}
    
    \item Если $T_{b \to c}$ - матрица перехода от старого базиса $b$ к новому базису $c$ линейного пространства, то $T^{-1}_{b \to c}$ - матрица перехода от базиса $c$ к базису $b$.
    \begin{proof}~

        Матрица $T_{b \to c}$ невырождена, и поэтому из равенства $c = bT_{b \to c}$ следует, что $cT^{-1}_{b \to c} = b$. Последнее равенство означает, что столбцы матрицы $T^{-1}_{b \to c}$ являются столбцами координат векторов $b$ относительно базиса $c$, т.е. согласно определению $\eqref{fig:definition_17_1}$ $T^{-1}_{b \to c}$ - это матрица перехода от базиса $c$ к базису $b$.
    \end{proof}

    \item Если в линейном пространстве заданы базисы $b, c$ и $d$, причем $T_{b \to c}$ - матрица перехода от базиса $b$ к новому базису $c$, а $T_{c \to d}$ - матрица перехода от базиса $c$ к базису $d$, то произведение этих матриц $T_{b \to c}T_{c \to d}$ - матрица перехода от базиса $b$ к базису $d$.
    \begin{proof}~

        Согласно определению $\eqref{fig:definition_17_1}$ матрицы перехода, имеем равенства
        $$c = bT_{b \to c}, \quad d = cT_{c \to d},$$
        откуда
        $$d = cT_{c \to d} = (b T_{b \to c}) \cdot T_{c \to d} = b(T_{b \to c} \cdot T_{c \to d}),$$
        т.е. $T_{b \to c} \cdot T_{c \to d} = T_{b \to d}$ - матрица перехода от базиса $b$ к базису $d$.
    \end{proof}

    \item Пусть $b_1, b_2, \ldots, b_n$ - это $n$ базисов линейного пространства $\mathcal{V}$ ($n \geq 4$). $T_k$ - матрица перехода от $b_k$ к $b_{k + 1}$, $k = \overline{1, n - 1}$. Тогда матрица перехода от $b_1$ к $b_n$ равна $T_1\cdot T_2 \cdot \ldots \cdot T_{n - 1}$.
    \begin{proof}
        Последовательное применение свойства 4°.
    \end{proof}
\end{enumerate}


\newpage
\section{
    Евклидовы и унитарные пространства. Три примера. Неравенство Коши-Буняковского (Шварца).
}

% Евклидовы и унитарные пространства. Три примера
\subsection{
    Евклидовы и унитарные пространства. Три примера.
}

\textbf{Примеры $\mathcal{E}$:}
\begin{enumerate}
    \item В линейных пространствах $\mathcal{V}_2$ и $\mathcal{V}_3 \colon (\vec{x}, \vec{y}) = |\vec{x}||\vec{y}|\cos \widehat{(\vec{x}, \vec{y})}$.
    \item В арифметическом линейном пространстве $\RR^n \colon (\vec{x}, \vec{y}) = x_1y_1 + \ldots + x_ny_n$. 
    \item Линейное пространство $C[0, 1]$ всех функций, непрерывных на отрезке $[0, 1]$ становится евклидовым, если в нем ввести скалярное произведение:
    $$(\vec{f}, \vec{g}) = \int_{0}^{1} f(x)g(x) \dd x.$$
\end{enumerate}

\textbf{Примеры $\mathcal{U}$:}

\begin{enumerate}
    \item $\CC^n \colon (\vec{x}, \vec{y}) = x_1\overline{y}_1 + x_2\overline{y}_2 + \ldots + x_n\overline{y}_n$.
\end{enumerate}



\newpage


% Неравенство Коши-Буняковского (Шварца)
\subsection{
    Неравенство Коши-Буняковского (Шварца).
}

\begin{theorem}
    Для любых векторов $\vec{x}, \vec{y} \in \mathcal{E}$ (или $\mathcal{U}$) справедливо неравенство Коши-Буняковского
    $$|(\vec{x}, \vec{y})|^2 \leq (\vec{x}, \vec{x}) (\vec{y}, \vec{y}),$$
    причем $|(\vec{x}, \vec{y})|^2 = (\vec{x}, \vec{x}) (\vec{y}, \vec{y}) \iff \vec{x} \parallel \vec{y}.$
\end{theorem}

\begin{corollary}~

    В случае линейного арифметического пр-ва $\RR^n$ неравенство Коши-Буняковского трансформируется в \textbf{неравенство Коши}:
    $$(a_1b_1 + \ldots + a_nb_n)^2 \leq (a_1^2 + \ldots + a_n^2)(b_1^2 + \ldots + b_n^2).$$
    Равенство достигается при линейной зависимости векторов, т.е. $\frac{a_i}{b_i} = const, \quad i = \overline{1, n}$.
\end{corollary}

\begin{corollary}~

    В евклидовом пространстве $C[0, 1]$, скалярное произведение в котором выражается определенным интегралом, неравенство Коши-Буняковского превращается в неравенство Буняковского-Шварца:
    $$\left( \int_0^1 f(x)g(x) \, dx \right)^2 \le \left( \int_0^1 f(x)^2 \, dx \right) \left( \int_0^1 g(x)^2 \, dx \right).$$
\end{corollary}



\newpage


% Неравенство Коши-Буняковского (доказательство для $\mathcal{U}$)
\subsection{
    Неравенство Коши-Буняковского (доказательство для $\mathcal{U}$).
}


\begin{proof}~
    
    При $\vec{x} = \vec{0}$ обе части неравенства равны нулю, значит, неравенство выполняется. Отбрасывая этот очевидный случай, будем считать, что $\vec{x}, \vec{y} \ne \vec{0}$, $(\vec{x}, \vec{y}) \ne 0$, $t \in \CC$.
    
    $$(\vec{x} + t\vec{y}, \vec{x} + t\vec{y}) = (\vec{x}, \vec{x}) + \underbrace{t(\vec{y}, \vec{x}) + \overline{t}(\vec{x}, \vec{y})}_{\star} + t \cdot \overline{t}(\vec{y}, \vec{y}) = (\vec{x}, \vec{x}) + \underbrace{2 \text{Re}(t(\vec{y}, \vec{x}))}_{\star} + |t|^2(\vec{y}, \vec{y}) = \underbrace{(\vec{y}, \vec{y})|t|^2}_{\in \RR} + \underbrace{2|(\vec{x}, \vec{y})||t|}_{\in \RR} + \underbrace{(\vec{x}, \vec{x})}_{\in \RR}.$$

    Выберем $t$ так, чтобы $\arg{t} = -\arg{(y, x)}$, тогда $\arg{(y, x)} = -\arg{t}$.

    \begin{gather*}
        t = |t|e^{i\varphi} \\
        \text{Так как }\arg{(y, x)} = -\arg{t}, \text{ то } \\
        (y, x) = |(x, y)|e^{i(-\varphi)}.
    \end{gather*}

    Итак, 
    
    $$2 \cdot \text{Re} (t \cdot (\vec{y}, \vec{x})) = 2 \cdot \text{Re} (|t|e^{i\varphi} \cdot |(\vec{x}, \vec{y})|e^{-i\varphi}) = 2|t||(\vec{y}, \vec{x})|.$$

    Тогда

    \begin{gather*}
        \underbrace{(\vec{y}, \vec{y})|t|^2}_{\in \RR} + \underbrace{2|(\vec{x}, \vec{y})||t|}_{\in \RR} + \underbrace{(\vec{x}, \vec{x})}_{\in \RR} \geq 0 \\
        D = |(\vec{x}, \vec{y})|^2 - (\vec{x}, \vec{x})(\vec{y}, \vec{y}) \leq 0.
    \end{gather*}

    \bigbreak
    
    $\text{\textbf{Примечание} } (\star) \colon$ Пусть $t = p + iq, (\vec{y}, \vec{x}) = a + ib$, $\overline{t} = p - iq, \overline{(\vec{y}, \vec{x})} = a - ib$. Тогда

    \begin{enumerate}
        \item \begin{align*}
            t(\vec{y}, \vec{x}) &+ \overline{t} \cdot (\vec{x}, \vec{y}) = t(\vec{y}, \vec{x}) + \overline{t} \cdot \overline{(\vec{y}, \vec{x})} = \\
            &= (p + iq)(a + ib) + (p - iq)(a - ib) = \\ 
            &= pa + pib + aiq - qb + pa - pib - aiq - qb = 2pa - 2qb = \\
            &= 2 \cdot (pa - qb).
        \end{align*}
        \item \begin{align*}
            2 \cdot \text{Re} (t \cdot (\vec{y}, \vec{x})) &= 2 \cdot \text{Re}((p + iq)(a + ib)) = \\
            &= 2 \cdot \text{Re}(pa - qb + i(pb + aq)) = \\
            &= 2 \cdot (pa - qb).
        \end{align*}
        \item \begin{align*}
            t(\vec{y}, \vec{x}) + \overline{t} \cdot (\vec{x}, \vec{y}) = 2 \cdot \text{Re} (t \cdot (\vec{y}, \vec{x})).
        \end{align*}
    \end{enumerate}
\end{proof}



\newpage


% Неравенство Коши-Буняковского (доказательство для $\mathcal{E}$)
\subsection{
    Неравенство Коши-Буняковского (доказательство для $\mathcal{E}$).
}


\begin{theorem}
    Для любых векторов $\vec{x}, \vec{y}$ евклидова пространства справедливо неравенство Коши-Буняковского

    \begin{equation}
        (\vec{x}, \vec{y})^2 \leq (\vec{x}, \vec{x}) (\vec{y}, \vec{y}).
        \label{equition:equition_18_1}
    \end{equation}
\end{theorem}

\begin{proof}~

    При $\vec{x} = \vec{0}$ обе части неравенства \eqref{equition:equition_18_1} равны нулю согласно свойствам скалярного произведения, значит, неравенство выполняется. Отбрасывая этот очевидный случай, будем считать, что $\vec{x} \ne \vec{0}$. Для любого действительного числа, в силу аксиомы 4 скалярного произведения, выполняется неравенство

    $$(\lambda \vec{x} - \vec{y}, \lambda \vec{x} - \vec{y}) \geq 0.$$

    Преобразуем левую часть неравенства, используя аксиомы и свойства скалярного произведения:

    \begin{align*}
        (\lambda \vec{x} - \vec{y}, \lambda \vec{x} - \vec{y}) &= \lambda (\vec{x}, \lambda \vec{x} - \vec{y}) - (\vec{y}, \lambda\vec{x} - \vec{y}) = \\
        &= \lambda^2\underbrace{(\vec{x}, \vec{x})}_{\ne 0} - 2\lambda(\vec{x}, \vec{y}) + (\vec{y}, \vec{y}) \geq 0.
    \end{align*}

    Мы получили квадратным трехчлен относительно параметра $\lambda$, неотрицательный при всех действительных значениях параметра. Следовательно, его дискриминант равен нулю или отрицательный, т.е.

    $$(\vec{x}, \vec{y})^2 - (\vec{x}, \vec{x})(\vec{y}, \vec{y}) \leq 0.$$
\end{proof}


\newpage
\section{
    Матрица Грама для системы векторов. Определитель и ранг матрицы Грама. Как изменится грамиан, если один из векторов заменить его ортогональной проекцией?
}

% Матрица Грама для системы векторов
\subsection{
    Матрица Грама для системы векторов.
}
    
\begin{definition}
    Пусть даны векторы \( \vec{x}_1, \vec{x}_2, \dots, \vec{x}_m \) в некотором евклидовом пространстве. \textit{\textbf{Матрицей Грама}} этой системы называется квадратная матрица \( \Gamma \) размера \( m \times m \), элементы которой задаются скалярными произведениями:

    \[
    \Gamma = \begin{pmatrix}
    (\vec{x}_1, \vec{x}_1) & (\vec{x}_1, \vec{x}_2) & \cdots & (\vec{x}_1, \vec{x}_m) \\
    (\vec{x}_2, \vec{x}_1) & (\vec{x}_2, \vec{x}_2) & \cdots & (\vec{x}_2, \vec{x}_m) \\
    \vdots     & \vdots     & \ddots & \vdots     \\
    (\vec{x}_m, \vec{x}_1) & (\vec{x}_m, \vec{x}_2) & \cdots & (\vec{x}_m, \vec{x}_m)
    \end{pmatrix},
    \]
    
    где \( (\vec{x}_i, \vec{x}_j) \) обозначает скалярное произведение векторов \( \vec{x}_i \) и \( \vec{x}_j \).
\end{definition}

Её определитель называется определителем Грама (или \textbf{\textit{грамианом}}).

% Определитель и ранг матрицы Грама
\subsection{
    Определитель и ранг матрицы Грама.
}


\begin{itemize}
    \item $\det \Gamma \geq 0$ (всегда неотрицателен).
    \item $\det \Gamma = 0$ $\iff$ векторы $\vec{x}_1, \vec{x}_2, \dots, \vec{x}_m$ линейно зависимы.
    \item Для линейно независимых векторов $\det \Gamma > 0$.
    \item Геометрический смысл: $\det \Gamma$ равен квадрату объёма параллелепипеда, натянутого на векторы.
\end{itemize}

Ранг матрицы Грама равен максимальному числу линейно независимых векторов в системе, т.е. 

$\rank \Gamma(\vec{x}_1, \vec{x}_2, \dots, \vec{x}_m) = \dim \Span (\vec{x}_1, \vec{x}_2, \dots, \vec{x}_m)$.

% Как изменится грамиан, если один из векторов заменить его ортогональной проекцией?
\subsection{
    Как изменится грамиан, если один из векторов заменить его ортогональной проекцией?
}

Рассмотрим линейное пространство $\mathcal{L} = \Span(\vec{e}_1, \ldots, \vec{e}_{k - 1}, \vec{e}_k, \vec{e}_{k + 1}, \ldots, \vec{e}_n)$ и его подпространство 

$\mathcal{H} = \Span(\vec{e}_1, \ldots, \vec{e}_{k - 1}, \vec{e}_{k + 1}, \ldots, \vec{e}_n)$.

\bigbreak

Представим вектор $\vec{e}_k \in \mathcal{L}$ в виде 
$$\vec{e}_k = \vec{e}^{\parallel}_k + \vec{e}^{\perp}_k,$$

где $\vec{e}^{\parallel}_k \in \mathcal{H}$ - ортогональная проекция, а $\vec{e}^{\perp}_k \in \mathcal{H}^\perp$ - ортогональная составляющая.

Значит,

\begin{gather*}
    \vec{e}^{\parallel}_k = \alpha_1\vec{e}_1 + \ldots + \alpha_{k - 1}\vec{e}_{k - 1} + \alpha_{k + 1}\vec{e}_{k + 1} + \ldots + \alpha_n\vec{e}_n. \\
    \downimplies \\
    \vec{e}_1, \ldots, \vec{e}_{k - 1}, \vec{e}^{\parallel}_k, \vec{e}_{k + 1}, \ldots, \vec{e}_n \text{ - линейно зависимы.} \\
    \downimplies \\
    \det \Gamma(\underbrace{\vec{e}_1, \ldots, \vec{e}_{k - 1}, \vec{e}^{\parallel}_k, \vec{e}_{k + 1}, \ldots, \vec{e}_n}_{\text{Система векторов, полученная заменой } \vec{e}_k \text{ на } \vec{e}^{\parallel}_k.}) = 0
\end{gather*}


\newpage
\section{
    Норма вектора в евклидовом пространстве. Привести три примера задания нормы. Свойства нормы (с доказательством).
}

% Норма вектора в евклидовом пространстве
\subsection{
    Норма вектора в евклидовом пространстве.
}

\begin{definition}
    Функция, заданная на линейном пространстве $\mathcal{V}$, которая каждому вектору ставит в соответствие вещественное число, называется \textbf{\textit{нормой}}, если выполнены 3 аксиомы:
    \begin{enumerate}[nosep]
        \item $\norm{\vec{x}} \geq 0$, причем $\norm{\vec{x}} = 0 \iff \vec{x} = 0$;
        \item $\norm{\lambda \vec{x}} = |\lambda| \cdot  \norm{\vec{x}}, \thinspace \lambda \in \RR$;
        \item $\norm{\vec{x} + \vec{y}} \leq \norm{\vec{x}} + \norm{\vec{y}}$ (неравенство треугольника).
    \end{enumerate}
\end{definition}

\begin{theorem}
    Всякое скалярное произведение в евклидовом пространстве определяет норму $\norm{\vec{x}} = \sqrt{(\vec{x}, \vec{x})}$.
\end{theorem}

% Три примера задания нормы
\subsection{
    Три примера задания нормы.
}


\begin{definition}
    Норма вида $\norm{\vec{x}}_2 = \sqrt{(\vec{x}, \vec{x})}$ называется \textbf{\textit{евклидовой}} $(l_2)$
\end{definition}

\begin{definition}
    Норма вида $\norm{\vec{x}}_1 = |x_1| + \dots + |x_n|$ называется \textbf{\textit{октаэдрической}} $(l_1)$
\end{definition}

\begin{definition}
    Норма вида $\norm{\vec{x}}_{\infty} = max\{|x_1|, \dots, |x_n|\}$ называется \textbf{\textit{кубической}} $(l_{\infty})$
\end{definition}

% Свойства нормы
\subsection{
    Свойства нормы.
}

\begin{enumerate}[label={\arabic*°.}]
    \item $\norm{\vec{x}} - \norm{\vec{y}} \leq \norm{\vec{x} \pm \vec{y}} \leq \norm{\vec{x}} + \norm{\vec{y}}$.
    
    \item $\norm{\alpha\vec{x} + \beta\vec{y}} \leq \norm{\alpha\vec{x}} + \norm{\beta\vec{y}} = |\alpha|\norm{\vec{x}} + |\beta|\norm{\vec{y}}$.

    \item $\norm{\vec{x} + \vec{y}}^2 + \norm{\vec{x} - \vec{y}}^2 = 2(\norm{\vec{x}}^2 + \norm{\vec{y}}^2)$.

    \begin{proof}
        \begin{gather*}
            \norm{\vec{x} + \vec{y}}^2 = (\vec{x} + \vec{y}, \vec{x} + \vec{y}) = (\vec{x}, \vec{x}) + (\vec{x}, \vec{y}) + (\vec{y}, \vec{x}) + (\vec{y}, \vec{y}),\\
            \norm{\vec{x} - \vec{y}}^2 = (\vec{x} - \vec{y}, \vec{x} - \vec{y}) = (\vec{x}, \vec{x}) - (\vec{x}, \vec{y}) - (\vec{y}, \vec{x}) + (\vec{y}, \vec{y}),\\
            \norm{\vec{x} + \vec{y}}^2 + \norm{\vec{x} - \vec{y}}^2 = 2(\vec{x}, \vec{x}) + 2(\vec{y}, \vec{y}) = 2(\norm{\vec{x}}^2 + \norm{\vec{y}}^2).
        \end{gather*}
    \end{proof}
    
    \item Если норма порождена скалярным произведением, т.е. $\norm{\vec{z}} = \sqrt{(\vec{z}, \vec{z})}$, то для неё определено
    
    $$(\vec{x}, \vec{y}) = \frac{1}{2}\cdot(\norm{\vec{x} + \vec{y}}^2 - \norm{\vec{x}}^2 - \norm{\vec{y}}^2).$$

    \begin{proof}
        \begin{align*}
            &\frac{1}{2}\cdot(\norm{\vec{x} + \vec{y}}^2 - \norm{\vec{x}}^2 - \norm{\vec{y}}^2) = \\
            &= \frac{1}{2}\cdot((\vec{x} + \vec{y}, \vec{x} + \vec{y}) - (\vec{x}, \vec{x}) - (\vec{y}, \vec{y})) = \\
            &= \frac{1}{2} \cdot ((\vec{x}, \vec{x}) + 2(\vec{x}, \vec{y}) + (\vec{y}, \vec{y})  - (\vec{x}, \vec{x}) - (\vec{y}, \vec{y})) = \\
            &= (\vec{x}, \vec{y}).
        \end{align*}
    \end{proof}
    
    \item В конечномерном пространстве любые две нормы эквивалентны.

    \begin{definition}[P.S.]
        Две нормы $p$ и $q$ на пространстве $\mathcal{V}$ называются эквивалентными, если
        
        $$\exists C_1, C_2 > 0 \colon C_1p(\vec{x}) \leq q(\vec{x}) \leq C_2p(\vec{x}), \forall \vec{x} \in \mathcal{V}.$$
    \end{definition}
\end{enumerate}



\part{Модуль 2.}

    \section{
    Линейный оператор, определение, три 
    примера. Матрица линейного оператора. 
    Вывести формулу для вычисления значений 
    линейного оператора (с помощью его 
    матрицы). Произведение линейных 
    операторов. Матрица для произведения 
    линейных операторов.
 }

% Линейный оператор, определение, три примера
\subsection{
    Линейный оператор, определение, три примера.
}

\begin{definition}
    Пусть $\mathcal{V}, \mathcal{W}$ - 
    линейные пространства над полем $\PP$, 
    $\dim \mathcal{V} = n, \dim \mathcal{W} = m$. 
    Отображение $\mathscr{A} \colon \mathcal{V} 
    \to \mathcal{W}$ называется 
    \textit{\textbf{линейным оператором}}, 
    если $\forall \vec{x}, \vec{y} \in 
    \mathcal{V}$ и $\forall \lambda \in \PP$ 
    выполнены следующие условия:
    
    \begin{enumerate}[nosep]
        \item $\mathscr{A}(\vec{x} + \vec{y}) = 
        \mathscr{A}(\vec{x}) + \mathscr{A}(\vec{y}),$
        \item $\mathscr{A}(\alpha\vec{x}) = 
        \alpha\mathscr{A}(\vec{x})$,
    \end{enumerate}
    где $\vec{x}$ - прообраз $\vec{y} = \mathscr{A}\vec{x}$,
    $\vec{y} = \mathscr{A}\vec{x}$ - образ $\vec{x}$.
\end{definition}

\begin{example}~

    \begin{enumerate}[nosep]
        \item Нулевой: $\zeroperator\vec{x} = \vec{0}$.
        \item Тождественный/единичный: 
        $\identityoperator\vec{x} = \vec{x}$.
        \item Оператор дифференцирования $\mathscr{D} \colon P_n(x) \to P_n(x)$, действующий по правилу $\mathscr{D}f = f'$.
    \end{enumerate}
\end{example}



\newpage


% Матрица линейного оператора
\subsection{
    Матрица линейного оператора.
}

Пусть $\mathcal{V}$ и $\mathcal{W}$ - два линейных пространства.

Пусть $e = (\vec{e_1}, \ldots, \vec{e_n})$ - некоторый базис в $\mathcal{V}$, $f = (\vec{f_1}, \ldots, \vec{f_m})$ - некоторый базис в $\mathcal{W}$. 

Тогда $\mathscr{A}\vec{e_1}, \ldots, \mathscr{A}\vec{e_n}$ - это некоторые векторы в $\mathcal{W}$; значит, их можно, причем единственным образом, разложить по базису $\vec{f_1}, \ldots, \vec{f_m}$:

$$\mathscr{A}\vec{e_1} = a_{11}\vec{f_1} + \ldots + a_{m1}\vec{f_m}$$
$$\ldots \ldots \ldots \ldots \ldots \ldots \ldots \ldots \ldots$$
$$\mathscr{A}\vec{e_n} = a_{1n}\vec{f_1} + \ldots + a_{mn}\vec{f_m},$$

где $(a_{ij} \in \PP)$.

\begin{definition}
    Матрицу, составленную из координатных столбцов векторов $\mathscr{A}\vec{e_1}, \ldots, \mathscr{A}\vec{e_n}$ в базисе $f = (\vec{f_1}, \ldots, \vec{f_m})$, называют \textit{\textbf{матрицей линейного оператора}} $\mathscr{A}$ в базисе $f$.
\end{definition}



\newpage

% Вывести формулу для вычисления значений линейного оператора (с помощью его матрицы)
\subsection{
    Вывести формулу для вычисления значений линейного оператора (с помощью его матрицы).
}

\begin{theorem}
    Пусть $\mathscr{A} \colon \mathcal{L} \to \mathcal{L}$ - линейный оператор. Тогда столбец $y_b$ координат вектора $\vec{y} = \mathscr{A}\vec{x}$ в данном базисе $b$ линейного пространства $\mathcal{L}$ равен произведению $A_bx_b$ матрицы $A_b$ оператора $\mathscr{A}$ в базисе $b$ на столбец $x_b$ координат вектора $\vec{x}$ в том же базисе: $y_b = A_bx_b$.
    \label{thm:theorem_21_1}
\end{theorem}

\begin{proof}
    Выберем произвольный вектор $\vec{x} = x_1\vec{b_1} + \ldots + x_n\vec{b_n}$. Его образом будет вектор

    \begin{align*}
        \vec{y} &= \mathscr{A}\vec{x} =  \mathscr{A}(x_1\vec{b_1} + \ldots + x_n\vec{b_n}) = x_1(\mathscr{A}\vec{b_1}) + \ldots + x_n(\mathscr{A}\vec{b_n}) = \\
        &= x_1(a_{11}\vec{b_1} + \ldots + a_{n1}\vec{b_n}) + \ldots + x_n(a_{1n}\vec{b_1} + \ldots + a_{nn}\vec{b_n}) = \\
        &= (a_{11}x_1 + \ldots + a_{1n}x_n)\vec{b_1} + \ldots + (a_{n1}x_1 + \ldots + a_{nn}x_n)\vec{b_n} = \\
        &= \begin{pmatrix}
            \vec{b}_1 & \ldots & \vec{b}_n
        \end{pmatrix} \cdot \begin{pmatrix} 
            a_{11}x_1 + \ldots + a_{1n}x_n \\
            \vdots \\
            a_{n1}x_1 + \ldots + a_{nn}x_n
        \end{pmatrix}
    \end{align*}

    Столбец координат вектора $\vec{y} = \mathscr{A}\vec{x}$ в базисе $b$ имеет вид

    \begin{equation*}
        y_b = (\mathscr{A}\vec{x})_b = \begin{pmatrix} 
            a_{11}x_1 + \ldots + a_{1n}x_n \\
            \vdots \\
            a_{n1}x_1 + \ldots + a_{nn}x_n
        \end{pmatrix} =
        \begin{pmatrix} 
            a_{11} \thinspace \thinspace \ldots \thinspace \thinspace a_{1n} \\
            \ldots \ldots \ldots \\
            a_{n1} \thinspace \thinspace \ldots \thinspace \thinspace a_{nn}
        \end{pmatrix}
        \begin{pmatrix} 
            x_1 \\
            \vdots \\
            x_n
        \end{pmatrix} = A_bx_b
    .\end{equation*}
\end{proof}

\begin{corollary}
    $\vec{y} = \mathscr{A}\vec{x} = bA_bx_b.$
\end{corollary}



\newpage


% Произведение линейных операторов. Матрица для произведения линейных операторов
\subsection{
    Произведение линейных операторов. Матрица для произведения линейных операторов.
}

\begin{definition}
    \textbf{\textit{Произведением операторов}} $\mathscr{A} \colon \mathcal{V} \to \mathcal{W}$ и $\mathscr{B} \colon \mathcal{L} \to \mathcal{V}$ называется оператор $(\mathscr{A}\mathscr{B}) \colon \mathcal{L} \to \mathcal{W}$, действующий по правилу $(\mathscr{A}\mathscr{B})\vec{x} = \mathscr{A}(\mathscr{B}\vec{x}), \forall \vec{x} \in \mathcal{L}$. 
    
    Этот оператор является линейным, так как $\forall \vec{x}, \vec{y} \in \mathcal{L},\thinspace \thinspace \forall \lambda, \mu \in \RR$:

    $$(\mathscr{A}\mathscr{B})(\lambda\vec{x} + \mu\vec{y}) = \mathscr{A}(\mathscr{B}(\lambda\vec{x} + \mu\vec{y})) = \mathscr{A}(\lambda\mathscr{B}\vec{x} + \mu\mathscr{B}\vec{y}) = \lambda\mathscr{A}(\mathscr{B}\vec{x}) + \mu\mathscr{A}(\mathscr{B}\vec{y}) = \lambda(\mathscr{A}\mathscr{B})\vec{x} + \mu(\mathscr{A}\mathscr{B})\vec{y}.$$
\end{definition}

\begin{theorem}
    Пусть в линейном пространстве $\mathcal{L}$ действуют линейные операторы $\mathscr{A}$ и $\mathscr{B}$, а $A_b$ и $B_b$ - матрицы этих линейных операторов в некотором базисе $b$. Тогда матрицей линейного оператора $\mathscr{B}\mathscr{A}$ в том же базисе $b$ является матрица $B_bA_b$.
\end{theorem}

\begin{proof}
    $\vec{y} = (\mathscr{B}\mathscr{A})\vec{x} = \mathscr{B}(\mathscr{A}\vec{x}) = \mathscr{B}(bA_bx_b) = b(B_b(A_bx_b)) = b(B_bA_b)x_b.$
\end{proof}


\newpage
\section{
    Линейный оператор, определение, три примера. Преобразование матрицы линейного оператора при переходе к новому базису (вывести формулу).
 }

% Преобразование матрицы линейного оператора при переходе к новому базису (вывести формулу)
\subsection{
    Преобразование матрицы линейного оператора при переходе к новому базису (вывести формулу).
}

\begin{theorem}
    Матрицы $A_b$ и $A_e$ линейного оператора $\mathscr{A} \colon \mathcal{L} \to \mathcal{L}$, записанные в базисах $b$ и $e$ линейного пространства $\mathcal{L}$, связаны друг с другом соотношением
    
    $$A_e = T^{-1}_{b \to e}A_bT_{b \to e}.$$
\end{theorem}

\begin{proof}~

    Пусть $\vec{y} = \mathscr{A}\vec{x}$. Обозначим координаты векторов $\vec{x}$ и $\vec{y}$ в старом базисе $b$ через $x_b$ и $y_b$, а в новом базисе $e$ - через $x_e$ и $y_e$. Поскольку действие линейного оператора $\mathscr{A}$ в матричной форме в базисе $b$ имеет вид $y_b = A_bx_b$ (\textbf{*}см. теорему \ref{thm:theorem_21_1}), а координаты векторов $\vec{x}$ и $\vec{y}$ в новом и старом базисах связаны между собой равенствами (\textbf{*}см. билет 17.)

    $$x_b = T_{b \to e}x_e, \quad \quad y_b = T_{b \to e}y_e,$$

    то получаем

    $$y_e = T^{-1}_{b \to e}y_b = T^{-1}_{b \to e}(A_bx_b) = T^{-1}_{b \to e}(A_bT_{b \to e}x_e) = (T^{-1}_{b \to e}A_bT_{b \to e})x_e.$$
    Равенство $y_e = (T^{-1}_{b \to e}A_bT_{b \to e})x_e$ является матричной формой записи действия линейного оператора $\mathscr{A}$ в базисе $e$ и поэтому, согласно теореме \ref{thm:theorem_21_1}, $T^{-1}_{b \to e}A_bT_{b \to e} = A_e$. 

    Изложенное доказательство теоремы хорошо иллюстрирует следующая диаграмма:

    \begin{figure}[H]
        \centering
        \includegraphics[scale=0.5]{images/module2/question22/1.jpg}
        \label{fig:picture_22_1}
    \end{figure}
\end{proof}


\newpage
\section{
    Линейный оператор, определение, три примера. Ранг и дефект, ядро и образ линейного оператора. Теорема про размерности. Инвариантное подпространство линейного оператора.  
}

% Ранг и дефект, ядро и образ линейного оператора
\subsection{
    Ранг и дефект, ядро и образ линейного оператора.
}

Пусть $\mathscr{A} \colon \mathcal{V} \to \mathcal{W}$ - линейный оператор.

\begin{definition}
    \textbf{\textit{Образом}} оператора $\mathscr{A}$ называется множество всех векторов $\vec{y} \in \mathcal{W}$, представимых в виде $\vec{y} = \mathscr{A}\vec{x}$.
\end{definition}

\begin{designation}
    $\Im \mathscr{A}$.
\end{designation}

\begin{definition}
    \textbf{\textit{Ядром}} оператора $\mathscr{A}$ называется множество всех векторов $\vec{x} \in \mathcal{V} \colon \mathscr{A}\vec{x} = \vec{0}_{\mathcal{W}}$.
\end{definition}

\begin{designation}
    $\ker \mathscr{A}$.
\end{designation}

\begin{definition}
    Размерность образа линейного оператора $\mathscr{A}$ называется \textbf{\textit{рангом}} линейного оператора $\mathscr{A}$. 
\end{definition}

\begin{designation}
    $\rank \mathscr{A}$.
\end{designation}

\begin{definition}
    Размерность ядра линейного оператора $\mathscr{A}$ называется \textbf{\textit{дефектом}} линейного оператора $\mathscr{A}$. 
\end{definition}

\begin{designation}
    $\defect \mathscr{A}$.
\end{designation}



\newpage


% Теорема про размерности
\subsection{
    Связь с решениями неоднородной СЛАУ.
}

Рассмотрим неоднородную СЛАУ

$$A\vec{x} = \vec{b},$$

где $A \in \RR^{m \times n}, \vec{x} = \begin{pmatrix} x_1 \\ \vdots \\ x_n \end{pmatrix} \in \RR^n, \vec{b} = \begin{pmatrix} b_1 \\ \vdots \\ b_m \end{pmatrix} \in \RR^m$.

Из курса Ан. Геом. известно, что если СЛАУ имеет бесконечно много решений, то они задаются так:

$$\vec{x} = \vec{x}_0 + c_1\vec{f}_1 + \ldots + c_k\vec{f}_k,$$

где $\vec{x}_0 \in \RR^n$ - вектор с постоянными коэффициентами, $\vec{f}_1, \ldots, \vec{f}_k \in \RR^n$ - векторы, образующие ФСР, $c_1, \ldots, c_k$ - произвольные константы.

Если рассмотреть $\mathcal{W} = \Span \{\vec{f}_1, \ldots, \vec{f}_k\}$, то получится, что множество $\mathcal{U}$ всех решений СЛАУ будет представлять из себя ЛАМ:

$$\mathcal{U} = \vec{x}_0 + \mathcal{W}.$$

Представляете!



\newpage


% Инвариантное подпространство линейного оператора
\subsection{
    Инвариантное подпространство линейного оператора.
}

\begin{definition}
    Пусть $\mathscr{A} \colon \mathcal{L} \to \mathcal{L}$ - линейный оператор. Подпространство $\mathcal{V} \in \mathcal{L}$ называется \textbf{\textit{инвариантным подпространством}} оператора $\mathscr{A}$, если оператор $\mathscr{A}$ отображает всякий вектор $\vec{x} \in \mathcal{V}$, в вектор, также принадлежащий подпространству $\mathcal{V}$, то есть $\forall \vec{x} \in \mathcal{V} \colon \vec{y} = \mathscr{A}\vec{x} \in \mathcal{V}$.
\end{definition}

\begin{example}~

    \begin{itemize}
        \item Тривиальными примерами являются: само пространство $\mathcal{L}$ и нулевое подпространство (состоящее из единственного нулевого вектора).
        \item Любой собственный вектор оператора порождает его одномерное инвариантное подпространство.
        \item Ядро линейного оператора $\ker \mathcal{L}$. 
    \end{itemize}
\end{example}


\newpage
\section{
    Операции с линейными операторами. Ранг произведения операторов. Линейное пространство линейных операторов.
}

% Операции с линейными операторами
\subsection{
    Операции с линейными операторами.
}

\begin{definition}
    Операторы $\mathscr{A} \colon \mathcal{V} \to \mathcal{W}$ и $\mathscr{B} \colon \mathcal{V} \to \mathcal{W}$ называются \textbf{\textit{равными}}, если $\mathscr{A}\vec{x} = \mathscr{B}\vec{x}, \forall \vec{x} \in \mathcal{V}$.
\end{definition}

\begin{definition}
    \textbf{\textit{Суммой операторов}} $\mathscr{A} \colon \mathcal{V} \to \mathcal{W}$ и $\mathscr{B} \colon \mathcal{V} \to \mathcal{W}$ называется оператор $(\mathscr{A} + \mathscr{B}) \colon \mathcal{V} \to \mathcal{W}$, действующий по правилу $(\mathscr{A} + \mathscr{B})\vec{x} = \mathscr{A}\vec{x} + \mathscr{B}\vec{x}, \forall \vec{x} \in \mathcal{V}$.
\end{definition}

\begin{definition}
    \textbf{\textit{Произведением оператора}} $\mathscr{A} \colon \mathcal{V} \to \mathcal{W}$ \textbf{\textit{на действительное число}} $\lambda$ называется оператор $(\lambda\mathscr{A}) \colon \mathcal{V} \to \mathcal{W}$, действующий по правилу $(\lambda\mathscr{A})\vec{x} = \lambda(\mathscr{A}\vec{x}), \forall \vec{x} \in \mathcal{V}$.
\end{definition}

\begin{definition}
    \textbf{\textit{Произведением операторов}} $\mathscr{A} \colon \mathcal{V} \to \mathcal{W}$ и $\mathscr{B} \colon \mathcal{L} \to \mathcal{V}$ называется оператор $(\mathscr{A}\mathscr{B}) \colon \mathcal{L} \to \mathcal{W}$, действующий по правилу $(\mathscr{A}\mathscr{B})\vec{x} = \mathscr{A}(\mathscr{B}\vec{x}), \forall \vec{x} \in \mathcal{L}$.
\end{definition}



\newpage


% Ранг произведения операторов
\subsection{
    Ранг произведения операторов.
}

Для любых двух линейных операторов $\mathscr{A}$ и $\mathscr{B}$, действующих в линейном пространстве $\mathcal{L}$, выполняется соотношение

$$\rank(\mathscr{A}\mathscr{B}) \leq \min\{\rank\mathscr{A}, \rank\mathscr{B}\}$$

\begin{proof}~

    Рассмотрим оператор $\mathscr{A}$ как линейный оператор $\mathscr{A}\colon \Im\mathscr{B} \to \mathcal{L}$. Размерность образа оператора не превосходит размерности линейного пространства, из которого он действует, так как сумма и дефекта и ранга совпадает с размерностью этого пространства.

    $$\rank(\mathscr{A}\mathscr{B}) = \dim \Im (\mathscr{A}\mathscr{B}) \leq \dim \Im \mathscr{B} = \rank \mathscr{B}.$$
    
    Так как образ линейного оператора $\mathscr{A}\mathscr{B}$ является линейным подпространством образа линейного оператора $\mathscr{A}$, то
    
    $$\rank(\mathscr{A}\mathscr{B}) \leq \rank \mathscr{A}.$$
\end{proof}


\begin{comment}~

    Доказанное соотношение можно перенести на квадратные матрицы. 
    
    Получаем, 
    $$\rank{(AB)} \leq \min\{\rank A, \rank B\}.$$

    Пусть $B$ - невырожденная. То есть ее ранг равен размерности матрицы. 
    
    Тогда $\rank{(AB)} \leq \rank A$ и одновременно $\rank A = \rank ((AB)B^{-1}) \leq \rank (AB)$.

    То есть 
    
    $$\rank (AB) \leq \rank A \leq \rank (AB).$$
    
    Следовательно, при умножении матрицы $A$ справа на невырожденную матрицу ее ранг не изменяется. 
    
    При умножении матрицы $A$ слева на невырожденную матрицу ранг также не изменяется, что доказывается аналогично.
    \label{comment:comment_24_2}
\end{comment}



\newpage


% Линейное пространство линейных операторов
\subsection{
    Линейное пространство линейных операторов.
}

\begin{definition}
    Линейное пространство $\mathcal{L}(\mathcal{V}, \mathcal{W})$ линейных операторов из линейного пространства $\mathcal{V}$ в линейное пространство $\mathcal{W}$ называют \textbf{\textit{линейным пространством линейных операторов}}.
\end{definition}

\begin{proof}[Проверка на линейность пространства $\mathcal{L}$]~

    Пусть даны линейные операторы $\mathscr{A}б \mathscr{B} \in \mathcal{L}(\mathcal{V}, \mathcal{W})$. 

    Поскольку
    \begin{align*}
        (\mathscr{A} + &\mathscr{B})(\alpha\vec{x} + \beta \vec{y}) = \mathscr{A}(\alpha\vec{x} + \beta \vec{y}) + \mathscr{B}(\alpha\vec{x} + \beta \vec{y}) = \\
        &= (\alpha\mathscr{A}\vec{x} + \beta\mathscr{A}\vec{y}) + (\alpha\mathscr{B}\vec{x} + \beta\mathscr{B}\vec{y}) = \\
        &= \alpha(\mathscr{A}\vec{x} + \mathscr{B}\vec{x}) + \beta(\mathscr{A}\vec{y} + \mathscr{B}\vec{y}) = \\
        &= \alpha(\mathscr{A} + \mathscr{B})\vec{x} + \beta(\mathscr{A} + \mathscr{B})\vec{y}
    \end{align*}

    и

    \begin{align*}
        (\lambda\mathscr{A})&(\alpha\vec{x} + \beta \vec{y}) = \lambda(\mathscr{A}(\alpha\vec{x} + \beta\vec{y})) = \lambda (\mathscr{A}(\alpha\vec{x}) + \mathscr{A}(\beta\vec{y})) = \\ 
        &= (\alpha \lambda)\mathscr{A}\vec{x} + (\beta \lambda)\mathscr{A}\vec{y} = \alpha(\lambda \mathscr{A}\vec{x}) + \beta(\lambda \mathscr{A}\vec{y}) = \\
        &= \alpha((\lambda\mathscr{A})\vec{x}) + \beta((\lambda\mathscr{A})\vec{y})
    \end{align*}

    отображения $\mathscr{A} + \mathscr{B}$ и $\lambda \mathscr{A}$ действительно являются линейными операторами. Таким образом, относительно введенных нами операций множество $\mathcal{L}(\mathcal{V}, \mathcal{W})$ замкнуто. Проверив аксиомы линейного пространства, можно убедиться, что $\mathcal{L}(\mathcal{V}, \mathcal{W})$ относительно этих операций является линейным пространством.
\end{proof}


\newpage
\section{
    Собственные векторы и собственные значения линейного оператора. Характеристическое уравнение и характеристический многочлен линейного оператора. Нахождение собственных значений линейного оператора (вывести характеристическое уравнение). Геометрическая и алгебраическая кратность. Жорданова нормальная форма.
}

% Собственные векторы и собственные значения линейного оператора
\subsection{
    Собственные векторы и собственные значения линейного оператора.
}

\begin{definition}
    Ненулевой вектор $\vec{x}$ в линейном пространстве $\mathcal{L}$ называют \textbf{\textit{собственным вектором}} линейного оператора $\mathscr{A}\colon \mathcal{L} \to \mathcal{L}$, если для некоторого действительного числа $\lambda$ выполняется соотношение $\mathscr{A}\vec{x} = \lambda\vec{x}$. При этом число $\lambda$ называют \textbf{\textit{собственным значением}} линейного оператора $\mathscr{A}$.
\end{definition}

% Характеристическое уравнение и характеристический многочлен линейного оператора
\subsection{
    Характеристическое уравнение и характеристический многочлен линейного оператора.
}

Для произвольной квадратной матрицы $A = (a_{ij})$ порядка $n$ рассмотрим определитель

$$\det(A - \lambda E) = \begin{vmatrix} 
    a_{11} - \lambda & a_{12} & \ldots & a_{1n} \\
    a_{21} & a_{22} - \lambda & \ldots & a_{2n} \\
    \vdots & \vdots & \ddots & \vdots \ \\
    a_{n1} & a_{12} & \ldots & a_{nn} - \lambda \\
\end{vmatrix},$$

где $E$ - единичная матрица, а $\lambda$ - действительное переменное.

\begin{definition}
    Многочлен $\chi_A(\lambda) = \det(A - \lambda E)$ называют \textbf{\textit{характеристическим многочленом}} матрицы $A$, а уравнение $\chi_A(\lambda) = 0$ — \textbf{\textit{характеристическим уравнением}} матрицы $A$.
\end{definition}

\begin{definition}
    \textbf{\textit{Характеристическим многочленом линейного оператора}} $\mathscr{A} \colon \mathcal{L} \to \mathcal{L}$ называют характеристический многочлен его матрицы $A$, записанной в некотором базисе, а \textbf{\textit{характеристическим уравнением}} этого \textbf{\textit{оператора}} - характеристическое уравнение матрицы $A$.
\end{definition}

% *Полезные факты, которые тоже могут быть на экзамене
\subsection{
    *Полезные факты, которые тоже могут быть на экзамене.
}

\begin{definition}
    Квадратные матрицы $A$ и $B$ порядка $n$ называются \textit{\textbf{подобными}}, если существует такая невырожденная матрица $P$, что $P^{-1}AP = B$.
    \label{def:definition_17_1}
\end{definition}

\begin{theorem}
    Если матрицы $A$ и $B$ подобны, то $\det A = \det B$.
    \label{thm:theorem_25_1}
\end{theorem}

\begin{proof}~

    Если матрицы подобны, то согласно определению \eqref{def:definition_17_1}, существует такая невырожденная матрица $P$, что $B = P^{-1}AP$. Так как определитель произведения квадратных матриц равен произведению определителей этих матриц, а $\det(P^{-1}) = (\det P)^{-1}$, то получаем
    $$\det B = \det(P^{-1}AP) = \det(P^{-1})\det A \det P = \det(P)^{-1}\det A \det P = \det A.$$
\end{proof}

\begin{corollary}
    Определитель матрицы линейного оператора не зависит от выбора базиса.
\end{corollary}

\begin{proof}
    Действительно, возьмем матрицы $A_b$ и $A_e$ линейного оператора $\mathscr{A}$ в двух различных базисах $b$ и $e$.

    $$A_e = T^{-1}_{b \to e}A_bT_{b \to e}.$$

    Согласно определению \eqref{def:definition_17_1}, матрицы $A_b$ и $A_e$ подобны. Поэтому $\det A_b = \det A_e$ по теореме \ref{thm:theorem_25_1}.
\end{proof}



\newpage


% Нахождение собственных значений линейного оператора (вывести характеристическое уравнение)
\subsection{
    Нахождение собственных значений линейного оператора (вывести характеристическое уравнение).
}

\begin{theorem}
    Для того чтобы действительное число $\lambda$ являлось собственным значением линейного оператора, необходимо и достаточно, чтобы оно было корнем характеристического уравнения этого оператора.
\end{theorem}

\begin{proof}~
    \begin{description}
        \item[$(\implies)$] 
            Пусть число $\lambda$ является собственным значением линейного оператора $\mathscr{A} \colon \mathcal{L} \to \mathcal{L}$. Это значит, что существует вектор $\vec{x} \ne \vec{0}$, для которого 

            $$\mathscr{A}\vec{x} = \lambda \vec{x}.$$

            Используя тождественный оператор $\mathscr{I}\vec{x} = \vec{x}$, преобразуем равенство: $\mathscr{A}\vec{x} = \lambda\mathscr{I}\vec{x}$, или
            
            $$(\mathscr{A} - \lambda\mathscr{I})\vec{x} = \vec{0}.$$

            Запишем векторное равенство выше в каком-либо базисе $b$. Матрицей линейного оператора $\mathscr{A} - \lambda\mathscr{I}$ будет матрица $A - \lambda E$, где $A$ - матрица линейного оператора $\mathscr{A}$ в базисе $b$, а $E$ - единичная матрица, и пусть $x$ - столбец координат собственного вектора $\vec{x}$. Тогда $x \ne 0$, а векторное равентсво выше равносильно матричному

            $$(A - \lambda E) = 0,$$

            которое представляет собой матричную форму записи ОСЛАУ с квадратной матрицей $A - \lambda E$ порядка $n$. Эта система имеет ненулевое решение, являющееся столбцом координат $x$ собственного вектора $\vec{x}$. Поэтому $\det(A - \lambda E) = 0$. А это означает, что $\lambda$ является корнем характеристического уравнения линейного оператора $\mathscr{A}$.
        \item[$(\impliedby)$]
            Приведенные рассуждения можно привести в обратном порядке. Если $\lambda$ является корнем характеристического уравнения, то в заданном базисе $b$ выполняется равенство $\det (A - \lambda E) = 0$. Следовательно, матрица ОСЛАУ, записанной в матричной форме, вырождена, и система имеет ненулевое решение $x$. Это ненулевое решение $x$ представляет собой набор координат в базисе $b$ некоторого ненулевого вектора $\vec{x}$, для которого выполняется равенство $(\mathscr{A} - \lambda\mathscr{I})\vec{x} = \vec{0}$. Значит, число $\lambda$ - собственное значение линейного оператора $\mathscr{A}$.
    \end{description}
\end{proof}



\newpage


% Геометрическая и алгебраическая кратность
\subsection{
    Геометрическая и алгебраическая кратность.
}

\begin{definition}
    \textbf{\textit{Геометрической кратностью}} собственного значения линейного оператора называется максимальное число линейно независимых собственных векторов, соответствующих данному собственному значению.
\end{definition}

\begin{definition}
    \textbf{\textit{Алгебраической кратностью}} собственного значения линейного оператора называется его кратность как корня характеристического многочлена.
\end{definition}



\newpage


% Жорданова нормальная форма
\subsection{
    Жорданова нормальная форма.
}

Для произвольного действительного числа $\mu$ введем обозначение матрицы порядка $s$:

$$J_s(\mu) = \begin{pmatrix} 
    \mu & 1 & 0 & \ldots & 0 & 0 \\
    0 & \mu & 1 & \ldots & 0 & 0 \\
    \hdotsfor6 \\
    0 & 0 & 0 & \ldots & \mu & 1 \\
    0 & 0 & 0 & \ldots & 0 & \mu
\end{pmatrix}$$

Для любого комплексного числа $\lambda = \alpha + i\beta (\beta \ne 0)$ введем обозначение блочной матрицы порядка $2r$:

$$C_r(\alpha, \beta) = \begin{pmatrix} 
    C(\alpha, \beta) & E & 0 & \ldots & 0 & 0 \\
    0 & C(\alpha, \beta) & E & \ldots & 0 & 0 \\
    \hdotsfor6 \\
    0 & 0 & 0 & \ldots & C(\alpha, \beta) & E \\
    0 & 0 & 0 & \ldots & 0 & C(\alpha, \beta)
\end{pmatrix},$$

где $C(\alpha, \beta) = \begin{pmatrix} 
    \alpha & \beta \\
    -\beta & \alpha
\end{pmatrix}$. Все остальные блоки также являются квадратными матрицами порядка 2, где $E$ - единичная матрица, 0 - нулевая.

Блочно-диагональную матрицу вида

\[
A = \begin{pmatrix}
    C_{r_1}(\alpha_1, \beta_1) &        &        &        &  \\
                               & \ddots &        &        & \scaleobj{4}{0}   \\
                               &        & C_{r_m}(\alpha_m, \beta_m) &        &   \\
                               &        &        & J_{s_1}(\mu_1) &   \\
                               & \scaleobj{4}{0} &        &     & \ddots  &  \\
                               &        &        &        & &J_{s_k}(\mu_k)
\end{pmatrix},
\]

где $\alpha_j, \beta_j (j = \overline{1, m})$ и $\mu_l (l = \overline{1, k})$ - действительные числа, называют \textbf{\textit{жордановой}}, ее диагональные блоки - \textbf{\textit{жордановыми клетками}}. Жорданову матрицу $A'$, подобную данной матрице $A$, называют \textbf{\textit{жордановой нормальной формой}} матрицы $A$.


\newpage
\section{
    Формулировка теоремы Гамильтона – Кэли. След линейного оператора. Инварианты.
}

% Формулировка теоремы Гамильтона – Кэли
\subsection{
    Формулировка теоремы Гамильтона – Кэли.
}

Квадратную матрицу можно использовать в качестве значения переменного в произвольном многочлене. Тогда значением многочлена от матрицы будет матрица того же порядка, что и исходная. Интерес представляют такие многочлены, значение которых от данной матрицы есть нулевая матрица. Их называют аннулирующими многочленами. Оказывается, что одним из таких аннулирующих многочленов для матрицы является ее характеристический многочлен.

\begin{theorem}
    Для любой квадратной матрицы характеристический многочлен является ее аннулирующим многочленом.
\end{theorem}

% След линейного оператора
\subsection{
    След линейного оператора.
}

\begin{definition}
    \textbf{\textit{Следом линейного оператора $\mathscr{A}$ (матрицы $A$)}} называется сумма диагональных элементов матрицы $A$ линейного оператора $\mathscr{A}$.
\end{definition}

\begin{designation}
    $\tr \mathscr{A}$ или $\Sp \mathscr{A}$.
\end{designation}

% Инварианты
\subsection{
    Инварианты.
}

Коэффициенты характеристического многочлена не зависят от выбора базиса (если представить в виде $\sum_{k=0}^{n} d_k \lambda^k
$), т.е. являются инвариантами относительно выбора базиса.

\begin{comment}
    Наиболее просто выражается коэффициент $d_{n - 1} = \tr \mathscr{A}$.
\end{comment}

\begin{comment}
    Коэффициент $d_0$ характеристического многочлена совпадает со значением этого многочлена при $\lambda = 0$ и равен определителю линейного оператора $\mathscr{A}$.
\end{comment}


\newpage
\section{
    Формулировка теоремы про ЖНФ. Алгоритм построения ЖНФ. Определение количества клеток. Нахождение базиса. 
}

Сначала сформулируем и докажем теоремы Фредгольма для операторов в линейных пространствах:

\subsection{
    Альтернатива Фредгольма.
}

\begin{theorem}["Альтернатива Фредгольма"] Пусть

    \begin{enumerate}
        \item $\left.\begin{array}{l}
            \mathcal{V} \text{ - линейное пространство, } \dim \mathcal{V} = n \\
            \mathcal{W} \text{ - линейное пространство, } \dim \mathcal{W} = m
        \end{array}\right\}$ \text{В обоих задано скалярное произведение.}
        \item $\mathscr{A} \colon \mathcal{V} \to \mathcal{W}$ - линейный оператор.
    \end{enumerate}

    Тогда справедливо ровно одно из двух:

    \begin{itemize}[nosep]
        \item либо уравнение $\mathscr{A}\vec{v} = \vec{w}$ имеет решение при любом $\vec{w} \in \mathcal{W}$,
        \item либо уравнение $\mathscr{A^*}\vec{w} = \vec{0}$ имеет нетривиальное (ненулевое) решение.
    \end{itemize}
\end{theorem}

\begin{proof}~

    Обозначим $r = \rank(\mathscr{A})$.

    \textbf{1 случай:}
    
    \begin{gather*}
        r = m \\
        \downimplies \\
        \dim(\Im \mathscr{A}) = m \\
        \downimplies \\
        \text{Т.к. } \Im \mathscr{A} \text{— } m\text{-мерное подпространство } m\text{-мерного пространства } \mathcal{W}, \text{то } \Im \mathscr{A} = \mathcal{W}. \\
        \downimplies \\
        \forall \vec{w} \in \mathcal{W} \exists \vec{v} \in \mathcal{V} \colon \mathscr{A}\vec{v} = \vec{w}. \\
        \text{Также заметим, что }\dim(\Im \mathscr{A^*}) + \dim(\ker \mathscr{A^*}) = m \\
        \downimplies \\
        \dim(\Im \mathscr{A}) + \dim(\ker \mathscr{A^*}) = m \\
        \downimplies \\
        m + \dim(\ker \mathscr{A^*}) = m \\
        \downimplies \\
        \dim(\ker \mathscr{A^*}) = 0 \\
        \downimplies \\
        \ker \mathscr{A^*} = \{\vec{0}\} \\
        \downimplies \\
        \text{Уравнение } \mathscr{A^*}\vec{w} = \vec{0} \text{ имеет только тривиальное решение.}
    \end{gather*}

    Получается, что второе "либо" не выполнено!
    
    \textbf{2 случай:}

    \begin{gather*}
        r < m \\
        \downimplies \\
        \dim(\Im \mathscr{A}) < \dim \mathcal{W} \\
        \downimplies \\
        \dim(\Im \mathscr{A^*}) < \dim \mathcal{W} \\
        \downimplies \\
        \text{Т.к. } \dim(\Im \mathscr{A^*}) + \dim(\ker \mathscr{A^*}) = \dim \mathcal{W}, \ker \mathscr{A^*} \ne \{\vec{0}\} \\
        \downimplies \\
        \exists \vec{w} \ne 0 \in \ker \mathscr{A^*} \\
        \downimplies \\
        \exists \vec{w} \ne 0 \colon \mathscr{A^*}\vec{w} = \vec{0}
    \end{gather*}
\end{proof}


\newpage
\section{
    Собственные векторы и собственные значения линейного оператора. Доказать независимость характеристического многочлена и характеристического уравнения линейного оператора от выбора базиса.
}

\begin{theorem}
    Характеристические многочлены (уравнения) подобных матриц совпадают.
    \label{thm:theorem_28_1}
\end{theorem}

\begin{proof}~

    Пусть квадратные матрицы $A$ и $A'$ одного порядка подобны, т.е. существует такая невырожденная матрица $P$ того же порядка, что $A' = P^{-1}AP$. Тогда в силу свойств определителей имеем

    \begin{align*}
        \chi_{A'}&(\lambda) = \det(A' - \lambda E) = \det(P^{-1}AP - \lambda P^{-1}EP) = \\
        &=\det(P^{-1}(A - \lambda E)P) = \det P^{-1}\det(A - \lambda E)\det P = \\
        &=\det(A - \lambda E) = \chi_A(\lambda).
    \end{align*}
\end{proof}

\begin{theorem}
    Характеристический многочлен и характеристическое уравнение линейного оператора не зависят от выбора базиса.
\end{theorem}

\begin{proof} Пусть 

    \begin{enumerate}[nosep]
        \item $\mathscr{A} \colon \mathcal{L} \to \mathcal{L}$ - линейный оператор,
        \item $A_b$ - матрица линейного оператора $\mathscr{A}$ в некотором ''старом'' базисе $b$, 
        \item $A_e$ - матрица линейного оператора $\mathscr{A}$  в некотором ''новом'' базисе $e$.
    \end{enumerate}
    
    Тогда 
    
    $$A_e = T^{-1}_{b \to e}A_bT_{b \to e},$$ где $T_{b \to e}$ - матрица перехода от базиса $b$ к базису $e$.

    $A_e$ и $A_b$ - две подобные матрицы.

    Значит, по теореме \ref{thm:theorem_28_1} характеристические многочлены двух подобных матриц $A_e$ и $A_b$ равны. Следовательно, и характеристический многочлен линейного оператора не зависит от выбора базиса.
\end{proof}


\newpage
\section{
    Собственные векторы и собственные значения линейного оператора. Свойство собственных векторов линейного оператора, соответствующих одному и тому же собственному значению (с доказательством).
}

% Свойство собственных векторов линейного оператора, соответствующих одному и тому же собственному значению (с доказательством)
\subsection{
    Свойство собственных векторов линейного оператора, соответствующих одному и тому же собственному значению (с доказательством).
}

Если $\vec{x}$ - собственный вектор оператора $\mathscr{A}$, отвечающий собственному значению, то для любого числа $k \ne 0$ вектор $k\vec{x}$ также является собственным вектором оператора $\mathscr{A}$, отвечающим собственному значению.

\begin{proof}~

    $\mathscr{A}\vec{x} = \lambda\vec{x}$,
    
    $\mathscr{A}(k\vec{x}) = k\mathscr{A}\vec{x} = k\lambda\vec{x}$.
\end{proof}

Если $\vec{x}$ и $\vec{y}$ - собственные векторы оператора $\mathscr{A}$, отвечающие собственному значению, то вектор $\vec{x} + \vec{y} \ne \vec{0}$ также является собственным вектором, отвечающим собственному значению.

\begin{proof}~

    $\mathscr{A}\vec{x} = \lambda\vec{x}$,
    
    $\mathscr{A}\vec{y} = \lambda\vec{y}$,
    
    $\mathscr{A}(\vec{x} + \vec{y}) = \mathscr{A}\vec{x} + \mathscr{A}\vec{y} = \lambda\vec{x} + \lambda\vec{y} = \lambda(\vec{x} + \vec{y})$.
\end{proof}

\begin{corollary}[свойство]~

    Множество всех собственных векторов, отвечающих данному собственному значению, с добавлением $\vec{0}$ является линейным подпространством для данного пространства $\mathcal{L}$. Такое подпространство называется \textbf{\textit{собственным подпространством $\mathcal{L}$}}.
\end{corollary}


\newpage
\section{
    Собственные векторы и собственные значения линейного оператора. Свойство собственных векторов линейного оператора, соответствующих различным собственным значениям (с доказательством).
}

% Свойство собственных векторов линейного оператора, соответствующих различным собственным значениям (с доказательством)
Сначала сформулируем и докажем теоремы Фредгольма для операторов в линейных пространствах:

\subsection{
    Альтернатива Фредгольма.
}

\begin{theorem}["Альтернатива Фредгольма"] Пусть

    \begin{enumerate}
        \item $\left.\begin{array}{l}
            \mathcal{V} \text{ - линейное пространство, } \dim \mathcal{V} = n \\
            \mathcal{W} \text{ - линейное пространство, } \dim \mathcal{W} = m
        \end{array}\right\}$ \text{В обоих задано скалярное произведение.}
        \item $\mathscr{A} \colon \mathcal{V} \to \mathcal{W}$ - линейный оператор.
    \end{enumerate}

    Тогда справедливо ровно одно из двух:

    \begin{itemize}[nosep]
        \item либо уравнение $\mathscr{A}\vec{v} = \vec{w}$ имеет решение при любом $\vec{w} \in \mathcal{W}$,
        \item либо уравнение $\mathscr{A^*}\vec{w} = \vec{0}$ имеет нетривиальное (ненулевое) решение.
    \end{itemize}
\end{theorem}

\begin{proof}~

    Обозначим $r = \rank(\mathscr{A})$.

    \textbf{1 случай:}
    
    \begin{gather*}
        r = m \\
        \downimplies \\
        \dim(\Im \mathscr{A}) = m \\
        \downimplies \\
        \text{Т.к. } \Im \mathscr{A} \text{— } m\text{-мерное подпространство } m\text{-мерного пространства } \mathcal{W}, \text{то } \Im \mathscr{A} = \mathcal{W}. \\
        \downimplies \\
        \forall \vec{w} \in \mathcal{W} \exists \vec{v} \in \mathcal{V} \colon \mathscr{A}\vec{v} = \vec{w}. \\
        \text{Также заметим, что }\dim(\Im \mathscr{A^*}) + \dim(\ker \mathscr{A^*}) = m \\
        \downimplies \\
        \dim(\Im \mathscr{A}) + \dim(\ker \mathscr{A^*}) = m \\
        \downimplies \\
        m + \dim(\ker \mathscr{A^*}) = m \\
        \downimplies \\
        \dim(\ker \mathscr{A^*}) = 0 \\
        \downimplies \\
        \ker \mathscr{A^*} = \{\vec{0}\} \\
        \downimplies \\
        \text{Уравнение } \mathscr{A^*}\vec{w} = \vec{0} \text{ имеет только тривиальное решение.}
    \end{gather*}

    Получается, что второе "либо" не выполнено!
    
    \textbf{2 случай:}

    \begin{gather*}
        r < m \\
        \downimplies \\
        \dim(\Im \mathscr{A}) < \dim \mathcal{W} \\
        \downimplies \\
        \dim(\Im \mathscr{A^*}) < \dim \mathcal{W} \\
        \downimplies \\
        \text{Т.к. } \dim(\Im \mathscr{A^*}) + \dim(\ker \mathscr{A^*}) = \dim \mathcal{W}, \ker \mathscr{A^*} \ne \{\vec{0}\} \\
        \downimplies \\
        \exists \vec{w} \ne 0 \in \ker \mathscr{A^*} \\
        \downimplies \\
        \exists \vec{w} \ne 0 \colon \mathscr{A^*}\vec{w} = \vec{0}
    \end{gather*}
\end{proof}


\newpage
\section{
    Дать определение сопряженного и самосопряженного линейного оператора. Доказать, что все корни характеристического многочлена самосопряженного оператора вещественны.
}

% Дать определение сопряженного и самосопряженного линейного оператора
\subsection{
    Дать определение сопряженного и самосопряженного линейного оператора.
}

Пусть $\mathcal{E}$ - евклидово пространство.

\begin{definition}
    Линейный оператор $\mathscr{A^*} \colon \mathcal{E} \to \mathcal{E}$ называют сопряженным к линейному оператору $\mathscr{A} \colon \mathcal{E} \to \mathcal{E}$, если для любых векторов $\vec{x}, \vec{y} \in \mathcal{E}$ верно равенство
    $$(\mathscr{A}\vec{x}, \vec{y}) = (\vec{x}, \mathscr{A^*}\vec{y}).$$
\end{definition}

\begin{example}~
    
    Вектор $\vec{a} \in \mathcal{V}_3$ порождает линейный оператор $\mathscr{A} \colon \mathcal{V}_3 \to \mathcal{V}_3$ согласно формуле
    
    $$\mathscr{A}\vec{x} = \vec{a} \times \vec{x}.$$

    Найдем оператор, сопряженный оператору $\mathscr{A}$:
    \begin{align*}
        (\mathscr{A}\vec{x}, &\vec{y}) = (\vec{a} \times \vec{x}, \vec{y}) = \vec{a}\vec{x}\vec{y} = \vec{y}\vec{a}\vec{x} = (\vec{y} \times \vec{a}, \vec{x}) = \\
        &= (\vec{x}, \vec{y} \times \vec{a}) = (\vec{x}, -\vec{a} \times \vec{y}) = (\vec{x}, -\mathscr{A}\vec{y}).
    \end{align*}

    Значит, $\mathscr{A^*} = -\mathscr{A}.$
\end{example}

\begin{definition}
    Линейный оператор $\mathscr{A}$, действующий в евклидовом пространстве, называют самосопряженным, если $\mathscr{A^*} = \mathscr{A}$. То есть для любых векторов $\vec{x}$ и $\vec{y}$ верно равенство

    $$(\mathscr{A}\vec{x}, \vec{y}) = (\vec{x}, \mathscr{A}\vec{y}).$$
\end{definition}

\begin{example}
    Тождественный $\mathscr{I}$ и нулевой $\mathscr{O}$.
\end{example}



\newpage


% Доказать, что все корни характеристического многочлена самосопряженного оператора вещественны
\subsection{
    Доказать, что все корни характеристического многочлена самосопряженного оператора вещественны.
}

\begin{theorem}
    Матрица самосопряженного оператора в любом ортонормированном базисе является симметрической.
\end{theorem}

\begin{theorem}
    Все корни характеристического многочлена самосопряженного оператора вещественны.
\end{theorem}

\begin{proof}~

    Будем доказывать, что все корни характеристического уравнения симметрической матрицы действительны.

    Предположим, что $\lambda \in \CC$ является корнем характеристического уравнения симметрической матрицы, т.е. $\det (A - \lambda E) = 0$. Тогда СЛАУ $(A - \lambda E)x = 0$ имеет некоторое ненулевое решение $x = \begin{pmatrix} x_1 & \cdots & x_n \end{pmatrix} ^ T$, состоящее из комплексных чисел $x_k, k = \overline{1, n}$. Рассмотрим столбец $\overline{x}$, комплексно сопряженный к столбцу $x$. Умножим равенство $(A - \lambda E)x = 0$ слева на строку $\overline{x}^T$. Тогда

    $$\overline{x}^T(A - \lambda E)x = 0,$$

    или

    $$\overline{x}^TAx = \lambda \overline{x}^Tx.$$

    Так как произведение комплексного числа на сопряженное к нему является действительным числом, равным квадрату модуля комплексного числа, а $x$ - ненулевое решение, то
    
    $$\overline{x}^Tx = \overline{x}_1x_1 + \ldots + \overline{x}_nx_n = |x_1|^2 + \ldots + |x_n|^2 > 0,$$
    То есть матричное произведение $\overline{x}^Tx$ - действительное положительное число.

    $$\lambda = \frac{\overline{x}^TAx}{\overline{x}^Tx},$$

    причем знаменатель дроби справа является действительным числом. Следовательно, число $\lambda$ будет действительным, если числитель этой дроби $w = \overline{x}^TAx$ будет действительным.

    В силу симметричности матрицы $A$

    $$w = w^T = (\overline{x}^TAx)^T = x^TA^T\overline{x} = x^TA\overline{x}.$$

    С учетом свойств операции комплексного сопряжения матриц и благодаря тому, что элементами матрицы $A$ являются действительные числа, получаем

    $$\overline{w} = \overline{\overline{x}^TAx} = (\overline{\overline{x}})^T\overline{A}\overline{x} = x^TA\overline{x} = w.$$

    Комплексное число, самосопряженное себе - это действительное число. Следовательно, и $w$ является действительным.
\end{proof}


\newpage
\section{
    Дать определение самосопряженного линейного  оператора. Свойство собственных векторов самосопряженного линейного оператора, отвечающих различным собственным значениям (с доказательством).
}

% Свойство собственных векторов самосопряженного линейного оператора, отвечающих различным собственным значениям (с доказательством)
Сначала сформулируем и докажем теоремы Фредгольма для операторов в линейных пространствах:

\subsection{
    Альтернатива Фредгольма.
}

\begin{theorem}["Альтернатива Фредгольма"] Пусть

    \begin{enumerate}
        \item $\left.\begin{array}{l}
            \mathcal{V} \text{ - линейное пространство, } \dim \mathcal{V} = n \\
            \mathcal{W} \text{ - линейное пространство, } \dim \mathcal{W} = m
        \end{array}\right\}$ \text{В обоих задано скалярное произведение.}
        \item $\mathscr{A} \colon \mathcal{V} \to \mathcal{W}$ - линейный оператор.
    \end{enumerate}

    Тогда справедливо ровно одно из двух:

    \begin{itemize}[nosep]
        \item либо уравнение $\mathscr{A}\vec{v} = \vec{w}$ имеет решение при любом $\vec{w} \in \mathcal{W}$,
        \item либо уравнение $\mathscr{A^*}\vec{w} = \vec{0}$ имеет нетривиальное (ненулевое) решение.
    \end{itemize}
\end{theorem}

\begin{proof}~

    Обозначим $r = \rank(\mathscr{A})$.

    \textbf{1 случай:}
    
    \begin{gather*}
        r = m \\
        \downimplies \\
        \dim(\Im \mathscr{A}) = m \\
        \downimplies \\
        \text{Т.к. } \Im \mathscr{A} \text{— } m\text{-мерное подпространство } m\text{-мерного пространства } \mathcal{W}, \text{то } \Im \mathscr{A} = \mathcal{W}. \\
        \downimplies \\
        \forall \vec{w} \in \mathcal{W} \exists \vec{v} \in \mathcal{V} \colon \mathscr{A}\vec{v} = \vec{w}. \\
        \text{Также заметим, что }\dim(\Im \mathscr{A^*}) + \dim(\ker \mathscr{A^*}) = m \\
        \downimplies \\
        \dim(\Im \mathscr{A}) + \dim(\ker \mathscr{A^*}) = m \\
        \downimplies \\
        m + \dim(\ker \mathscr{A^*}) = m \\
        \downimplies \\
        \dim(\ker \mathscr{A^*}) = 0 \\
        \downimplies \\
        \ker \mathscr{A^*} = \{\vec{0}\} \\
        \downimplies \\
        \text{Уравнение } \mathscr{A^*}\vec{w} = \vec{0} \text{ имеет только тривиальное решение.}
    \end{gather*}

    Получается, что второе "либо" не выполнено!
    
    \textbf{2 случай:}

    \begin{gather*}
        r < m \\
        \downimplies \\
        \dim(\Im \mathscr{A}) < \dim \mathcal{W} \\
        \downimplies \\
        \dim(\Im \mathscr{A^*}) < \dim \mathcal{W} \\
        \downimplies \\
        \text{Т.к. } \dim(\Im \mathscr{A^*}) + \dim(\ker \mathscr{A^*}) = \dim \mathcal{W}, \ker \mathscr{A^*} \ne \{\vec{0}\} \\
        \downimplies \\
        \exists \vec{w} \ne 0 \in \ker \mathscr{A^*} \\
        \downimplies \\
        \exists \vec{w} \ne 0 \colon \mathscr{A^*}\vec{w} = \vec{0}
    \end{gather*}
\end{proof}


\newpage
\section{
    Ортогональные матрицы и их свойства.
}

\begin{definition}
    Квадратную матрицу $O$ называют \textbf{\textit{ортогональной}}, если она удовлетворяет условию

    \begin{equation}
        O^TO = E,
        \label{eq:equation_33_1}
    \end{equation}

    где $E$ — единичная матрица.
\end{definition}

\begin{example}
    Простейший пример - единичная матрица $E$, так как $E^TE = EE = E$.
\end{example}

\begin{example}
    $U = \begin{pmatrix}
    \cos \varphi & -\sin \varphi \\
    \sin \varphi & \cos \varphi
    \end{pmatrix}$.
\end{example}

\subsection*{Свойства ортогональных матриц.}

Пусть $O$ - ортогональная матрица.

\begin{enumerate}[label={\arabic*°.}]
    \item $\det O = \pm 1$.
    
    \begin{proof}~
    
        $\det(O^TO) = \det O^T \det O = (\det O)^2$.
        
        Так как $\det E = 1$, то и $(\det O)^2 = 1$. Следовательно, $\det O = \pm 1$.
    \end{proof}
    
    \item $O^{-1} = O^T$.

    \begin{proof}~
    
        Согласно свойству 1, ортогональная матрица невырождена и поэтому имеет обратную $O^{-1}$. Умножая равенство \eqref{eq:equation_33_1} справа на $O^{-1}$, получаем

        $$(O^TO)O^{-1} = EO^{-1},$$

        откуда $O^T(OO^{-1}) = O^{-1}$. Но $OO^{-1} = E$, поэтому $O^T = O^{-1}$.
    \end{proof}

    \item $OO^T = E$.

    \begin{proof}
        Согласно свойству 2 и определению обратной матрицы, $OO^T = OO^{-1} = E$.
    \end{proof}

    \item $O^T$ - тоже ортогональная.

    \begin{proof}~
    
        Нужно для произвольной ортогональной матрицы $O$ доказать равенство

        $$(O^T)^TO^T = E,$$

        представляющее собой запись соотношения \eqref{eq:equation_33_1} для предполагаемой ортогональной матрицы $O^T$ (\textbf{*}вместо $O$). Так как, согласно свойству операции транспонирования, $(O^T)^T = O$, равенство выше эквивалентно $OO^T = E$, которое верно в силу свойства 3.
    \end{proof}

    \item Произведение двух ортогональных матриц $O$ и $Q$ одного порядка является ортогональной матрицей.

    \begin{proof}
        Для доказательства достаточно проверить выполнение равенства \eqref{eq:equation_33_1} для матрицы $OQ$:

        $$(OQ)^T(OQ) = (Q^TO^T)OQ = Q^T(O^TO)Q = Q^TEQ = Q^TQ = E.$$
    \end{proof}

    \item $O^{-1}$ - тоже ортогональная.

    \begin{proof}~
    
        Согласно свойству 1, ортогональная матрица вырождена, а потому имеет обратную. Согласно свойству 2, матрица, обратная к ортогональной, совпадает с транспонированной. Наконец, согласно свойству 4, матрица, транспонированная к ортогональной, является ортогональной.
    \end{proof}
\end{enumerate}


\newpage
\section{
    Ортогональное преобразование евклидова пространства. Свойства ортогональных преобразований (с доказательством).
}

% Ортогональное преобразование евклидова пространства
\subsection{
    Ортогональное преобразование евклидова пространства.
}

Пусть $\mathcal{E}$ - евклидово пространство, $\mathscr{A} \colon \mathcal{E} \to \mathcal{E}$ - некоторый линейный оператор.

\begin{definition}
     Говорят, что $\mathscr{A}$ задает \textbf{\textit{ортогональное преобразование (называется ортогональным оператором)}}, если $\forall \vec{x}, \vec{y} \in \mathcal{V} \colon (\mathscr{A}\vec{x}, \mathscr{A}\vec{y}) = (\vec{x}, \vec{y})$.
\end{definition}

% Свойства ортогональных преобразований
\subsection{
    Свойства ортогональных преобразований.
}

\begin{enumerate}[label={\arabic*°.}]
    \item Сохраняет ортогональность, т.е. $\vec{x} \perp \vec{y} \Rightarrow \mathscr{A}\vec{x} \perp \mathscr{A}\vec{y}$.

    \begin{proof}
        Очевидно.
    \end{proof}
    
    \item Сохраняет норму вектора, т.е. $\norm{\mathscr{A}\vec{x}} = \norm{\vec{x}}$.

    \begin{proof}
        $\norm{\mathscr{A}\vec{x}} = \sqrt{(\mathscr{A}\vec{x}, \mathscr{A}\vec{x})} = \sqrt{(\vec{x}, \vec{x})} = \norm{\vec{x}}$.
    \end{proof}

    \item Сохраняет углы между ненулевыми векторами.

    \begin{proof}
        $\widehat{(\mathscr{A}\vec{x}, \mathscr{A}\vec{y})} = \arccos \frac{(\mathscr{A}\vec{x}, \mathscr{A}\vec{y})}{\norm{\mathscr{A}\vec{x}}\cdot\norm{\mathscr{A}\vec{y}}} = \arccos \frac{(\vec{x}, \vec{y})}{\norm{\vec{x}}\cdot\norm{\vec{y}}} = \widehat{(\vec{x}, \vec{y})}$.
    \end{proof}

    \item Пусть $\mathscr{A} \colon \mathcal{E} \to \mathcal{E}$ - ортогональный оператор, $\vec{e}_1, \ldots, \vec{e}_n$ - ОНБ в $\mathcal{E}$. Тогда $\mathscr{A}\vec{e}_1, \ldots, \mathscr{A}\vec{e}_n$ - ОНБ в $\mathcal{E}$.

    \begin{proof}~
    
        $(\mathscr{A}\vec{e}_i, \mathscr{A}\vec{e}_j) = (\vec{e}_i, \vec{e}_j) = \delta_{ij} \text{ (символ Кронекера)}$.

        $\norm{\mathscr{A}\vec{e}_i} = 1$; $\mathscr{A}\vec{e}_1, \ldots, \mathscr{A}\vec{e}_n$ попарно ортогональны, а значит, ЛНЗ. К тому же, $\dim \mathcal{E} = n$, а $\mathscr{A}\vec{e}_1, \ldots, \mathscr{A}\vec{e}_n$ - тоже $n$. Значит, $\mathscr{A}\vec{e}_1, \ldots, \mathscr{A}\vec{e}_n$ - ОНБ.
    \end{proof}

    \item Пусть 
    
    \begin{itemize}
        \item $\mathcal{E}$ - $n$-мерное евклидово пространство;
        \item $\vec{e}_1, \ldots, \vec{e}_n$ - некоторый ОНБ $\mathcal{E}$;
        \item $\mathscr{A} \colon \mathcal{E} \to \mathcal{E}$ - некоторый линейный оператор.
        \item $\mathscr{A}\vec{e}_1, \ldots, \mathscr{A}\vec{e}_n$ - тоже ОНБ $\mathcal{E}$.
    \end{itemize}

    Тогда $\mathscr{A}$ - ортогональный оператор.

    \begin{proof}~
    
        Требуется доказать, что

        $$\forall \vec{x}, \vec{y} \in \mathcal{E} \colon (\mathscr{A}\vec{x}, \mathscr{A}\vec{y}) = (\vec{x}, \vec{y}).$$

        Возьмем произвольные $\vec{x}, \vec{y} \in \mathcal{E}$.

        Разложим по базису $\vec{e}_1, \ldots, \vec{e}_n$:

        \begin{gather*}
            \vec{x} = x_1\vec{e}_1 + \ldots + x_n\vec{e}_n \\
            \vec{y} = y_1\vec{e}_1 + \ldots + y_n\vec{e}_n
        \end{gather*}

        Найдем:

        \begin{align*}
            (\mathscr{A}\vec{x}, \mathscr{A}\vec{y}) &= (\mathscr{A}(x_1\vec{e}_1 + \ldots + x_n\vec{e}_n), \mathscr{A}(y_1\vec{e}_1 + \ldots + y_n\vec{e}_n)) = \\
            &= (x_1\cdot\mathscr{A}\vec{e}_1 + \ldots + x_n\cdot\mathscr{A}\vec{e}_n, y_1\cdot\mathscr{A}\vec{e}_1 + \ldots + y_n\cdot\mathscr{A}\vec{e}_n) = \\
            &= \left\{ 
            \begin{array}{l}
                \text{Получилось, что} \\
                \text{у вектора } \mathscr{A}\vec{x} \\
                \text{в базисе } \mathscr{A}\vec{e}_1, \ldots, \mathscr{A}\vec{e}_n \\
                \text{координаты } \begin{pmatrix}
                    x_1 \\
                    \vdots \\
                    x_n
                \end{pmatrix}; \\
                \text{у вектора } \vec{y} \text{ аналогично}.
            \end{array} \right\} \underbrace{=}_{\text{Следствие } \ref{corollary:corollary_1}} \\
            &\underbrace{=}_{\text{Следствие } \ref{corollary:corollary_1}} x_1y_1 + \ldots + x_ny_n = (\vec{x}, \vec{y}).
        \end{align*}
    \end{proof}

    \item Пусть 
    
    \begin{itemize}
        \item $\mathcal{E}$ - $n$-мерное евклидово пространство;
        \item $\vec{e}_1, \ldots, \vec{e}_n$ - некоторый ОНБ $\mathcal{E}$;
        \item $\mathscr{A} \colon \mathcal{E} \to \mathcal{E}$ - ортогональный оператор.
    \end{itemize}

    Тогда матрица $A_e$ линейного оператора $\mathscr{A}$ в базисе $\vec{e}_1, \ldots, \vec{e}_n$ - ортогональная.

    \begin{proof}~
    
        \begin{align*}
            (\mathscr{A}\vec{x}, \mathscr{A}\vec{y}) &\underbrace{=}_{\text{Лемма } \ref{lemma:lemma_1}} (\mathscr{A}\vec{x})^T_e\cdot\Gamma_e\cdot(\mathscr{A}\vec{y})_e = \\
            &= (A_e\vec{x}_e)^T\cdot\Gamma_e\cdot(A_e\vec{y}_e) = \\
            &= \vec{x}^T\cdot A^T_e\Gamma_eA_e\vec{y}_e.
        \end{align*}

        При этом

        $$(\vec{x}, \vec{y}) = \vec{x}^T_e\cdot\Gamma_e\cdot\vec{y}_e.$$

        Из этого ясно, что 
        $$A^T_e\Gamma_eA_e = \Gamma_e.$$

        Так как $e$ - ОНБ, то $\Gamma_e = E \in \RR^{n \times n}$.

        И тогда $A^T_e \cdot A_e = E$.
    \end{proof}

    \item Справедлив и обратный факт свойству 6:

    Если линейный оператор $\mathscr{A} \colon \mathcal{E} \to \mathcal{E}$ в ОНБ $\vec{e}_1, \ldots, \vec{e}_n$ имеет ортогональную матрицу, то $\mathscr{A}$ задает ортогональное преобразование.

    \begin{proof}~
    
        Действительно, 

        \begin{align*}
            (\mathscr{A}\vec{x}, \mathscr{A}\vec{y}) &\underbrace{=}_{\text{Лемма } \ref{lemma:lemma_1}} (\mathscr{A}\vec{x})^T_e\cdot\Gamma_e\cdot(\mathscr{A}\vec{y})_e = \\
            &= (A_e\vec{x}_e)^T\cdot E\cdot(A_e\vec{y}_e) = \\
            &= \vec{x}^T\cdot \underbrace{A^T_eA_e}_{ E}\vec{y}_e = \\
            &= \vec{x}^T\vec{y}_e = \\
            &= (\vec{x}, \vec{y}).
        \end{align*}
    \end{proof}
\end{enumerate}


\newpage
\section{
    Дать определение квадратичной формы. Матрица квадратичной формы и ее преобразование при переходе к новому базису (вывести формулу).
}

\begin{definition}
    \textbf{\textit{Квадратичной формой}} называется сумма вида $$f(x_1, \ldots, x_n) = \sum_{i=1}^n b_{ii}x_i^2 + \sum_{1 \leq i < j \leq n} 2b_{ij}x_ix_j,$$

    где $b_{ij}$ - заданные числа, $1 \leq i < j < n$.
\end{definition}

Ясно, что

$$
f(x_1, x_2, \ldots, x_n) = 
\begin{pmatrix}
x_1 & x_2 & \cdots & x_n
\end{pmatrix}
\begin{pmatrix}
    b_{11} & b_{12} & \cdots & b_{1n} \\
    b_{21} & b_{22} & \cdots & b_{2n} \\
    \vdots & \vdots & \ddots & \vdots \\
    b_{n1} & b_{n2} & \cdots & b_{nn}
\end{pmatrix}
\begin{pmatrix}
    x_1 \\
    x_2 \\
    \vdots \\
    x_n
\end{pmatrix}.
$$

Столбец $\begin{pmatrix}
    x_1 \\
    x_2 \\
    \vdots \\
    x_n
\end{pmatrix}$ можно рассматривать как координаты вектора $\vec{x}$ в некотором 'первичном' базисе $\vec{e}_1, \vec{e}_2, \ldots, \vec{e}_n$, а полученную матрицу $\begin{pmatrix}
    b_{11} & b_{12} & \cdots & b_{1n} \\
    b_{21} & b_{22} & \cdots & b_{2n} \\
    \vdots & \vdots & \ddots & \vdots \\
    b_{n1} & b_{n2} & \cdots & b_{nn}
\end{pmatrix}$ - как матрицу квадратичной формы в этом 'первичном' базисе $\vec{e}_1, \vec{e}_2, \ldots, \vec{e}_n$. 

Как она изменится при переходе к новому базису?

Мы знаем, что если $\begin{pmatrix}
    x_1 \\
    x_2 \\
    \vdots \\
    x_n
\end{pmatrix}$ - координаты вектора $\vec{x}$ в базисе $e$, а $\begin{pmatrix}
    y_1 \\
    y_2 \\
    \vdots \\
    y_n
\end{pmatrix}$ - координаты того же вектора $\vec{x}$ в базисе $f$, то

$$\begin{pmatrix}
    x_1 \\
    x_2 \\
    \vdots \\
    x_n
\end{pmatrix} = T_{e \to f}\begin{pmatrix}
    y_1 \\
    y_2 \\
    \vdots \\
    y_n
\end{pmatrix}\text{ (см. \ref{subsection:subsection_17_3}).}$$

Тогда 

$$\begin{pmatrix}
x_1 & x_2 & \cdots & x_n
\end{pmatrix} = \left(T_{e \to f}\begin{pmatrix}
    y_1 \\
    y_2 \\
    \vdots \\
    y_n
\end{pmatrix}\right)^T = \begin{pmatrix}
y_1 & y_2 & \cdots & y_n
\end{pmatrix} \cdot T_{e \to f}^T.$$

Значит,

$$
f(x_1, x_2, \ldots, x_n) = 
\begin{pmatrix}
y_1 & y_2 & \cdots & y_n
\end{pmatrix}
\underbrace{
T_{e \to f}^T
\begin{pmatrix}
    b_{11} & b_{12} & \cdots & b_{1n} \\
    b_{21} & b_{22} & \cdots & b_{2n} \\
    \vdots & \vdots & \ddots & \vdots \\
    b_{n1} & b_{n2} & \cdots & b_{nn}
\end{pmatrix}
T_{e \to f}}_{\text{матрица кв. формы в новом базисе.}}
\begin{pmatrix}
    y_1 \\
    y_2 \\
    \vdots \\
    y_n
\end{pmatrix}.
$$

Значит,

$$\underbracket{B_f}_{\mathclap{\substack{\text{матр. кв.} \\ \text{формы} \\ \text{в новом} \\ \text{базисе } f.}}} = T_{e \to f}^T\underbracket{B_e}_{\mathclap{\substack{\text{исх. матр.} \\ \text{кв. формы} \\ \text{(в старом} \\ \text{базисе } e).}}}T_{e \to f}.$$


\newpage
\section{
    Ранг квадратичной формы, его независимость от выбора базиса. Закон инерции квадратичных форм (с доказательством). 
}

% Ранг квадратичной формы, его независимость от выбора базиса
\subsection{
    Ранг квадратичной формы, его независимость от выбора базиса.
}

\begin{definition}
    Ранг матрицы $A$ квадратичной формы называют \textbf{\textit{рангом квадратичной формы}}.
\end{definition}

При изменении базиса линейного пространства матрица $A$ квадратичной формы преобразуется по формуле $A' = U^TAU$, где $U$ - матрица перехода. Матрица $U$, как матрица перехода, является невырожденной, поэтому ранг $A'$ совпадает с рангом $A$, так как при умножении на невырожденную матрицу ранг не меняется. (см. замечание \ref{comment:comment_24_2}). 

То есть ранг квадратичной формы не зависит от выбора базиса.



\newpage


% Закон инерции квадратичных форм
\subsection{
    Закон инерции квадратичных форм (с доказательством).
}

\begin{theorem} Пусть
    \begin{enumerate}
        \item $\mathcal{V}$ - $n$-мерное линейное пространство.
        \item $\mathscr{B}: \mathcal{V} \times \mathcal{V} \to \RR$ - симметрическая билинейная форма.
        \item $\text{КФ}(\vec{x}) = \mathscr{B}(\vec{x}, \vec{x})$ - соответствующий ей функционал, который может быть записан в виде различных квадратичных форм в зависимости от базиса.
    \end{enumerate}

    Тогда как бы мы ни выбирали канонический базис, количество положительных, количество отрицательных коэффициентов и количество нулевых коэффициентов в каноническом виде КФ будут всегда одними и теми же.
\end{theorem}

\begin{proof}~

    Пусть $\vec{f}_1, \ldots, \vec{f}_n$ и $\vec{g}_1, \ldots, \vec{g}_n$ - два канонических базиса, причем в первом базисе 

    $$\vec{x} = \sum_{k = 1}^{n}x_k\vec{f}_k \text{ и КФ}(\vec{x}) = \sum_{k = 1}^{n}\lambda_kx^2_k.$$

    а во втором базисе

    $$\vec{x} = \sum_{k = 1}^{n}y_k\vec{g}_k \text{ и КФ}(\vec{x}) = \sum_{k = 1}^{n}\mu_ky^2_k.$$

    Будем считать, что:

    \begin{enumerate}
        \item Среди $\lambda_1, \ldots, \lambda_n$ первые $p$ положительные, а остальные $\leq 0$.
        \item Среди $\mu_1, \ldots, \mu_n$ первые $s$ положительные, а остальные $\leq 0$.
    \end{enumerate}

    Отделим положительные слагаемые

    $$\sum_{k = 1}^{p}\underbrace{\lambda_k}_{> 0}x^2_k + \sum_{k = p + 1}^{n}\underbrace{\lambda_k}_{\leq 0}x^2_k = \sum_{k = 1}^{s}\underbrace{\mu_k}_{> 0}y^2_k + \sum_{k = s + 1}^{n}\underbrace{\mu_k}_{\leq 0}y^2_k.$$

    Заметим, что любая из этих сумм может оказаться пустой.

    Перенесем

    \begin{equation}
        \sum_{k = 1}^{p}\underbrace{\lambda_k}_{> 0}x^2_k + \sum_{k = s + 1}^{n}\underbrace{(-\mu_k)}_{\geq 0}y^2_k = \sum_{k = 1}^{s}\underbrace{\mu_k}_{> 0}y^2_k + \sum_{k = p + 1}^{n}\underbrace{(-\lambda_k)}_{\geq 0}x^2_k.
        \label{equation:equation_36_2_1}
    \end{equation}

    Предположим, что $p \ne s$. Без ограничения общности будет считать, что $p < s$ (случай $p > s$ рассматривается аналогично).

    Вопрос: можно ли для заданных базисов $f$ и $g$ найти такой ненулевой вектор $\vec{x}$, чтобы 
    $$x_1 = x_2 = \ldots = x_p = 0, \text{ и } \vec{y}_{s + 1} = \ldots = \vec{y}_n = \vec{0}?$$

    Мы знаем, что 

    $$\vec{x}_f = T_{f \to g}\vec{x}_g.$$

    Наш вопрос переформулируется так:

    Можно ли для заданных $f$ и $g$ найти такой ненулевой $\vec{x}$, чтобы

    \begin{equation}
        \begin{pmatrix}
            0 \\
            0 \\
            \vdots \\
            0 \\
            x_{p + 1} \\
            x_{p + 1}
            \\
            \vdots \\
            x_n
        \end{pmatrix} = T_{f \to g}\begin{pmatrix}
            y_1 \\
            y_2 \\
            \vdots \\
            y_s \\
            0 \\
            0
            \\
            \vdots \\
            0
        \end{pmatrix}.
        \label{equation:equation_36_2_2}
    \end{equation}
    Если справа домножить матрицу на вектор, записать равенство векторов как $n$ уравнений и в них перенести все слагаемые в одну часть, то получим однородную систему из $n$ уравнений с $(n - p) + s$ неизвестными.

    Т.к. $p < s$, то количество уравнений $(n)$ меньше числа неизвестных $(n + (s - p))$.

    По теореме из 1-го семестра такая однородная система имеет нетривиальное решение, т.е.

    $$\exists \underbrace{\thinspace x_{p + 1}, \ldots, x_n, y_1, \ldots, y_s}_{\text{не все равны }0 \thinspace (\star\star\star).},$$
    удовлетворяющее системе \eqref{equation:equation_36_2_2}.

    Если бы все $y_1, y_2, \ldots, y_s$ равнялись $0$, то из \eqref{equation:equation_36_2_2} следовало бы, что $x_{p + 1} = \ldots = x_n = 0$. Тогда бы мы получили противоречие с $(\star\star\star)$. 
    
    Следовательно, среди $y_1, y_2, \ldots, y_s$ обязательно есть хотя бы одно ненулевое.

    Получается, что в \eqref{equation:equation_36_2_1} левая часть состоит сплошь из $0$, а в правой части стоит отрицательное число. Противоречие.

    Значит, $p = s$. Т.е. количество положительных коэффициентов в КФ не зависит от базиса.

    Аналогично, умножив $\mathscr{B}(\vec{x}, \vec{x})$ на $(-1)$, можно доказать, что количество отрицательных коэффициентов в КФ не зависит от базиса.
\end{proof}


\newpage
\section{
    Знакоопределенность квадратичной формы. Критерий Сильвестра (без доказательства). Примеры.
}

% Знакоопределенность квадратичной формы
\subsection{
    Знакоопределенность квадратичной формы.
}

\begin{definition}
    Квадратичная форма $f(x) = x^TAx$, $x = \begin{pmatrix}
        x_1 \\
        \vdots \\
        x_n
    \end{pmatrix}$, называется:

    \begin{itemize}
        \item \textbf{\textit{положительно (отрицательно) определенной}}, если для любого ненулевого столбца $x$ выполняется неравенство $f(x) > 0 \thinspace (f(x) < 0)$;
        \item \textbf{\textit{неотрицательно (неположительно) определенной}}, если $f(x) \geq 0 \thinspace (f(x) \leq 0)$ для любого столбца $x$, причем существует ненулевой столбец $x$, для которого $f(x) = 0$;
        \item \textbf{\textit{знакопеременной (неопределенной)}}, если существуют такие столбцы $x$ и $y$, что $f(x) > 0$ и $f(y) < 0$;
    \end{itemize}
\end{definition}

% Критерий Сильвестра (без доказательства)
\subsection{
    Критерий Сильвестра (без доказательства).
}

\begin{enumerate}
    \item Для того чтобы квадратичная форма от $n$ переменных была положительно определена, необходимо и достаточно, чтобы выполнялись неравенства $\Delta_1 > 0, \Delta_2 > 0, \Delta_3 > 0, \ldots, \Delta_n > 0$.
    \item Для того чтобы квадратичная форма от $n$ переменных была отрицательно определена, необходимо и достаточно, чтобы выполнялись неравенства $-\Delta_1 > 0, \Delta_2 > 0, -\Delta_3 > 0, \ldots, (-1)^n\Delta_n > 0$ (знаки угловых миноров чередуются, начиная с минуса).
    \item Невырожденная квадратичная форма знакопеременная тогда и только тогда, когда для матрицы квадратичной формы выполнено хотя бы одно из условий:
    \begin{itemize}
        \item один из угловых миноров равен нулю;
        \item один из угловых миноров четного порядка отрицателен;
        \item два угловых минора нечетного порядка имеют разные знаки.
    \end{itemize}
\end{enumerate}

\begin{example}~

    $f(x, y) = 2xy$: $\left(\begin{array}{cc}
        0 & 1 \\
        1 & 0
    \end{array}\right)$ - знакопеременная.

    $f(x, y) = 2x^2 + 2xy + y^2$: $\left(\begin{array}{cc}
        2 & 1 \\
        1 & 1
    \end{array}\right)$ - положительная.

    $f(x, y) = -x^2 - 2xy$: $\left(\begin{array}{cc}
        -1 & -1 \\
        -1 & 0
    \end{array}\right)$ - отрицательная.
\end{example}


\newpage
\section{
    Приведение квадратичной формы к каноническому виду ортогональным преобразованием (обосновать возможность такого приведения).
}

Сначала сформулируем и докажем теоремы Фредгольма для операторов в линейных пространствах:

\subsection{
    Альтернатива Фредгольма.
}

\begin{theorem}["Альтернатива Фредгольма"] Пусть

    \begin{enumerate}
        \item $\left.\begin{array}{l}
            \mathcal{V} \text{ - линейное пространство, } \dim \mathcal{V} = n \\
            \mathcal{W} \text{ - линейное пространство, } \dim \mathcal{W} = m
        \end{array}\right\}$ \text{В обоих задано скалярное произведение.}
        \item $\mathscr{A} \colon \mathcal{V} \to \mathcal{W}$ - линейный оператор.
    \end{enumerate}

    Тогда справедливо ровно одно из двух:

    \begin{itemize}[nosep]
        \item либо уравнение $\mathscr{A}\vec{v} = \vec{w}$ имеет решение при любом $\vec{w} \in \mathcal{W}$,
        \item либо уравнение $\mathscr{A^*}\vec{w} = \vec{0}$ имеет нетривиальное (ненулевое) решение.
    \end{itemize}
\end{theorem}

\begin{proof}~

    Обозначим $r = \rank(\mathscr{A})$.

    \textbf{1 случай:}
    
    \begin{gather*}
        r = m \\
        \downimplies \\
        \dim(\Im \mathscr{A}) = m \\
        \downimplies \\
        \text{Т.к. } \Im \mathscr{A} \text{— } m\text{-мерное подпространство } m\text{-мерного пространства } \mathcal{W}, \text{то } \Im \mathscr{A} = \mathcal{W}. \\
        \downimplies \\
        \forall \vec{w} \in \mathcal{W} \exists \vec{v} \in \mathcal{V} \colon \mathscr{A}\vec{v} = \vec{w}. \\
        \text{Также заметим, что }\dim(\Im \mathscr{A^*}) + \dim(\ker \mathscr{A^*}) = m \\
        \downimplies \\
        \dim(\Im \mathscr{A}) + \dim(\ker \mathscr{A^*}) = m \\
        \downimplies \\
        m + \dim(\ker \mathscr{A^*}) = m \\
        \downimplies \\
        \dim(\ker \mathscr{A^*}) = 0 \\
        \downimplies \\
        \ker \mathscr{A^*} = \{\vec{0}\} \\
        \downimplies \\
        \text{Уравнение } \mathscr{A^*}\vec{w} = \vec{0} \text{ имеет только тривиальное решение.}
    \end{gather*}

    Получается, что второе "либо" не выполнено!
    
    \textbf{2 случай:}

    \begin{gather*}
        r < m \\
        \downimplies \\
        \dim(\Im \mathscr{A}) < \dim \mathcal{W} \\
        \downimplies \\
        \dim(\Im \mathscr{A^*}) < \dim \mathcal{W} \\
        \downimplies \\
        \text{Т.к. } \dim(\Im \mathscr{A^*}) + \dim(\ker \mathscr{A^*}) = \dim \mathcal{W}, \ker \mathscr{A^*} \ne \{\vec{0}\} \\
        \downimplies \\
        \exists \vec{w} \ne 0 \in \ker \mathscr{A^*} \\
        \downimplies \\
        \exists \vec{w} \ne 0 \colon \mathscr{A^*}\vec{w} = \vec{0}
    \end{gather*}
\end{proof}


\newpage
\section{
    Метод Лагранжа для приведения квадратичной формы к диагональному виду. Описать алгоритм для разных случаев и обосновать возможность применения.
}

По определению, квадратичной формой называется сумма вида

$$f(x_1, \ldots, x_n) = \sum_{i=1}^n b_{ii}x_i^2 + \sum_{1 \leq i < j \leq n} 2b_{ij}x_ix_j,$$

где $b_{ij}$ - заданные числа, $1 \leq i < j < n$.

Таким образом, у нас есть упорядоченный набор переменных $x_1, x_2, \ldots, x_n$, причем каждая переменная одного из трех типов:

\begin{itemize}
    \item \textbf{1 типа:} есть ненулевое слагаемое с квадратом это переменной и хотя бы одно слагаемое с первой степенью этой переменной;
    \item \textbf{2 типа:} нет слагаемого с квадратом этой переменной, но есть хотя бы одно ненулевое слагаемое с первой степенью этой переменной;
    \item \textbf{3 типа:} есть ненулевое слагаемое с квадратом этой переменной, но нет слагаемых с первой степенью этой переменной.
\end{itemize}

Суммарное количество переменных 1-го и 2-го типа назовем \textit{дефектом КФ}.

Например, 

\begin{enumerate}
    \item $f(x, y, z) = 2x^2 - y^2 + xz$.

    \begin{itemize}[nosep]
        \item $x$ — 1-го типа.
        \item $z$ — 2-го типа.
        \item $y$ — 3-го типа.
    \end{itemize}
    Дефект КФ равен 2.

    \item $f(x, y, z, t) = 4y^2-z^2-xy+3xt+2yt$.

    \begin{itemize}[nosep]
        \item $y$ — 1-го типа.
        \item $x, t$ — 2-го типа.
        \item $z$ — 3-го типа.
    \end{itemize}
    Дефект КФ равен 3.
\end{enumerate}

Докажем, что если дефект КФ больше нуля (т.е. в КФ есть хотя бы одна переменная 1-го или 2-го типа), то его можно понизить, сделав подходящую линейную замену.

\textbf{Случай 1: В КФ есть переменные 1-го типа.}

Соберем все слагаемые с этой переменной в скобку и выделим полный квадрат:

\begin{align}
    \text{КФ } &= (\underbrace{a}_{\ne 0}x^2 + \underbrace{2b_1xy_1 + 2b_2xy_2 + \ldots + 2b_kxy_k}_{\text{все эти слагаемые (их }\geq \text{ 1) ненулевые.}}) + \underbrace{\ldots}_{\star} = \\
    &= a(x^2 + 2\cdot x\cdot\frac{b_1}{a}y_1 + 2\cdot x\cdot\frac{b_2}{a}y_2 + \ldots + 2\cdot x\cdot\frac{b_k}{a}y_k) + \underbrace{\ldots}_{\star} = \\
    &= a\left((x + \frac{b_1}{a}y_1 + \ldots + \frac{b_k}{a}y_k)^2 - \sum_{i = 1}^k(\frac{b_i}{a}y_i)^2 - \sum_{1 \leq i < j \leq k}2\cdot\frac{b_i}{a}y_i\cdot \frac{b_j}{a}y_j\right) + \underbrace{\ldots}_{\star} = \\
    &= a(x + \frac{b_1}{a}y_1 + \ldots + \frac{b_k}{a}y_k)^2 + \sum_{i = 1}^k\underbrace{\left(-\frac{b^2_i}{a}\right)}_{\lambda_i \ne 0}y^2_i + \sum_{1 \leq i < j \leq k}\underbrace{\left(-\frac{2b_ib_j}{a}\right)}_{\mu_{ij} \ne 0}y_iy_j + \underbrace{\ldots}_{\star} = \\
    &=\left\{ 
            \begin{array}{l}
                \text{Замена} \\
                x' = x + \frac{b_1}{a}y_1 + \ldots + \frac{b_k}{a}y_k
            \end{array} \right\} = \\
    &= a(x')^2 + \underbrace{\sum_{i = 1}^k\lambda_iy^2_i}_{\star\star} + \underbrace{\sum_{1 \leq i < j \leq k}\mu_{ij}y_iy_j}_{\star\star\star} + \underbrace{\ldots}_{\text{нет }x'}.
\end{align}

\newpage

"$\star$" - В этом месте:

\begin{enumerate}
    \item нет $x$.
    \item могут быть $Ay_iy_j, 1\leq i < j \leq k$.
    \item могут быть $Ay^2_i, 1 \leq i \leq k$.
    \item могут быть слагаемые $Az_pz_q, 1 \leq p \leq q \leq m$ целиком из незадействованных в скобке переменных $z_1, z_2, \ldots, z_m$.
    \item могут быть $Ay_iz_j, 1\leq i\leq k, 1\leq j\leq m$.
\end{enumerate}

В новой квадратичной форме $x'$ - переменная 3-го типа.

\textbf{А что произошло с другими переменными - т.е. с $y_1, y_2, \ldots, y_k$ и "незадействованными" переменными $z_1, z_2, \ldots, z_m$?}

\begin{enumerate}
    \item Если в исходной КФ $y_1$ была переменной 1-го типа, то она станет либо 1-го, либо 2-го, либо 3-го типа (в зависимости от результата сложения $\star\star$ и $\star\star\star$);
    \item Если в исходной КФ $y_1$ была переменной 2-го типа, то она станет переменной либо 1-го, либо 3-го типа;
    \item $y_1$ не могла быть переменной 3-го типа;
    \item Если $z_1$ была переменной 1-го типа, то она ею и останется;
    \item Если $z_1$ была переменной 2-го типа, то она ею и останется;
    \item Если $z_1$ была переменной 3-го типа, то она ею и останется;
\end{enumerate}

Получается, что наша процедура как минимум на 1 увеличило число переменных 3-го типа $\Rightarrow$ дефект КФ уменьшился на 1.

\bigbreak

\textbf{Случай 2: В КФ нет переменных 1-го типа.} 

\bigbreak

\textbf{Подслучай 2а: В КФ есть переменная 2-го типа.}

В этом случае КФ содержит только слагаемые с разноименными переменными. Так как КФ $\ne 0$, то в ней есть хотя бы одно такое слагаемое; пусть, это в примеру, $ax_1x_2 (a \ne 0)$. Тогда КФ будет иметь вид:

\begin{align*}
    \text{КФ} &= ax_1x_2 + \\
    &+ b_3x_1x_3 + b_4x_1x_4 + \ldots + b_nx_1x_n + \\
    &+ c_3x_2x_3 + c_4x_2x_4 + \ldots + c_nx_2x_n + \\
    &+ \underbrace{\ldots}_{\text{Здесь нет }x_1, x_2\text{ но, возм., есть какие-то другие переменные.}} = \\
    &= \left\{\begin{array}{c}
        x_1 = v_1 + v_2 \\
        x_2 = v_1 - v_2 
    \end{array}\right\} = \\
    &= av^2_1 + (-a)v^2_2 + \\
    &+ b_3(v_1 + v_2)x_3 + \ldots + b_n(v_1 + v_2)x_n + \\
    &+ c_3(v_1 - v_2)x_3 + \ldots + c_n(v_1 - v_2)x_n + \\
    & + \underbrace{\ldots}_{\text{ничего не изменится.}} = \\
    &= av^2_1 + (-a)v^2_2 + \\
    &+ (b_3 - c_3)v_1x_3 + (b_4 - c_4)z_1x_4 + \ldots + (b_n - c_n)z_1x_n + \\ &+ \underbrace{\ldots}_{\text{ничего не изменится.}}
\end{align*}

где $b_i, c_j$ - любые (в том числе и нулевые)

Видим, что количество переменных 3-го типа выросло $\Rightarrow$ дефект КФ уменьшился.

\bigbreak

\textbf{Подслучай 2б: В КФ нет переменных 2-го типа.}

Этот случай невозможен, т.к. по условию дефект КФ положителен.
\bigbreak

Итак, мы предположили процедуру, благодаря которой в любой КФ с положительным дефектом можно уменьшить этот дефект. Последовательно применяя эту процедуру, можно получить КФ с нулевым дефектом, т.е. КФ канонического вида.


\newpage
\section{
    Теоремы Фредгольма для систем линейных уравнений*.
}

% Альтернатива Фредгольма
Сначала сформулируем и докажем теоремы Фредгольма для операторов в линейных пространствах:

\subsection{
    Альтернатива Фредгольма.
}

\begin{theorem}["Альтернатива Фредгольма"] Пусть

    \begin{enumerate}
        \item $\left.\begin{array}{l}
            \mathcal{V} \text{ - линейное пространство, } \dim \mathcal{V} = n \\
            \mathcal{W} \text{ - линейное пространство, } \dim \mathcal{W} = m
        \end{array}\right\}$ \text{В обоих задано скалярное произведение.}
        \item $\mathscr{A} \colon \mathcal{V} \to \mathcal{W}$ - линейный оператор.
    \end{enumerate}

    Тогда справедливо ровно одно из двух:

    \begin{itemize}[nosep]
        \item либо уравнение $\mathscr{A}\vec{v} = \vec{w}$ имеет решение при любом $\vec{w} \in \mathcal{W}$,
        \item либо уравнение $\mathscr{A^*}\vec{w} = \vec{0}$ имеет нетривиальное (ненулевое) решение.
    \end{itemize}
\end{theorem}

\begin{proof}~

    Обозначим $r = \rank(\mathscr{A})$.

    \textbf{1 случай:}
    
    \begin{gather*}
        r = m \\
        \downimplies \\
        \dim(\Im \mathscr{A}) = m \\
        \downimplies \\
        \text{Т.к. } \Im \mathscr{A} \text{— } m\text{-мерное подпространство } m\text{-мерного пространства } \mathcal{W}, \text{то } \Im \mathscr{A} = \mathcal{W}. \\
        \downimplies \\
        \forall \vec{w} \in \mathcal{W} \exists \vec{v} \in \mathcal{V} \colon \mathscr{A}\vec{v} = \vec{w}. \\
        \text{Также заметим, что }\dim(\Im \mathscr{A^*}) + \dim(\ker \mathscr{A^*}) = m \\
        \downimplies \\
        \dim(\Im \mathscr{A}) + \dim(\ker \mathscr{A^*}) = m \\
        \downimplies \\
        m + \dim(\ker \mathscr{A^*}) = m \\
        \downimplies \\
        \dim(\ker \mathscr{A^*}) = 0 \\
        \downimplies \\
        \ker \mathscr{A^*} = \{\vec{0}\} \\
        \downimplies \\
        \text{Уравнение } \mathscr{A^*}\vec{w} = \vec{0} \text{ имеет только тривиальное решение.}
    \end{gather*}

    Получается, что второе "либо" не выполнено!
    
    \textbf{2 случай:}

    \begin{gather*}
        r < m \\
        \downimplies \\
        \dim(\Im \mathscr{A}) < \dim \mathcal{W} \\
        \downimplies \\
        \dim(\Im \mathscr{A^*}) < \dim \mathcal{W} \\
        \downimplies \\
        \text{Т.к. } \dim(\Im \mathscr{A^*}) + \dim(\ker \mathscr{A^*}) = \dim \mathcal{W}, \ker \mathscr{A^*} \ne \{\vec{0}\} \\
        \downimplies \\
        \exists \vec{w} \ne 0 \in \ker \mathscr{A^*} \\
        \downimplies \\
        \exists \vec{w} \ne 0 \colon \mathscr{A^*}\vec{w} = \vec{0}
    \end{gather*}
\end{proof}

% Теорема Фредгольма
\subsection{
    Теорема Фредгольма.
}


\begin{theorem}[Фредгольма] Пусть
    \begin{enumerate}
        \item $\left.\begin{array}{l}
                \mathcal{V} \text{ - линейное пространство, } \dim \mathcal{V} = n \\
                \mathcal{W} \text{ - линейное пространство, } \dim \mathcal{W} = m
            \end{array}\right\}$ \text{оба конечномерны, в обоих задано скалярное произведение.}   
        \item $\mathscr{A} \colon \mathcal{V} \to \mathcal{W}$ - линейный оператор.
        \item $\vec{w} \in \mathcal{W}$ - фиксированный элемент, причем $\vec{w} \ne \vec{0}$.
    \end{enumerate}

    Тогда уравнение $\mathscr{A}\vec{v} = \vec{w}$ имеет решение $\iff$ вектор $\vec{w}$ ортогонален всем решением уравнения $\mathscr{A^*}\vec{u} = \vec{0}$.
\end{theorem}

\begin{proof}~

    \begin{description}
        \item[$(\implies)$]~
        
        Пусть $\mathscr{A}\vec{v} = \vec{w}$ имеет решение, т.е. $\exists \vec{v}_0 \in \mathcal{V} \colon \mathscr{A}\vec{v}_0 = \vec{w}$.

            Рассмотрим произвольное решение $\vec{u}_0$ уравнения $\mathscr{A^*}\vec{u} = \vec{0}$ (т.е. $\mathscr{A^*}\vec{u}_0 = \vec{0}$).

            Получим:

            $$(\vec{w}, \vec{u}_0) = (\mathscr{A}\vec{v}_0, \vec{u}_0) = (\vec{v}_0, \mathscr{A^*}\vec{u}_0) = (\vec{v}_0, \vec{0}) = 0.$$

            Значит, $\vec{w} \perp \vec{u}_0$, ч.т.д.
        \item[$(\impliedby)$]~
        
        \textbf{($\star$)} Пусть $\forall \vec{u} \in \mathcal{W} \left(\mathscr{A^*}\vec{u} = \vec{0} \Rightarrow (\vec{u}, \vec{w}) = 0\right)$.

        Докажем, что $\mathscr{A}\vec{v} = \vec{w}$ имеет хотя бы одно решение.

        \textbf{($\star\star\star$)} Предположим, что $\mathscr{A}\vec{v} = \underbrace{\vec{w}}_{\mathclap{\substack{\text{фикс. в условии} \\ \text{теоремы} \\ \text{ненулевой} \\ \text{элемент}.}}}$ несовместна.

        Так как $\mathcal{W}$ - конечномерное (пусть $m$-мерное), то в нем найдется некоторый базис $\vec{e}_1, \ldots, \vec{e}_m$. Так как в $\mathcal{W}$ задано скалярное произведение, то базис $e$ можно ортогонализовать, а потом нормировать, т.е. в $\mathcal{W}$ найдется ОНБ $\vec{f}_1, \vec{f}_2, \ldots, \vec{f}_m$.

        Рассмотрим какой-нибудь $\vec{u} \in \mathcal{W}$ такой, что $\mathscr{A}\vec{u} = \vec{0}$.

         По условию \textbf{$(\star)$} из этого будет следовать, что $(\vec{u}, \underbrace{\vec{w}}_{\mathclap{\substack{\text{фикс. в условии} \\ \text{теоремы} \\ \text{ненулевой} \\ \text{элемент}.}}}) = 0$.

         Пусть в базисе $\vec{f}_1, \ldots, \vec{f}_m$ вектор $\vec{u}$, вектор $\vec{w}$ и оператор $\mathscr{A^*}$ имеют координаты $\vec{u}_f = \begin{pmatrix}
             u_1 \\
             u_2 \\
             \vdots \\
             u_m
         \end{pmatrix}$, $\vec{w}_f = \begin{pmatrix}
             w_1 \\
             w_2 \\
             \vdots \\
             w_m
         \end{pmatrix}$ и матрицу $A^*_f = \begin{pmatrix}
             A_1 & A_2 & \ldots & A_m
         \end{pmatrix} = \begin{pmatrix}
             a_{11} & a_{12} & \ldots & a_{1m} \\
             a_{21} & a_{22} & \cdots & a_{2m} \\
             \vdots & \vdots & \ddots & \vdots \\
             a_{n1} & a_{n2} & \ldots & a_{nm}
         \end{pmatrix}$.

         Так как $\mathscr{A^*}\vec{u} = \vec{0}$, то $A^*_f\vec{u}_f = \begin{pmatrix}
             0 \\
             0 \\
             \vdots \\
             0
         \end{pmatrix} \in \RR^n$, т.е.

         $$\begin{pmatrix}
             a_{11} & a_{12} & \ldots & a_{1m} \\
             a_{21} & a_{22} & \cdots & a_{2m} \\
             \vdots & \vdots & \ddots & \vdots \\
             a_{n1} & a_{n2} & \ldots & a_{nm}
         \end{pmatrix}\begin{pmatrix}
             u_1 \\
             u_2 \\
             \vdots \\
             u_m
         \end{pmatrix} = \begin{pmatrix}
             0 \\
             0 \\
             \vdots \\
             0
         \end{pmatrix}.$$

        $$\textbf{(}\star\star\textbf{)} \thinspace u_1 \cdot A^T_1 + u_2 \cdot A^T_2 + \ldots + u_m \cdot A^T_m = \underbrace{\begin{pmatrix}
             0 & \ldots & 0
         \end{pmatrix}}_{n}.$$

         Так как $(\vec{u}, \vec{w}) = 0$, то в силу ортонормированности $\vec{f}_1, \vec{f}_2, \ldots, \vec{f}_m$ имеем
         $$u_1w_1 + u_2w_2 + \ldots + u_mw_m = 0$$

         Значит, соотношение \textbf{$(\star \star)$} можно дополнить:

         $$u_1\cdot(A^T_1, \vec{w}_1) + u_2\cdot(A^T_2, \vec{w}_2) + \ldots + u_m\cdot(A^T_m, \vec{w}_m) = \underbrace{\begin{pmatrix}
             0 & 0 & \ldots & 0
         \end{pmatrix}}_{n + 1}.$$

         Это можно переписать иначе

         $$\begin{pmatrix}
                u_1 & u_2 & \ldots & u_m
            \end{pmatrix}
            \cdot
            \left(
                \begin{array}{c|c}
                    A_1^T & w_1 \\
                    A_2^T & w_2 \\
                    \vdots & \vdots \\
                    A_m^T & w_m
                \end{array}
            \right)
            =
            \underbrace{
                \begin{pmatrix}
                    0 & 0 & \ldots & 0
                \end{pmatrix}
            }_{n + 1}.$$
            Вспомним, что $\mathscr{A}\vec{v} = \vec{w}$ несовместно (предположение \textbf{$(\star\star\star)$} метода от противного). По т. Кронекера-Капелли 

            $$\rank\begin{pmatrix}
                    A^T_1 \\ 
                    A^T_2 \\ 
                    \vdots \\
                    A^T_m
                \end{pmatrix} \ne \rank \left(
                \begin{array}{c|c}
                    A_1^T & w_1 \\
                    A_2^T & w_2 \\
                    \vdots & \vdots \\
                    A_m^T & w_m
                \end{array}
            \right).$$

            Это означает, что если приводить обе матрицы одинаковыми преобразованиями над строками к ступенчатому виду, то количество "ступенек" у левой будет на 1 меньше, чем количество ступенек у правой. Это означает, что в ступенчатом виде правой матрицы будет строка

            $$\underbrace{\left(
                \begin{array}{cccc|c}
                    0 & 0 & \ldots & 0 & 1
                \end{array}
            \right)}_{n \text{ нулей и одна единица.}}.$$

            Мы знаем, что если над строками матрицы были совершены элементарные преобразования, то любую строку полученной матрицы можно представить в виде линейной комбинации строк исходной матрицы. Следовательно, найдутся такие $\lambda_1, \dots, \lambda_m$, что 

            \begin{equation}
                \lambda_1\cdot(A^T_1, \vec{w}_1) + \lambda_2\cdot(A^T_2, \vec{w}_2) + \ldots + \lambda_m\cdot(A^T_m, \vec{w}_m) = \begin{pmatrix}
                     0 & 0 & \ldots & 0 & 1
                 \end{pmatrix}.
                 \label{eq:equation_40_1}
            \end{equation}

            Откуда, беря только последний элемент строк, получим

            \begin{equation}
                \lambda_1\vec{w}_1 + \lambda_2\vec{w}_2 + \ldots + \lambda_m\vec{w}_m = 1.
                \label{eq:equation_40_2}
            \end{equation}
            Возьмем $\vec{z}_f = \begin{pmatrix}
                     \lambda_1 \\
                     \lambda_2 \\
                     \vdots \\
                     \lambda_m
                 \end{pmatrix}.$
            С одной стороны, из \eqref{eq:equation_40_1} следует, что $A^*_f\vec{z}_f = \left. \begin{pmatrix}
                 0 \\
                 0 \\
                 \vdots \\
                 0
             \end{pmatrix} \right\} n$, т.е. $A^*\vec{z} = \vec{0}$. А значит, согласно условию $(\vec{z}, \vec{w}) = 0$. С другой стороны, из \eqref{eq:equation_40_2} следует, что $(\vec{z}, \vec{w}) = 1$. 
             
             Противоречие.
    \end{description}
\end{proof}



\newpage


% Теоремы о СЛАУ, которые являются следствиями из теорем Фредгольма
\subsection{
    Теоремы о СЛАУ, которые являются следствиями из теорем Фредгольма.
}

\begin{theorem} Пусть 
    \begin{enumerate}
        \item $A \in \RR^{n \times m}$ - некоторая матрица;
        \item $A^* \in \RR^{m \times n}$ - сопряженная к ней.
    \end{enumerate}

    Тогда справедливо ровно одно из следующих двух утверждений:

    \begin{itemize}
        \item либо СЛАУ $A\vec{x} = \vec{b}$ имеет решение при любом $\vec{b} \in \RR^n$, 
        \item ОСЛАУ $A^*\vec{y} = \vec{0}$ имеет нетривиальное решение $\vec{y} = \vec{y}_0 \in \RR^m$.
    \end{itemize}
\end{theorem}

\begin{theorem} Пусть
    \begin{enumerate}
        \item $A \in \RR^{n \times m}$ - некоторая матрица;
        \item $A^* \in \RR^{m \times n}$ - сопряженная к ней;
        \item $\vec{b} \in \RR^{m}$ - заданный вектор-столбец.
    \end{enumerate}

    Тогда СЛАУ $A\vec{x} = \vec{b}$ совместна $\iff$ вектор $\vec{b}$ ортогонален всем решениям ОСЛАУ $A^*\vec{y} = \vec{0}$.
\end{theorem}




\end{document}
