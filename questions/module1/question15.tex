\section{
    Псевдообратная матрица. Алгоритм нахождения.
}

\begin{definition}
    Для любой матрицы $A$ существует такая матрица $A^+$, что нормальное псевдорешение СЛАУ $A\vec{x} = \vec{b}$ с произвольным вектор-столбцом $\vec{b}$ имеет вид $\vec{x} = A^+\vec{b}$. Матрицу $A^+$ называют \textit{\textbf{псевдообратной}} по отношению к матрице $A$.
\end{definition}

\subsection{
    Алгоритм нахождения, опираясь на нормальное псевдорешение.
}

Рассмотрим СЛАУ $A\vec{x} = \vec{e}_i$, в правой части которой записан $i$-й вектор стандартного базиса в пространстве $\RR^n$. Ее нормальное псевдорешение $A^+\vec{e}_i$ - это $i$-й столбец псевдообратной матрицы. Вычислив все $n$ столбцов, получим $A^+$. Таким образом, $i$-й столбец псевдообратной матрицы $a_i$ можно найти из системы

\begin{equation*}
    \left(\begin{array}{c}
        A^TA \\
        F^T
    \end{array}\right)a_i
    =
    \left(\begin{array}{c}
        A^T\vec{e}_i \\
        \vec{0}
    \end{array}\right)
,\end{equation*}

где $F$ - матрица, составленная из столбцов ФСР СЛАУ $A\vec{x} = \vec{0}$. Объединив по правилам действий с блочными матрицами $n$ систем в одно матричное уравнение, получим

\begin{equation*}
    \left(\begin{array}{c}
        A^TA \\
        F^T
    \end{array}\right)A^+
    =
    \left(\begin{array}{c}
        A^T \\
        \vec{0}
    \end{array}\right)
.\end{equation*}

\textbf{Итак, алгоритм состоит в следующем:}

\begin{enumerate}
    \item Найти $F$ - матрицу, составленную из столбцов ФСР СЛАУ $A\vec{x} = \vec{0}$.
    \item Транспонируем матрицу $F$, чтобы в дальнейшем добавить $F^T$ как строки новой матрицы.
    \item Найдем $A^T$.
    \item Найдем $A^TA$.
    \item Составим матричное уравнение ниже и решим его методом элементарных преобразований.

    \begin{equation*}
        \left(\begin{array}{c}
            A^TA \\
            F^T
        \end{array}\right)A^+
        =
        \left(\begin{array}{c}
            A^T \\
            \vec{0}
        \end{array}\right)
    .\end{equation*}
    \begin{equation*}
        \left(\begin{array}{c|c}
            A^TA & A^T \\
            F^T & \vec{0}
        \end{array}\right)
        \sim \ldots \sim
        \left(\begin{array}{c|c}
            E & A^+
        \end{array}\right)
    .\end{equation*}
\end{enumerate}


\newpage

\subsection{
    *Алгоритм, основанный на понятии скелетного разложения.
}

\textit{Скелетное разложение} заключается в представлении матрицы $A$ типа $m \times n$ ранга $k$ в виде $A = UV$, где матрица $U$ имеет тип $m \times k$ и ранг $k$ (матрица максимального столбцового ранга), а матрица $V$ - тип $k \times n$ и ранг $k$ (матрица максимального строчного ранга).

\textbf{Итак, алгоритм состоит в следующем:}


\begin{enumerate}
    \item Матрицу $A$ с помощью элементарных преобразований приводим к ступенчатому виду. Выбрав базисный минор, элементарыми преобразованиями строк приводим его к единичному виду. Отбросив в полученной матрице нулевые строки, получим матрицу $V$. Матрицу $U$ составляют столбцы матрицы $A$, которые стоят на тех же местах, что и базисные столбцы матрицы ступенчатого вида.
    \item Если $U$ - матрица максимального столбцового ранга, то $U^+ = (U^TU)^{-1}U^T$.
    \item Если $V$ - матрица максимального строчного ранга, то $V^+ = V^T(VV^T)^{-1}$.
    \item Если $A$ - ненулевая матрица, а $UV$ - ее скелетное разложение, то
    $$A^+ = V^+U^+ = V^T(VV^T)^{-1}(U^TU)^{-1}U^T.$$
\end{enumerate}

\begin{example}~
    
    A = $\left(\begin{array}{ccc}
        2 & 2 & -2 \\
        -2 & -2 & 2 \\
        -1 & -2 & -1
    \end{array}\right)$

    1) $V_{k \times n}$ - матрица max строчного ранга.
    
    \begin{equation*}
        \left(\begin{array}{ccc}
            2 & 2 & -2 \\
            -2 & -2 & 2 \\
            -1 & -2 & -1
        \end{array}\right)
        \sim 
        \left(\begin{array}{ccc}
            1 & 2 & 1 \\
            1 & 1 & -1 \\
            0 & 0 & 0
        \end{array}\right)
        \sim 
        \left(\begin{array}{ccc}
            1 & 2 & 1 \\
            0 & 1 & -2
        \end{array}\right)
        \sim 
        \left(\begin{array}{ccc}
            1 & 0 & -3 \\
            0 & 1 & 2
        \end{array}\right) = V
    .\end{equation*}

    \begin{equation*}
        VV^T = \left(\begin{array}{ccc}
            1 & 0 & -3 \\
            0 & 1 & 2
        \end{array}\right)
        \left(\begin{array}{cc}
            1 & 0 \\
            0 & 1 \\
            -3 & 2
        \end{array}\right)
        =
        \left(\begin{array}{cc}
            10 & -6 \\
            -6 & 5
        \end{array}\right)
    .\end{equation*}

    Найдем $(VV^T)^{-1}$ с помощью элементарых преобразований:

    \begin{equation*}
        \left(\begin{array}{cc|cc}
            10 & -6 & 1 & 0 \\
            -6 & 5 & 0 & 1
        \end{array}\right)
        \sim \ldots \sim
        \left(\begin{array}{cc|cc}
            1 & 0 & 5/14 & 6/14 \\
            0 & 1 & 6/14 & 10/14
        \end{array}\right)
    .\end{equation*}

    $$(VV^T)^{-1} = \frac{1}{14}\left(\begin{array}{cc}
        5 & 6 \\
        6 & 10
    \end{array}\right)$$

    $$V^+ = V^T(VV^T)^{-1} = \frac{1}{14}\left(\begin{array}{cc}
        1 & 0 \\
        0 & 1 \\
        -3 & 2
    \end{array}\right)
    \left(\begin{array}{cc}
        5 & 6 \\
        6 & 10
    \end{array}\right) = \frac{1}{14}\left(\begin{array}{cc}
        5 & 6 \\
        6 & 10 \\
        -3 & 2
    \end{array}\right)$$

    2) $U_{m \times k}$ - матрица max столбцового ранга.

    U = $\left(\begin{array}{cc}
        2 & 2 \\
        -2 & -2 \\
        -1 & -2
    \end{array}\right)$

    \begin{equation*}
        U^TU = \left(\begin{array}{ccc}
            2 & -2 & -1 \\
            2 & -2 & -2 \\
        \end{array}\right)
        \left(\begin{array}{cc}
            2 & 2 \\
            -2 & -2 \\
            -1 & -2
        \end{array}\right)
        =
        \left(\begin{array}{cc}
            9 & 10 \\
            10 & 12 \\
        \end{array}\right)
    .\end{equation*}

    Найдем $(U^TU)^{-1}$ с помощью элементарых преобразований:

    \begin{equation*}
        \left(\begin{array}{cc|cc}
            9 & 10 & 1 & 0 \\
            10 & 12 & 0 & 1
        \end{array}\right)
        \sim \ldots \sim
        \left(\begin{array}{cc|cc}
            1 & 0 & 12/8 & -10/8 \\
            0 & 1 & -10/8 & 9/8
        \end{array}\right)
    .\end{equation*}

    $$(U^TU)^{-1} = \frac{1}{8}\left(\begin{array}{cc}
        12 & -10 \\
        -10 & 9
    \end{array}\right)$$

    $$U^+ = (U^TU)^{-1}U^T = \frac{1}{8}\left(\begin{array}{cc}
        12 & -10 \\
        -10 & 9
    \end{array}\right)
    \left(\begin{array}{ccc}
        2 & -2 & -1 \\
        2 & -2 & -2
    \end{array}\right) = \frac{1}{4}\left(\begin{array}{ccc}
        2 & -2 & -4 \\
        -1 & 1 & -4 \\
    \end{array}\right)$$

    $$A^+ = V^+U^+ = \frac{1}{56}\left(\begin{array}{cc}
        5 & 6 \\
        6 & 10 \\
        -3 & 2
    \end{array}\right)
    \left(\begin{array}{ccc}
        2 & -2 & -4 \\
        -1 & 1 & -4 \\
    \end{array}\right)
    = \frac{1}{56}\left(\begin{array}{ccc}
        4 & -4 & -4 \\
        2 & -2 & -16 \\
        -8 & 8 & -20
    \end{array}\right)
    = \frac{1}{28}\left(\begin{array}{ccc}
        2 & -2 & -2 \\
        1 & -1 & -8 \\
        -4 & 4 & -10
    \end{array}\right).$$
\end{example}
