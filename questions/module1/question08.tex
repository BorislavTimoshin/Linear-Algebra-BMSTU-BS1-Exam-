\section{
    Линейное аффинное многообразие. Вектор сдвига. Пересечение линейного аффинного многообразия с подпространством, дополнительным к его направляющему подпространству.
}

\subsection{
    Пересечение линейного аффинного многообразия с подпространством, дополнительным к его направляющему подпространству.
}

\begin{theorem}
    Пусть
    \begin{enumerate}
        \item $\mathcal{V}$ - линейное пространство.
        \item $\mathcal{W}$ - его подпространство.
        \item $\mathcal{W'}$ - прямое дополнение к $\mathcal{W}$.
        \item $\vec{a} \in \mathcal{V}$ - фиксированный вектор.
    \end{enumerate}
    Тогда $(\vec{a} + \mathcal{W}) \cap \mathcal{W}'$ - одноэлементное множество (т.е. множество, состоящее из одного вектора).
\end{theorem}

\begin{proof}~

    Рассмотрим произвольный вектор $\vec{\varphi} \in (\vec{a} + \mathcal{W}) \cap \mathcal{W}'$. 
    
    Тогда $\vec{\varphi} = \vec{a} + \vec{w}$, для некоторого $\vec{w} \in \mathcal{W}$ и $\vec{\varphi} \in \mathcal{W}'$.

    Так как $\mathcal{W}'$ - прямое дополнение к $\mathcal{W}$, то вектор $\vec{\varphi}$ единственным образом раскладывается в сумму векторов из $\mathcal{W}$ и из $\mathcal{W}'$:

    \begin{equation}
        \underbrace{\vec{\varphi}}_{\in \mathcal{W'}} = \underbrace{\vec{0}}_{\in \mathcal{W}} + \underbrace{\vec{\varphi}}_{\in \mathcal{W'}}\text{ - вот оно, это единственное разложение.}
        \label{eq:equation_8_1}
    \end{equation}

    Такое же разложение рассмотрим для $\vec{a}$:

    \begin{equation}
        \vec{a} = \underbrace{\vec{z}}_{\in \mathcal{W}} + \underbrace{\vec{z'}}_{\in \mathcal{W}'}\text{ - оба вектора определены единственным образом}
        \label{eq:equation_8_2}
    \end{equation}
    
    Тогда 

    \begin{equation}
        \vec{\varphi} = \vec{a} + \vec{w} = (\vec{z} + \vec{z'}) + \vec{w} = \underbrace{(\vec{z} + \vec{w})}_{\in \mathcal{W}} + \underbrace{\vec{z'}}_{\in \mathcal{W}'}.
        \label{eq:equation_8_3}
    \end{equation}
    Из \eqref{eq:equation_8_1} и \eqref{eq:equation_8_3} имеем: $\vec{\varphi} = \vec{z'}$, а $\vec{z'}$ определяется единственным образом из \eqref{eq:equation_8_2}.
\end{proof}
