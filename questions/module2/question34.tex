\section{
    Ортогональное преобразование евклидова пространства. Свойства ортогональных преобразований (с доказательством).
}

Пусть $\mathcal{E}$ - евклидово пространство, $\mathscr{A} \colon \mathcal{E} \to \mathcal{E}$ - некоторый линейный оператор.

\begin{definition}
     Говорят, что $\mathscr{A}$ задает \textbf{\textit{ортогональное преобразование (называется ортогональным оператором)}}, если $\forall \vec{x}, \vec{y} \in \mathcal{V} \colon (\mathscr{A}\vec{x}, \mathscr{A}\vec{y}) = (\vec{x}, \vec{y})$.
\end{definition}

\subsection*{Свойства ортогональных преобразований.}

\begin{enumerate}[label={\arabic*°.}]
    \item Сохраняет ортогональность, т.е. $\vec{x} \perp \vec{y} \Rightarrow \mathscr{A}\vec{x} \perp \mathscr{A}\vec{y}$.

    \begin{proof}
        Очевидно.
    \end{proof}
    
    \item Сохраняет норму вектора, т.е. $\norm{\mathscr{A}\vec{x}} = \norm{\vec{x}}$.

    \begin{proof}
        $\norm{\mathscr{A}\vec{x}} = \sqrt{(\mathscr{A}\vec{x}, \mathscr{A}\vec{x})} = \sqrt{(\vec{x}, \vec{x})} = \norm{\vec{x}}$.
    \end{proof}

    \item Сохраняет углы между ненулевыми векторами.

    \begin{proof}
        $\widehat{(\mathscr{A}\vec{x}, \mathscr{A}\vec{y})} = \arccos \frac{(\mathscr{A}\vec{x}, \mathscr{A}\vec{y})}{\norm{\mathscr{A}\vec{x}}\cdot\norm{\mathscr{A}\vec{y}}} = \arccos \frac{(\vec{x}, \vec{y})}{\norm{\vec{x}}\cdot\norm{\vec{y}}} = \widehat{(\vec{x}, \vec{y})}$.
    \end{proof}

    \item Пусть $\mathscr{A} \colon \mathcal{E} \to \mathcal{E}$ - ортогональный оператор, $\vec{e}_1, \ldots, \vec{e}_n$ - ОНБ в $\mathcal{E}$. Тогда $\mathscr{A}\vec{e}_1, \ldots, \mathscr{A}\vec{e}_n$ - ОНБ в $\mathcal{E}$.

    \begin{proof}~
    
        $(\mathscr{A}\vec{e}_i, \mathscr{A}\vec{e}_j) = (\vec{e}_i, \vec{e}_j) = \delta_{ij} \text{ (символ Кронекера)}$.

        $\norm{\mathscr{A}\vec{e}_i} = 1$; $\mathscr{A}\vec{e}_1, \ldots, \mathscr{A}\vec{e}_n$ попарно ортогональны, а значит, ЛНЗ. К тому же, $\dim \mathcal{E} = n$, а $\mathscr{A}\vec{e}_1, \ldots, \mathscr{A}\vec{e}_n$ - тоже $n$. Значит, $\mathscr{A}\vec{e}_1, \ldots, \mathscr{A}\vec{e}_n$ - ОНБ.
    \end{proof}

    \item Пусть 
    
    \begin{itemize}
        \item $\mathcal{E}$ - $n$-мерное евклидово пространство;
        \item $\vec{e}_1, \ldots, \vec{e}_n$ - некоторый ОНБ $\mathcal{E}$;
        \item $\mathscr{A} \colon \mathcal{E} \to \mathcal{E}$ - некоторый линейный оператор.
        \item $\mathscr{A}\vec{e}_1, \ldots, \mathscr{A}\vec{e}_n$ - тоже ОНБ $\mathcal{E}$.
    \end{itemize}

    Тогда $\mathscr{A}$ - ортогональный оператор.

    \begin{proof}~
    
        Требуется доказать, что

        $$\forall \vec{x}, \vec{y} \in \mathcal{E} \colon (\mathscr{A}\vec{x}, \mathscr{A}\vec{y}) = (\vec{x}, \vec{y}).$$

        Возьмем произвольные $\vec{x}, \vec{y} \in \mathcal{E}$.

        Разложим по базису $\vec{e}_1, \ldots, \vec{e}_n$:

        \begin{gather*}
            \vec{x} = x_1\vec{e}_1 + \ldots + x_n\vec{e}_n \\
            \vec{y} = y_1\vec{e}_1 + \ldots + y_n\vec{e}_n
        \end{gather*}

        Найдем:

        \begin{align*}
            (\mathscr{A}\vec{x}, \mathscr{A}\vec{y}) &= (\mathscr{A}(x_1\vec{e}_1 + \ldots + x_n\vec{e}_n), \mathscr{A}(y_1\vec{e}_1 + \ldots + y_n\vec{e}_n)) = \\
            &= (x_1\cdot\mathscr{A}\vec{e}_1 + \ldots + x_n\cdot\mathscr{A}\vec{e}_n, y_1\cdot\mathscr{A}\vec{e}_1 + \ldots + y_n\cdot\mathscr{A}\vec{e}_n) = \\
            &= \left\{ 
            \begin{array}{l}
                \text{Получилось, что} \\
                \text{у вектора } \mathscr{A}\vec{x} \\
                \text{в базисе } \mathscr{A}\vec{e}_1, \ldots, \mathscr{A}\vec{e}_n \\
                \text{координаты } \begin{pmatrix}
                    x_1 \\
                    \vdots \\
                    x_n
                \end{pmatrix}; \\
                \text{у вектора } \vec{y} \text{ аналогично}.
            \end{array} \right\} \underbrace{=}_{\text{Следствие } \ref{corollary:corollary_1}} \\
            &\underbrace{=}_{\text{Следствие } \ref{corollary:corollary_1}} x_1y_1 + \ldots + x_ny_n = (\vec{x}, \vec{y}).
        \end{align*}
    \end{proof}

    \item Пусть 
    
    \begin{itemize}
        \item $\mathcal{E}$ - $n$-мерное евклидово пространство;
        \item $\vec{e}_1, \ldots, \vec{e}_n$ - некоторый ОНБ $\mathcal{E}$;
        \item $\mathscr{A} \colon \mathcal{E} \to \mathcal{E}$ - ортогональный оператор.
    \end{itemize}

    Тогда матрица $A_e$ линейного оператора $\mathscr{A}$ в базисе $\vec{e}_1, \ldots, \vec{e}_n$ - ортогональная.

    \begin{proof}~
    
        \begin{align*}
            (\mathscr{A}\vec{x}, \mathscr{A}\vec{y}) &\underbrace{=}_{\text{Лемма } \ref{lemma:lemma_1}} (\mathscr{A}\vec{x})^T_e\cdot\Gamma_e\cdot(\mathscr{A}\vec{y})_e = \\
            &= (A_e\vec{x}_e)^T\cdot\Gamma_e\cdot(A_e\vec{y}_e) = \\
            &= \vec{x}^T\cdot A^T_e\Gamma_eA_e\vec{y}_e.
        \end{align*}

        При этом

        $$(\vec{x}, \vec{y}) = \vec{x}^T_e\cdot\Gamma_e\cdot\vec{y}_e.$$

        Из этого ясно, что 
        $$A^T_e\Gamma_eA_e = \Gamma_e.$$

        Так как $e$ - ОНБ, то $\Gamma_e = E \in \RR^{n \times n}$.

        И тогда $A^T_e \cdot A_e = E$.
    \end{proof}

    \item Справедлив и обратный факт свойству 6:

    Если линейный оператор $\mathscr{A} \colon \mathcal{E} \to \mathcal{E}$ в ОНБ $\vec{e}_1, \ldots, \vec{e}_n$ имеет ортогональную матрицу, то $\mathscr{A}$ задает ортогональное преобразование.

    \begin{proof}~
    
        Действительно, 

        \begin{align*}
            (\mathscr{A}\vec{x}, \mathscr{A}\vec{y}) &\underbrace{=}_{\text{Лемма } \ref{lemma:lemma_1}} (\mathscr{A}\vec{x})^T_e\cdot\Gamma_e\cdot(\mathscr{A}\vec{y})_e = \\
            &= (A_e\vec{x}_e)^T\cdot E\cdot(A_e\vec{y}_e) = \\
            &= \vec{x}^T\cdot \underbrace{A^T_eA_e}_{ E}\vec{y}_e = \\
            &= \vec{x}^T\vec{y}_e = \\
            &= (\vec{x}, \vec{y}).
        \end{align*}
    \end{proof}
\end{enumerate}
