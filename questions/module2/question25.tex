\section{
    Собственные векторы и собственные значения линейного оператора. Характеристическое уравнение и характеристический многочлен линейного оператора. Нахождение собственных значений линейного оператора (вывести характеристическое уравнение). Геометрическая и алгебраическая кратность. Жорданова нормальная форма.
}
\subsection{
    Собственные векторы и собственные значения линейного оператора.
}

\begin{definition}
    Ненулевой вектор $\vec{x}$ в линейном пространстве $\mathcal{L}$ называют \textbf{\textit{собственным вектором}} линейного оператора $\mathscr{A}\colon \mathcal{L} \to \mathcal{L}$, если для некоторого действительного числа $\lambda$ выполняется соотношение $\mathscr{A}\vec{x} = \lambda\vec{x}$. При этом число $\lambda$ называют \textbf{\textit{собственным значением}} линейного оператора $\mathscr{A}$.
\end{definition}

\subsection{
    Характеристическое уравнение и характеристический многочлен линейного оператора.
}

Для произвольной квадратной матрицы $A = (a_{ij})$ порядка $n$ рассмотрим определитель

$$\det(A - \lambda E) = \begin{vmatrix} 
    a_{11} - \lambda & a_{12} & \ldots & a_{1n} \\
    a_{21} & a_{22} - \lambda & \ldots & a_{2n} \\
    \vdots & \vdots & \ddots & \vdots \ \\
    a_{n1} & a_{12} & \ldots & a_{nn} - \lambda \\
\end{vmatrix},$$

где $E$ - единичная матрица, а $\lambda$ - действительное переменное.

\begin{definition}
    Многочлен $\chi_A(\lambda) = \det(A - \lambda E)$ называют \textbf{\textit{характеристическим многочленом}} матрицы $A$, а уравнение $\chi_A(\lambda) = 0$ — \textbf{\textit{характеристическим уравнением}} матрицы $A$.
\end{definition}

\begin{definition}
    \textbf{\textit{Характеристическим многочленом линейного оператора}} $\mathscr{A} \colon \mathcal{L} \to \mathcal{L}$ называют характеристический многочлен его матрицы $A$, записанной в некотором базисе, а \textbf{\textit{характеристическим уравнением}} этого \textbf{\textit{оператора}} - характеристическое уравнение матрицы $A$.
\end{definition}


\subsection{
    *Полезные факты, которые тоже могут быть на экзамене.
}

\begin{definition}
    Квадратные матрицы $A$ и $B$ порядка $n$ называются \textit{\textbf{подобными}}, если существует такая невырожденная матрица $P$, что $P^{-1}AP = B$.
    \label{def:definition_17_1}
\end{definition}

\begin{theorem}
    Если матрицы $A$ и $B$ подобны, то $\det A = \det B$.
    \label{thm:theorem_25_1}
\end{theorem}

\begin{proof}~

    Если матрицы подобны, то согласно определению \eqref{def:definition_17_1}, существует такая невырожденная матрица $P$, что $B = P^{-1}AP$. Так как определитель произведения квадратных матриц равен произведению определителей этих матриц, а $\det(P^{-1}) = (\det P)^{-1}$, то получаем
    $$\det B = \det(P^{-1}AP) = \det(P^{-1})\det A \det P = \det(P)^{-1}\det A \det P = \det A.$$
\end{proof}

\begin{corollary}
    Определитель матрицы линейного оператора не зависит от выбора базиса.
\end{corollary}

\begin{proof}
    Действительно, возьмем матрицы $A_b$ и $A_e$ линейного оператора $\mathscr{A}$ в двух различных базисах $b$ и $e$.

    $$A_e = T^{-1}_{b \to e}A_bT_{b \to e}.$$

    Согласно определению \eqref{def:definition_17_1}, матрицы $A_b$ и $A_e$ подобны. Поэтому $\det A_b = \det A_e$ по теореме \ref{thm:theorem_25_1}.
\end{proof}


\newpage


\subsection{
    Нахождение собственных значений линейного оператора (вывести характеристическое уравнение).
}

\begin{theorem}
    Для того чтобы действительное число $\lambda$ являлось собственным значением линейного оператора, необходимо и достаточно, чтобы оно было корнем характеристического уравнения этого оператора.
\end{theorem}

\begin{proof}~
    \begin{description}
        \item[$(\implies)$] 
            Пусть число $\lambda$ является собственным значением линейного оператора $\mathscr{A} \colon \mathcal{L} \to \mathcal{L}$. Это значит, что существует вектор $\vec{x} \ne \vec{0}$, для которого 

            $$\mathscr{A}\vec{x} = \lambda \vec{x}.$$

            Используя тождественный оператор $\mathscr{I}\vec{x} = \vec{x}$, преобразуем равенство: $\mathscr{A}\vec{x} = \lambda\mathscr{I}\vec{x}$, или
            
            $$(\mathscr{A} - \lambda\mathscr{I})\vec{x} = \vec{0}.$$

            Запишем векторное равенство выше в каком-либо базисе $b$. Матрицей линейного оператора $\mathscr{A} - \lambda\mathscr{I}$ будет матрица $A - \lambda E$, где $A$ - матрица линейного оператора $\mathscr{A}$ в базисе $b$, а $E$ - единичная матрица, и пусть $x$ - столбец координат собственного вектора $\vec{x}$. Тогда $x \ne 0$, а векторное равентсво выше равносильно матричному

            $$(A - \lambda E) = 0,$$

            которое представляет собой матричную форму записи ОСЛАУ с квадратной матрицей $A - \lambda E$ порядка $n$. Эта система имеет ненулевое решение, являющееся столбцом координат $x$ собственного вектора $\vec{x}$. Поэтому $\det(A - \lambda E) = 0$. А это означает, что $\lambda$ является корнем характеристического уравнения линейного оператора $\mathscr{A}$.
        \item[$(\impliedby)$]
            Приведенные рассуждения можно привести в обратном порядке. Если $\lambda$ является корнем характеристического уравнения, то в заданном базисе $b$ выполняется равенство $\det (A - \lambda E) = 0$. Следовательно, матрица ОСЛАУ, записанной в матричной форме, вырождена, и система имеет ненулевое решение $x$. Это ненулевое решение $x$ представляет собой набор координат в базисе $b$ некоторого ненулевого вектора $\vec{x}$, для которого выполняется равенство $(\mathscr{A} - \lambda\mathscr{I})\vec{x} = \vec{0}$. Значит, число $\lambda$ - собственное значение линейного оператора $\mathscr{A}$.
    \end{description}
\end{proof}



\newpage


\subsection{
    Геометрическая и алгебраическая кратность.
}

\begin{definition}
    \textbf{\textit{Геометрической кратностью}} собственного значения линейного оператора называется максимальное число линейно независимых собственных векторов, соответствующих данному собственному значению.
\end{definition}

\begin{definition}
    \textbf{\textit{Алгебраической кратностью}} собственного значения линейного оператора называется его кратность как корня характеристического многочлена.
\end{definition}


\newpage


\subsection{
    Жорданова нормальная форма.
}

Для произвольного действительного числа $\mu$ введем обозначение матрицы порядка $s$:

$$J_s(\mu) = \begin{pmatrix} 
    \mu & 1 & 0 & \ldots & 0 & 0 \\
    0 & \mu & 1 & \ldots & 0 & 0 \\
    \hdotsfor6 \\
    0 & 0 & 0 & \ldots & \mu & 1 \\
    0 & 0 & 0 & \ldots & 0 & \mu
\end{pmatrix}$$

Для любого комплексного числа $\lambda = \alpha + i\beta (\beta \ne 0)$ введем обозначение блочной матрицы порядка $2r$:

$$C_r(\alpha, \beta) = \begin{pmatrix} 
    C(\alpha, \beta) & E & 0 & \ldots & 0 & 0 \\
    0 & C(\alpha, \beta) & E & \ldots & 0 & 0 \\
    \hdotsfor6 \\
    0 & 0 & 0 & \ldots & C(\alpha, \beta) & E \\
    0 & 0 & 0 & \ldots & 0 & C(\alpha, \beta)
\end{pmatrix},$$

где $C(\alpha, \beta) = \begin{pmatrix} 
    \alpha & \beta \\
    -\beta & \alpha
\end{pmatrix}$. Все остальные блоки также являются квадратными матрицами порядка 2, где $E$ - единичная матрица, 0 - нулевая.

Блочно-диагональную матрицу вида

\[
A = \begin{pmatrix}
    C_{r_1}(\alpha_1, \beta_1) &        &        &        &  \\
                               & \ddots &        &        & \scaleobj{4}{0}   \\
                               &        & C_{r_m}(\alpha_m, \beta_m) &        &   \\
                               &        &        & J_{s_1}(\mu_1) &   \\
                               & \scaleobj{4}{0} &        &     & \ddots  &  \\
                               &        &        &        & &J_{s_k}(\mu_k)
\end{pmatrix},
\]

где $\alpha_j, \beta_j (j = \overline{1, m})$ и $\mu_l (l = \overline{1, k})$ - действительные числа, называют \textbf{\textit{жордановой}}, ее диагональные блоки - \textbf{\textit{жордановыми клетками}}. Жорданову матрицу $A'$, подобную данной матрице $A$, называют \textbf{\textit{жордановой нормальной формой}} матрицы $A$.
