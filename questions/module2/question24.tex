\section{
    Операции с линейными операторами. Ранг произведения операторов. Линейное пространство линейных операторов.
}

\subsection{
    Операции с линейными операторами.
}

\begin{definition}
    Операторы $\mathscr{A} \colon \mathcal{V} \to \mathcal{W}$ и $\mathscr{B} \colon \mathcal{V} \to \mathcal{W}$ называются \textbf{\textit{равными}}, если $\mathscr{A}\vec{x} = \mathscr{B}\vec{x}, \forall \vec{x} \in \mathcal{V}$.
\end{definition}

\begin{definition}
    \textbf{\textit{Суммой операторов}} $\mathscr{A} \colon \mathcal{V} \to \mathcal{W}$ и $\mathscr{B} \colon \mathcal{V} \to \mathcal{W}$ называется оператор $(\mathscr{A} + \mathscr{B}) \colon \mathcal{V} \to \mathcal{W}$, действующий по правилу $(\mathscr{A} + \mathscr{B})\vec{x} = \mathscr{A}\vec{x} + \mathscr{B}\vec{x}, \forall \vec{x} \in \mathcal{V}$.
\end{definition}

\begin{definition}
    \textbf{\textit{Произведением оператора}} $\mathscr{A} \colon \mathcal{V} \to \mathcal{W}$ \textbf{\textit{на действительное число}} $\lambda$ называется оператор $(\lambda\mathscr{A}) \colon \mathcal{V} \to \mathcal{W}$, действующий по правилу $(\lambda\mathscr{A})\vec{x} = \lambda(\mathscr{A}\vec{x}), \forall \vec{x} \in \mathcal{V}$.
\end{definition}

\begin{definition}
    \textbf{\textit{Произведением операторов}} $\mathscr{A} \colon \mathcal{V} \to \mathcal{W}$ и $\mathscr{B} \colon \mathcal{L} \to \mathcal{V}$ называется оператор $(\mathscr{A}\mathscr{B}) \colon \mathcal{L} \to \mathcal{W}$, действующий по правилу $(\mathscr{A}\mathscr{B})\vec{x} = \mathscr{A}(\mathscr{B}\vec{x}), \forall \vec{x} \in \mathcal{L}$.
\end{definition}


\newpage


\subsection{
    Ранг произведения операторов.
}

Для любых двух линейных операторов $\mathscr{A}$ и $\mathscr{B}$, действующих в линейном пространстве $\mathcal{L}$, выполняется соотношение

$$\rank(\mathscr{A}\mathscr{B}) \leq \min\{\rank\mathscr{A}, \rank\mathscr{B}\}$$

\begin{proof}~

    Рассмотрим оператор $\mathscr{A}$ как линейный оператор $\mathscr{A}\colon \Im\mathscr{B} \to \mathcal{L}$. Размерность образа оператора не превосходит размерности линейного пространства, из которого он действует, так как сумма и дефекта и ранга совпадает с размерностью этого пространства.

    $$\rank(\mathscr{A}\mathscr{B}) = \dim \Im (\mathscr{A}\mathscr{B}) \leq \dim \Im \mathscr{B} = \rank \mathscr{B}.$$
    
    Так как образ линейного оператора $\mathscr{A}\mathscr{B}$ является линейным подпространством образа линейного оператора $\mathscr{A}$, то
    
    $$\rank(\mathscr{A}\mathscr{B}) \leq \rank \mathscr{A}.$$
\end{proof}


\begin{comment}~

    Доказанное соотношение можно перенести на квадратные матрицы. 
    
    Получаем, 
    $$\rank{(AB)} \leq \min\{\rank A, \rank B\}.$$

    Пусть $B$ - невырожденная. То есть ее ранг равен размерности матрицы. 
    
    Тогда $\rank{(AB)} \leq \rank A$ и одновременно $\rank A = \rank ((AB)B^{-1}) \leq \rank (AB)$.

    То есть 
    
    $$\rank (AB) \leq \rank A \leq \rank (AB).$$
    
    Следовательно, при умножении матрицы $A$ справа на невырожденную матрицу ее ранг не изменяется. 
    
    При умножении матрицы $A$ слева на невырожденную матрицу ранг также не изменяется, что доказывается аналогично.
    \label{comment:comment_24_2}
\end{comment}


\newpage


\subsection{
    Линейное пространство линейных операторов.
}

\begin{definition}
    Линейное пространство $\mathcal{L}(\mathcal{V}, \mathcal{W})$ линейных операторов из линейного пространства $\mathcal{V}$ в линейное пространство $\mathcal{W}$ называют \textbf{\textit{линейным пространством линейных операторов}}.
\end{definition}

\begin{proof}[Проверка на линейность пространства $\mathcal{L}$]~

    Пусть даны линейные операторы $\mathscr{A}б \mathscr{B} \in \mathcal{L}(\mathcal{V}, \mathcal{W})$. 

    Поскольку
    \begin{align*}
        (\mathscr{A} + &\mathscr{B})(\alpha\vec{x} + \beta \vec{y}) = \mathscr{A}(\alpha\vec{x} + \beta \vec{y}) + \mathscr{B}(\alpha\vec{x} + \beta \vec{y}) = \\
        &= (\alpha\mathscr{A}\vec{x} + \beta\mathscr{A}\vec{y}) + (\alpha\mathscr{B}\vec{x} + \beta\mathscr{B}\vec{y}) = \\
        &= \alpha(\mathscr{A}\vec{x} + \mathscr{B}\vec{x}) + \beta(\mathscr{A}\vec{y} + \mathscr{B}\vec{y}) = \\
        &= \alpha(\mathscr{A} + \mathscr{B})\vec{x} + \beta(\mathscr{A} + \mathscr{B})\vec{y}
    \end{align*}

    и

    \begin{align*}
        (\lambda\mathscr{A})&(\alpha\vec{x} + \beta \vec{y}) = \lambda(\mathscr{A}(\alpha\vec{x} + \beta\vec{y})) = \lambda (\mathscr{A}(\alpha\vec{x}) + \mathscr{A}(\beta\vec{y})) = \\ 
        &= (\alpha \lambda)\mathscr{A}\vec{x} + (\beta \lambda)\mathscr{A}\vec{y} = \alpha(\lambda \mathscr{A}\vec{x}) + \beta(\lambda \mathscr{A}\vec{y}) = \\
        &= \alpha((\lambda\mathscr{A})\vec{x}) + \beta((\lambda\mathscr{A})\vec{y})
    \end{align*}

    отображения $\mathscr{A} + \mathscr{B}$ и $\lambda \mathscr{A}$ действительно являются линейными операторами. Таким образом, относительно введенных нами операций множество $\mathcal{L}(\mathcal{V}, \mathcal{W})$ замкнуто. Проверив аксиомы линейного пространства, можно убедиться, что $\mathcal{L}(\mathcal{V}, \mathcal{W})$ относительно этих операций является линейным пространством.
\end{proof}