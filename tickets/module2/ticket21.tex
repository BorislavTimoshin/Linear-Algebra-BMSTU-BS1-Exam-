\section{
    Линейный оператор, определение, три 
    примера. Матрица линейного оператора. 
    Вывести формулу для вычисления значений 
    линейного оператора (с помощью его 
    матрицы). Произведение линейных 
    операторов. Матрица для произведения 
    линейных операторов.
 }

 \subsection{
    Линейный оператор, определение, три 
    примера.
 }

\begin{definition}
    Пусть $\mathcal{V}, \mathcal{W}$ - 
    линейные пространства над полем $\PP$, 
    $\dim \mathcal{V} = n, \dim \mathcal{W} = m$. 
    Отображение $\mathscr{A} \colon \mathcal{V} 
    \to \mathcal{W}$ называется 
    \textit{\textbf{линейным оператором}}, 
    если $\forall \vec{x}, \vec{y} \in 
    \mathcal{V}$ и $\forall \lambda \in \PP$ 
    выполнены следующие условия:
    
    \begin{enumerate}[nosep]
        \item $\mathscr{A}(\vec{x} + \vec{y}) = 
        \mathscr{A}(\vec{x}) + \mathscr{A}(\vec{y}),$
        \item $\mathscr{A}(\alpha\vec{x}) = 
        \alpha\mathscr{A}(\vec{x})$,
    \end{enumerate}
    где $\vec{x}$ - прообраз $\vec{y} = \mathscr{A}\vec{x}$,
    $\vec{y} = \mathscr{A}\vec{x}$ - образ $\vec{x}$.
\end{definition}

\begin{example}~

    \begin{enumerate}[nosep]
        \item Нулевой: $\zeroperator\vec{x} = \vec{0}$.
        \item Тождественный/единичный: 
        $\identityoperator\vec{x} = \vec{x}$.
        \item Оператор дифференцирования. Пусть $\mathcal{V} = P_n(x)$ - множество многочленов от переменной $x$ степени не выше $n$ с действительными коэффициентами. Пусть $p \in \mathcal{V}$. Тогда $\mathscr{A} \colon p \to p'$ - линейный оператор.
    \end{enumerate}
\end{example}


\newpage


\subsection{
    Матрица линейного оператора.
}

Пусть $\mathcal{V}$ и $\mathcal{W}$ - два линейных пространства.

Пусть $e = (\vec{e_1}, \ldots, \vec{e_n})$ - некоторый базис в $\mathcal{V}$, $f = (\vec{f_1}, \ldots, \vec{f_m})$ - некоторый базис в $\mathcal{W}$. 

Тогда $\mathscr{A}\vec{e_1}, \ldots, \mathscr{A}\vec{e_n}$ - это некоторые векторы в $\mathcal{W}$; значит, их можно, причем единственным образом, разложить по базису $\vec{f_1}, \ldots, \vec{f_m}$:

$$\mathscr{A}\vec{e_1} = a_{11}\vec{f_1} + \ldots + a_{m1}\vec{f_m}$$
$$\ldots \ldots \ldots \ldots \ldots \ldots \ldots \ldots \ldots$$
$$\mathscr{A}\vec{e_n} = a_{1n}\vec{f_1} + \ldots + a_{mn}\vec{f_m},$$

где $(a_{ij} \in \PP)$.

\begin{definition}
    Матрицу, составленную из координатных столбцов векторов $\mathscr{A}\vec{e_1}, \ldots, \mathscr{A}\vec{e_n}$ в базисе $f = (\vec{f_1}, \ldots, \vec{f_m})$, называют \textit{\textbf{матрицей линейного оператора}} $\mathscr{A}$ в базисе $f$.
\end{definition}


\newpage


\subsection{
    Вывести формулу для вычисления значений 
    линейного оператора (с помощью его 
    матрицы).
}

\begin{theorem}
    Пусть $\mathscr{A} \colon \mathcal{L} \to \mathcal{L}$ - линейный оператор. Тогда столбец $y_b$ координат вектора $\vec{y} = \mathscr{A}\vec{x}$ в данном базисе $b$ линейного пространства $\mathcal{L}$ равен произведению $A_bx_b$ матрицы $A_b$ оператора $\mathscr{A}$ в базисе $b$ на столбец $x_b$ координат вектора $\vec{x}$ в том же базисе: $y_b = A_bx_b$.
    \label{thm:theorem_21_1}
\end{theorem}

\begin{proof}
    Выберем произвольный вектор $\vec{x} = x_1\vec{b_1} + \ldots + x_n\vec{b_n}$. Его образом будет вектор

    \begin{align*}
        \vec{y} &= \mathscr{A}\vec{x} =  \mathscr{A}(x_1\vec{b_1} + \ldots + x_n\vec{b_n}) = x_1(\mathscr{A}\vec{b_1}) + \ldots + x_n(\mathscr{A}\vec{b_n}) = \\
        &= x_1(a_{11}\vec{b_1} + \ldots + a_{n1}\vec{b_n}) + \ldots + x_n(a_{1n}\vec{b_1} + \ldots + a_{nn}\vec{b_n}) = \\
        &= (a_{11}x_1 + \ldots + a_{1n}x_n)\vec{b_1} + \ldots + (a_{n1}x_1 + \ldots + a_{nn}x_n)\vec{b_n} = \\
        &= b \cdot \begin{pmatrix} 
            a_{11}x_1 + \ldots + a_{1n}x_n \\
            \vdots \\
            a_{n1}x_1 + \ldots + a_{nn}x_n
        \end{pmatrix}
    \end{align*}

    Столбец координат вектора $\vec{y} = \mathscr{A}\vec{x}$ в базисе $b$ имеет вид

    \begin{equation*}
        y_b = (\mathscr{A}\vec{x})_b = \begin{pmatrix} 
            a_{11}x_1 + \ldots + a_{1n}x_n \\
            \vdots \\
            a_{n1}x_1 + \ldots + a_{nn}x_n
        \end{pmatrix} =
        \begin{pmatrix} 
            a_{11} \thinspace \thinspace \ldots \thinspace \thinspace a_{1n} \\
            \ldots \ldots \ldots \\
            a_{n1} \thinspace \thinspace \ldots \thinspace \thinspace a_{nn}
        \end{pmatrix}
        \begin{pmatrix} 
            x_1 \\
            \vdots \\
            x_n
        \end{pmatrix} = A_bx_b
    .\end{equation*}
\end{proof}

\begin{corollary}
    $\vec{y} = \mathscr{A}\vec{x} = bA_bx_b.$
\end{corollary}


\newpage


\subsection{
    Произведение линейных операторов. Матрица для произведения 
    линейных операторов.
}

\begin{definition}
    \textbf{\textit{Произведением операторов}} $\mathscr{A} \colon \mathcal{V} \to \mathcal{W}$ и $\mathscr{B} \colon \mathcal{L} \to \mathcal{V}$ называется оператор $(\mathscr{A}\mathscr{B}) \colon \mathcal{L} \to \mathcal{W}$, действующий по правилу $(\mathscr{A}\mathscr{B})\vec{x} = \mathscr{A}(\mathscr{B}\vec{x}), \forall \vec{x} \in \mathcal{L}$. 
    
    Этот оператор является линейным, так как $\forall \vec{x}, \vec{y} \in \mathcal{L},\thinspace \thinspace \forall \lambda, \mu \in \RR$:

    $$(\mathscr{A}\mathscr{B})(\lambda\vec{x} + \mu\vec{y}) = \mathscr{A}(\mathscr{B}(\lambda\vec{x} + \mu\vec{y})) = \mathscr{A}(\lambda\mathscr{B}\vec{x} + \mu\mathscr{B}\vec{y}) = \lambda\mathscr{A}(\mathscr{B}\vec{x}) + \mu\mathscr{A}(\mathscr{B}\vec{y}) = \lambda(\mathscr{A}\mathscr{B})\vec{x} + \mu(\mathscr{A}\mathscr{B})\vec{y}.$$
\end{definition}

\begin{theorem}
    Пусть в линейном пространстве $\mathcal{L}$ действуют линейные операторы $\mathscr{A}$ и $\mathscr{B}$, а $A_b$ и $B_b$ - матрицы этих линейных операторов в некотором базисе $b$. Тогда матрицей линейного оператора $\mathscr{B}\mathscr{A}$ в том же базисе $b$ является матрица $B_bA_b$.
\end{theorem}

\begin{proof}
    $\vec{y} = (\mathscr{B}\mathscr{A})\vec{x} = \mathscr{B}(\mathscr{A}\vec{x}) = \mathscr{B}(bA_bx_b) = b(B_b(A_bx_b)) = b(B_bA_b)x_b.$
\end{proof}
