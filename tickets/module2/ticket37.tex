\section{
    Знакоопределенность квадратичной формы. Критерий Сильвестра (без доказательства). Примеры.
}

\subsection{
    Знакоопределенность квадратичной формы.
}

\begin{definition}
    Квадратичная форма $f(x) = x^TAx$, $x = \begin{pmatrix}
        x_1 \\
        \vdots \\
        x_n
    \end{pmatrix}$, называется:

    \begin{itemize}
        \item \textbf{\textit{положительно (отрицательно) определенной}}, если для любого ненулевого столбца $x$ выполняется неравенство $f(x) > 0 \thinspace (f(x) < 0)$;
        \item \textbf{\textit{неотрицательно (неположительно) определенной}}, если $f(x) \geq 0 \thinspace (f(x) \leq 0)$ для любого столбца $x$, причем существует ненулевой столбец $x$, для которого $f(x) = 0$;
        \item \textbf{\textit{знакопеременной (неопределенной)}}, если существуют такие столбцы $x$ и $y$, что $f(x) > 0$ и $f(y) < 0$;
    \end{itemize}
\end{definition}

\subsection{
    Критерий Сильвестра (без доказательства).
}

\begin{enumerate}
    \item Для того чтобы квадратичная форма от $n$ переменных была положительно определена, необходимо и достаточно, чтобы выполнялись неравенства $\Delta_1 > 0, \Delta_2 > 0, \Delta_3 > 0, \ldots, \Delta_n > 0$.
    \item Для того чтобы квадратичная форма от $n$ переменных была отрицательно определена, необходимо и достаточно, чтобы выполнялись неравенства $-\Delta_1 > 0, \Delta_2 > 0, -\Delta_3 > 0, \ldots, (-1)^n\Delta_n > 0$ (знаки угловых миноров чередуются, начиная с минуса).
    \item Невырожденная квадратичная форма знакопеременная тогда и только тогда, когда для матрицы квадратичной формы выполнено хотя бы одно из условий:
    \begin{itemize}
        \item один из угловых миноров равен нулю;
        \item один из угловых миноров четного порядка отрицателен;
        \item два угловых минора нечетного порядка имеют разные знаки.
    \end{itemize}
\end{enumerate}

\begin{example}~

    $f(x, y) = 2xy$: $\left(\begin{array}{cc}
        0 & 1 \\
        1 & 0
    \end{array}\right)$ - знакопеременная.

    $f(x, y) = 2x^2 + 2xy + y^2$: $\left(\begin{array}{cc}
        2 & 1 \\
        1 & 1
    \end{array}\right)$ - положительная.

    $f(x, y) = -x^2 - 2xy$: $\left(\begin{array}{cc}
        -1 & -1 \\
        -1 & 0
    \end{array}\right)$ - отрицательная.
\end{example}
