\section{
    Дать определение сопряженного и самосопряженного линейного оператора. Доказать, что все корни характеристического многочлена самосопряженного оператора вещественны.
}

\subsection{
    Дать определение сопряженного и самосопряженного линейного оператора.
}

Пусть $\mathcal{E}$ - евклидово пространство.

\begin{definition}
    Линейный оператор $\mathscr{A^*} \colon \mathcal{E} \to \mathcal{E}$ называют сопряженным к линейному оператору $\mathscr{A} \colon \mathcal{E} \to \mathcal{E}$, если для любых векторов $\vec{x}, \vec{y} \in \mathcal{E}$ верно равенство
    $$(\mathscr{A}\vec{x}, \vec{y}) = (\vec{x}, \mathscr{A^*}\vec{y}).$$
\end{definition}

\begin{example}~
    
    Вектор $\vec{a} \in \mathcal{V}_3$ порождает линейный оператор $\mathscr{A} \colon \mathcal{V}_3 \to \mathcal{V}_3$ согласно формуле
    
    $$\mathscr{A}\vec{x} = \vec{a} \times \vec{x}.$$

    Найдем оператор, сопряженный оператору $\mathscr{A}$:
    \begin{align*}
        (\mathscr{A}\vec{x}, &\vec{y}) = (\vec{a} \times \vec{x}, \vec{y}) = \vec{a}\vec{x}\vec{y} = \vec{y}\vec{a}\vec{x} = (\vec{y} \times \vec{a}, \vec{x}) = \\
        &= (\vec{x}, \vec{y} \times \vec{a}) = (\vec{x}, -\vec{a} \times \vec{y}) = (\vec{x}, -\mathscr{A}\vec{y}).
    \end{align*}

    Значит, $\mathscr{A^*} = -\mathscr{A}.$
\end{example}

\begin{definition}
    Линейный оператор $\mathscr{A}$, действующий в евклидовом пространстве, называют самосопряженным, если $\mathscr{A^*} = \mathscr{A}$. То есть для любых векторов $\vec{x}$ и $\vec{y}$ верно равенство

    $$(\mathscr{A}\vec{x}, \vec{y}) = (\vec{x}, \mathscr{A}\vec{y}).$$
\end{definition}

\begin{example}
    Тождественный $\mathscr{I}$ и нулевой $\mathscr{O}$.
\end{example}


\newpage


\subsection{
    Доказать, что все корни характеристического многочлена самосопряженного оператора вещественны.
}

\begin{theorem}
    Матрица самосопряженного оператора в любом ортонормированном базисе является симметрической.
\end{theorem}

\begin{theorem}
    Все корни характеристического многочлена самосопряженного оператора вещественны.
\end{theorem}

\begin{proof}~

    Будем доказывать, что все корни характеристического уравнения симметрической матрицы действительны.

    Предположим, что $\lambda \in \CC$ является корнем характеристического уравнения симметрической матрицы, т.е. $\det (A - \lambda E) = 0$. Тогда СЛАУ $(A - \lambda E)x = 0$ имеет некоторое ненулевое решение $x = \begin{pmatrix} x_1 & \cdots & x_n \end{pmatrix} ^ T$, состоящее из комплексных чисел $x_k, k = \overline{1, n}$. Рассмотрим столбец $\overline{x}$, комплексно сопряженный к столбцу $x$. Умножим равенство $(A - \lambda E)$x = 0 слева на строку $\overline{x}^T$. Тогда

    $$\overline{x}^T(A - \lambda E)x = 0,$$

    или

    $$\overline{x}^TAx = \lambda \overline{x}^Tx.$$

    Так как произведение комплексного числа на сопряженное к нему является действительным числом, равным квадрату модуля комплексного числа, а $x$ - ненулевое решение, то
    
    $$\overline{x}^Tx = \overline{x}_1x_1 + \ldots + \overline{x}_nx_n = |x_1|^2 + \ldots + |x_n|^2 > 0,$$
    То есть матричное произведение $\overline{x}^Tx$ - действительное положительное число.

    $$\lambda = \frac{\overline{x}^TAx}{\overline{x}^Tx},$$

    причем знаменатель дроби справа является действительным числом. Следовательно, число $\lambda$ будет действительным, если числитель этой дроби $w = \overline{x}^TAx$ будет действительным.

    В силу симметричности матрицы $A$

    $$w = w^T = (\overline{x}^TAx)^T = x^TA^T\overline{x} = x^TA\overline{x}.$$

    С учетом свойств операции комплексного сопряжения матриц и благодаря тому, что элементами матрицы $A$ являются действительные числа, получаем

    $$\overline{w} = \overline{\overline{x}^TAx} = (\overline{\overline{x}})^T\overline{A}\overline{x} = x^TA\overline{x} = w.$$

    Комплексное число, самосопряженное себе - это действительное число. Следовательно, и $w$ является действительным.
\end{proof}
