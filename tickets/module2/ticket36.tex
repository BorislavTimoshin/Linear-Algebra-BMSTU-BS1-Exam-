\section{
    Ранг квадратичной формы, его независимость от выбора базиса. Закон инерции квадратичных форм (с доказательством). 
}

\subsection{
    Ранг квадратичной формы, его независимость от выбора базиса.
}

\begin{definition}
    Ранг матрицы $A$ квадратичной формы называют \textbf{\textit{рангом квадратичной формы}}.
\end{definition}

При изменении базиса линейного пространства матрица $A$ квадратичной формы преобразуется по формуле $A' = U^TAU$, где $U$ - матрица перехода. Матрица $U$, как матрица перехода, является невырожденной, поэтому ранг $A'$ совпадает с рангом $A$, так как при умножении на невырожденную матрицу ранг не меняется. (см. замечание \ref{comment:comment_24_2}). 

То есть ранг квадратичной формы не зависит от выбора базиса.

\newpage


\subsection{
    Закон инерции квадратичных форм (с доказательством).
}

\begin{theorem} Пусть
    \begin{enumerate}
        \item $\mathcal{V}$ - $n$ - мерное линейное пространство.
        \item $\mathscr{B}: \mathcal{V} \times \mathcal{V} \to \RR$ - симметрическая билинейная форма.
        \item $\text{КФ}(\vec{x}) = \mathscr{B}(\vec{x}, \vec{x})$ - соответствующий ей функционал, который может быть записан в виде различных квадратичных форм в зависимости от базиса.
    \end{enumerate}

    Тогда как бы мы ни выбирали канонический базис, количество положительных, количество отрицательных коэффициентов и количество нулевых коэффициентов в каноническом виде КФ будут всегда одними и теми же.
\end{theorem}

\begin{proof}~

    Пусть $\vec{f}_1, \ldots, \vec{f}_n$ и $\vec{g}_1, \ldots, \vec{g}_n$ - два канонических базиса, причем в первом базисе 

    $$\vec{x} = \sum_{k = 1}^{n}x_k\vec{f}_k \text{ и КФ}(\vec{x}) = \sum_{k = 1}^{n}\lambda_kx^2_k.$$

    а во втором базисе

    $$\vec{x} = \sum_{k = 1}^{n}y_k\vec{g}_k \text{ и КФ}(\vec{x}) = \sum_{k = 1}^{n}\mu_ky^2_k.$$

    Будем считать, что:

    \begin{enumerate}
        \item Среди $\lambda_1, \ldots, \lambda_n$ первые $p$ положительные, а остальные $\leq 0$.
        \item Среди $\mu_1, \ldots, \mu_n$ первые $s$ положительные, а остальные $\leq 0$.
    \end{enumerate}

    Отделим положительные слагаемые

    $$\sum_{k = 1}^{p}\underbrace{\lambda_k}_{> 0}x^2_k + \sum_{k = p + 1}^{n}\underbrace{\lambda_k}_{\leq 0}x^2_k = \sum_{k = 1}^{s}\underbrace{\mu_k}_{> 0}y^2_k + \sum_{k = s + 1}^{n}\underbrace{\mu_k}_{\leq 0}y^2_k.$$

    Заметим, что любая из этих сумм может оказаться пустой.

    Перенесем

    \begin{equation}
        \sum_{k = 1}^{p}\underbrace{\lambda_k}_{> 0}x^2_k + \sum_{k = s + 1}^{n}\underbrace{(-\mu_k)}_{\geq 0}y^2_k = \sum_{k = 1}^{s}\underbrace{\mu_k}_{> 0}y^2_k + \sum_{k = p + 1}^{n}\underbrace{(-\lambda_k)}_{\geq 0}x^2_k.
        \label{equation:equation_36_2_1}
    \end{equation}

    Предположим, что $p \ne s$. Без ограничения общности будет считать, что $p < s$ (случай $p > s$ рассматривается аналогично).

    Вопрос: можно ли для заданных базисов $f$ и $g$ найти такой ненулевой вектор $\vec{x}$, чтобы 
    $$x_1 = x_2 = \ldots = x_p = 0, \text{ и } \vec{y}_{s + 1} = \ldots = \vec{y}_n = \vec{0}?$$

    Мы знаем, что 

    $$\vec{x}_f = T_{f \to g}\vec{x}_g.$$

    Наш вопрос переформулируется так:

    Можно ли для заданных $f$ и $g$ найти такой ненулевой $\vec{x}$, чтобы

    \begin{equation}
        \begin{pmatrix}
            0 \\
            0 \\
            \vdots \\
            0 \\
            x_{p + 1} \\
            x_{p + 1}
            \\
            \vdots \\
            x_n
        \end{pmatrix} = T_{f \to g}\begin{pmatrix}
            y_1 \\
            y_2 \\
            \vdots \\
            y_s \\
            0 \\
            0
            \\
            \vdots \\
            0
        \end{pmatrix}.
        \label{equation:equation_36_2_2}
    \end{equation}
    Если справа домножить матрицу на вектор, записать равенство векторов как $n$ уравнений и в них перенести все слагаемые в одну часть, то получим однородную систему из $n$ уравнений с $(n - p) + s$ неизвестными.

    Т.к. $p < s$, то количество уравнений $(n)$ меньше числа неизвестных $(n + (s - p))$.

    По теореме из 1-го семестра такая однородная система имеет нетривиальное решение, т.е.

    $$\exists \underbrace{\thinspace x_{p + 1}, \ldots, x_n, y_1, \ldots, y_s}_{\text{не все равны }0 \thinspace (\star\star\star).},$$
    удовлетворяющее системе \eqref{equation:equation_36_2_2}.

    Если бы все $y_1, y_2, \ldots, y_s$ равнялись $0$, то из \eqref{equation:equation_36_2_2} следовало бы, что $x_{p + 1} = \ldots = x_n = 0$. Тогда бы мы получили противоречие с $(\star\star\star)$. 
    
    Следовательно, среди $y_1, y_2, \ldots, y_s$ обязательно есть хотя бы одно ненулевое.

    Получается, что в \eqref{equation:equation_36_2_1} левая часть состоит сплошь из $0$, а в правой части стоит отрицательное число. Противоречие.

    Значит, $p = s$. Т.е. количество положительных коэффициентов в КФ не зависит от базиса.

    Аналогично, умножив $\mathscr{B}(\vec{x}, \vec{x})$ на $(-1)$, можно доказать, что количество отрицательных коэффициентов в КФ не зависит от базиса.
\end{proof}
