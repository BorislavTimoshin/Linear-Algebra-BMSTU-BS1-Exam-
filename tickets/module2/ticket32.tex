\section{
    Дать определение самосопряженного линейного  оператора. Свойство собственных векторов самосопряженного линейного оператора, отвечающих различным собственным значениям (с доказательством).
}

\subsection{
    Свойство собственных векторов самосопряженного линейного оператора, отвечающих различным собственным значениям (с доказательством).
}

\begin{theorem}
    Собственные векторы самосопряженного оператора, отвечающие различным собственным значениям, ортогональны.
\end{theorem}

\begin{proof}~

    Рассмотрим самосопряженный оператор $\mathscr{A}$ и два его собственных вектора $\vec{x}_1$ и $\vec{x}_2$, отвечающие различным собственным значениям $\lambda_1$ и $\lambda_2$. Тогда $\mathscr{A}\vec{x}_1 = \lambda_1\vec{x}_1$ и $\mathscr{A}\vec{x}_2 = \lambda_2\vec{x}_2$. Поэтому

    \begin{equation}
        (\mathscr{A}\vec{x}_1, \vec{x}_2) = (\lambda_1\vec{x}_1, \vec{x}_2) = \lambda_1(\vec{x}_1, \vec{x}_2).
        \label{eq:equation_32_1}
    \end{equation}

    Но так как $\mathscr{A}$ является самосопряженным оператором, то $(\mathscr{A}\vec{x}_1, \vec{x}_2) = (\vec{x}_1, \mathscr{A}\vec{x}_2)$. Значит,

    \begin{equation}
        (\mathscr{A}\vec{x}_1, \vec{x}_2) = (\vec{x}_1, \mathscr{A}\vec{x_2}) = (\vec{x}_1, \lambda_2\vec{x}_2) = \lambda_2(\vec{x}_1, \vec{x}_2).
        \label{eq:equation_32_2}
    \end{equation}

    Приравнивая правые части соотношений \eqref{eq:equation_32_1} и \eqref{eq:equation_32_2}, получаем

    $$\lambda_1(\vec{x}_1, \vec{x}_2) = \lambda_2(\vec{x}_1, \vec{x}_2),$$

    или

    $$(\lambda_1 - \lambda_2)(\vec{x}_1, \vec{x}_2) = 0.$$

    Так как $\lambda_1 \ne \lambda_2$, $(\vec{x}_1, \vec{x}_2) = 0$, что и означает ортогональность векторов $\vec{x}_1$ и $\vec{x}_2$.
\end{proof}
