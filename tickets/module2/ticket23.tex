\section{
    Линейный оператор, определение, три примера. Ранг и дефект, ядро и образ линейного оператора. Теорема про размерности. Инвариантное подпространство линейного оператора.  
}

\subsection{
    Ранг и дефект, ядро и образ линейного оператора.
}
Пусть $\mathscr{A} \colon \mathcal{V} \to \mathcal{W}$ - линейный оператор.

\begin{definition}
    \textbf{\textit{Образом}} оператора $\mathscr{A}$ называется множество всех векторов $\vec{y} \in \mathcal{W}$, представимых в виде $\vec{y} = \mathscr{A}\vec{x}$.
\end{definition}

\begin{designation}
    $\Im \mathscr{A}$.
\end{designation}

\begin{definition}
    \textbf{\textit{Ядром}} оператора $\mathscr{A}$ называется множество всех векторов $\vec{x} \in \mathcal{V} \colon \mathscr{A}\vec{x} = \vec{0}_{\mathcal{W}}$.
\end{definition}

\begin{designation}
    $\ker \mathscr{A}$.
\end{designation}

\begin{definition}
    Размерность образа линейного оператора $\mathscr{A}$ называется \textbf{\textit{рангом}} линейного оператора $\mathscr{A}$. 
\end{definition}

\begin{designation}
    $\rank \mathscr{A}$.
\end{designation}

\begin{definition}
    Размерность ядра линейного оператора $\mathscr{A}$ называется \textbf{\textit{дефектом}} линейного оператора $\mathscr{A}$. 
\end{definition}

\begin{designation}
    $\defect \mathscr{A}$.
\end{designation}


\newpage


\subsection{
    Теорема про размерности.
}

\begin{theorem}
    Пусть $\mathscr{A} \colon \mathcal{V} \to \mathcal{W}$ - линейный оператор, $\dim \mathcal{V} = n$. Тогда $\defect \mathscr{A} + \rank \mathscr{A} = \dim \mathcal{V}$.
\end{theorem}

\begin{proof}~

    Пусть $\vec{e}_1, \ldots, \vec{e}_r$ - базис в $\ker \mathscr{A}$. Дополним его до базиса $\vec{e}_1, \ldots, \vec{e}_r, \vec{e}_{r + 1}, \ldots, \vec{e}_n$ всего пространства $\mathcal{V}$. Докажем, что $\dim \Im \mathscr{A} = n - r$. Для этого рассмотрим набор векторов $\mathscr{A}(\vec{e}_{r + 1}), \ldots, \mathscr{A}(\vec{e}_n)$ и докажем, что он является базисом в $\Im \mathscr{A}$.

    \bigbreak

    $\Im \mathscr{A} = \Span(\underbrace{\mathscr{A}(\vec{e}_1), \ldots, \mathscr{A}(\vec{e}_r)}_{\vec{0}}, \mathscr{A}(\vec{e}_{r + 1}), \ldots, \mathscr{A}(\vec{e}_n)) = \Span(\mathscr{A}(\vec{e}_{r + 1}), \ldots, \mathscr{A}(\vec{e}_n))$.

    Предположим, что $\lambda_{r + 1}\mathscr{A}(\vec{e}_{r + 1}) + \ldots + \lambda_n\mathscr{A}(\vec{e}_n) = \mathscr{A}(\lambda_{r + 1}\vec{e}_{r + 1} + \ldots + \lambda_n\vec{e}_n) = \vec{0}_{\mathcal{W}}.$

    Значит, $\lambda_{r + 1}\vec{e}_{r + 1} + \ldots + \lambda_n\vec{e}_n \in \ker \mathscr{A}$, но тогда $\lambda_{r + 1}\vec{e}_{r + 1} + \ldots + \lambda_n\vec{e}_n = \mu_1\vec{e}_1 + \ldots + \mu_r\vec{e}_r$, для некоторых $\mu_1, \ldots, \mu_r$. Так как векторы $\vec{e}_1, \ldots, \vec{e}_n$ - линейно независимы, то все $\lambda_i = 0$ (и $\mu_j$ тоже), следовательно, векторы $\vec{e}_{r + 1}, \ldots, \vec{e}_n$ - линейно независимы.

    \bigbreak

    При этом любой вектор $\vec{y} \in \Im \mathscr{A}$ является линейной комбинацией векторов $\mathscr{A}(\vec{e}_{r + 1}), \ldots, \mathscr{A}(\vec{e}_n)$.

    \bigbreak

    Следовательно, $\mathscr{A}(\vec{e}_{r + 1}), \ldots, \mathscr{A}(\vec{e}_n)$ - базис в $\Im \mathscr{A} \Rightarrow \dim \Im \mathscr{A} = n - r$.

    \bigbreak

    Значит, $\dim \ker \mathscr{A} + \dim \Im \mathscr{A} = \defect \mathscr{A} + \rank \mathscr{A} = r + n - r = n = \dim \mathcal{V}$.
\end{proof}

\bigbreak

\begin{comment}
    Можно ли утверждать, что если $\mathscr{A} \colon \mathcal{V} \to \mathcal{V}$, то $\Im \mathscr{A} + \ker \mathscr{A} = \mathcal{V}$?

    \bigbreak

    \textbf{Ответ:} нельзя. 
    
    Например, $\mathscr{D} \colon p \longmapsto p'$.

    $p \in P_n(x)$

    $\Im \mathscr{D} = P_{n - 1}(x)$
    
    $\ker \mathscr{D} = P_0(x)$

    Но $P_0(x) + P_{n - 1}(x) \ne P_n(x)$.
    
\end{comment}


\newpage


\subsection{
    Инвариантное подпространство линейного оператора.
}

\begin{definition}
    Пусть $\mathscr{A} \colon \mathcal{L} \to \mathcal{L}$ - линейный оператор. Подпространство $\mathcal{V} \in \mathcal{L}$ называется \textbf{\textit{инвариантным подпространством}} оператора $\mathscr{A}$, если оператор $\mathscr{A}$ отображает всякий вектор $\vec{x} \in \mathcal{V}$, в вектор, также принадлежащий подпространству $\mathcal{V}$, то есть $\forall \vec{x} \in \mathcal{V} \colon \vec{y} = \mathscr{A}\vec{x} \in \mathcal{V}$.
\end{definition}

\begin{example}~

    \begin{itemize}
        \item Тривиальными примерами являются: само пространство $\mathcal{L}$ и нулевое подпространство (состоящее из единственного нулевого вектора).
        \item Любой собственный вектор оператора порождает его одномерное инвариантное подпространство.
        \item Ядро линейного оператора $\ker \mathcal{L}$. 
    \end{itemize}
\end{example}
