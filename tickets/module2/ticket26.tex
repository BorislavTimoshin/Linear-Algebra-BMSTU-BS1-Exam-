\section{
    Формулировка теоремы Гамильтона – Кэли*. След линейного оператора. Инварианты.
}

\subsection{
    *Формулировка теоремы Гамильтона – Кэли.
}

Квадратную матрицу можно использовать в качестве значения переменного в произвольном многочлене. Тогда значением многочлена от матрицы будет матрица того же порядка, что и исходная. Интерес представляют такие многочлены, значение которых от данной матрицы есть нулевая матрица. Их называют аннулирующими многочленами. Оказывается, что одним из таких аннулирующих многочленов для матрицы является ее характеристический многочлен.

\begin{theorem}
    Для любой квадратной матрицы характеристический многочлен является ее аннулирующим многочленом.
\end{theorem}

\subsection{
    След линейного оператора.
}

\begin{definition}
    \textbf{\textit{Следом линейного оператора $\mathscr{A}$ (матрицы $A$)}} называется сумма диагональных элементов матрицы $A$ линейного оператора $\mathscr{A}$.
\end{definition}

\begin{designation}
    $\tr \mathscr{A}$ или $\Sp \mathscr{A}$.
\end{designation}

\subsection{
    Инварианты.
}

Коэффициенты характеристического многочлена не зависят от выбора базиса (если представить в виде $\sum_{k=0}^{n} d_k \lambda^k
$), т.е. являются инвариантами относительно выбора базиса.

\begin{comment}
    Наиболее просто выражается коэффициент $d_{n - 1} = \tr \mathscr{A}$.
\end{comment}

\begin{comment}
    Коэффициент $d_0$ характеристического многочлена совпадает со значением этого многочлена при $\lambda = 0$ и равен определителю линейного оператора $\mathscr{A}$.
\end{comment}
