\section{
    Приведение квадратичной формы к каноническому виду ортогональным преобразованием (обосновать возможность такого приведения).
}
    
\begin{definition}
    Квадратичную форму
    $$\alpha_1x_1^2 + \ldots + \alpha_nx_n^2, \quad \alpha_i \in \RR, \quad i = \overline{1, n},$$
    не имеющую попарных произведений переменных, называют \textit{\textbf{квадратичной формой канонического вида}}. Переменные $x_1, \ldots, x_n$, в которых квадратичная форма имеет канонический вид, называют \textbf{\textit{каноническими переменными}}.
\end{definition}