\section{
    Матрица перехода от базиса к базису, вывод формулы для преобразования координат вектора при переходе к новому базису. Обратный переход. Работа с тремя и более базисами.
}

\subsection{
    Матрица перехода от базиса к базису.
}

\begin{definition}
    \textit{\textbf{Матрицей перехода}} от старого базиса к новому называется матрица, элементами \textit{\textbf{столбцов}} которой являются координаты векторов нового базиса, разложенных по старому базису.
    \label{fig:definition_17_1}
\end{definition}

\subsection{
    Вывод формулы для преобразования координат вектора при переходе к новому базису.
}

Пусть в $n$-мерном линейном пространстве $\mathcal{L}$ заданы два базиса: старый $b = (\vec{b_1}, \ldots, \vec{b_n})$ и новый $c = (\vec{c_1}, \ldots, \vec{c_n})$.

Разложим векторы базиса $c$ по базису $b$:

$$\vec{c_i} = \alpha_{1i}\vec{b_1} + \ldots + \alpha_{ni}\vec{b_n}, \quad i = \overline{1, n}.$$

Запишем эти представления в матричной форме:

$$\vec{c_i} = b \begin{pmatrix} \alpha_{1i} \\ \vdots \\ \alpha_{ni} \end{pmatrix}, \quad  i = \overline{1, n},$$

или

$$c = bU,$$

где

\begin{equation*}
    U = \left(\begin{array}{ccc}
        \alpha_{11} & \ldots & \alpha_{1n} \\
        \hdotsfor{3} \\
        \alpha_{n1} & \ldots & \alpha_{nn}
    \end{array}\right).
\end{equation*}


\newpage


\subsection{
    Обратный переход. Работа с тремя и более базисами.
}

\textbf{Свойства.}

\begin{enumerate}[label={\arabic*°.}]
    \item Матрица перехода невырождена и всегда имеет обратную.
    \begin{proof}~

        Столбцы матрицы перехода - столбцы координат векторов нового \textbf{базиса} в старом. Следовательно, они, как и векторы базиса, линейно независимы. Значит, матрица $U$ невырожденная и имеет обратную матрицу $U^{-1}$.
    \end{proof}
    
    \item Если в $n$-мерном линейном пространстве задан базис $b$, то для любой невырожденной квадратной матрицы $U$ порядка $n$ существует такой базис $c$ в этом линейном пространстве, что $U$ будет матрицей перехода то базиса $b$ к базису $c$.
    \begin{proof}~

        Из невырожденности матрицы $U$ следует, что ее ранг ранг равен $n$, и поэтому ее столбцы, будучи базисными, линейно независимы. Эти столбцы являются столбцами координат векторов системы $c = bU$. Линейная независимость столбцов матрицы $U$ равносильная линейной независимости системы векторов $c$. Так как система $c$ содержит $n$ векторов, причем линейное пространство $n$-мерно, то согласно теореме $\eqref{thm:theorem_2_1}$, эта система является базисом.
    \end{proof}
    
    \item Если $U$ - матрица перехода от старого базиса $b$ к новому базису $c$ линейного пространства, то $U^{-1}$ - матрица перехода от базиса $c$ к базису $b$.
    \begin{proof}~

        Матрица $U$ невырождена, и поэтому из равенства $c = bU$ следует, что $cU^{-1} = b$. Последнее равенство означает, что столбцы матрицы $U^{-1}$ являются столбцами координат векторов $b$ относительно базиса $c$, т.е. согласно определению $\eqref{fig:definition_17_1}$ $U^{-1}$ - это матрица перехода от базиса $c$ к базису $b$.
    \end{proof}

    \item Если в линейном пространстве заданы базисы $b, c$ и $d$, причем $U$ - матрица перехода от базиса $b$ к новому базису $c$, а $V$ - матрица перехода от базиса $c$ к базису $d$, то произведение этих матриц $UV$ - матрица перехода от базиса $b$ к базису $d$.
    \begin{proof}~

        Согласно определению $\eqref{fig:definition_17_1}$ матрицы перехода, имеем равенства
        $$c = bU, \quad d = cV,$$
        откуда
        $$d = cV = (bU)V = b(UV),$$
        т.е. $UV$ - матрица перехода от базиса $b$ к базису $d$.
    \end{proof}

    \item Пусть $b_1, b_2, \ldots, b_n$ - это $n$ базисов линейного пространства $V$ ($n \geq 4$). $U_k$ - матрица перехода от $b_k$ к $b_{k + 1}$, $k = \overline{1, n - 1}$. Тогда матрица перехода от $b_1$ к $b_n$ равна $U_1U_2 \ldots U_{n - 1}$.
    \begin{proof}
        Последовательное применение свойства 4°.
    \end{proof}
\end{enumerate}

