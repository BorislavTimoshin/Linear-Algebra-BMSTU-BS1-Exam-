\section{
    Матрица Грама для системы векторов. Определитель и ранг матрицы Грама. Как изменится грамиан, если один из векторов заменить его ортогональной проекцией?
}

\subsection{
    Матрица Грама для системы векторов.
}
    
\begin{definition}
    Пусть даны векторы \( \vec{x}_1, \vec{x}_2, \dots, \vec{x}_m \) в некотором евклидовом пространстве. \textit{\textbf{Матрицей Грама}} этой системы называется квадратная матрица \( \Gamma \) размера \( m \times m \), элементы которой задаются скалярными произведениями:

    \[
    \Gamma = \begin{pmatrix}
    (\vec{x}_1, \vec{x}_1) & (\vec{x}_1, \vec{x}_2) & \cdots & (\vec{x}_1, \vec{x}_m) \\
    (\vec{x}_2, \vec{x}_1) & (\vec{x}_2, \vec{x}_2) & \cdots & (\vec{x}_2, \vec{x}_m) \\
    \vdots     & \vdots     & \ddots & \vdots     \\
    (\vec{x}_m, \vec{x}_1) & (\vec{x}_m, \vec{x}_2) & \cdots & (\vec{x}_m, \vec{x}_m)
    \end{pmatrix},
    \]
    
    где \( (\vec{x}_i, \vec{x}_j) \) обозначает скалярное произведение векторов \( \vec{x}_i \) и \( \vec{x}_j \).
\end{definition}

Её определитель называется определителем Грама (или \textbf{\textit{грамианом}}).


\subsection{
    Определитель и ранг матрицы Грама.
}


\begin{itemize}
\item $\det \Gamma \geq 0$ (всегда неотрицателен).
\item $\det \Gamma = 0$ $\iff$ векторы $\vec{x}_1, \vec{x}_2, \dots, \vec{x}_m$ линейно зависимы.
\item Для линейно независимых векторов $\det \Gamma > 0$.
\item Геометрический смысл: $\det \Gamma$ равен квадрату объёма параллелепипеда, натянутого на векторы.
\end{itemize}

Ранг матрицы Грама равен максимальному числу линейно независимых векторов в системе, т.е. 

$\rank \Gamma(\vec{x}_1, \vec{x}_2, \dots, \vec{x}_m) = \dim \Span (\vec{x}_1, \vec{x}_2, \dots, \vec{x}_m)$.


\subsection{
    Как изменится грамиан, если один из векторов заменить его ортогональной проекцией?
}

Рассмотрим линейное пространство $\mathcal{L} = \Span(\vec{e}_1, \ldots, \vec{e}_{k - 1}, \vec{e}_k, \vec{e}_{k + 1}, \ldots, \vec{e}_n)$ и его подпространство 

$\mathcal{H} = \Span(\vec{e}_1, \ldots, \vec{e}_{k - 1}, \vec{e}_{k + 1}, \ldots, \vec{e}_n)$.

\bigbreak

Представим вектор $\vec{e}_k \in \mathcal{L}$ в виде 
$$\vec{e}_k = \vec{e}^{\parallel}_k + \vec{e}^{\perp}_k,$$

где $\vec{e}^{\parallel}_k \in \mathcal{H}$ - ортогональная проекция, а $\vec{e}^{\perp}_k \in \mathcal{H}^\perp$ - ортогональная составляющая.

Значит,

\begin{gather*}
    \vec{e}^{\parallel}_k = \alpha_1\vec{e}_1 + \ldots + \alpha_{k - 1}\vec{e}_{k - 1} + \alpha_{k + 1}\vec{e}_{k + 1} + \ldots + \alpha_n\vec{e}_n. \\
    \downimplies \\
    \vec{e}_1, \ldots, \vec{e}_{k - 1}, \vec{e}^{\parallel}_k, \vec{e}_{k + 1}, \ldots, \vec{e}_n \text{ - линейно зависимы.} \\
    \downimplies \\
    \det \Gamma(\underbrace{\vec{e}_1, \ldots, \vec{e}_{k - 1}, \vec{e}^{\parallel}_k, \vec{e}_{k + 1}, \ldots, \vec{e}_n}_{\text{Система векторов, полученная заменой } \vec{e}_k \text{ на } \vec{e}^{\parallel}_k.}) = 0
\end{gather*}
