\section{
    Матрица Грама для системы векторов. Определитель и ранг матрицы Грама. Как изменится грамиан, если один из векторов заменить его ортогональной проекцией?
}

\subsection{
    Матрица Грама для системы векторов.
}
    
\begin{definition}
    Пусть даны векторы \( \vec{x}_1, \vec{x}_2, \dots, \vec{x}_m \) в некотором евклидовом пространстве. \textit{\textbf{Матрицей Грама}} этой системы называется квадратная матрица \( \Gamma \) размера \( m \times m \), элементы которой задаются скалярными произведениями:

    \[
    \Gamma = \begin{pmatrix}
    (\vec{x}_1, \vec{x}_1) & (\vec{x}_1, \vec{x}_2) & \cdots & (\vec{x}_1, \vec{x}_m) \\
    (\vec{x}_2, \vec{x}_1) & (\vec{x}_2, \vec{x}_2) & \cdots & (\vec{x}_2, \vec{x}_m) \\
    \vdots     & \vdots     & \ddots & \vdots     \\
    (\vec{x}_m, \vec{x}_1) & (\vec{x}_m, \vec{x}_2) & \cdots & (\vec{x}_m, \vec{x}_m)
    \end{pmatrix},
    \]
    
    где \( (\vec{x}_i, \vec{x}_j) \) обозначает скалярное произведение векторов \( \vec{x}_i \) и \( \vec{x}_j \).
\end{definition}

Её определитель называется определителем Грама (или \textbf{\textit{грамианом}}).


\subsection{
    Определитель и ранг матрицы Грама.
}


\begin{itemize}
\item $\det \Gamma \geq 0$ (всегда неотрицателен).
\item $\det \Gamma = 0$ $\iff$ векторы $\vec{x}_1, \vec{x}_2, \dots, \vec{x}_m$ линейно зависимы.
\item Для линейно независимых векторов $\det \Gamma > 0$.
\item Геометрический смысл: $\det \Gamma$ равен квадрату объёма параллелепипеда, натянутого на векторы.
\end{itemize}

Ранг матрицы Грама равен максимальному числу линейно независимых векторов в системе, т.е. 

$\rank \Gamma(\vec{x}_1, \vec{x}_2, \dots, \vec{x}_m) = \dim \Span (\vec{x}_1, \vec{x}_2, \dots, \vec{x}_m)$.


\subsection{
    Как изменится грамиан, если один из векторов заменить его ортогональной проекцией?
}
