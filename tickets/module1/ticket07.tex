\section{
    Линейное аффинное многообразие. Вектор сдвига. Представление k-мерного линейного аффинного многообразия. Решения неоднородной СЛАУ. 
}

\subsection{
    Представление k-мерного линейного аффинного многообразия.
}

Пусть даны $\mathcal{V}$ - линейное пространство над $\PP$, $\mathcal{W}$ - его подпространство, $\vec{a} \in \mathcal{V}$ - некоторый вектор.

\begin{definition}
    \textbf{\textit{Аффинной линейной комбинацией}} произвольных $s$ векторов $\vec{b}_1, \vec{b}_2, \ldots, \vec{b}_s \in \mathcal{V}$ называется вектор

    $$\lambda_0\vec{a} + \lambda_1\vec{b}_1 + \lambda_2\vec{b}_2 +  \ldots + \lambda_s\vec{b}_s,$$

    где числа $\lambda_i$ удовлетворяют соотношению $\lambda_0 + \lambda_1 + \ldots + \lambda_s = 1$.
\end{definition}

\begin{definition}
    \textbf{\textit{Аффинной оболочкой заданных векторов $\vec{b}_1, \vec{b}_2, \ldots, \vec{b}_s \in \mathcal{V}$}} называется множество всех их аффинных линейных комбинаций.
\end{definition}

\begin{designation}
    $\Aff(\vec{b}_1, \vec{b}_2, \ldots, \vec{b}_s)$.
\end{designation}

\begin{theorem}
    Всякое $k$-мерное ЛАМ $(\vec{a} + \mathcal{W})$ линейного пространства $\mathcal{V}$ может быть представлено как аффинная линейная оболочка $\leq k$ векторов.
\end{theorem}

\begin{proof}~

    \begin{gather*}
        (\vec{a} + \mathcal{W})\text{ - это $k$-мерное ЛАМ.} \\
        \downimplies \\
        \dim \mathcal{W} = k. \\
        \downimplies \\
        \text{Можно зафиксировать базис }\mathcal{W}\text{ - векторы }\vec{e}_1, \vec{e}_2, \ldots, \vec{e}_k. \\
        \text{Тогда рассмотрим векторы: } \\
        \vec{v}_1 = \vec{a} + \vec{e}_1 \\
        \vec{v}_2 = \vec{a} + \vec{e}_2 \\
        \vdots \\
        \vec{v}_k = \vec{a} + \vec{e}_k. \\
        \text{И докажем, что } \vec{a} + \mathcal{W} = \Aff(\vec{v}_1, \vec{v}_2, \ldots, \vec{v}_k).
    \end{gather*}

    \begin{enumerate}
        \item[$\subseteq$] Возьмем любой $\vec{a} + \vec{w} \in \vec{a} + \mathcal{W}$.

        Тогда 

        \begin{align*}
            \vec{a} + \vec{w} &= \vec{a} + \lambda_1\vec{e}_1 + \ldots + \lambda_k\vec{e}_k = \\
            &= \vec{a} + \lambda_1(\vec{v}_1 - \vec{a}) + \ldots + \lambda_k(\vec{v}_k - \vec{a}) = \\
            &= \vec{a} + \lambda_1\vec{v}_1 - \lambda_1\vec{a} + \ldots + \lambda_k\vec{v}_k - \lambda_k\vec{a} = \\
            &= (\underbrace{1 - \lambda_1 - \lambda_2 - \ldots - \lambda_k}_{\lambda_0})\vec{a} + \lambda_1\vec{v}_1 + \ldots + \lambda_k\vec{v}_k = \\
            &= \lambda_0\vec{a} + \lambda_1\vec{v}_1 + \ldots + \lambda_k\vec{v}_k,
        \end{align*}
        при этом $\lambda_0 + \lambda_1 + \ldots + \lambda_k = 1$.
        
        Значит, $\vec{a} + \vec{w} \in \Aff(\vec{v}_1, \vec{v}_2, \ldots, \vec{v}_k)$.
        
        \item[$\supseteq$] Рассмотрим любой $\vec{x} \in \Aff(\vec{v}_1, \vec{v}_2, \ldots, \vec{v}_k)$.

        Тогда $\vec{x} = \lambda_0\vec{a} + \lambda_1\vec{v}_1 + \ldots + \lambda_k\vec{v}_k$, где $\lambda_0 + \lambda_1 + \ldots + \lambda_k = 1$. Выразим $\lambda_0 = 1 - \lambda_1 - \lambda_2 - \ldots - \lambda_k$.

        Тогда

        \begin{align*}
            \vec{x} &= (1 - \lambda_1 - \lambda_2 - \ldots - \lambda_k)\vec{a} + \lambda_1\vec{v}_1 + \ldots + \lambda_k\vec{v}_k = \\
            &= \vec{a} + \lambda_1(\vec{v}_1 - \vec{a}) + \ldots + \lambda_k(\vec{v}_k - \vec{a}) = \\
            &= \vec{a} + \underbrace{\lambda_1\vec{e}_1 + \ldots + \lambda_k\vec{e}_k}_{\in \mathcal{W}} \in \vec{a} + \mathcal{W}.
        \end{align*}
    \end{enumerate}
\end{proof}


\subsection{
    Связь с решениями неоднородной СЛАУ.
}

Рассмотрим неоднородную СЛАУ

$$A\vec{x} = \vec{b},$$

где $A \in \RR^{m \times n}, \vec{x} = \begin{pmatrix} x_1 \\ \vdots \\ x_n \end{pmatrix} \in \RR^n, \vec{b} = \begin{pmatrix} b_1 \\ \vdots \\ b_m \end{pmatrix} \in \RR^m$.

Из курса Ан. Геом. известно, что если СЛАУ имеет бесконечно много решений, то они задаются так:

$$\vec{x} = \vec{x}_0 + c_1\vec{f}_1 + \ldots + c_k\vec{f}_k,$$

где $\vec{x}_0 \in \RR^n$ - вектор с постоянными коэффициентами, $\vec{f}_1, \ldots, \vec{f}_k \in \RR^n$ - векторы, образующие ФСР, $c_1, \ldots, c_k$ - произвольные константы.

Если рассмотреть $\mathcal{W} = \Span \{\vec{f}_1, \ldots, \vec{f}_k\}$, то получится, что множество $\mathcal{U}$ всех решений СЛАУ будет представлять из себя ЛАМ:

$$\mathcal{U} = \vec{x}_0 + \mathcal{W}.$$

Представляете!
