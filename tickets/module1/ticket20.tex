\section{
    Норма вектора в евклидовом пространстве. Привести три примера задания нормы. Свойства нормы (с доказательством).
}

\begin{definition}
    Функция, заданная на линейном пространстве $\mathcal{V}$, которая каждому вектору ставит в соответствие вещественное число, называется \textbf{\textit{нормой}}, если выполнены 3 аксиомы:
    \begin{enumerate}[nosep]
        \item $\norm{\vec{x}} \geq 0$, причем $\norm{\vec{x}} = 0 \iff \vec{x} = 0$;
        \item $\norm{\lambda \vec{x}} = |\lambda| \cdot  \norm{\vec{x}}, \thinspace \lambda \in \RR$;
        \item $\norm{\vec{x} + \vec{y}} \leq \norm{\vec{x}} + \norm{\vec{y}}$ (неравенство треугольника).
    \end{enumerate}
\end{definition}

\begin{theorem}
    Всякое скалярное произведение в евклидовом пространстве определяет норму $\norm{\vec{x}} = \sqrt{(\vec{x}, \vec{x})}$.
\end{theorem}


\subsection{
    Привести три примера задания нормы.
}


\begin{definition}
    Норма вида $\norm{\vec{x}}_2 = \sqrt{(\vec{x}, \vec{x})}$ называется \textbf{\textit{евклидовой}} $(l_2)$
\end{definition}

\begin{definition}
    Норма вида $\norm{\vec{x}}_1 = |x_1| + \dots + |x_n|$ называется \textbf{\textit{октаэдрической}} $(l_1)$
\end{definition}

\begin{definition}
    Норма вида $\norm{\vec{x}}_{\infty} = max\{|x_1|, \dots, |x_n|\}$ называется \textbf{\textit{кубической}} $(l_{\infty})$
\end{definition}


\subsection{
    Свойства нормы (с доказательством).
}

\begin{enumerate}
    \item $\norm{\vec{x}} - \norm{\vec{y}} \leq \norm{\vec{x} \pm \vec{y}} \leq \norm{\vec{x}} + \norm{\vec{y}}$.
    
    \item $\norm{\alpha\vec{x} + \beta\vec{y}} \leq \norm{\alpha\vec{x}} + \norm{\beta\vec{y}} = |\alpha|\norm{\vec{x}} + |\beta|\norm{\vec{y}}$.

    \item $\norm{\vec{x} + \vec{y}}^2 + \norm{\vec{x} - \vec{y}}^2 = 2(\norm{\vec{x}}^2 + \norm{\vec{y}}^2)$.

    \begin{proof}~
    
        $\norm{\vec{x} + \vec{y}}^2 = (\vec{x} + \vec{y}, \vec{x} + \vec{y}) = (\vec{x}, \vec{x}) + (\vec{x}, \vec{y}) + (\vec{y}, \vec{x}) + (\vec{y}, \vec{y})$,

        $\norm{\vec{x} - \vec{y}}^2 = (\vec{x} - \vec{y}, \vec{x} - \vec{y}) = (\vec{x}, \vec{x}) - (\vec{x}, \vec{y}) - (\vec{y}, \vec{x}) + (\vec{y}, \vec{y})$,

        $\norm{\vec{x} + \vec{y}}^2 + \norm{\vec{x} - \vec{y}}^2 = 2(\vec{x}, \vec{x}) + 2(\vec{y}, \vec{y}) = 2(\norm{\vec{x}}^2 + \norm{\vec{y}}^2)$.
    \end{proof}
    
    \item Если норма порождена скалярным произведением, т.е. $\norm{\vec{z}} = \sqrt{(\vec{z}, \vec{z})}$, то для неё определено
    
    $$(\vec{x}, \vec{y}) = \frac{1}{2}\cdot(\norm{\vec{x} + \vec{y}}^2 - \norm{\vec{x}}^2 - \norm{\vec{y}}^2).$$

    \begin{proof}
        $$\frac{1}{2}\cdot(\norm{\vec{x} + \vec{y}}^2 - \norm{\vec{x}}^2 - \norm{\vec{y}}^2) = \frac{1}{2}\cdot((\vec{x} + \vec{y}, \vec{x} + \vec{y}) - (\vec{x}, \vec{x}) - (\vec{y}, \vec{y})) = \frac{1}{2} \cdot ((\vec{x}, \vec{x}) + 2(\vec{x}, \vec{y}) + (\vec{y}, \vec{y})  - (\vec{x}, \vec{x}) - (\vec{y}, \vec{y})) = (\vec{x}, \vec{y})$$
    \end{proof}
    
    \item В конечномерном пространстве любые две нормы эквивалентны.

    \begin{definition}[P.S.]
        Две нормы $p$ и $q$ на пространстве $\mathcal{V}$ называются эквивалентными, если
        
        $\exists C_1, C_2 > 0 \colon C_1p(\vec{x}) \leq q(\vec{x}) \leq C_2p(\vec{x}), \forall \vec{x} \in \mathcal{V}$.
    \end{definition}
\end{enumerate}
