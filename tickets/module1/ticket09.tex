\section{
    Скалярное произведение, примеры (привести три примера). Косинус. Евклидовы пространства. Понятие метрики и нормы, способы задания норм (привести три примера). 
}

\subsection{
    Скалярное произведение, примеры (привести три примера). Евклидовы пространства.
}

\begin{definition}
    Пусть дано линейное пространство $\mathcal{V} = \{\vec{a}, \vec{b}, \vec{c}, \vec{d}, \dots\}$. Множество вида 
    
    $\{(\vec{a}, \vec{a}), (\vec{a}, \vec{b}), (\vec{a}, \vec{c}), (\vec{a}, \vec{d}), \dots, (\vec{b}, \vec{a}), (\vec{b}, \vec{b}), (\vec{b}, \vec{c}), (\vec{b}, \vec{d}), \dots\}$ называется \textbf{\textit{декартовым квадратом}} $\mathcal{V} \times \mathcal{V}$.
\end{definition}

\begin{definition}
    Отображение $\mathcal{V} \times \mathcal{V} \to \RR$, где $\mathcal{V}$ - линейное пространство над полем $\RR$, называется \textbf{\textit{скалярным произведением}}, если выполнены 4 аксиомы:
    \begin{enumerate}[nosep]
        \item $(\vec{x}, \vec{y}) = (\vec{y}, \vec{x})$.
        \item $(\vec{x} + \vec{y}, \vec{z}) = (\vec{x}, \vec{z}) + (\vec{y}, \vec{z})$ - аддитивность по первому аргументу.
        \item $(\alpha \vec{x}, \vec{y}) = \alpha(\vec{x}, \vec{y})$ - однородность по первому аргументу.
        \item $(\vec{x}, \vec{x}) \geq 0$, причем $(\vec{x}, \vec{x}) = 0 \iff \vec{x} = 0$.
    \end{enumerate}
\end{definition}

\begin{definition}
    Вещественное линейное пространство с так введенным скалярным произведением называется \textbf{\textit{евклидовым пространством}}.
\end{definition}

\begin{definition}
    Отображение $\mathcal{V} \times \mathcal{V} \to \CC$, где $\mathcal{V}$ - линейное пространство над полем $\CC$, называется \textbf{\textit{скалярным произведением}}, если выполнены 4 аксиомы:
    \begin{enumerate}[nosep]
        \item $(\vec{x}, \vec{y}) = \overline{(\vec{y}, \vec{x})}$.
        \item $(\vec{x} + \vec{y}, \vec{z}) = (\vec{x}, \vec{z}) + (\vec{y}, \vec{z})$ - аддитивность по первому аргументу.
        \item $(\alpha \vec{x}, \vec{y}) = \alpha(\vec{x}, \vec{y})$ - однородность по первому аргументу.
        \item $(\vec{x}, \vec{x}) \geq 0$, причем $(\vec{x}, \vec{x}) = 0 \iff \vec{x} = 0$.
    \end{enumerate}
\end{definition}

\begin{definition}
    Комплексное линейное пространство с так введенным скалярным произведением называется \textbf{\textit{унитарным пространством}}.
\end{definition}

\begin{designation}
    $\mathcal{E}$, $\mathcal{U}$.
\end{designation}

\begin{example}~
    \begin{enumerate}[nosep]
        \item В линейных пространствах $\mathcal{V}_2$ и $\mathcal{V}_3 \colon (\vec{x}, \vec{y}) = |\vec{x}||\vec{y}|\cos \widehat{(\vec{x}, \vec{y})}$.
        \item В арифметическом линейном пространстве $\RR^n \colon (\vec{x}, \vec{y}) = x_1y_1 + \dots + x_ny_n$. 
        \item Линейное пространство $C[0, 1]$ всех функций, непрерывных на отрезке $[0, 1]$ становится евклидовым, если в нем ввести скалярное произведение:
        $$(\vec{f}, \vec{g}) = \int_{0}^{1} f(x)g(x) \dd x.$$
    \end{enumerate}
\end{example}

\textbf{Свойства скалярного произведения.}

Пусть $\vec{x}, \vec{y}, \vec{z}$ - произвольные векторы евклидова пространства, а $\lambda$ - действительное число.

\begin{enumerate}[label={\arabic*°.}]
    \item $(\vec{x}, \vec{y} + \vec{z}) = (\vec{x}, \vec{y}) + (\vec{x}, \vec{z}).$
    
    $(\vec{x}, \vec{y} + \vec{z}) = (\vec{y} + \vec{z}, \vec{x}) = (\vec{y}, \vec{x}) + (\vec{z}, \vec{x}) = (\vec{x}, \vec{y}) + (\vec{x}, \vec{z}).$
    
    \item $(\vec{x}, \lambda \vec{y}) = \overline{\lambda}(\vec{x}, \vec{y}).$

    $(\vec{x}, \lambda \vec{y}) = \overline{(\lambda \vec{y}, \vec{x})} = \overline{\lambda \cdot (\vec{y}, \vec{x})} = \overline{\lambda} \cdot \overline{(\vec{y}, \vec{x})} = \overline{\lambda} (\vec{x}, \vec{y}).$
    
    \item $(\vec{x}, \vec{0}) = 0.$

    $(\vec{x}, \vec{0}) = (\vec{x}, 0 \cdot \vec{0}) = \overline{0} \cdot (\vec{x}, \vec{0}) = 0 \cdot (\vec{x}, \vec{0}) = 0.$
    
    \item $(\forall \vec{y} \colon(\vec{x}, \vec{y}) = 0 )\implies \vec{x} = \vec{0}.$
    
    Возьмем $\vec{y} = \vec{x}$. Тогда $(\vec{x}, \vec{x}) = 0$. Значит, по определению $(\vec{x}, \vec{x}) = 0$.
    
    \item Любое подпространство $\mathcal{E}$ ($\mathcal{U}$) само является евклидовым (унитарным).
    
    Непосредственная проверка всез аксиом ленейного пространства и скалярного произведения.
\end{enumerate}


\newpage


\subsection{
    Понятие нормы, способы задания норм (привести три примера). 
}

\begin{definition}
    Функция, заданная на линейном пространстве $\mathcal{V}$, которая каждому вектору ставит в соответствие вещественное число, называется \textbf{\textit{нормой}}, если выполнены 3 аксиомы:
    \begin{enumerate}[nosep]
        \item $\norm{\vec{x}} \geq 0$, причем $\norm{\vec{x}} = 0 \iff \vec{x} = 0$;
        \item $\norm{\lambda \vec{x}} = |\lambda| \cdot  \norm{\vec{x}}, \thinspace \lambda \in \RR$;
        \item $\norm{\vec{x} + \vec{y}} \leq \norm{\vec{x}} + \norm{\vec{y}}$ (неравенство треугольника).
    \end{enumerate}
\end{definition}

\begin{definition}
    Линейное пространство с заданной нормой называется \textbf{\textit{нормированным}}.
\end{definition}

\begin{theorem}
    Всякое скалярное произведение в евклидовом пространстве определяет норму $\norm{\vec{x}} = \sqrt{(\vec{x}, \vec{x})}$.
\end{theorem}

\begin{proof}~

    Проверим норму с помощью трех аксиом:
    \begin{enumerate}[nosep]
        \item $(\vec{x}, \vec{x}) \geq 0 \implies$ заданная функция определена для любого вектора $\vec{x}$ евклидова пространства.
        \item $\norm{\lambda \vec{x}} = \sqrt{(\lambda \vec{x}, \lambda \vec{x})} = \sqrt{\lambda^2(\vec{x}, \vec{x})} = \sqrt{\lambda^2}\sqrt{(\vec{x}, \vec{x})} = |\lambda| \cdot \norm{\vec{x}}$.
        \item Воспользуемся неравенством Коши-Буняковского: 
        
        $(\vec{x}, \vec{y}) \leq \sqrt{(\vec{x}, \vec{x})}\cdot\sqrt{(\vec{y}, \vec{y})}$,
        
        $(\vec{x}, \vec{y}) \leq \norm{\vec{x}} \cdot \norm{\vec{y}}$.

        Используя это неравенство, получаем:

        $\norm{\vec{x} + \vec{y}} ^2 = (\vec{x} + \vec{y}, \vec{x} + \vec{y}) = (\vec{x}, \vec{x}) + 2(\vec{x}, \vec{y}) + (\vec{y}, \vec{y}) \leq (\vec{x}, \vec{x}) + 2 \norm{\vec{x}} \cdot \norm{\vec{y}} + (\vec{y}, \vec{y}) = (\norm{\vec{x}} + \norm{\vec{y}})^2 \implies \norm{\vec{x} + \vec{y}} \leq \norm{\vec{x}} + \norm{\vec{y}}$
    \end{enumerate}
\end{proof}

\subsubsection{
    Способы задания норм (привести три примера)
}

\begin{definition}
    Норма вида $\norm{\vec{x}}_2 = \sqrt{(\vec{x}, \vec{x})}$ называется \textbf{\textit{евклидовой}} $(l_2)$
\end{definition}

\begin{definition}
    Норма вида $\norm{\vec{x}}_1 = |x_1| + \dots + |x_n|$ называется \textbf{\textit{октаэдрической}} $(l_1)$
\end{definition}

\begin{definition}
    Норма вида $\norm{\vec{x}}_{\infty} = max\{|x_1|, \dots, |x_n|\}$ называется \textbf{\textit{кубической}} $(l_{\infty})$
\end{definition}


\newpage


\subsection{
    Понятие метрики.
}

\begin{definition}
    Пусть $M$ - произвольное непустое множество. Отображение декартова квадрата $M \times M$ на поле $\RR$ называется метрикой, если оно удовлетворяет трем аксиомам:
    \begin{enumerate}
        \item $\rho(\vec{x}, \vec{y}) = \rho(\vec{y}, \vec{x})$.
        \item $\rho(\vec{x}, \vec{y}) \geq 0$, причем $\rho(\vec{x}, \vec{y}) = 0 \iff \vec{x} = \vec{y}$.
        \item $\rho(\vec{x}, \vec{y}) \leq \rho(\vec{x}, \vec{z}) + \rho(\vec{z}, \vec{y})$ - неравенство треугольника.
    \end{enumerate}
\end{definition}

\begin{example}~

    \begin{enumerate}
        \item $M = \RR, \rho(\vec{x}, \vec{y}) = |\vec{x} - \vec{y}|$.
        \item $M$ - произвольное непустое множество. Тогда дискретная метрика: 
        
        $\rho(\vec{x}, \vec{y}) = 
        \begin{cases}
        1, & x \ne y \\
        0, & x = y
        \end{cases}$.
    \end{enumerate}
\end{example}


\newpage


\subsection{
    Косинус.
}

\begin{definition}
    \textbf{\textit{Косинусом угла между}} $\vec{x}$ и $\vec{y} \in \mathcal{V}$ называется величина $\cos \widehat{(\vec{x}, \vec{y})} = \frac{(\vec{x}, \vec{y})}{\norm{\vec{x}} \cdot\norm{\vec{y}}}, \thinspace \varphi \in [0, \pi]$.
\end{definition}

\begin{definition}
    Пусть $\vec{x} \ne \vec{0}$ и $\vec{y} \ne \vec{0}$. Тогда \textbf{\textit{углом}} $\widehat{(\vec{x}, \vec{y})}$ называется число $\arccos{\frac{(\vec{x}, \vec{y})}{\sqrt{(\vec{x}, \vec{x})} \sqrt{(\vec{y}, \vec{y})}}}$.
\end{definition}


\newpage


\subsection{
    *Полезные факты, которые тоже могут быть на экзамене.
}

\begin{lemma}~

    Пусть

    \begin{enumerate}
        \item $\mathcal{E}$ - $n$-мерное евклидово пространство;
        \item $\vec{e}_1, \ldots, \vec{e}_n$ - некоторый базис $\mathcal{E}$;
        \item $\vec{x}_e = \begin{pmatrix}
            x_1 \\
            \vdots \\
            x_n
        \end{pmatrix}$ - столбец координат $\vec{x}$ в базисе $\vec{e}_1, \ldots, \vec{e}_n$;
        \item $\vec{y}_e = \begin{pmatrix}
            y_1 \\
            \vdots \\
            y_n
        \end{pmatrix}$ - столбец координат $\vec{y}$ в базисе $\vec{e}_1, \ldots, \vec{e}_n$;
    \end{enumerate}

    Тогда 
    $$(\vec{x}, \vec{y}) = \vec{x}^T_e\Gamma_e\vec{y}_e.$$
    \label{lemma:lemma_1}
\end{lemma}

\begin{proof}~

    \begin{align*}
        (\vec{x}, \vec{y}) &= (\sum_{i = 1}^nx_i\vec{e}_i, \sum_{j = 1}^ny_j\vec{e}_j) = \\ 
        &=\sum_{i = 1}^n\sum_{j =  1}^n(x_iy_j(\vec{e}_i, \vec{e}_j)) = \\
        &= \vec{x}^T_e\Gamma_e\vec{y}_e,
    \end{align*}

    где $\Gamma_e$ - матрица Грама для системы векторов $\vec{e}_1, \ldots, \vec{e}_n$.
\end{proof}

\begin{corollary}~

    Пусть $e$ - ОНБ $\mathcal{E}$. Тогда матрица Грама для этого базиса является единичной. Поэтому

    $$(\vec{x}, \vec{y}) = \vec{x}^T_eE\vec{y}_e = \vec{x}^T_e\vec{y}_e = x_1y_1 + x_2y_2 + \ldots + x_ny_n.$$

    В частности,

    $$\norm{\vec{x}} = \sqrt{(\vec{x}, \vec{x})} = \sqrt{x^T_ex_e} = \sqrt{x^2_1 + \ldots + x^2_n}.$$

    \label{corollary:corollary_1}
\end{corollary}
