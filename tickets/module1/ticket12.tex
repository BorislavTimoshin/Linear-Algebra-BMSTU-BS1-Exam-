\section{
    Проекция вектора на подпространство вдоль другого подпространства. Расстояние от вектора до подпространства и угол между вектором и подпространством (для случая евклидовых пространств). 
}

Рассмотрим $\mathcal{L} = \mathcal{L}_1 \oplus \mathcal{L}_2$.

\subsection{
    Проекция вектора на подпространство вдоль другого подпространства.
}

\begin{definition}
    \textit{\textbf{Проекцией вектора $\vec{x}$ на подпространство $\mathcal{L}_1$ вдоль подпространства $\mathcal{L}_2$}} называется вектор $x_1$ из представления $\vec{x} = \vec{x_1} + \vec{x_2}$, $\vec{x_1} \in \mathcal{L}_1, \vec{x_2} \in \mathcal{L}_2$.
\end{definition}

\subsection{
    Расстояние от вектора до подпространства и угол между вектором и подпространством (для случая евклидовых пространств). 
}

\begin{definition}
    \textit{\textbf{Расстоянием от вектора $\vec{a}$ до подпространства $\mathcal{L}_1$}} называется норма вектора, опущенного из конца вектора $\vec{a}$ на линейное подпространство $\mathcal{L}$ и ортогонального ему, т.е. норма ортогональной составляющей вектора $\vec{a}$ относительно подпространства $\mathcal{L}$.
\end{definition}

\begin{definition}
    \textit{\textbf{Углом между вектором $\vec{a}$ и подпространством $\mathcal{L}_1$}} называется угол между вектором $\vec{a}$ и его ортогональной проекцией на подпространство $\mathcal{L}_1$. (имеется в виду векторной ортогональной проекцией, не скалярной, как в АГ).

    $\varphi = \widehat{\vec{a}, \mathcal{L}} = \widehat{\vec{a}, \vec{b}} = \arccos(\cos(\widehat{\vec{a}, \vec{b}})) = \arccos \frac{(\vec{a}, \vec{b})}{\norm{\vec{a}} \cdot \norm{\vec{b}}}$, где $\vec{b}$ - ортогональная проекция вектора $\vec{a}$ на подпространство $\mathcal{L}_1$.
\end{definition}
