\section{
    Евклидовы и унитарные пространства. Три примера. Неравенство Коши-Буняковского (Шварца).
}

\subsection{
    Евклидовы и унитарные пространства. Три примера.
}

\textbf{Примеры $\mathcal{E}$:}
\begin{enumerate}
    \item В линейных пространствах $\mathcal{V}_2$ и $\mathcal{V}_3 \colon (\vec{x}, \vec{y}) = |\vec{x}||\vec{y}|\cos \widehat{(\vec{x}, \vec{y})}$.
    \item В арифметическом линейном пространстве $\RR^n \colon (\vec{x}, \vec{y}) = x_1y_1 + \ldots + x_ny_n$. 
    \item Линейное пространство $C[0, 1]$ всех функций, непрерывных на отрезке $[0, 1]$ становится евклидовым, если в нем ввести скалярное произведение:
    $$(\vec{f}, \vec{g}) = \int_{0}^{1} f(x)g(x) \dd x.$$
\end{enumerate}

\textbf{Примеры $\mathcal{U}$:}

\begin{enumerate}
    \item $\CC^n \colon (\vec{x}, \vec{y}) = x_1\overline{y}_1 + x_2\overline{y}_2 + \ldots + x_n\overline{y}_n$.
\end{enumerate}


\newpage


\subsection{
    Неравенство Коши-Буняковского (Шварца).
}

\begin{theorem}
    Для любых векторов $\vec{x}, \vec{y} \in \mathcal{E}$ (или $\mathcal{U}$) справедливо неравенство Коши-Буняковского
    $$|(\vec{x}, \vec{y})|^2 \leq (\vec{x}, \vec{x}) (\vec{y}, \vec{y}),$$
    причем $|(\vec{x}, \vec{y})|^2 = (\vec{x}, \vec{x}) (\vec{y}, \vec{y}) \iff \vec{x} \parallel \vec{y}.$
\end{theorem}

\begin{corollary}~

    В случае линейного арифметического пр-ва $\RR^n$ неравенство Коши-Буняковского трансформируется в \textbf{неравенство Коши}:
    $$(a_1b_1 + \ldots + a_nb_n)^2 \leq (a_1^2 + \ldots + a_n^2)(b_1^2 + \ldots + b_n^2).$$
    Равенство достигается при линейной зависимости векторов, т.е. $\frac{a_i}{b_i} = const, \quad i = \overline{1, n}$.
\end{corollary}

\begin{corollary}~

    В евклидовом пространстве $C[0, 1]$, скалярное произведение в котором выражается определенным интегралом, неравенство Коши-Буняковского превращается в неравенство Буняковского-Шварца:
    $$\left( \int_0^1 f(x)g(x) \, dx \right)^2 \le \left( \int_0^1 f(x)^2 \, dx \right) \left( \int_0^1 g(x)^2 \, dx \right).$$
\end{corollary}


\newpage


\subsection{
    Неравенство Коши-Буняковского (доказательство для $\mathcal{U}$).
}


\begin{proof}~
    
    При $\vec{x} = \vec{0}$ обе части неравенства равны нулю, значит, неравенство выполняется. Отбрасывая этот очевидный случай, будем считать, что $\vec{x}, \vec{y} \ne \vec{0}$, $(\vec{x}, \vec{y}) \ne 0$, $t \in \CC$.
    
    $$(\vec{x} + t\vec{y}, \vec{x} + t\vec{y}) = (\vec{x}, \vec{x}) + \underbrace{t(\vec{y}, \vec{x}) + \overline{t}(\vec{x}, \vec{y})}_{\star} + t \cdot \overline{t}(\vec{y}, \vec{y}) = (\vec{x}, \vec{x}) + \underbrace{2 \text{Re}(t(\vec{y}, \vec{x}))}_{\star} + |t|^2(\vec{y}, \vec{y}) = \underbrace{(\vec{y}, \vec{y})|t|^2}_{\in \RR} + \underbrace{2|(\vec{x}, \vec{y})||t|}_{\in \RR} + \underbrace{(\vec{x}, \vec{x})}_{\in \RR}.$$

    Выберем $t$ так, чтобы $\arg{t} = -\arg{(y, x)}$, тогда $\arg{(y, x)} = -\arg{t}$.

    \begin{gather*}
        t = |t|e^{i\varphi} \\
        \text{Так как }\arg{(y, x)} = -\arg{t}, \text{ то } \\
        (y, x) = |(x, y)|e^{i(-\varphi)}.
    \end{gather*}

    Итак, 
    
    $$2 \cdot \text{Re} (t \cdot (\vec{y}, \vec{x})) = 2 \cdot \text{Re} (|t|e^{i\varphi} \cdot |(\vec{x}, \vec{y})|e^{-i\varphi}) = 2|t||(\vec{y}, \vec{x})|.$$

    Тогда

    \begin{gather*}
        \underbrace{(\vec{y}, \vec{y})|t|^2}_{\in \RR} + \underbrace{2|(\vec{x}, \vec{y})||t|}_{\in \RR} + \underbrace{(\vec{x}, \vec{x})}_{\in \RR} \geq 0 \\
        D = |(\vec{x}, \vec{y})|^2 - (\vec{x}, \vec{x})(\vec{y}, \vec{y}) \leq 0.
    \end{gather*}

    \bigbreak
    
    $\text{\textbf{Примечание} } (\star) \colon$ Пусть $t = p + iq, (\vec{y}, \vec{x}) = a + ib$, $\overline{t} = p - iq, \overline{(\vec{y}, \vec{x})} = a - ib$. Тогда

    \begin{enumerate}
        \item \begin{align*}
            t(\vec{y}, \vec{x}) &+ \overline{t} \cdot (\vec{x}, \vec{y}) = t(\vec{y}, \vec{x}) + \overline{t} \cdot \overline{(\vec{y}, \vec{x})} = \\
            &= (p + iq)(a + ib) + (p - iq)(a - ib) = \\ 
            &= pa + pib + aiq - qb + pa - pib - aiq - qb = 2pa - 2qb = \\
            &= 2 \cdot (pa - qb).
        \end{align*}
        \item \begin{align*}
            2 \cdot \text{Re} (t \cdot (\vec{y}, \vec{x})) &= 2 \cdot \text{Re}((p + iq)(a + ib)) = \\
            &= 2 \cdot \text{Re}(pa - qb + i(pb + aq)) = \\
            &= 2 \cdot (pa - qb).
        \end{align*}
        \item \begin{align*}
            t(\vec{y}, \vec{x}) + \overline{t} \cdot (\vec{x}, \vec{y}) = 2 \cdot \text{Re} (t \cdot (\vec{y}, \vec{x})).
        \end{align*}
    \end{enumerate}
\end{proof}


\subsection{
    Неравенство Коши-Буняковского (доказательство для $\mathcal{E}$).
}


\begin{proof}~

    
\end{proof}
